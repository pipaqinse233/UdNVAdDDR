Den Auftrag,in weiterer Angleichung an die Uni-
formen der Armeen des Warschauer Vertrages das
äußere Bild der Offiziere,Generale und Admirale
der NVA in der Offentlichkeit,besonders zu Fest-
veranstaltungen, Empfängen und bei der Wahrneh
mung von Aufgaben auf protokollarischem Gebiet.
attraktiver zu gestalten, hatte der inister für Na-
tionale Verteidigung seinem Stellvertreter und Chef
Rückwärtige Dienste,Generalleutnant H. Poppe
am 4.April 1975 erteilt, Bei der Aufgabenstellung
ging der Minister für Nationale Verteidigung davon
aus, daB die angestrebte Vervollkommnung der
Uniformierung aufgrund der verfügbaren ökono
mischen Potenzen der Volkswirtschaft der DDR
möglich sei.
Uniformen mit größerem Repräsentationsvermó
gen einzuführen hatte noch weitere Gr¼nde. Im Er
gebnis der weltweiten Anerkennung der DDR ver-
sahen immer mehr Angehörige der NVA ihren
Dienst in diplomatischen Vertretungen der DDR
im Ausland, Ihnen eine entsprechende Gesell-
schaftsuniform zu geben, die nicht Sonderuniform,
sondern die Uniform aller Offiziere der NVA war,
entsprach internationalen Gepflogenheiten.
Die Einf¼hrung der Gesellschaftsuniform zum
20.lahrestag der Nationalen Volksarmee wurde mit
dem Befehl Nr.9/75 des Stellvertreters des Mini-
sters für Nationale Verteidigung und Chefs Rück
wärtige Dienste vom 21.0ktober 1975 geregelt. Da.
nach waren Generale,Admirale und Offiziere mit
einer Gesellschaftsjacke und Halbschuhen aus
Lackleder auszustatten.Tragebeginn für Generale
Admirale,Offiziere im Auslandsdienst und Offi
ziere auf protokollarischem Gebiet war der 20.Fe
bruar 1976.Die Ausstattung der übrigen Offiziere
erfolgte etappenweise, wobei weibliche Offziere bei
der Ausstattung noch nicht berücksichtigt wurden.
Die neue niformart Gesellschaftsuniform trugen Offiziere,Generale und Admirale zu Festveran.
staltungen, offiziellen Empfangen,Theater- und
Konzertbesuchen und auf Weisung zu besonderen
Veranstaltungen.Sie bestand jahreszeitlich variier
aus Schirm-bzw,Wintermütze,bei den Fallschirm
jägern aus der orangefarbenen Baskenmütze in der
vorhandenen Ausführung, dem Uniformmantel. de
Gesellschaftsjacke,der steingrauen niformhose
weißem Oberhemd,dunkelgrauem- bei der Volks.
marine schwarzem-Binder,schwarzen Lackschu
hen sowie der Achselschnur.
Prinzipiell neu waren Gesellschaftsjacke, Ach.
selschnur und Lackschuhe.Die Gesellschaftsiacke
für Generale und Admirale wurde aus hellgrauer
Gabardine gefertigt. Die zweireihige Jacke mit
einem Schlieknopf und einem Blindknopfpaar
zwei geraden Seitentaschen mit Patte, glatten Ar
meln mit Schlitz und drei Knöpfen wiesen für Ge.
nerale am Kragen Biesen und Kragenspiegel mit
Arabesken auf. Die acke der Admirale war mit
goldfarben gestickten Eichenlaubranken auf dun
kelblauer Tuchunterlage auf dem Kragenrevers und
Dienstgradabzeichen auf hellgrauer Tuchunterlage
auf beiden Armeln versehen.Die Gesellschaftsiacke
für Offiziere der Landstreitkräfte und der LSK/LV
war aus graugrüner Gabardine mit einem Anteil
von 45 Prozent Wolle und 55 Prozent Polyester ge-
arbeitet. Sie war ebenfalls zweireihig mit einem
Schließknopf und einem Blindknopfpaar, zwei ge
raden Seitentaschen mit Patte und Doppelpaspel,
glatten Armeln mit Schlitz und drei Zierknöpfen
gefertigt. Die Farbe der Biesen am Kragen und der
Kragenspiegel war bei Offizieren der Landstreit-
kräfte Weiß, bei den Angehörigen der LSK/LV
Hellblau.
Der Schnitt und die aterialzusammensetzung
der Gesellschaftsjacke für Offiziere der Volksma
rine entsprach der der anderen Teilstreitkräfte.Of
fiziere der Volksmarine führten iedoch auf beiden
Armeln ihrer cremefarbenen Geselschaftsiacke aul
gleichfarbigem Untergrund ihre goldfarbenen Tres.
sen -von Naht zu Naht angebracht- als Dienst
gradabzeichen.
Generale, Admirale und Offziere legten zur Pa-
radeuniform, auf besonderen Befehl zur Gesell-
schaftsuniform und nach eigenem Ermessen zur
Ausgangsuniform die neu eingeführte Achselschnur
an, Sie war sehr dekorativ und bestand für Gene-
rale und Admirale aus zwei goldfarbenen, dickge-
flochtenen Schnüren sowie zwei dünneren goldfarbenen Doppelschnüren aus glatten Flirettfäden, dic
in zwei goldfarbenen Metallspitzen ausliefen. Offi
ziere erhielten Achselschnüre aus silberfarbenem
glattem Aluminiumgespinst. Die Metallspitzen wa
ren versilbert,bei ffizieren der Volksmarine ver
goldet.
Ebenfalls am 1.März 1976 wurden Angehörigc
der Ehrenkompanien des Wachregiments Berlin
der Ehreneinheit des Wachregiments 2, des Zentra.
len Orchesters der NVA,des Stabsmusikkorps Ber
lin und des Spielmannszuges des Stabsmusikkorps
Berlin für zentrale militärische Zeremonielle mit
Repräsentationsschnüren ausgestattet. Die anderer
Stabsmusikkorps erhielten diese Schnüre 1977. Di
Repräsentationsschnüre,die nicht von ffizieren
getragen wurden,waren aus silberfarbenem Alumi
niumgespinst gefertigt und liefen in silbernen Ei
cheln aus. Sie wurden auf der rechten Sèite der Pa
rade-/Ausgangsjacke angebracht. Wahrend der
Feierlichkeiten zum 20.]ahrestag der NVA trugen
die obengenannten Armeeangehörigendie neuen
Repräsentationsschnüre erstmals in der @ffentlich.
keit.
Mit seiner Anordnung Nr.7/76 vom 28.Mai
1976 traf der Stellvertreter des Ministers für Natio
nale Verteidigung und Chef RÃückwärtige Dienste
weitere Festlegungen im,Zusammenhang mit de
Ausstattung des Offizierskorps und der Generale
und Admirale der NVA mit Gesellschaftsuniformen
sowie zurlrageweise einzelner Uniformstücke.
Es wurden die weiteren Etappen der Ausstattung
aller Offiziere mit der Gesellschaftsiacke und
schwarzen Halbschuhen festgelegt,und es wurd
angewiesen,daß bis zum 30.[uni 1976 eine Gesell
schaftsuniform für weibliche Offiziere der NVA zu
entwickeln ist. Dazu wurden dem Minister für Na
tionale Verteidigung am 28. Mai des Jahres dre
Versionen von Gesellschaftsuniformen für weibli.
che Offiziere der NVA vorgeführt, die sich im Zu
schnitt,der Taschenausstattung und dem Uniform
zubeh'r unterschieden.Im Ergebnis dieser Vorfiih
rung bestätigte Armeegeneral H,Hoffmann die
Gesellschaftsuniform für weibliche Offiziere der NVA. Sie bestand aus einer einreihigen Jacke (auf
drei Kn¶pfen zu schlieBen), die schräge Seitenta-
schen mit Patten aufwies, einer weißen Bluse mit
zweiteiligem Kragen und 9cm langen Kragenek-
ken, einem steingrauen bzw.dunkelblauen Rock
ohne Falte und Lacklederpumps mit 8 cm hohem
Absatz.
Die Gesellschaftsjacken der weiblichen Offiziere
der Landstreitkräfte, der Luftstreitkräfte und der
Luftverteidigung hatten im Unterschied zu den Jakken der männlichen Offiziere an den Kragen keine
Paspelierung, dafür aber Kragenspiegel. Die Farbe
der Jacken war Graugrün, Weibliche Offiziere der
Volksmarine hatten an ihren cremefarbenen ]acken
keine Kragenspiegel und auch keine Ärmelstreifen.
Im Ergebnis der Beratung wurde festgelegt, daß
zur Ausstattung der weiblichen Offiziere keine Ach-
selschnur gehört. Die Ausgangsuniform für weibli-
che Offiziere wurde ab 1. März 1978 getragen.
Im Zusammenhang mit der Einführung der Ge-
sellschaftsuniform durfte von Generalen die zwei-
reihige Ausgangsjacke und von Admiralen, Militär-,
Luftwaffen- und Marineattachés, deren Gehilfen sowie von Offizieren, die auf protokollarischem Ge-
biet tätig waren, die weiBe Uniformjacke nicht
mehr getragen werden, Generale trugen anstelle der
bisherigen Ausgangsuniform ab 1. Juni 1976 die
Stabsdienstuniform mit weißem Hemd. Eingeführt
wurden weiße Uniformjacken für Matrosen, Maate,
Meister, Fahnriche und Offiziere der Musikkorps
der Volksmarine.
Bereits seit dem 15.M¤rz 1975 verkauften die
Spezialausstatter in den Bezirksstadten und die
Verkaufseinrichtungen in milit¤rischen Objekten
neue kombinierfahige Interimsspangen. Anstelle
der Ordensb¤nder aus textilem Material wurden
nun farbechte Darstellungen der Ordensbänder auf
Kunstdruckpapier, geschützt durch eine Abdeck-
platte aus Piacryl, verwendet. Die neuen Interims.
spangen waren zweckmäßiger, da es jetzt unkompli-
ziert möglich war, die Interimsspangen selbst
variabel zusammenzustellen. Sie sind auch heute
noch in Gebrauch.