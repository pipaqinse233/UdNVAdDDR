

Bereits in den ersten Jahren der Existenz der NVA
war die Taätigkeit des B/A-Dienstes darauf gerich-
tet, notwendige und zweckmäßige Verbesserungen
der Uniformierung der Armeeangeh¤rigen in Ein-
klang mit der strengsten Einhaltung der Sparsam-
keitsprinzipien zu bringen,So berichtete die Zei-
tung «Volksarmee» in ihrer Ausgabe vom 25.]anuar
1958 über Arbeitsmethoden des B/A-Dienstes, die
die Umarbeitung von abgetragenen Drillichunifor-
men in Arbeitsuniformen ermöglichten. .Jene Initia-
toren erbrachten Einsparungen in Höhe von na-
hezu 30 000 Mark.

In einer durch den B/A-Dienst im April 1960
vorgenommenen Analyse konnte eine positive Bi-
lanz gezogen werden.Durch zielstrebige Gemein-
schaftsarbeit und durch die Verwirklichung der Ra-
tionalisatorenvorschläge aus der Truppe,
der
Abteilung B/A und den Betrieben wurden bei der
Fertigung von Uniformen allein im Jahre 1959 ins-
gesamt 51 268 Arbeitsstunden, 21 959 m? Uniform-
und 34 746 m? Futterstoffe sowie 384 099 Mark ein-
gespart. Der gesamte ökonomische Nutzen lag bei
1 Million Mark, da die genannten Vorschläge auch
bei der Verarbeitung der Dienstbekleidung anderer
bewaffneter Organe der DDR übernommen werden
konnten.
Erste Maßnahmen, um unter den gegebenen öko-
nomischen Möglichkeiten auch den Soldaten und
Unteroffizieren verbesserte, bequemere und modi-
schere Ausgangsuniformen anzubieten, wurden Ende 1958 eingeleitet. Vom 20. bis zum 28.Septem-
ber 1958 fand die l.Sommerspartakiade der Armee.
sportler der befreundeten sozialistischen L¤nder in
der DDR statt. Bei der Eroffnungsveranstaltung im
Leipziger Zentralstadion trat die Mannschaft der
NVA ausnahmslos in einer Ausgangsuniform mi
offener,zweireihiger Ausgangsjacke an. Von diesem
Zeitpunkt an bestand für die Soldaten und Unter
offiziere die Möglichkeit,sich eine eigene Aus
gangsuniform aus dem Stoff der Offiziersuniform
anfertigen zulassen und zum Ausgang zu tragen
Der Stoff mußte gekauft werden.
Die Ausgangsjacke wurde auf der Grundlage der
Herstellungsvorschrift xAusgangsrock, offen,zwei
reihig fiir Offiziere》 ohne farbige Kragenpaspelie
rung gefertigt. Die dazu passende Uniformhose ent
sprach der Herstellungsvorschrift der Uniformhosc
für Soldaten und Unteroffiziere,Zur Ausgangsjacke
gehörten bei denLandstreitkräften silberfarbene
Kragenspiegel und Armelplatten fiir Soldaten und
Unteroffiziere, Hinzu kamen ein silbergraues Ober
hemd und ein dunkelgrauer Binder. Der Kragen
des Uniformmantels konnte zum Ausgang geöffnet
werden.
Eine weitere Verbesserung betraf die Bekleidung
des Oberdeckpersonals der Torpedoschnellboote
und konnte 1960 realisiert werden. Nach einer etwa
zweiahrigen Entwicklungs-undErprobungszeit
wurden zweiteilige graue Spezialanziige aus Dede
ronmischgewebe mit PVC-Beschichtung und Web.
pelzfitterung eingefihrt,Wahrend der Erprobung
waren die Anzüge den unterschiedlichen,fiir die
Witterungsbedingungen und
Ostsee typischen
Lufttemperaturen zwischen maximal plus1°
und minus 7,4°C ausgesetzt. Diese neue Spezialbe.
kleidung war günstiger als die bis dahin an Bord
übliche Lederbekleidung und die Watteanzüge, die
sich bei Regen oder überkommendem Wasser voll.
saugten,dann schwer und unhandlich wurden und
die Bedienung von Waffen und Gerät erschwerten
Die 1960 cingeführten Spezialanzüge einschlieB
lich Unterleibsschutz mit Steppwatte fiir den Win.
ter- dies ist dem AbschluBbericht des Stellvertreters des Chefs der Seestreitkräfte für Rückwärtige
Dienste,Kapitän zur See WEhm,vom 4. M¤rz
1959 zu entnehmen - boten auch bei l¤ngerer und
intensiver Nässeeinwirkung guten Wind- und Wet-
terschutz, ausreichende W¤rmehaltung in allen .Jah-
reszeiten und genügende Luftdurchlässigkeit, Zu-
dem waren sie bequem, sehr haltbar, widerstanden
der Einwirkung von Ölen und Treibstoffen und wa-
ren leicht zu reinigen. Durch die Einführung der
Spezialanzüge entfielen auf den TS-Booten fol-
gende Bekleidungsstücke: Sidwester, Wattejacke,
Wattehose, Ölzeugmantel, Ölzeugjacke, Ölzeug
hose und Pelzmütze. Diese Neuerung war damit
sehr rationell und verbesserte durch höhere Ge-
brauchseigenschaften vor allem die Gefechtsbereit-
schaft der Stoßkräfte der Seestreitkräfte der NVA
mit den spezifischen Mitteln, die dem B/A-Dienst
zur Verfügung standen, Allerdings stellte der Spe-
zialanzug für die TS-Boot-Besatzungen nur einen
Wetterschutzanzug dar und war somit kein Vorläu-
fer des Kampfanzuges der Volksmarine. Er hatte
keine Schwimmweste und bot keinen Schutz vor
Massenvernichtungsmitteln.