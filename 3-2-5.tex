

Am 1.Mai 1975 war damit begonnen worden, Ar
meeangeh¶rige etappenweise mit verbesserten Trai.
ningsanzügen in der Grundfarbe Braun auszustat
ten,Generale und Admirale trugen diese neuen
Trainingsanzüge ab 1.Mai 1975,weibliche Armee.
angehörige, Offiziere auf Zeit, Berufsunteroffiziere.
Fahnriche und Berufsoffiziere ab 1.Mai 1977. Soldaten im Grundwehrdienst, Soldaten auf Zeit,Un.
teroffiziere und Offiziersschüler erhielten etappen.
weise in der Zeit vom 1.Mai 1977 bis zum 1. Mai
1978 die neue Sportbekleidung.
Für Soldaten im Grundwehrdienst, Soldaten auf
Zeit und Unteroffiziere auf Zeit bestand der neue
zweiteilige Trainingsanzug aus einem belastbaren
Baumwollgewebe.Die Jacke war mit Rollkragen
halbhohem, nicht durchgängigem ReiBverschluB
einer Brusttasche links mit Reißverschluß und rot-
gelben Armelstreifen l¤ngs der Naht gearbeitet. Die
dazugehörige moderne Keilhose mit Steg und
Strickbund an den Beinen hatte zwei in die Seiten-
nähte eingearbeitete Taschen.
Die Trainingsanzüge für Berufssoldaten waren
aus atmungsaktivem synthetischem Material herge
stellt.Im Unterschied zu den Trainingsjacken der
Soldaten und Unteroffiziere im Grundwehrdienst
und auf Zeit hatten die der Berufssoldaten einen
durchgehenden Reißverschluß. Die Trainingshosen
der Berufssoldaten zierten lampassenartige rotgelbe
Streifen an den Beinen. Sie konnten mit ReiBver.
schlüssen an den Beinen geschlossen werden. An
den neuen braunen Trainingsanzügenwurden
keine Dienstgradabzeichen getragen, dafür einheit
lich auf der linken Brustseite das Emblem der Ar.
meesportvereinigung Vorwärts》angebracht.Nur
die unterschiedliche
Gestaltung der Trainingsan
züge wies die Träger als Soldaten im Grundwehr
dienst oder auf Zeit, als Unteroffiziere auf Zeit oder
Berufssoldaten aus.
Die etappenweise Einführung des neuen Trainingsanzuges und der damit verbundene Wegfall
der Dienstgradabzeichen am Trainingsanzug wur-
den in der 1.Ãnderung der Uniformvorschrift, Aus-
gabejahr 1977, vom 1.Juli 1978 berücksichtigt. Nur
die Träger von graumelierten Trainingsanzügen
hatten die Pflicht, sich dienstgradmäßig auszuwei-
sen. Unteroffiziere n¤hten in der Mitte des linken
Oberärmels der Trainingsjacke eine mattsilber-
graue 10 cm lange und 7 mm breite Tresse auf. Der
Abstand von der Schulternaht betrug 18 cm, Für
Maate und Meister der Volksmarine galt bis zur
Ausstattung mit neuen Trainingsanzügen, daß sie
am Trainingsanzug einen gewebten klaren Anker
auf ovaler Unterlage aus blauem Stoff trugen.
F¤hnriche und Offiziere der Volksmarine verwen-
deten das gleiche Symbol, jedoch mit einer gelben
Umrandung.
Es war für die Armeeangehörigen in dieser Zeit
schon selbstverständlich, daß sie die gesamte Sportbekleidung einschlieBlich der festgelegten Ergän-
zungen kostenlos erhielten und sie ihnen bei der
Beendigung der aktiven Dienstzeit übereignet
wurde. Soldaten auf Zeit, Unteroffiziere auf Zeit
und Offiziersschüler erhielten nach jeweils 18 Mo-
naten einen neuen Trainingsanzug,Offizieren auf
Zeit, Berufsunteroffizieren, F¤hnrichen und Berufs
offizieren in Truppenteilen, Verbänden, Unteroffi-
ziersschulen und Offiziershochschulen wurde nach
2 Jahren ein neuer Trainingsanzug ausgehändigt. Weibliche Armeeangehörige, Offiziere auf Zeit, Be-
rufsunteroffiziere, Fahnriche und Berufsoffiziere in
St¤ben und Einrichtungen, außer den obengenann-
ten, bekamen ihre Sportbekleidung nach 3 Jahren
erneuert.