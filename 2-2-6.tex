

W¤hrend des gemeinsamen Manövers «Quartett»
der NVA, der Polnischen Armee, der Sowjetarmee
und der Tschechoslowakischen Volksarmee, das in
der Zeit vom 9,bis zum 14.September 1963 im
Rahmen der Vereinten Streitkräfte der Teilnehmer-
staaten des Warschauer Vertrages im Süden der
DDR durchgeführt wurde, traten NVA-Angehö-
rige, die Aufgaben der Ordnung und Sicherung zu
erfüllen hatten, mit besonderen Kennzeichnungen in Erscheinung. Sie waren mit weißen Koppeln, mit
ebensolchen Schulterriemen,weiBen Armbinden
mit der sichtbaren Aufschrift <Wachregiment》 und
einem weißen Streifen an Schirmmütze und Stahl-
helm gekennzeichnet. Auch Fotos der abschlieBen-
den Parade zum Manöver Oktobersturm》- es
fand zwischen dem 16.und 22.Oktober 1965 in der
DDR und mit Verbänden und Truppenteilen der-
selben vier Bruderarmeen statt - zeigen derartig ge-
kennzeichnete Angehörige der NVA als Sicherungs-
posten am Straßenrand.

Der Minister für Nationale Verteidigung der DDR bestimmte mit seinem Befehl Nr. 2/66 vom
10.Januar 1966 endgültig die Einführung von Be
kleidung und Ausrüstung für Angehörige der NVA.
die besondere Ordnungs- und Sicherungsaufgaben
zu erfüllen haben. An Bekleidung und Ausrüstung
kamen zusätzlich hinzu: weißes Koppel mit Schul
terriemen, weiBe Pistolentasche für Pistolenträger,
weiße Handschuhe für die Angeh¶rigen von Ehren-
einheiten und -wachen,weiße Stulpenhandschuhe
für Kradfahrer,ein Stahlhelm mit einem 40 mm
breiten weißen Streifen, der 50 mm von der Unter.
kante des Helmes entfernt angebracht war, und für
Soldaten und Unteroffiziere eine Schirmmütze mit
einem 40 mm breiten weißen Band sowie für Offi.
ziere eine Schirmmütze mit weißem Mützenbezug
Diese zus¤tzlichen Uniform- und Ausrüstungsteile
waren nur zur Parade-und zur Dienstuniform zu
tragen,Bezüglich des Stahlhelms ist anzumerken,
daà die auf Befehl des Ministers für Nationale Ver
teidigung der DDR eingesetzten Armeeangehörigen
vorn in der Helmmitte das KD-Emblem(KD=
Kommandantendienst)anzubringen hatten.
Diese neuen,zusammengefaBten Regelungen wa
ren in die DV-10/5 und die DV-98/4 in den Ausga.
ben von 1965 mit der Ergänzung vom 27.Oktober
1966 eingearbeitet worden. In der DV-98/4. Beklei.
dungs- und Ausristungsnormen der Nationalen
Volksarmee waren ebenfalls die zus¤tzlichen Beklei-
dungs- und Ausrüstungsstücke für die Angehörigen
der Kommandantendiensteinheiten aufgefihrt. Die
nebenstehende Tabelle gibt darüber Auskunft. 