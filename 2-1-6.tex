Die Uniformierung der NVA unterlag in allen Teil
streitkräften bestimmten witterungsbedingten und
klimatischen Einflüssen, Diesen hatte die Armee
führung bereits bei der Sicherung bestimmter mili.
t¤rischer Tätigkeiten,z.B. des Wachdienstes im
Winter, des Flugdienstes von Flugzeugfihrern in
der kalten Jahreszeit und von Arbeiten des Ober-
deckpersonals von Schiffen und Booten, durch ent-
sprechende Bekleidung Rechnung getragen. An-
fang der 6er Jahre waren die ökonomischen
Möglichkeiten gegeben und die berlegungen so
weit gediehen, alle Armeeangehörigen mit zweck-
mäßigerer Winterbekleidung auszustatten.
Ab 1961 erhielten die Angehörigen der Teilstreit.
kräfte schrittweise steingraue bzw, dunkelblaue
Watteanzüge vor allem als Felddienstuniform für
die Winterperiode.Der Watteanzug bestand aus
Jacke und Hose, gearbeitet aus Baumwoll-Zelt
bahngewebe.Beide waren mit Steppwattine und
Kunstseide abgefüttert. Die Verarbeitung gewahr-
leistete bei geringem Gewicht und Volumen eine
hohe Wärmehaltung und eine große Beweglichkeit.
Waren Ausristungsgegenstände mitzuführen,
wurde das Tragegestell angelegt. Bis Ende 1965
konnte die erstmalige Ausstattung mit diesen Wat-
teanzügen in den Landstreitkräften und den
LSK/LV vollständig und in der Volksmarine zu
80 Prozent abgeschlossen werden. Vereinzelt vor-
handene blaue Watteanzüge mit Karosteppmuster
wurden aufgetragen.Weiterhin konnte nun der
zweite Uniformmantel der Soldaten entfallen.

In der Volksmarine zogen die Matrosen, Maate,
Meister und Offiziere der fahrenden Einheiten der
Watteanzug unter ihren speziellen Kampfanzug
wenn die dazu sonst verwendete blaue Arbeitsuni.
form nicht mehr ausreichend vor derKalte
schützte.Die Angehörigen der Landeinheiten tru
gen den Watteanzug als Felddienstuniform sowie
auch als Dienstuniform,wenn die Temperaturen
unter minus 3'C sanken, iedoch nicht im Streifen-
Stabs-und Lazarettdienst.
Ebenfalls noch 1961 erhielten die Schiffsoffiziere
der Volksmarine einen kurzen Wattemantel,d.h
einen zweireihigen Mantel mit einknöpfbarer Web
pelzf¼tterung und Webpelzkragen, als zusätzliche
Wetterschutzbekleidung.
In die Ausstattung der Armeeangehörigen wurde
in diesem Jahr der Pullover aufgenommen. Seinc
Materialzusammensetzung aus je 40 Prozent Woll
und Wolpryla sowie 20 Prozent Zellwolle gewährlei.
stete gute Trageeigenschaften. Soldaten und Unter
offiziere erhielten Pullover mit spitzem Ausschnitt.
das fliegende Personal hochgeschlossene Pullover
und die Matrosen und Maate Pullover mit Rollkra
gen.
Nach 1961 wurde der schlauchförmige Kopf
schützer durch einen solchen in Haubenform er
setzt,Er hatte eine Materialzusammensetzung von
60 Prozent Wolpryla, 20 Prozent Wolle und 20 Pro-
zent Zellwolle.Erwognurnoch60statt
100Gramm.Insgesamt bot der neue Kopfschützer
für die Angeh¶rigen aller Waffengattungen cinen
besseren Kalteschutz,da der Kopf vollständig be
deckt wurde.
In der ersten H¤lfte der 60er |ahre ver¤nderte jc
doch ein anderes Uniformstück das ¤uBere Erschei
nungsbild aller Armeeangehörigen in der Winterpe
riode: die neue Wintermütze. Sie wurde bis 1965
aufgrund des Befehls Nr.86/63 des Ministers für
Nationale Verteidigung der DDR vom 2. Oktober
1963 schrittweise eingeführt, Die Entwicklung und
die Erprobung dieser neuen Wintermitze began
nen bereits Ende der 50er Jahre, als sich immer kla.
rer herausstellte, daß die bis dahin verwendete Wintermütze nicht mehr den Anforderungen entsprach,
da sie bei strengen Frösten nur unzureichenden
Schutz bot. Auch wurde die alte Wintermütze beim
Mitführen im Sturmgepäck- und dies geschah.
wenn der Stahlhelm aufzusetzen war- stark defor
miert, Trotzdem trugen die Angehörigen der Land.
streitkräfte und der LSK/LV der NVA die Skimüt-
zen noch auf,ebenfalls das Deck-und das
Briickenpersonal auf Schiffen und Booten der
Volksmarine ihre bisherige Pelzmitze.
Die neue Wintermütze,die in ihrer äuBeren
Form sowjetischem Vorbild folgte,bestand im
Grundmaterial aus steingrauem bzw, dunkelblauem
Uniformtuch fiir Soldaten und Unteroffiziere und
gleichfarbigem Uniformtrikot für die Offiziere, Ge.
nerale und Admirale sowie fir die weiblichen Ar.
meeangehörigen. Die Wintermütze der Soldaten
und Unteroffiziere war mit Pelzbesatz aus Woll.
mischgarn, die der Frauen und Offiziere sowie der
Meister der Volksmarine mit Pelzbesatz aus Grisu-
ten versehen.Der Pelzbesatz der Wintermütze der
Generale und Admirale war aus Naturpersianer ge.
arbeitet. Silber-bzw, goldfarbene Metallkokarden
mit dem Staatsemblem waren an den Wintermit.
zen der Soldaten und Unteroffiziere, maschinenge
stickte Mützenkränze mit metallener Kokarde und
Staatsemblem an denen der Offiziere und handge
stickte goldfarbene Kränze,Kokarde mit Staatsem.
blem aus Tombak,alles auf hochroter, hell- oder
dunkelblauer Unterlage,an denen der Generale
und Admirale angebracht.
Die Bestimmungen zur Verwendung der neuen
Wintermüitze sahen vor, diese innerhalb der Win
tertrageperiode (1.November bis 15, April)in de
Zeit vom 1.Dezember bis zum 15.Februar aufzuset
zen, Bei Temperaturen von unter minus 10 °C durf
ten die Ohrenklappen -in geschlossenen Forma-
tionen 'nur auf Befehl des Kommandeurs
heruntergeschlagen werden.