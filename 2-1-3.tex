

Mit dem bercits geuannten Befchl Nr.51/61 des
Ministers für Nationale Verteidigung der DDR
vom 9.August 196l erhielten zunächst auch die An-
gehörigen der Luftstrcitkrafte hellblau paspelierte
Armelpatten für die Parade- und Ausgangsuniform,
Vorher führten sie keine derartigen Patten an ihrer
Uniform.
Als gravierender erwiesen sich jedoch die Festle-
gungen iber neue Uniformen der Truppen der
Luftverteidigung.Diese drückten äuBerlich einen
interessanten EntwicklungsprozeB in der NVA aus.
Komplizierte Bedingungen fǔr die moderne Luft.
verteidigung,die aus der wissenschaftlich-techni-
schen Entwicklung mit ihren Wirkungen auf das
Militärwesen resultierten, geboten auch in dieser
Teilstreitkraft, die Kräfte und Mittel zu konzentrie-
ren.Deshalb wurden Ende 196l aus den bisherigen
Flieger- und Flakartillerieverbänden einheitliche
Luftverteidigungsdivisionen mit Fla-Raketentrup
pen, Fliegerkräften, funktechnischen Truppen, Spe-
zialtruppen und Diensten gebildet. In bezug auf die
Uniformen erachtete es die Führung der NVA als
zweckmäßig, diesen Prozeß durch eine ¤ußere An-
gleichung der Uniformen der Luftstreitkräfte und
der Luftverteidigung abzurunden.
Während der 50er Jahre trugen die Soldaten,
Unteroffiziere, Offiziere und Generale der Flakar-
tillerie Uniformen der Landstreitkräfte mit der ro-
ten Waffenfarbe der Artillerie. Jetzt erhielten die Truppen der Luftverteidigung wiederum steingraue
Uniformen, die an den Armelaufschlägen, an den
Kragen der Uniformjacken und an der Schirm
mütze hellblau wie die der Luftstreitkräfte paspe.
liert waren, Als eigentliche Waffenfarbe wurde aber
fr die Soldaten, Unteroffiziere und Offiziere der
Luftverteidigung Hellgrau für die Paspelierung der
Kragenspiegel, Schulterklappen, -stücke und Ãr
melpatten bestimmt, Fǔr Generale der Luftverteidi.
gung blieb jedoch bis Anfang der 70er Jahre Hoch.
rot die Waffenfarbe, Die Offiziere der Truppen der
Luftverteidigung trugen wie die der Luftstreitkrafte
ebenfalls schon Uniformjacken offener Fasson, d. h.
mit Uniformhemd und cinem dunkelgrauen Bin
der, Die gesamte Umkleidung der Angehörigen der
Luftverteidigung fand 1962 und 1963 statt.

Mit dem bercits geuannten Befchl Nr.51/61 des
Ministers für Nationale Verteidigung der DDR
vom 9.August 196l erhielten zunächst auch die An-
gehörigen der Luftstrcitkrafte hellblau paspelierte
Armelpatten für die Parade- und Ausgangsuniform,
Vorher führten sie keine derartigen Patten an ihrer
Uniform.
Als gravierender erwiesen sich jedoch die Festle-
gungen iber neue Uniformen der Truppen der
Luftverteidigung.Diese drückten äuBerlich einen
interessanten EntwicklungsprozeB in der NVA aus.
Komplizierte Bedingungen fǔr die moderne Luft.
verteidigung,die aus der wissenschaftlich-techni-
schen Entwicklung mit ihren Wirkungen auf das
Militärwesen resultierten, geboten auch in dieser
Teilstreitkraft, die Kräfte und Mittel zu konzentrie-
ren.Deshalb wurden Ende 196l aus den bisherigen
Flieger- und Flakartillerieverbänden einheitliche
Luftverteidigungsdivisionen mit Fla-Raketentrup
pen, Fliegerkräften, funktechnischen Truppen, Spe-
zialtruppen und Diensten gebildet. In bezug auf die
Uniformen erachtete es die Führung der NVA als
zweckmäßig, diesen Prozeß durch eine ¤ußere An-
gleichung der Uniformen der Luftstreitkräfte und
der Luftverteidigung abzurunden.
Während der 50er Jahre trugen die Soldaten,
Unteroffiziere, Offiziere und Generale der Flakar-
tillerie Uniformen der Landstreitkräfte mit der ro-
ten Waffenfarbe der Artillerie. Jetzt erhielten die Truppen der Luftverteidigung wiederum steingraue
Uniformen, die an den Armelaufschlägen, an den
Kragen der Uniformjacken und an der Schirm
mütze hellblau wie die der Luftstreitkräfte paspe.
liert waren, Als eigentliche Waffenfarbe wurde aber
fr die Soldaten, Unteroffiziere und Offiziere der
Luftverteidigung Hellgrau für die Paspelierung der
Kragenspiegel, Schulterklappen, -stücke und Ãr
melpatten bestimmt, Fǔr Generale der Luftverteidi.
gung blieb jedoch bis Anfang der 70er Jahre Hoch.
rot die Waffenfarbe, Die Offiziere der Truppen der
Luftverteidigung trugen wie die der Luftstreitkrafte
ebenfalls schon Uniformjacken offener Fasson, d. h.
mit Uniformhemd und cinem dunkelgrauen Bin
der, Die gesamte Umkleidung der Angehörigen der
Luftverteidigung fand 1962 und 1963 statt.