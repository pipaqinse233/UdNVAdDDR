

Am 1.April 1977 trat die neue Dienstvorschrift
010/0/005.Uniformen und ihre Trageweise in
Kraft.

Das nunmehr gültige Dokument,in dem alle
Veränderungen seit 1972 eingearbeitet worden wa
ren, enthielt die Ausstattung der Armeeangehörigen
mit neuen und besseren Bekleidungs- und Ausri
stungsgegenständen.Es regelte die Trageweise alle)
Uniformarten der NVA, war aber mehr als nur ein(
Vorschrift für das Tragen der verschiedenen ni
formarten in richtiger Kombination, zum befohle
nen Anlaß und entsprechend der .ahreszeit.
Die neue Dienstvorschrift schrieb Ergebnisse
einer fünfjährigen positiven Entwicklung auf dem
Gebiet der Bekleidung und Ausrüstung der Armee
angehorigen fest.
Mit der Vervollkommnung der Uniformierung
weiblicher Armeeangehöriger, der Schaffung und
Uniformierung des Fahnrichkorps der NVA,der
großzüigigen Ausstattung der Generale, Admiralc
und Offiziere mit Parade- und Dienstuniformen of
fener Fasson,mit reprasentativen Gesellschaftsuni
formen und verbesserten Felddienstanzügen,de
kostenlosen bereignung des Sportzeugs und de
Erhöhung der Grundausstattungsnormen für Be
kleidung und Ausriistung für alle Armeeangehöri
gen wurden persÃnliche und gesellschaftliche Be
dürfnisse gleichermaBen befriedigt.
Die Trageweise der neuen oder verbesserten ni.
formarten,Uniformsticke und Ausrüstungsgegen
stände war prinzipiell in den entsprechenden An
ordnungen bei ihrer zeitlich gestaffelten Einfüh-
rung geregelt worden.Diese Festlegungen wurden
ebenfalls in diese neue Dienstvorschrift übernom.
1nen.
Im Detail gab es jedoch neue Festlegungen. So
wurden ffiziersschüler nicht mehr mit silber.
grauen Oberhemden ausgestattet. Die Vorschrif
legte fest, daß dieser Personenkreis zur Ausgangs
und Paradeuniform das weiße Oberhemd, im Som
mer zur Ausgangsuniform auch die silbergrauc
Hemdbluse zu tragen hatte. Hohe Schnürschuh
entfielen und wurden durch schwarze Halbschuhe
ersetzt.
Offiziersschülern im 3. Lehrjahr gestattete die
neue Dienstvorschrift nicht mehr das Tragen der Schirmmütze zum Felddienstanzug. Als dazugehö-
rige Kopfbedeckung wurde nunmehr einheitlich,
jahreszeitlich bedingt, Feld- bzw, Wintermütze an-
geordnet.
Festgelegt wurde der Personenkreis, der berech-
tigt war, den neuen Webpelzkragen zu tragen: Offi-
ziere auf Zeit,Berufsunteroffiziere, Unteroffiziere
im Reservistenwehrdienst, die aktiv als Berufsunter-
offiziere gedient hatten, Offiziere im Reservisten-
wehrdienst,Offiziersschüler im 3. Lehrjahr, alle
weiblichen Armeeangehörigen und Offiziere sowie
die Generale und Admirale der NVA. Weiter galt,
daß Wintermütze und Webpelzkragen nur zusam-
men getragen werden durften.

Nach der Vorschrift von 1972 trugen Generale
und Admirale nur zur Ausgangsuniform ein weies
Oberhemd,ietzt gehörte es zur Parade-,Ausgangs
und Gesellschaftsuniform, Erweitert wurden dic
Anlässe,zu denen die Gesellschaftsuniform angezo.
gen werden konnte.Die neue Vorschrift ließ auch
zu,die repräsentative Gesellschaftsuniformzu
Staatsempfängen, Festveranstaltungen, protokolla-
rischen Empfangen anläßlich von Staatsfeiertagen
und nach besonderer Festlegung auch zu ,Jahresta
gen von Armeen,zu privaten Theater- und Kon.
zertbesuchen und familiären Festlichkeiten wie .Ju
gendweihen und EheschlieBungen u.a. zu tragen.
Die Achselschnur wurde als Bestandteil der Para.
deuniform bestätigt.ZurGesellschaftsuniform wa
ren Achselschnur und Dolch nur dann anzulegen,
wenn Orden und Medaillen anläßlich der Staatsfei.
ertage 1.Mai und 7.Oktober, zum Jahrestag der Na
tionalen Volksarmee sowie auf Befehl von Vorge
setzten ab Regimentskommandeur und Gleichge-
stellten aufw¤rts Orden und Medaillen am Band
befohlen wurden.
Matrosen,Maate, F¤hnriche und Offiziere der
Musikkorps der Volksmarine traten vom 1. Mai bis
zum 30.September einheitlich in den neueingeführ.
ten weiBen Uniformjacken auf. Die Angehörigen
der usikkorps derolksmarine trugen keine
Schwalbennester am Armel.Neu war die Festle
gung,daß der hellgraue bzw. cremefarbene Müt.
zenbezug für Admirale und Offiziere der Volksma-
rine nur in der Zeit vom 16,April bis zum
30.September und nur in Verbindung mit der Ge
sellschaftsuniform auf die Schirmmütze aufgezogen
werden durfte.
Die Vorschrift enthielt jetzt auch die Dienstgrad.
abzeichen für ffiziersschüler in der Berufs- bzw.
Hochschulreifeausbildung. Sie waren analog den
Schulterklappen der Unteroffiziere mit Litze um-
randet und ohne Querstreifen ausgeführt. Das me.
tallgeprägte «S》 wies sie als ffiziersschüiler aus.
An Offiziersschüler in der Berufs-bzw.Hochschul
reifeausbildung wurde keine Felddienst- bzw, Ge
fechtsuniform ausgegeben.