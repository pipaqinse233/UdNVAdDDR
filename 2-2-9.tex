

Eine der bemerkenswertesten Ãnderungen ihrer
Uniformierung erfuhren die Fallschirmjäger der
NVA Ende der 60er Jahre. Anfang September 1969
faBte die Leitung des Ministeriums für Nationale
Verteidigung der DDR den Beschluß, Uniformen
offener Fasson für die Angehörigen dieser Waffen.
gattung der Landstreitkräfte in diesem Jahr einzu.
führen.Der entsprechende Befehl Nr.124/69 des
Ministers für Nationale Verteidigung der DDR, Ar
meegeneral H, Hoffmann, und die Durchführungs.
anordnung seines Stellvertreters und Chefs Rück-
wärtige Dienste der NVA, Generalmajor W. Allen-
stein, datierten vom 9.Oktober 1969.
Was hatte sich an den Uniformen geändert? Von
nun an gehörte eine orangefarbene Baskenmiütze
als sichtbarstes ¤uBeres Zeichen zur Parade- und
Ausgangsuniform aller Fallschirmjäger. Auch die
steingraue BaskenmÃütze, die zum Dienst aufgesetzt wurde, erhielt eine veränderte Form und war jetzt
aus Streichgarn gefertigt.Die neue Uniformjacke
für Soldaten und Unteroffiziere war aufgrund ihrer
Fassonausfihrung im Ausgang offen und zusam
men mit einem silbergrauen niformhemd und
einem dunkelgrauen Binder, im Dienst ohne Hemd
geschlossen zu tragen. Die Offiziere erhielten eben.
falls, aber für alle Uniformarten, eine Uniformjacke
offener Fasson, wie sie die Offiziere der LSK/LV
der NVA bereits hatten. Hinzu kamen auch das sil.
bergraue Uniformhemd und der dunkelgraue Bin
der. Sowohl bei der Uniformjacke als auch beim
Uniformmantelwar der Kragen aus dem Grund
tuch des antels gefertigt.
Neugestaltete Kragenspiegel machten die an den
Armeln der Uniformiacken angebrachten Dienst.
laufbahnabzeichen überflüssig. Es handelte sich um
die auch heute noch üblichen Kragenspiegel aus
orangefarbenem Tuch mit einem silberfarbenen sti.
lisierten Fallschirm und einer Schwinge für die Sol
daten und Unteroffiziere sowie einer zusätzlichen
Umrandung der Kragenspiegel mit silberfarbener
Kordel fiir die Offiziere. Diese Kragenspiegel wa-
ren wie bei den Luftstreitkräften der NVA auch auf
die Kragen der Uniformmäntel genäht. Ansonsten
blieb die weiße Paspelierung an den Uniformen der
Fallschirmiäger unverändert erhalten.
Durch die Einführung dieser Uniform wurden
die Fallschirmjäger - wie es in der Begründung
hieß -entsprechend ihrer Bedeutung weiter her
vorgehoben.Die neueniform verlieh den Fall
schirmjägern im Ausgang ein sehr repräsentatives
Aussehen.ZugleichwardamitdieRichtungbestimmt
die fǔr die niformierung der Landstreitkräfte
im neuen ahrzehnt auf der Tagesordnung stand
Generell unterstreicht das Bild, das die Soldaten,
Matrosen, Unteroffiziere, Maate, Meister, Offiziere.
Generale und Admirale 1970 boten, daß die Staats.
führung der DDR auch in den 60er Jahren der Ver
sorgung der Streitkräfte der Republik mit zweckmä
Biger Kampf-, Dienst- und Repräsentationsbeklei
dung stets die erforderliche Aufmerksamkeit ge
schenkt hat.