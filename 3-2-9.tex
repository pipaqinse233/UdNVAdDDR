Eine Kontrolle des Ministers für Nationale Vertei-
digung der DDR, Armeegeneral H. Hoffmann, an
der Offiziershochschule der Landstreitkräfte «Ernst
Th¤lmann» im ahre 1978 löste eine weitere Ande-
rung der Uniformierung aus.In einem Gespräch
unterbreiteten Offziersschüler dem inister den
Vorschlag, die Litze am Kragen ihrer Uniformjak-
ken zu entfernen. Erstens, meinten sie, wirden die
Uniformen der Offiziersschüler mehr den Offiziers-
uniformen angeglichen, zweitens, und das zahlte
noch mehr, k¶nnten die Uniformjacken nach Er-
nennung der Offiziersschüler zum Offizier weiter-
genutzt werden, und drittens entfiel der hohe Ar-
beitsaufwand für das Aufnähen der Litze in den
Werkstätten des Bekleidungs- und Ausrüstungs-
dienstes zugunsten notwendiger Instandsetzungsar-
beiten.
Dieser Vorschlag überzeugte durch seine Origi-
nalitÃät und den ökonomischen Nutzen, Er wurde in
allen Teilstreitkräften geprüft und schließlich mit
der Anordnung Nr.11/79 des Stellvertreters des
Ministers für Nationale Verteidigung und Chefs
Rückwärtige Dienste vom 18.Juli 1979 zur Realisierung angeordnet. Es wurde angewiesen, daß mit Be-
ginn des Lehrjahres 1979/80 die Litze am Kragen
der Uniformjacke der Offiziersschüler der Land-
streitkr¤fte und der LSK/LV nicht mehr zu tragen
ist, Weiter wurde festgelegt, daß die Litze von den
bisher getragenen Jacken der Parade- und Aus-
gangsuniform abzutrennen ist und diese ,Jacken als
Dienstjacken aufzutragen sind, An Offiziersschüler,
die mit Beginn des Lehrjahres 1979/80 in das
2. Lehrjahr versetzt wurden, erfolgte die vorfristige
Ausgabe einer Parade-/Ausgangsjacke aus Kamm-
garn.