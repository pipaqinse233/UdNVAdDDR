

Einige Maßnahmen zur Veränderung der Unifor-
men wurden noch vor der Unterzeichnung der
neuen Bekleidungsvorschrift getroffen, m auch an
den Uniformen in der ffentlichkeit deutlicher
zwischen Soldaten auf Zeit und Berufssoldaten mit
Unteroffiziersdienstgraden unterscheiden zu kön
nen, ordnete der Stellvertreter des Ministers für Na-
tionale Verteidigung der DDR und Chef Rückw¤r.
tige Dienste der NVA am 5.Juni 1964 an, ab
1.Januar 1965 anstelle der spitzen Winkel und glei-
cher Doppelwinkel für eine mehr als dreijährige
bzw. mehr als finfjährige Dienstzeit stumpfe Win-
kel zur Kennzeichnung des Dienstverhältnisses am
rechten Unterärmel der Uniformjacken zu befesti-
gen,Soldaten auf Zeit führten nunmehr einen solchen stumpfen Winkel aus silberfarbener Tresse
(bei der Volksmarine goldfarben oder blau, auch
am Ärel der Kieler Hemden aufgenǎht)und die
Berufssoldaten einen entsprechenden Doppelwin
kel in diesen Ausführungen.
Mit derselben Anordnung Nr.6/64 wurden di
schon vorhandenen dunkelgrauen Schulterklappen
und sticke genutzt,uman Kampfanzügen und
Watteanzügen der Landstreitkräfte und der
LSK/LY die ijeweiligen Dienstgrade sichtbar zu ma-
chen.Die Angeh¶rigen der Volksmarine, die über
blaue
Watteanzüge verfigten,
behielten ihre
Schulterklappen und -stiicke bei.
Ein halbes Jahr später folgte die Anordnung
Nr.18/64 vom 1.Dezember 1964.Der Stellvertreter
des Ministers für Nationale Verteidigung der DDR
und Chef Rückwärtige Dienste der NVA legte in
ihr für die Angehörigen der Landstreitkräfte und
der LSK/LV fest,zur zweireihigen, offenen Aus
gangsjacke statt des silbergrauen ein weiBes Obcr
hemd zu tragen,Weibliche Angch¶rige dieser bei
den Teilstreitkräfte konnten ebenfalls eine weiße
Oberhemdbluse zur Ausgangsuniform anziehen.Im
folgenden Monat, am 23,[anuar 1965, gestattete es
eine Ergänzung zur genannten Anordnung allen
diesen Armeeangehörigen bei Aufenthalt im Aus
land und einem bestimmten Personenkreis,2.B
Militarattachés für den taglichen Dienst,Mitglie
dern des Erich-Weinert-Ensembles als Reiseuni
form,Sportinstrukteuren der Armeesportklubs bei
sportlichen Veranstaltungen, die zweireihige offenc
Ausgangsuniform noch mit dem silbergrauen Uni
formhemd und dunkelgrauem Binder zu tragen.
Seit Herbst 1965,unmittelbar vor dem Inkraft
treten der neuen Bekleidungsvorschrift der NVA
konnten alle Armeeangehörigen einen grauen Re
genumhang aus PVC-Folie kauflich erwerben. Da
mit wurde insbesondere auch den Wehrpflichtigen
im Grundwehrdienst und den Soldaten auf Zeit eine zweckmäBige Regenschutzbekleidung fir den
Ausgang und für den Urlaub geboten.
Als schlieBlich die neue Bekleidungsvorschrift
der NVA,die DV-10/5,Ausgabe 1965, mit Beginn
des.Ausbildungsjahres 1965/66 in Kraft trat, galt
es,eine Vielzahl von Modifizierungen der Unifor.
men durchzusetzen.Zunächst veränderten sich die
Trageperioden fiir die Uniformen der NVA. Die
Sommeruniformen waren in der Zeit vom 16. April
bis 31.0ktober und die Winteruniformen in den
verbleibenden Wochen und Monaten anzuziehen.
Als Tragezeit für die neue Wintermütze galten die
Tage vom 1.Dezember bis zum 15.Februar.
Nicht nur die Offiziere der Reserve und auße
Dienst,sondern auch die Unteroffiziere der Re
serve, die mehr als 10 |ahre aktiven Wehrdienst ge
leistet hatten,erhielten das Recht, bei besonderen
Anlässen wie Staatsfeiertagen,Empfängen, Festver
anstaltungen sowie Feierlichkeiten der NVA die
niform zu tragen.Zu bemerken ist,daà die in der
Bekleidungsvorschrift von 1960 für Offiziere der
Reserve und auBer Dienst getroffene Regelung, in
der Mitte der Schulterstüicke einen querlaufenden
Silberstreifen(Generale und Admirale der Reserve
und auBer Dienst cinen goldfarbenen Streifen)an-
zubringen, jetzt entfiel. Dieser Streifen war jeweils
9 mm breit gewesen, verlief unterhalb der Tuchun
terlage und ragte unter dem Schulterstüick an bei.
den Seiten um je 5 mm heraus, Ebenfalls galt nicht
mehr die Regelung, daß Soldaten, die sich für cine
Offiziersschule beworben hatten, nach ihrer Best¤.
tigung eine in der Mitte der Schulterklappen aufge
schobene Schlaufe aus 9 mm breiter Aluminiumge
spinsttresse fuhrten.
In seiner Anordnung Nr.2/65 vom 14.Mai 1965
traf der Stellvertreter des Ministers für Nationale
Verteidigung der DDR und Chef Rückwärtige
Dienste der NVA cinige Festlegungen bis zum In.
krafttreten der neuen Bekleidungsvorschrift.
So
wurde das schwarze Lederkoppel eingezogen. Die
Soldaten im Grundwehrdienst und die auf Zeit so-
wie zunachst auch noch die Berufsunteroffiziere
trugen die Ausgangsuniform ohne Koppel und schnallten zur Paradcuniform das Gurtkoppel um.
Nur bei zentralen Paraden wurde auf Befehl das
schwarze Lederkoppel anstelle des Gurtkoppels zur
Paradeuniform getragen.