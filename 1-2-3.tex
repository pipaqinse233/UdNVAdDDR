

Zur persönlichen Ausr¼stung des Soldaten und zur
Vervollständigung der Uniform gehört in der NVA
wie auch in anderen modernen Armeen der Stahl.
helm. Er wird insbesondere bei Handlungen in der
Gefechtsausbildung, aber auch beim Wachdienst getragen,Seine ZweckmäBigkeit hatte sich schon
wahrend der äuBerst verlustreichen K¤mpfe des er.
sten Weltkrieges sehr rasch als unverzichtbarer
Schutz des Kopfes vor Geschossen, Splittern und
Schlageinwirkung erwiesen.
Auch die bewaffneten Organe der jungen DDR
die Bereitschaften der Volkspolizei, waren schon
teilweise mit einem Stahlhelm ausgerüstet worden.
Er war in seiner Form Sturzhelmen nachempfun.
den und blieb deshalb für die Belange von Streit-
kräften unzureichend, Aus diesem Grunde wurden
bereits durch die Führungsorgane der KVP Ãberle-
gungen angestellt,einen neuen Stahlhelm zu ent
wickeln, Vereinzelt verwendeten Einheiten der KVP
und spÃäter der NVA - beispielsweise in Übungen
auch den Stahlhelm der Sowjetarmee. SchlieBlich
entschloB sich die Führung der NVA aber, einen
auf die Uniform der Volksarmee abgestimmten
Helm herstellen zu lassen.Demzufolge vereinbar
ten die rückwärtigen Dienste der NVA und das
Amt für echnik gemeinsam mit dem VEB Eisen
Hüttenwerk Thale und dem VEB Sattler- und Le
derwarenfabrik Taucha Ende |anuar 1956 Maßnah-
men,um die Entwicklung eines für die NVA
geeigneten Stahlhelms abzuschlieBen und diesen
unverzüglich in die Verb¤nde und Truppenteile
einzufihren.
In diesem Prozeß griffen die Konstrukteure, ver
antwortlicher Ingenieur war Erich Kiesan,auf eine
der letzten Entwicklungen des Stahlhelms der fa.
schistischen deutschen Wehrmacht zurick,die bis
1943 vorangetrieben worden war. Dieser Helmtyp
wurde aber nicht mehr hergestellt und eingesetzt
Bei den bekannten Stahlhelmen der Wehrmacht
Modell 1935 und Modell 1942 traten insbesondere
an den Knickstellen des felms an Stirn und Nak
ken infolge von Durchschlägen häufig Kopfschüsse
auf. Deshalb wählten die Konstrukteure nun eine
überschräge Form, die Geschosse und Splitter im
wesentlichen abgleiten ließ, Auch die Innenausstat
tung des Helms und die Metallegierung wurden
weiter verbessert.
Die Arbeiten an dem Stahlhelm der NVA gingen rasch und erfolgreich voran.Die Erprobung des
neuen Helms hatte gerade erst begonnen, als anl¤ß.
lich der ersten Parade der NVA am 1.Mai 1956
Teile der über den Marx-Engels-Platz paradieren
den Einheiten schon mit diesem Stahlhelm an die
Öffentlichkeit traten.
Die systematischen Erprobungen des Stahlhelms
setzten erst Mitte Mai 1956 ein, Insbesondere die
Beschußproben zogen sich- äuBerste Sorgfalt war
geboten - bis Ende des |ahres hin., In seiner Anord.
nung Nr.15/56 vom 14. Mai 1956 regelte der Mini.
ster für Nationale Verteidigung der DDR, General.
oberst W.Stoph,die Erprobung des Stahlhelms
VM 1/56 (Versuchsmodell 1/56).Zwischen dem 16.
und 19,Mai sollten mittels BeschuB- und Festig
keitsproben die Formgebung und die Materialhalt.
barkeit getestet und im ,Juni die Tragemöglichkei-
ten über längere Zeit festgestellt werden.
Beschuß-und Festigkeitsproben erfolgten durch
direkten Beschuß mit der Pistole TT33(10 m bis
25 m), mit der MPi PPSch 41 (25 m bis 100 m), mit
dem Scharfschützengewehr D(300 m bis 600 m)
und mit dem sMG(600 m)sowie mit Handgrana-
ten am 16.und 17.Mai. Weitere Beschußproben
fanden am 17.und 18.Juli sowie am 27.Dezember
1956 statt, W¤hrend der Erprobung im Juli wurde
der Stahlhelm auch der Wirkung von Artillerie-
munition-des 82-mm-Granatwerfers,der 76-mm-
Kanone und der 122-mm-Haubitze- ausgesetzt.
Alle diese Versuche zeigten eindeutig: Der Stahl-
helm bot seinem Träger mit absoluter Sicherheit
Schutz vor der Schußeinwirkung durch Pistolen ab
10m Entfernung und vor der von aschinenpisto
len ab 50m.Noch 1m von der Detonationsstelle
einer Handgranate entfernt, hielt der Stahlhelm der
Splittereinwirkung stand, Auch beim Detonieren
von Artilleriemunition konnte der Soldat mit dem
Stahlhelm vor Kopfverletzungen geschützt werden.
So bestand beim genannten Granatwerfer ab 10 m,
bei der Kanone ab 20 m und bei der Haubitze ab
25 m von der Detonationsstelle entfernt Sicherheit
Ebenfalls erfolgreich verliefen die Versuche zur
Feststellung der Druck- und Schlagfestigkeit.

Um die Zweckmäßigkeit des Stahlhelms hinsicht.
lich der Tragfahigkeit bei den verschiedensten mili-
tarischen Tatigkeiten zu erproben, fuhrten ein
Schützenzug,ein Aufklärungszug, ein Granatwer-
ferzug (82-mm-Granatwerfer), eine Geschützbedie-
nung (122-mm-Haubitze), ein Zug des Wachregi-
ments, ein Nachrichtenzug, cin Pionierzug und ein
Zug der Truppen der chemischen Abwehr Trage-
versuche durch, Sie fanden bei der Grundausbil-
dung, bei Marschübungen und bei der SchieBaus
bildung sowie während der Fahrten mit dem SPW
mit Kfz und Krad statt. Es galt festzustellen, ob die
Innenausstattung des Stahlhelms auch bei Wen-
dungen, beim Hinlegen oder Aufstehen und beim
Exerzier- oder Laufschritt stets einen einwandfreien
Sitz gewährleistete. Der Helm durfte keine Druck-
schmerzen hervorrufen, Meteorologische Bedingun-
gen wie Sonne, Regen und Wind sollten sich nicht störend auswirken,z.B,Regen und Wind nicht das
Wahrnehmen von Geräuschen beeinträchtigen.
Insgesamt erbrachten die Materialerprobungen
und die Trageversuche sowie vorgenommene Ver.
besserungen am Stahlhelm hinsichtlich der Legie.
rung des Stahlblechs sowie bei der Innenausstat.
tung des Helms bis zum Ende des lahres 1956 sehr
gute Ergebnisse, Sie erlaubten es, die Massenferti-
gung des Stahlhelms M56 aufzunehmen und die
Teilstreitkräfte mit ihnen auszustatten. Bereits vor
den Erprobungen des Stahlhelms rechtfertigten es
Foringestaltung undMaterialzusammensetzung.
Helme dieses Typs in drei Größen bis Mitte April
1956 in sehr kleiner Stückzahl an die Paradetrup
pen auszuliefern, Sie waren farblich steingrau-matt
gehalten, an der linken Seite mit einem schwarz
rot-goldenen Wappen als Abzichbild versehen und
von innen mit einem Stempel <S 1/56» als beson-
dere Serie gekennzeichnet.
Anfang 1957 wurden die Herstellungs- und Ab
nahmevorschriften bestätigt und die Produktion
der Stahlhelme in den drei GröBen 60 cm, 64 cm
und 68cm Bezugsmaß aufgenommen. Bis Ende
September 1957lieferte die Industrie ungefahr
50 000 Stahlhelme an die Truppe aus. Etwa ein .Jahr
später gewährlcistete sie die restlose Versorgung der
Verbände und Truppenteile der NVA mit diesem
Stahlhelm.
Den Hauptanteil an der Entwicklung des Stahl-
helms in weniger als einem Jahr, den der Ingenieui
Erich Kiesan geleistet hatte, anerkannte die Füh
rung der NVA zum 1.Mai 1957 mit der Verdienst
medaille der NVA in Bronze.
Ergänzend zur Entwicklung und Einführung des
Stahlhelms M 56 sei noch angemerkt, daß es diesen
Helm bis heute in zwei Formen gibt. Bei der ersten
Form ist das Helmfutter durch drei auBen sichtbare
Nieten befestigt, wihrend bei der zweiten Form das
Futter innen durch sechs Metallknöpfe aufgehängt
ist. Die Ehrenformationen der NVA verfügen, wie
international vielfach üblich, über einen am 17.Tuni
1957 bestätigten Kunststoffhelm mit cinem Ge
wicht von 500 Gramm.