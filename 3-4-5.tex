

Die Paradeuniform wird zu Paraden und Ehrenwa-
chen, in Ehrenformationen und bei Kranzdelegatio-
nen, zu Appellen an Staatsfeiertagen, zum Tag der
NVA und zu militärischen Zeremoniellen befohlen.
Soldaten im Grundwehrdienst, Soldaten und Unteroffiziere auf Zeit und im Reservistendienst tra.
gen diese Uniform auch zum Standortstreifendienst
und zu Dienstreisen.
Bei den Wehrpflichtigen sowie den Soldaten und
Unteroffizieren auf Zeit gehören zur Paradeuni
form in allen Trageperioden Uniformjacke, Uni
formhose, die in den Halbschaftstiefeln eingeschla
gen getragen wird, graues Oberhemd mit Binder
und Lederkoppel mit SchloB. In der Regel ist der
Stahlhelm, nur auf Befehl die Schirmmütze aufzu
setzen.Im Winter vervollst¤ndigen Uniformmantel
und Wintermütze die Paradeuniform.
Besonderheiten gibt es in der Uniformierung de)
Fallschirmjäger und der Wachregimenter der NVA
Soldaten und Unteroffiziersdienstgrade auf Zeit dei
Fallschirmjäger tragen zur Parade- und Ausgangs
uniform die orangefarbene,zur Felddienst-.Dienst
und Arbeitsuniform die dunkelgraue Baskenmütze
Zur Parade-und Dienstuniform werden Keilhose
zur Felddienst-,Dienst- und Paradeuniform an
stelle der Halbschaftstiefel Sprungschuhe angezo
gen.
Alle Angehörige der Wachregimenter sind wie
Berufssoldaten mit genarbten Schaftstiefeln, Stie
felhosen und Lederkoppel mit Schnalle ausgestat
tet. Sie tragen deshalb zur Felddienst-, Dienst-, Pa
rade- und Arbeitsuniform genarbte Schaftstiefel
zur Felddienst-、Dienst-und Paradeuniform die
Stiefelhose anstelle der Uniformhose und zur Pa.
rade-und Ausgangsuniform das braune Lederkop
pel mit Schnalle.
Matrosen im Grundwehrdienst,Unteroffiziers.
schiüler,Matrosen und Maate auf Zeit tragen zur
Paradeuniform Nr.1(Sommer)Tellermütze mit
weißem oder blauem Mützenbezug, Kieler Hemd
mit Kieler Kragen und Kieler Knoten, Klapphos
auf Befehl mit halbem Schlag, Halbschaftstiefe!
und schwarzes Lederkoppel mit SchloB. In der
ÃJbergangszeit wird die Paradeuniform Nr. 2 befoh
len, die gegenüber der ersteren durch Überzieher
mit umgelegtem Lederkoppel und Wirkhandschu
hen ergänzt wird, Im Winter setzen sie zur Parade
uniform Nr.3 die Wintermütze auf.

Die Paradeuniform für Berufssoldaten besteht
aus Schirmmütze (auf Befehl Stahlhelm), Uniform-
jacke, Stiefelhose, weißer Hemdbluse bzw. weißem
Hemd, Binder und glatten Schaftstiefeln. Offiziere
haben dazu Feldbinde,Achselschnur und Ehren-
dolch anzulegen. In der Ãbergangsperiode erg¤n-
zen Uniformmantel und Lederhandschuhe,im
Winter die Wintermütze die Paradeuniform für Of-
fiziere der Landstreitkräfte und der LSK/LV. Be-
rufssoldaten ohne Offziersdienstgrad schnallen an-
stelle der Feldbinde ihr Lederkoppel mit Schnalle
um.
Die Paradeuniform der Berufssoldaten
der
Volksmarine gleicht in ihrer Zusammenstellung der
der anderen Teilstreitkräfte, nur daß anstatt
der
Stiefelhose die lange Uniformhose und anstelle der
Schaftstiefel Halbschuhe oder Zugstiefel angezogen
werden.
Neu für weibliche Armeeangeh¶rige ist in der 1986er Vorschrift ebenfalls eine Paradeuniform.
(Siehe nebenstehende Tabelle)
Generale und Admirale erscheinen zu Paraden
und Ehrenwachen,inEhrenformationen und
Kranzdelegationen, zu milit¤rischen Appellen und
Zeremoniellen in Paradeuniform. Im Sommer be-
steht sie aus Schirmmütze (auf Befehl Stahlhelm),
Uniformjacke ohne Arabesken, bei Generalen aus
Stiefelhose, bei Admiralen aus langer Uniformhose,
aus glatten Schaftstiefeln bzw. Halbschuhen, wei-
ßem Oberhemd mit Binder, Feldbinde, Achsel-
schnur und Dolch.In der Übergangsperiode wer-
den darüber der Uniformmantel und dazu die Lederhandschuhe, im Winter anstelle der Schirm-
mütze die Wintermütze getragen, Admirale ziehen
dann anstatt der Halbschuhe Zugstiefel an.