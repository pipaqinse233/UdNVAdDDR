Bislang trugen die Fallschirmjäger des Truppenteils
«Willi S¤nger» im Sommer die für ihre speziellen
Aufgaben modifizierten Felddienstanziüge und im
Winter Watteanzüge, Nach 5 Jahren h¤rtester Ein-
satzbedingungen war es an der Zeit, die Frage zu
stellen, wie die Bekleidung und Ausrüstung
der
Fallschirmjäger noch besser dazu beitragen k¶nne,
ihre Ausbildungs- und Gefechtsaufgaben zu lösen.
Bewährt hatte sich im Sommerhalbiahr der Feld-
dienstanzug, wie ihn auch die Aufklärer besaBen. MMit dem Watteanzug waren die Fallschirmjäger we
niger zufrieden,bot er doch im Einsatz unter Win
terbedingungen zuwenig Bewegungsfreiheit.Pro
bleme,vor allem beim Sprung,gab es mit der
ungenügend fixierten Schutzmaskentasche und der
Magazintasche.Die Sprungschuhe entsprachen vol
den Anforderungen von Fallschirmspringern, waren
jedoch den weitaus höheren Belastungen der Ge
fechtsausbildung von Fallschirmjägern bzw. der Er
füllung von Gefechtsaufgaben nicht gewachsen. Bei
der Durchführung der Sprungausbildung erwies
sich zudem die bisher übliche Art der Befestigung
der Schulterklappen bzw.-stücke als unzweckmäßig
und führte zu einem hohen Verschleiß der Effek
ten.Probleme gab es auch mit dem Regenschutz
Die bisher verwendete Zeltbahn schützte den Ein
zelk¤mpfer bei der Erfüllung seiner Gefechtsaufga.
ben unter schwierigen Gelände- und Witterungsbe
dingungen nur ungenügend.Dermhang
schränkte beim Tragen die Bewegungsfreiheit stark
ein und bot vor allem keinen ausreichenden Schutz
bei längerer Nässeeinwirkung. Wurde die Zeltplane
nicht benötigt,belastete sie den Einzelk¤mpfer
durch das relativ hohe Gewicht.
Die Erhöhung der Gefechts- und Einsatzbereit.
schaft war und ist in der Nationalen Volksarmee die
Sache aller. Neuerer des Truppenteils «Willi S¤n-
ger» verstanden die in Dienst- und FD]-Versamm
lungen ge¤uBerte Kritik an der Bekleidung und
Ausrüstung als Aufforderung, zu knobeln und Ver
besserungen vorzuschlagen.Das Ergebnis konnte
sich sehen lassen. Drei Neuerervorschläge, unter
den Registrier-Nummern 02/72,03/72 und 04/72
eingetragen,lagen sehr bald auf dem Tisch des
Stellvertreters des inisters für Nationale Verteidi.
gung und Chefs Rückwärtige Dienste. Sie besta.
chen durch Sachkenntnis und Originalität,und sie
waren in die Praxis umzusetzen,ohne daß ein we
sentlicher Mehraufwand notwendig gewesen wire.
Die Vorschläge wurden geprüft und befüirwortet.
Am 29.September 1972 wies der MMinister für Na
tionale Verteidigung an,entsprechend den einge
reichten Vorschlägen Muster eines neuenKampfanzuges fúr Fallschirmjager,
verbesserter Fall
schirmjäger-Schnürstiefel,zweckmäBiger Regenbe
kleidung und eines speziellen berziehers analog
dem für Matrosen und Maate der Volksmarine so-
wie flexibler Effekten zu fertigen.
Dies geschah umgehend.Die
neuentwickelten
Uniformstücke wurden durch die Fallschirmjäger
12 Monate unter allen Einsatzbedingungen getestet
und noch geringfügig verändert. Das Ergebnis aller
Bemühungen war ein neuer fünfteiliger Felddienst-
anzug für Sommer und Winter,der den besonderen
Einsatzbedingungen der Fallschirmjäger optimal
entsprach,Der neue Felddienstanzug wurde durch
eine Kampfweste und neueFallschirmjäger
Schnürstiefel ergänzt.
Er bestand aus einer einreihig zu kn¶pfenden
Jacke mit verdeckter Knopfleiste und Keiliiberfall.
hosen aus einem Spezialgewebe im Stricheldruck
Im Winter trugen die Fallschirmjäger darunter eine
wattierte Unterziehjacke und ebensolche Unterzieh.
hosen. Der Felddienstanzug wurde vervollständigt
durch eine beschichtete Regenkutte mit Kapuze aus
Dederongewebe,die in einer dafǔr vorgesehenen
Tasche der Kampfanzugjacke untergebracht wer.
den konnte.Die Kampfweste,aus dem gleichen
Material wie der Felddienstanzug gefertigt, war mit
ie einer ankonfektionierten Schutzmasken-,aga
zin-und Rickentasche versehen.
Die neuen Fallschirmjäger-Schnürstiefel bestan
den aus einem strapazierfähigen hydrophobierten
Oberleder.Der Schaftschnitt war durch eine ange
schnittene Gamasche mit durchgehender Staubla
sche,Geröllklappe und Klemmringverschluß der
harten und breitgefächerten Einsatzbedingungen
der Fallschirmjäger angepaßt. Die weite und be
queme Leistenform, die flexible abriebfeste Poro
kreppsohle und die insgesamt zwiegenähte Ausfih
rung entsprachen derForderung nacheinem
Fallschirmjäger-Schn¼rstiefel, der auf Grund seinei
Form,Gestaltung und seines Materials die Einsatz
-fahigkeit des Fallschirmjägers
bereitschaft und
über einen langen Zeitraum gewährleistete und sehr
strapazierlahig war.

Mit dem neuen Kampfanzug, der Kampfweste
und den neuen Fallschirmjäger-Schnürstiefeln wur-
den die Fallschirmj¤ger der NVA bis zum Oktober
1975 ausgerüstet.