

Im Abschnitt über spezielle Bekleidung der NVA
wurden die Fallschirmjäger, die die jiingste Waffen-
gattung der Landstreitkräfte bildeten, bereits im
Hinblick auf ihre Gefechtsbekleidung erwähnt. 【m
Jahre 1962 im Bataillon <Willi ¤nger》 formiert,
bewährten sich die Fallschirmjäger schon ein Jahr
späiter im gemeinsamen Manöver «Quartett» der
NVA, der Sowjetarmce, der Polnischen Armee und der Tschechoslowakischen Volksarmee (9,bis
14.September 1963)als taktische Luftlandetruppe
Sehr rasch kam die Führung der NVA zu dem
SchluB, die spezielle Rolle der Fallschirmjäger in
den Landstreitkräften auch durch cine besondere
Uniform zum Ausdruck zu bringen. Eine Anfang
November 1963 im Militärbezirk Neubrandenburg
durchgeführte Beratung leitender Offiziere des
B/A-Dienstes der NVA,der Verwaltung Ausbil
dung im Ministerium für Nationale Verteidigung
der DDR und des Fallschirmjägerbataillons befaßte
sich mit diesen Fragen. Die Offiziere stellten u. a
fest, daB die Gefechtsausbildung der Fallschirmjä
ger überwiegend im Kampfanzug fǔr Aufkl¤rer
und mit Schnirstiefeln
Sprungkombination)
'Sprungschuhen)durchgefihrt wird. Anstelle des
Stahlhelms,der auch beim Absprung nicht itge.
führt werden konnte, hatte sich die vorhandene Le
derkopfhaube alszweckmäBiger erwiesen.Das
Sturmgepack muÃte einem Transportsack weichen.
Im Interesse der Gew'hnung an die Schnürschuhe
wurden die Halbschaftstiefel nicht in die Grund
ausstattungsnorn aufgenommen.
Am 15.Februar 1964 beschloß die Leitung des
Ministeriums für Nationale Verteidigung der DDR
die Einfihrung einer speziellen Uniform für die
Fallschirmjäger, Die Uniform war durch die Aus
stattung mit Keilüberfallhosen und Sprungschuhen
so gestaltet, daß sie den spezifischen Bedingungen
der Ausbildung und Dienstdurchführung in dieser
Waffengattung entsprach, Als Kopfbedeckung er
hielten die Fallschirmiäger eine steingraue Basken
mütze und als Waffenfarbe wurde Orange be
stimmt, Die Ausstattung mit den neuen Uniformen
fand bis Ende September 1964 statt. Die nebenste.
hende Tabelle faBt die in der DV-98/4. Beklei.
dungs-und Ausr¼stungsnormen der Nationalen
Volksarmee vom 2.November 1964(Inkraftsetzung
mit Wirkung vom 1.Mai 1965)enthaltenen Uni
form- und Ausrüstungsstücke fǔr Fallschirmjäger
Zusammen.