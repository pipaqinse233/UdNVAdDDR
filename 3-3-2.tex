

Am 25.März 1982 beschloß die Volkskammer der
DDR ein neues Wehrdienstgesetz, das am 1. Mai
1982 in Kraft trat, Dieses Wehrdienstgesetz, das
auf der Grundlage der Verfassung und des Vertei.
digungsgesetzes vom 13.Oktober 1978 alle Fragen
der Organisation und Gestaltung des Wehrdienstes
in der DDR regelte, löste das aus dem Jahre 1962
stammende Wehrpflichtgesetz ab.
Als Folgebestimmung zum Wehrdienstgesetz ver.
abschiedete der Staatsrat der DDR den BeschluB
über milit¤rische Dienstgrade, der erstmals die
Dienstgrade Flottenadmiral und Marschall der
DDR enthielt. Der Nationale Verteidigungsrat er.
ließ auf neuer gesetzlicher Grundlage die Einberu.
fungs-und Reservistenordnung sowie die Dienst
laufbahnordnung der NVA.Der Ministerrat der
DDR beschloß die Besoldungsordnung, die Unter.
halts-, die Wiedergutmachungs- und die Förde.
rungsverordnung.Zur Reservisten- und Förde
rungsverordnung erließ der Minister für Nationale
Verteidigung Durchführungsbestimmungen,die
ebenfalls am 1.MMai 1982 wirksam wurden, Auf der
Grundlage des neuen Gesetzwerkes zur Organisa.
tion der Landesverteidigung der DDR war es not
wendig,Dienstvorschriften zu überarbeiten, so auch
die DV010/0/005, die Bekleidungsvorschrift der
NVA. Die überarbeitete Fassung wurde mit dem
1. März 1983 für alle Uniformträger der NVA ver
bindlich.
Zu den wichtigsten Festlegungen der 1983er Vor
schrift geh¶rte die Festlegung neuer Zeitabschnitte
für das Tragen der Bekleidung und Ausrüstung
Danach
waren die Sommertrageperiode vom 16. April bis 31. Oktober, die bergangstragepe-
riode vom 1.M¤rz bis 15.April und vom 1.Novem-
ber bis 30.November sowie die Wintertrageperiode
vom 1.Dezember bis zum 28./29.Februar festgelegt.
Mit der Einführung der Übergangsperioden wur-
den für die Armeeangeh¶rigen umfangreiche M¶g-
lichkeiten der Kombination der Uniformarten ge-
stattet.
Neu waren auch Festlegungen über die Einfüh-
rung und die Zusammensetzung
einer Arbeitsuni-
form.