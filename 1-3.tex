

Mit dem Inkrafttreten der <Vorl¤äufigen Beklei-
dungsordnung der`Nationalen Volksarmee》 war
nun keinesfalls die Uniformierung der Armee für
lange Zeit festgeschrieben worden. Viele Dinge im
Bereich der Bekleidung und Ausristung befanden
sich noch im Stadium der Überlegungen und Er-
probungen, Sie mußten üiberwiegend in engem Zu-
sammenhang mit ebenfalls noch anstehenden Auf.
gaben in der gesamten Entwicklung der Nationalen
Volksarmee gelöst werden, Dies betraf Bereiche wie
die Militärtechnik, Gefechtsausbildung und Mili-
tärökonomie.
Gerade hinsichtlich der Ausr¼stung
der NVA
mit der notwendigen Militärtechnik gab es in deh
Aufbaujahren der Armee viel Bewegung. Die Land
streitkräfte erhielten Ende der 50er Jahre mittlere
sowjetische Panzer der Typen T-34/85 und 'T-54 so-
wie den Schwimmpanzer PT-76, Die mot, Schützen
bekamen durchgängig automatische Schitzenwaf.
fen, vor allem die Maschinenpistole Kalaschnikow
(MPi K).In den Luftstreitkräften vollzog sich der
Ãbergang zu Strahlflugzeugen der Typen MiG-15
und MiG-17. Die Seestreitkrafte stellten Küsten-
schutzschiffe des Typs 50 und Torpedoschnellboote
des Typs 183 in Dienst.