\begin{figure}
\includegraphics[width = \columnwidth]{./media/page (9).jpg}
\end{figure}

德意志民主共和国(以下简称「民主德国」或「东德」)国家人民军制服的起源和发展,目前几乎未曾在出版物中提及;然而,国家武装部队成员主要穿着这一身制服,向社会展示着自己的形象。几乎每个青年男子和一些青年女子都要在服兵役期间穿上至少18个月。
% Die Entstehung und Entwicklung der Uniformen der Nationalen Volksarmee der Deutschen Demokratischen Republik ist bislang kaum in Publikationen behandelt worden, dabei ist es vor allem die Uniform, in der sich die Angehörigen der Streitkräfte des Landes optisch der Gesellschaft präsentieren. Fast jeder junge Mann und auch manche junge Frau trägt sie wahrend des Wehrdienstes mindestens 18 Monate.

许多市民(尤其是年轻人)质疑国家人民军制服及其个别部分的起源和意义。门外汉很难意识到国家人民军的制服尽管在某些基本要素上较为稳定,但一直都有较大的发展。军服的变化反映了国家人民军的许多特点和变化,而国家人民军在成立三十多年后的今天,其本身也成为历史关注的主题。
% Viele Bürger, insbesondere junge Menschen, stellen die Frage nach Herkunft und Sinn der Uniformen der NVA und einzelner Teile derselben. Einem außenstehenden Beobachter wird kaum bewußt, daß die Uniformierung der NVA bei aller Stabilität in gewissen Grundelementen doch einer erheblichen Entwicklung unterworfen war und ist. Im Wandel der Uniformen spiegelt sich mancher Wesenszug und manche Veränderung in der NVA wider, die - mehr als drei Jahrzehnte nach ihrer Griindung - selbst Gegenstand historischen Interesses ist.

本书基本完整概述了截至1986年底东德军队制服的历史发展。截至本书出版时,东德军队制服的新变化已无法囊括。本书展现了制服的发展与东德全社会的发展、军事需求、国民经济的可能性以及军事习俗和时尚潮流之间始终存在的密切联系。书中主要配有原版制服、军衔和技术岗位袖章的图片,以及民主德国陆军博物馆的藏品和当代照片。
% Die vorliegende Publikation gibt einen weitgehend geschlossenen Überblick über die geschichtliche Entwicklung der NVA-Uniformen bis Ende 1986. Weitere bis zum Erscheinen des Buches vorgenommene Veränderungen an den Uniformen der NVA konnten nicht mehr berücksichtigt werden. Die stets engen Bezüge der Uniformentwicklung zur gesamtgesellschaftlichen Entwicklung in der DDR, zu den militärischen Erfordernissen, den Möglichkeiten der Volkswirtschaft sowie auch zum militärischen Brauchtum und zu modischen Einflüssen werden dargestellt. Das Buch wurde vor allem mit Abbildungen originaler Uniformen, Dienstgrad- und Dienstlaufbahnabzeichen sowie Effekten aus dem Bestand des Armeemuseums der DDR sowie mit zeitgenössischen Fotos illustriert.

最初的军衔和技术岗位袖章、荣誉短佩剑等物品往往是国家人民军成员在服役期间长期佩戴的,但并不总是符合规定(如军衔星的位置)。由于生产地不同,尺寸、颜色等方面可能存在误差,确切的日期并不总能确定。不过作者可不想放弃呈现原品。第1章前14张图上的制服插图取自一本画册,该画册于1956年初提交给东德人民议会,供其决定是否统一国家人民军制服。
% Die originalen Dienstgrad-, Dienstlaufbahnabzeichen, Effekten, Ehrendolche u.a.m. wurden häufig von Angehörigen der NVA lange Zeit im Dienst getragen und entsprechen nicht immer und in allem den Vorschriften (z.B. die Stellung der Dienstgradsterne). Aufgrund unterschiedlicher Fertigungsstätten und möglicher Toleranzen in Abmessungen, Farben usw. ist auch nicht in jedem Falle eine exakte Zeitbestimmung möglich. Trotzdem wollten die Autoren nicht auf die Wiedergabe von Originalen verzichten. Die Uniformabbildungen auf den ersten 14 Tafeln im 1.Kapitel sind einem Album entnommen und nachgestaltet worden, das Anfang 1956 den Abgeordneten der Volkskammer der DDR zur Beschlußfassung über die Uniformierung der NVA vorlag.

由于大量细节和材料已经遗失,因此在处理国家人民军制服这一看似简单的主题时,不得不敢于留下空白。由于部分资料来源情况复杂,有关国家人民军制服发展的一些问题暂未得到解答。本书的读者来信追随着国家人民军制服设计的步伐,作者希望这些来信将有助于解答上述问题,甚至使其成为可能。
% Bei der Behandlung dieses scheinbar überschaubaren Gegenstandes der NVA-Uniformen war es aufgrund einer Vielzahl von Details und auch schon verlorengegangenen Materials unumginglich, Mut zur Lücke aufzubringen. Eine teilweise komplizierte Quellenlage läßt noch immer manche Fragen der Uniformentwicklung der NVA offen. Zuschriften von Lesern des Buches, die Schritte der Uniformgestaltung der NVA mitgegangen sind und auf die die Autoren hoffen,  könnten die Beantwortung solcher Fragen erleichtern oder gar erst ermöglichen.

\begin{figure}
\includegraphics[width = \columnwidth]{./media/page (10).jpg}
\end{figure}

以下机构和顾问为本书的开发和完成提供了重要的支持,在此作者衷心感谢:民主德国国防部服装与装备总署、国家人民军各军种服装与装备处官员、民主德国陆军博物馆、民主德国军事档案馆、民主德国军事历史研究所和民主德国军事出版社。
% Die Autoren danken den Institutionen und Ratgebern, die die Entstehung und Fertigstellung des Vorhabens kritisch begleitet und gefördert haben: der Hauptabteilung Bekleidung und Ausrüstung im Ministerium für Nationale Verteidigung der DDR, Offizieren des Bekleidungs- und Ausrüstungsdienstes in den Teilstreitkräften der NVA, dem Armeemuseum der DDR, dem Militärarchiv der DDR, dem Militärgeschichtlichen Institut der DDR und dem Militärverlag der DDR.

作者尤其要感谢教授Karl Greese博士,其有益建议和支持为本书的创作做出了重要贡献。作者还感谢民主德国陆军博物馆博物学家Renate Siegel的支持。
% Besonders danken die Autoren Professor Dr. sc. Karl Greese und Oberstleutnant Dieter Möricke, deren hilfreicher Rat und fördernde Tat wesentlich zum Zustandekommen des Buches beigetragen haben. Ebenfalls wissen die Autoren die Unterstützung der Museologin im Armeemuseum der DDR, Renate Siegel, zu schätzen.

\hfill 本书作者% Die Autoren