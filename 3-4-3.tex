

Sehr differenziert und den Aufgaben der jeweiligen
Dienstgradgruppen angepaßt sind die Uniformar-
ten Dienstuniform und Stabsdienstuniform.
Soldaten im Grundwehrdienst, Soldaten, Unteroffiziere auf Zeit und Unteroffiziersschüler sowie
Unteroffiziere auf Zeit im Reservistendienst kön-
nen die Dienstuniform nur im Ãbergang und im
Winter tragen, und zwar zum Unterricht in ge-
schlossenen R¤umen sowie zum Tages- und Innen-
dienst. Die Dienstuniform besteht aus hochge-
schlossener Uniformjacke und eingeschlagener
Hose, Halbschaftstiefeln und Gurtkoppel. Bei
Dienst in $täben k¶nnen anstelle der Halbschaft-
stiefel Halbschuhe angezogen werden. Kopfbedek-
kung zur Dienstuniform Nr, 1 für die Übergangspe-
riode ist für Soldaten die Feldmütze, für Unteroffi-
ziere die Schirmmütze und für Fallschirmjäger die
dunkelgraue Baskenmütze. Bei niedrigeren Tempe-
raturen wird die Dienstuniform Nr.2 befohlen. Den notwendigen Kalteschutz erreicht man dann durch
das Anlegen des Wintermantels und der Wirkhand-
schuhe.
Den besonderen Anforderungen des Dienstes
sind die Dienstuniformen der Angehörigen der
Wachregimenter zur Durchführung des Wachdien-
stes angepaBt. Dazu gehören Uniformstiicke, die
entsprechend der befohlenen Uniformart, wie in
der nebenstehenden Tabelle dargestellt, kombiniert
werden können.
Angehörige der Wachregimenter sind, bedingt
durch die Besonderheiten ihres Dienstes und die hohen Ansprüche an das Repräsentationsverm¶gen
ihrer Uniform, auch w¤hrend der Wachdienst
durchführung entsprechend einer Sondernorm mit
Bekleidung und Ausrüstung wie Berufssoldaten
ausgerustet.
Die Dienstuniformen der Offiziere auf Zeit und
Berufssoldaten der Landstreitkrafte undder
LSK/LV können im Sommer und in den iber
gangsperioden variiert werden,im Winter gibt es
diese Möglichkeit nicht. Generell gehören zur
Dienstuniform Stiefelhose und Schaftstiefel. Als
Kopfbedeckung dient im Sommer und in den
Ubergangsperioden die Schirmmütze,im Winter
die Wintermütze.
Die Dienstuniform Nr.1 für den Sommer besteht
aus Stiefelhose, Uniformjacke, Uniformhemd ode
silbergrauer Hemdbluse mit geschlossenem Kragen
Binder und Lederkoppel mit Schnalle. Dazu kann
der Sommermantel mit geschlossenem Rinksgurt
getragen werden,Bei der Dienstuniform Nr.2 fir
wärmere Witterung wird auf die Uniformjacke ver.
zichtet und der Kragen der Hemdbluse geöffnet
Das Lederkoppel wird dann durch die Schlaufen
der Hemdbluse gezogen.
Die Dienstuniform Nr,3 berücksichtigt kühle
Tage während der Übergangszeiten. Unter dem
Uniformmantel werden dann Stiefelhose,Uniform
iacke,Uniformhemd bzw.-bluse und Binder, im
Bedarfsfall der Pullover mit V-Ausschnitt getragen
Dazu gehören auch schwarze Lederhandschuhe.
Die Dienstuniform Nr.4 sieht statt des Uniform
mantels den Sommermantel vor.Zur Dienstuni
form Nr, 5 z¤hlen Wintermütze und Uniformman
tel, über dem das Lederkoppel mit Schnalle
getragen wird.
Die Stabsdienstuniform für Berufssoldaten un
terscheidet sich von den Arten der Dienstuniform
dadurch, daß anstelle der Stiefelhosen lange Uni-
formhosen, statt der genarbten Schaftstiefel Halb
schuhe oder die neu eingefihrten Zugstiefel ange.
zogen werden und kein Lederkoppel getragen wird
Berufsunteroffiziersschüler, Fahnrich- und Offi-
ziersschüler tragen die Dienstuniform zum Tages dienst und zum Standortstreifendienst,die Stabs.
dienstuniform zumInnendienst;Berufssoldaten
zum Wach-und Tagesdienst, zu Inspektionen und
Kontrollen und zum Standortstreifendienst. Die
Stabsdienstuniform tragen Berufssoldaten zumIn
nendienst,auf Befehl zu Kontrollen, Inspektionen
sowie auf dem Weg vom und zum Dienst.
Füir weibliche Armeeangehörige gibt es die Uni.
formart Dienstuniform nicht. Die vielfaltigen a.
riationsmöglichkeiten für das Tragen der Stabs
dienstuniform - fǔr weibliche Armeeangehörige
die Hauptuniformart für die Dienstdurchführung
zeigt die Tabelle auf S.257.
Innerhalb geschlossener Geb¤ude dürfen weibli-
che Armeeangehörige die Uniformjacke gegen die
Uniformweste auswechseln.
Generale tragen zur Dienstuniform immer Stie-
felhosen mit Lampassen, glatte Schaftstiefel und
braunes Lederkoppel mit Schnalle.Zur Dienstuni.
form Nr.1 für den Sommer gehören Uniformjacke.
graues Oberhemd bzw.silbergraue Hemdbluse.
Binder und Schirmmütze.Dazu darf der Sommer-
mantel getragen werden.In der zweiten Sommerva.
riante tritt an die Stelle der Uniformjacke die
Hemdbluse mit geöffnetem Kragen.Ist der Som-
mermantel notweridig, wird der Kragen geschlossen
und der Binder umgelegt.
In der bergangsperiode können die betreffen-
den Armeeangehörigen zwischen niformmantel
und Sommermantel wahlen.Wird der Uniform
mantel im bergang wie auch im Winter zu
Dienstuniform getragen, muß das Lederkoppel dar.
über angelegt werden.In der Winterzeit ergänzen
Wintermütze und Lederhandschuhe die Dienstuni.
form der Generale. Seit dem 1.Dezember 1986 er
halten-wie schon angedeutet- alle Generale der
NVA eine neue einreihige Uniformjacke, die im
Schnitt ihrer Paradeiacke entspricht.Diese einrei.
hige Uniformjacke löst die zweireihige Stabsdienst
jacke ab, die noch bis zum 30.November 1988 auf-
getragen werden durfte.
Auch bei den Generalen unterscheidet sich die
Dienstuniform von der Stabsdienstuniform vor allem dadurch, daB die Uniformhose anstelle der
Stiefelhose und die schwarzen Halbschuhe bzw. die
Zugstiefel anstatt der glatten Stiefel getragen wer-
den. Auf das Koppel wird verzichtet.
Matrosen im Grundwehrdienst, Unteroffiziers-
schiüler, Matrosen und Maate auf Zeit und im Re-
servistenwehrdienst, die in Gefechtseinheiten und
Stäben der Volksmarine an Land dienen, tragen die
Dienstuniform zum Standortstreifendienst,zum
Tages- und Innendienst, zum Dienst in St¤ben und
zu Dienstreisen.Gefechtseinheiten an Bord verrich-
ten ihren Wach- und Tagesdienst sowie ihren Strei-
fendienst im Standort in Dienstuniform.
Grundelemente der Dienstuniform sind blaues
Kieler Hemd, Kieler Kragen mit Kieler Knoten,
Klapphose, Halbschaftstiefel und Lederkoppel mit
SchloB. Diese Uniformstücke z¤hlen auf jeden Fall zur Dienstuniform im Sommer, in den Ãbergangs
zeiten und im Winter. Die Tellermütze wird im
Sommer und in der Ãbergangsperiode, die Winter-
mütze aus schwarzem Webpelz im Winter aufge-
setzt. Das Seemannshemd wird nur während der
Ãbergangsperioden und im Winter untergezogen.
Der Ãberzieher wird zur Dienstuniform Nr. 3 für
die Ãbergangsperiode und Nr.4 für den Winter ge-
tragen, Hinzu kommen dann die Wirkhandschuhe.
Bei Dienst in Stäben kónnen Halbschuhe anstatt
Halbschaftstiefel angezogen werden.
Zur Borduniform tragen Matrosen im Grund.
wehrdienst,Unteroffiziersschüler, Matrosen und
Maate auf Zeit im Sommer und in der {bergangs-
periode das Bordk¤ppi und im Winter die Winter-
mütze sowie den Ãberzieher. Zum weißen Bordan-
zug gehört generell der Kieler Kragen, In der Übergangszeit werden darunter das Seemanns-
hemd, im Winter das Seemannshemd und der Pul-
lover mit Rollkragen gezogen. Bordschuhe vervoll.
ständigen die Borduniform.
Die Dienstuniform der Berufssoldaten der Volks-
marine entspricht in ihrer Zusammenstellung der
Stabsdienstuniform der Berufssoldaten der Land-
streitkräfte und der LSK/LV,Im Unterschied zu
diesen schnallen sie jedoch bei der Durchführung
von Tagesdiensten das Lederkoppel um. Für Be-
rufsunteroffiziers-,Fahnrich- und Offziersschüler
legt die Vorschrift fest, daß diese innerhalb der mi-
litärischen Lehreinrichtungen anstelle der Schirm-
mütze das Bordkäppi zu tragen haben, Berufsunter-
offiziersschüler, Berufsunteroffiziere, Fahnrichschü-
ler, Fahnriche und Offiziere von Gefechtseinheiten
an Bord tragen zur Ausbildung und zum t¤glichen Dienst an Bord und zur Esseneinnahme eine spe-
zielle Dienstuniform - die Borduniform. Diese
Uniformart bietet für jede Trageperiode zwei M¶g
lichkeiten, die in nebenstehender Tabelle verdeut-
licht werden.
Die Dienstuniform der Admirale entspricht in
ihrer Zusammenstellung und in der Zuordnung der
Uniformstücke der der Stabsdienstuniform der Ge-
nerale. Admirale führen in der Dienstuniform In-
spektionen und Truppenbesichtigungen durch und
tragen sie zum täglichen Dienst.
Die Entwicklung der Dienstuniform für alle Ar-
meeangeh¶rigen und der Stabsdienstuniform für
Berufssoldaten und weibliche Armeeangehörige ist
ein Ergebnis der gewachsenen Möglichkeiten der
Volkswirtschaft der DDR, die NVA mit mehr und
besseren Uniformstücken auszustatten. Sie ist ein
wesentlicher Beitrag zur Verbesserung der Dienst-
bedingungen der Armeeangehörigen.