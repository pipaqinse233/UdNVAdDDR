

Die Entwicklung der Felddienstuniform gestaltete
sich in nahezu drei Jahrzehnten vom Kampfanzug
im Flächendruck über weitgeschnittene und für un-
terschiedliche Dienstgradgruppen auch anders ge.
staltete Felddienstanzüge für den Sommer und
steingraue Watteanzüge für den Winter hin zum
körpernahen Felddienstanzug der 80er,Jahre. In seiner äußeren Form und Schnittgestaltung hat er für
alle Dienstgradgruppen, für männliche und weibli-
che Armeeangehörige, für Sommer und Winter ein
gleiches Aussehen. Damit wurde eine allgemeine
Uniform für die gesamte Ausbildung geschaffen,
und der Sommerdienstanzug, der vordem zur Aus-
bildung getragen wurde, fiel weg. Viele Verände-
rungen wurden in den 70er und 80er Jahren vorge-
nommen, um den Kämpfern bessere Trageeigen-
schaften, höhere Beweglichkeit und bekleidungshy-
gienisches Wohlbefinden zu sichern. Die Weiter-
entwicklung zum körpernahen Felddienstanzug für
den Sommer und für den Winter entspricht diesem
Ziel und pr¤gt das Bild der Felddienstbekleidung
der Armeeangehörigen in der zweiten Halfte der
80er Jahre. Die Felddienstuniform Nr.1 für den
Sommer besteht aus Felddienstanzug im Tarnstri.
cheldruck, Feldmütze für alle Dienstgradgruppen
der Landstreitkräfte und der LSK/LV(nur Fallschirmjäger und weibliche Armeeangehörige tragen
dazu die dunkelgraue Baskenmütze), Bordkäppi fir
alle Angehörigen der Volksmarine, Gurtkoppel und
Halbschaftstiefeln, Berufsoffiziere ziehen statt der
Halbschaftstiefel genarbte, Generale und Admirale
glatte Schaftstiefel an, Letztere tragen anstelle des
Gurtkoppels das Lederkoppel.
Hauptbekleidungsstück der. Felddienstuniform
Nr, 2 für die Ãbergangsperiode ist gleichfalls ein-
heitlich für alle Armeeangehörigen der Felddienstanzug
(Sommer). Soldaten im Grundwehrdienst,
Unteroffiziersschüler, Soldaten und Unteroffiziere
auf Zeit und im Reservistenwehrdienst der Land.
streitkräfte und der LSK/LV ziehen darunter Uni-
formjacke und Uniformhose und ergänzen diese
Uniform generell mit Feldmütze, Gurtkoppel und
Halbschaftstiefeln. Wenn angeordnet, werden auch
Vierfingerhandschuhe, Stahlhelm, Tragegestell und
Ausrüstung dazu getragen, Berufsunteroffiziers
schüler, Berufsunteroffiziere, Fahnrichschüler, Of.
fiziersschüler, Fahnriche und Offiziere ziehen zur
Felddienstuniform Nr.2 in der Übergangsperiode
unter den Felddienstanzug(Sommer)Uniform
jacke, Stiefelhose und silbergraue Hemdbluse mit
Binder, Weiter kommen genarbte Stiefel, Gurtkop
pel, Leder- bzw.Vierfingerhandschuhe und Feldmüitze oder Bordkäppi dazu. Die analoge Beklei-
dungsanordnung zur Felddienstuniform Nr.2 in der
Ubergangsperiode gilt auch für Generale, nur daß
diese Lederkoppel und glatte Schaftstiefel dazu um-
schnallen bzw.tragen.
Weibliche Armeeangehörige und Berufssoldaten
der Volksmarine,zu deren Ausstattung keine Stie
felhose gehört, ziehen zur Felddienstuniform Nr. 2
in der Übergangsperiode unter den Felddienstan-
zug(Sommer)Uniformjacke und Uniformhose, sil
bergraue Hemdbluse und dunkelgrauen Binder
Wie auch zur Felddienstuniform Nr.3 der ber
gangsperiode gehört bei den Frauen die dunkel-
graue Baskenmütze zur Anzugsordnung.
Matrosen im Grundwehrdienst, Unteroffiziers
schiler, Matrosen und MMaate auf Zeit und im Re
servistendienst derVolksmarine tragen im Unter
schied zu den Angehörigen der gleichen Dienst-
gradgruppen bei den Landstreitkräften und den
LSK/LV unter dem Felddienstanzug (Sommer) den
Arbeitsanzug und das Seemannshemd.
Zum Felddienstanzug Nr.3 für die Ãbergangspe.
riode tragen alle Armeeangehörige den in der ¤uße.
ren Form dem Felddienstanzug (Sommer) angepaß
ten Felddienstanzug(Winter).Dazu kommen
Feldmütze, Baskenmütze oder Bordkäppi.Vom
1.Dezember bis Ende Februar, der Winterperiode
ist der Felddienstanzug Nr.4 oder Nr.5 immer mit
aufgeknöpftem Webpelzkragen und Wintermütze
befohlen. Im äuBeren Erscheinungsbild gleichen
sich diese für den Winter bestimmten Felddienst.
uniformen, Wahrend bei der Felddienstuniform
Nr.4 für mildere Wintertage unter dem Felddienst
anzug (Winter) nur Unterwäsche getragen wird, zie
hen die Armeeangehörigen bei der Felddienstuni.
form Nr.5 unter den Felddienstanzug (Winter) dic
gleichen Kleidungsstücke wie bei der Felddienst
uniform Nr. 2 für den Übergang.
Nur Angeh¶rige der Volksmarine tragen bei Aus
lösung höherer Stufen der Gefechtsbereitschaft, bei
taktischen bungen auf See und auf besonderen
Befehl an Bord der Kampfschiffe die Gefechtsuni
form,Zu dieser Uniformart gehört als Hauptbekleidungsstück der orangefarbene Kampfanzug
der
Volksmarine, der durch
wasserundurchlässige
Nahtgestaltung weiter verbessert worden ist. Im
Sommer, in den Ãbergangszeiten und im Winter
kommen Gummistiefel hinzu. Im Winter und in
den Ãbergangsperioden wird der Arbeits- bzw. von
Berufssoldaten der Bordanzug untergezogen. Im
Sommer und in den Ãbergangsperioden sind Bord-
käppi, im Winter Wintermütze, auf Befehl auch
Kopfschützer bzw. Stahlhelm, die dazugehörige
Kopfbedeckung.