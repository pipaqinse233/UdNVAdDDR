

Am 3.Januar 1974 wurden in der NVA die ersten
Stabsfeldwebel und Stabsobermeister in einem fei-
erlichen Akt zum Fahnrich ernannt. Damit ent-
stand gemäß dem Beschluß des Nationalen Vertei-
digungsrates der DDR vom 17. Mai 1973 über die
Schaffung eines Fahnrichkorps und dem Befehl
Nr,168/73 des Ministers für Nationale Verteidi-
gung eine neue Kategorie von milit¤rischen Ka-
dern.

Die Entwicklung im Milit¤rwesen hatte Anfang
der 70er Jahre den Bedarf an Militärspezialisten
sprunghaft anwachsen lassen.Militärspezialisten
mit der Befahigung, milit¤rische, militärtechnische
und administrative Aufgaben relativ selbständig
und eigenverantwortlich zu l¶sen, waren rar. Des-
halb wurde mit dem Ausbildungsjahr 1973/74 be-
gonnen, das Fähnrichkorps der NVA aufzustellen.
Erfahrungen der Sowjetarmee und anderer Bru-
derarmeen nutzend, waren und sind die Fahnriche
der NVA nicht wie in früheren deutschen Armeen
selbständige
Offiziersanwärter,
sondern eine
Dienstgradgruppe zwischen Berufsunteroffizieren (Meisterabschluß) und Offizieren (Hochschulab
schluß). F¤hnriche mit abgeschlossener Ausbildung
verfügen über eine militärische Fachschulqualifika-
tion und schließen somit eine Lücke im Bildungsge-
füge der NVA.
Fähnriche kamen dort zum Einsatz, wo langjäh-
rige militärische, milit¤rtechnische, militärökono-
mische und administrative Erfahrungen Vorausset-
zung für die Lösung abgegrenzter komplizierter
Aufgaben waren und über l¤ngere Zeiten ein stabi-
ler Bestand von Kommandeuren und Militärspezia-
listen der unteren Führungsebenen benötigt wurde.
Voraussetzung für die Einstufung als Fähnrich wa-
ren hohe politische, militärische und spezialfachli-
che Bildung und langjährige praktische Erfahrun-
gen.
Entsprechend der Stellung der Fahnriche zwi-
schen den Berufsunteroffizieren und den Offizieren
und der erreichten Qualifikation konnten Fähnriche Dienststellungen in drei Verantwortungsebenen
einnehmen.In die erste Stufe eingeordnet wurden
Fahnriche,die als Truppführer, Leiter von Werk
staätten,Flugleitstellen,Nachrichtenzentralen.VS
Stellen,Küchen und Wartungspunkten sowie als
technische Spezialisten, Ausbilder und Instrukteure
ihren verantwortungsvollen Dienst versahen. Hǒ
here Anforderungen wurden an jene gestellt, die die
Dienststellung eines Zugführers oder Gleichgestell.
ten wie Hauptfeldwebel, Leiter von Werkstätten
Nachrichtenzentralen und Lagern,denenjeweils
weitere Einheiten und Einrichtungen unterstellt wa-
ren,innehatten, In die dritte Kategorie eingeordne
und somit am höchsten gefordert wurden Fähnri.
che, die als Stellvertreter des Kompaniechefs für
technische Ausrüstung,Techniker,eiter des Dien
stes eines Truppenteils und Gehilfen für Nachrich.
tenverbindungen eines 'Truppenteils eingesetzt wur
den.
Die Achtung vor den Leistungen dieser in der
NVA neuen Dienstgradgruppe, die sich anfänglich
wie die Berufsoffiziere zu einer 25jährigen Dienst
zeit verpflichteten, kam auch in der Ausstattung der
Faähnriche mit Bekleidung und Ausrüstung zum
Ausdruck,Fähnriche erhielten die gleiche Beklei.
dung und Ausrüstung wie Berufsoffiziere. Nu
Feldbinde und Ehrendolch blieben nach wie vor al
leinige Attribute des Offiziers. Fahnriche der Land.
streitkräfte und der LSK/LV trugen zu ihrer Kenn-
zeichnung
Mützenabzeichen,Mützenkordel,Kra
genspiegel und Armelpatten wie ffiziere.Dic
Schulterst¼cke bestanden aus silberfarbenen, außen
steingrauen Plattschnüren mit je zwei goldfarbenen
Sternen. Für die Felddienstbekleidung gab es
Schulterstücke aus mattgrauen Plattschnüren mi
ebensolchen Sternen.
Dienstlaufbahnabzeichen
wurden wie bei den Ünteroffizieren an der Uniform
befestigt, Hauptfeldwebel mit Fahnrichdienstgrad
waren an den silberfarbenen Armelstreifen erkennt
lich; Fahnriche in der Technikerkombination an
einer 12mm breiten und 1cm langen lresse am
Armel.
Fähnriche der Volksmarine führten Mitzenabzeichen wie ffiziere bis einschließlich Kapitän
leutnant und Armelabzeichen aus einer 7mm brei
ten und 10cm langen goldfarbenen Tresse.Dic
Schulterstücke bestanden aus silberfarbenen un
dunkelblauen Plattschn¼ren mit je zwei goldfarbe
nen Sternen und den Dienstlaufbahnabzeichen da
zwischen. Hauptfeldwebel waren zus¤tzlich mit
goldfarbenen Armelstreifen gekennzeichnet.
Die Fahnriche aller Teilstreitkräfte nahten zu
besonderen KennzeichnungihrerDienstgrad
gruppe auf dem linken Ober¤rmel der Uniform
jacke und des Uniformmantels ein in der Farbe des
Mantels gehaltenes Armelabzeichen mit dem
Staatswappen der DDR auf. Zus¤tzlich waren au
dem Arelabzeichen zurKennzeichnungde
Dienstalters Sterne angebracht, ab 1l. Dienstjahr
1 Stern,ab 16.Dienstiahr 2 Sterne und mit Beginn
des 21.Dienstjahres 3 Sterne.
Fahnrichschüler der Landstreitkräfte und de
LSK/Ly erhielten Bekleidung und Ausrüstung wie
Soldaten,ihre Parade-und Ausgangsuniform be
stand jedoch aus Schirmmütze bzw. Wintermütze
Uniformmantel,Parade-/Ausgangsjacke, Hemd
bluse,Stiefelhose und glatten Schaftstiefeln wie Be
rufssoldaten,Damit wurden die Fähnrichschüler in
der Ausstattung den Offiziersschülern beider Teil
streitkräfte gleichgestellt, Sie trugen Schulterklap
pen wie Unteroffiziere,allerdings mit einem stili
sierten «F».An der Dienstjacke,der Parade-/Aus
gangsjacke sowie am Uniformmantel befand sich
wie bei den Fahnrichen ein Armelabzeichen. Im
Unterschied zu den Offziersschiülern wurde keine
Kragenlitze aufgenäht.
Fahnrichschüler der Volksmarine erhielten eben.
falls Uniformen wie die Offziersschüler des 1. und
2.Lehrjahres dieser Teilstreitkraft. Sie trugen Effek
ten einschlieBlich Dienstlaufbahnabzeichen wie dic
Matrosen.Zus¤tzlich befestigten sie ein rmelab
zeichen am Kieler Hemd, an der weißen Bordbluse
und am Überzieher. Darunter wurden die Dienst
laufbahnabzeichen in der Ausf¼hrung wie für Ma
trosen aufgenäht.
Der Fahnrichberuf bot für junge Menschen eine interessante und anspruchsvolle berufliche Ent-
wicklung. Nicht zuletzt die den Offizieren gleichge.
stellte Ausstattung dieser neuen Dienstgradgruppe
mit Bekleidung und Ausrüstung und ihre gute ma-
terielle und finanzielle Versorgung beschleunigten
bei vielen jungen Leuten den Entschluß, einen mili-
tärischen Beruf zu ergreifen. Fahnrichschüler und
F¤hnriche waren dazu aufgefordert, sich am Kampf
um die fünf Soldatenauszeichnungen zu beteiligen.
Für das Tragen des Bestenabzeichens, der Schüt-
zenschnur,des Militärsportabzeichens, des Abzei-
chens *Für gutes Wissen» und von Qualifizierungs-
spangen galten die gleichen Festlegungen wie für
Soldaten und Unteroffiziere.