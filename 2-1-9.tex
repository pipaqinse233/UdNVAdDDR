

Im Herbst 1960 entwickelt und seit 1961 einge-
fihrt, war der Ehrendolch fǔr die Offiziere, Gene-
rale und Admirale erstmals wahrend der Parade
zum 1.Mai 1961 in Berlin, der Hauptstadt der
DDR, in der Ãffentlichkeit zu sehen. Die Anord
nung Nr,22/62 des Ministers für Nationale Vertei-
digung der DDR vom 26. April 1961 regelte jedoch
umfassend die Einführung, den Verkauf und die
Registrierung des Ehrendolches.
Diesen Dolch für Offiziere (ausgenommen blie-
ben die weiblichen Offiziere), Generale und Admi-
rale der NVA gab und gibt es in mehreren Ausfüh-
rungen.Sie unterscheiden sich jeweils durch ihre
Oberflächen- und Gehängefarben voneinander. Für
Offiziere der Landstreitkräfte und der LSK/LV ist
der Dolch an Knopf, Griffring, Parierstange sowie
Mund- und Ortblech der Scheide silberfarben. In
den Landstreitkräften besteht das Gehänge aus sil
berfarbenen Litzen auf steingrauem Untergrund,
bei den LSK/LV auf hellblauem Untergrund. Offi-
ziere der Volksmarine haben einen Dolch, bei dem dieser Untergrund dunkelblau ist, Gehänge und
Ehrendolch sind goldfarben.
Bei Generalen und Admiralen sind Knopf, Griff.
ring,Parierstange sowie Mund- und Ortblech der
Scheide des Dolches feuervergoldet. Das Gehinge
des Dolches besteht aus Goldlitzen und befindet
sich bei den Generalen der Landstreitkräfte auf ro-
tem, bei denen der LSK/LV auf hellblauem und bei
Admiralen auf dunkelblauem Untergrund.
Zun¤chst war der Dolch zur Parade mit und
ohne Uniformmantel an der Feldbinde,zum Ur-
laub und im Ausgang zur Ausgangsjacke unterge-
schnallt bzw,beim Uniformmantel durch den Tascheneingriff gezogen zu tragen, Am 16, März 1964
präzisierte der Stellvertreter des inisters für Na-
tionale Verteidigung der DDR und Chef Rückwär.
tige Dienste der NVA die Trageweise des Dolches.
Danach war der Dolch zur Parade- und Ausgangs-
uniform an Staatsfeiertagen, am Jahrestag der NVA
und auf besonderen Befehl der Kommandeure (ab
Regimentskommandeur aufw¤rts)und der Stand-
ortältesten zu tragen,Der Dolch konnte auch zur
Ausgangsuniform im Ausgang und im Urlaub um-
geschnallt werden.