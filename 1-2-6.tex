

Die Uniformierung einer Armee vollzieht sich nicht
einfach dadurch, daß Uniformen hergestellt und an
die Soldaten ausgegeben werden. Es galt, die ver-
schiedenen Uniformen militärischen Tätigkeiten
zuzuordnen, die Trageweise zu bestimmen und
Normen für die Tragezeiten der einzelnen Uniform-
stücke festzulegen, Dazu bedurfte es umfangreicher
und sorgsam abgewogener ÃJberlegungen von Ange-
hörigen vieler Dienstbereiche. Der Minister für Na-
tionale Verteidigung der DDR wies in einer Anord-
nung über Vorbereitungsmaßnahmen für die
Einführung der Uniform der Land-, Luft- und See-
streitkräfte der Nationalen Volksarmee》 bereits am
4.Februar 1956 an, daß durch den Chef des Haupt-
stabes der NVA bis Ende des Monats der Entwurf
einer Dienstvorschrift über die Bekleidungsord-
nung und durch den Chef Rückwärtige Dienste die Normen und Tragezeiten für alle in die NVA einge-
führten Bekleidungs- und Ausrüstungsgegenst¤nde
auszuarbeiten sind.Des weiteren sollte bis Ende
M¤rz der Entwurf einer Instruktion fǔr die Rege
lung der B/A-Wirtschaft erarbeitet werden.
Viele Vorschläge sachkundiger Offiziere aus na-
hezu allen Dienstbereichen gingen im Ministerium
ein, nachdem Ende August 1956 der Entwurf der
Bekleidungsvorschrift zur Stellungnahme in die
Truppe versandt worden war.Am 15.Juni 1957
setzte der Minister für Nationale Verteidigung der
DDR die «Vorläufige Bekleidungsordnung der Na-
tionalen Volksarmee» in Kraft, Er hatte schon vier
Monate zuvor die Chefs der Luftstreitkräfte, der
Luftverteidigung, der Seestreitkräfte sowie die der beiden Militärbezirke der Landstreitkräfte angewie.
sen, alle Angehörigen ihrer Dienstbereiche mit die-
ser Ordnung vertraut zu machen.Die ständige
Pflege und Instandhaltung der Bekleidung und
Ausr¼stung sowie die Einhaltung der Bekleidungs.
ordnung galt es durchzusetzen, um das diszipli-
nierte Auftreten und das ¤uBere Bild der NVA-An-
gehörigen in der Öffentlichkeit zu verbessern.
Im ersten Punkt der <Allgemeinen Grundsätze》
dieser Bekleidungsordnung unterstrich die Führung
der NVA den Charakter der Uniform der NVA. Sie
ist, hieß es, «das Ehrenkleid aller Angehörigen der
Nationalen Volksarmee der Deutschen Demokrati-
schen Republik. Jeder Angehörige der bewaffneten
Kräfte der Deutschen Demokratischen Republik ist
verpflichtet, die Ehre und Würde der Uniform zu
wahren».
Weitere Punkte jener Grunds¤tze regelten das
Tragen der Uniformen der NVA für Soldaten, Matrosen,Unteroffiziere,Maate,Offiziere,Generale
und Admirale, die im aktiven Dienst der NVA ste.
hen bzw, für diejenigen, die aus dem aktiven Dienst
entsprechend denDienstlaufbahnbestimmungen
ausgeschieden sind.In mehreren Kapiteln
be
stimmte die Vorschrift die Uniformarten, die Art
und die Trageweise der Uniformen sowie die Trage.
weise der Orden, Medaillen und Abzeichen. Festge-
legt wurde,welche Uniformarten für welche Dien.
ste galten (s. Tabelle S.14); innerhalb der Einheiten
muBte die Uniform bei gleicher Dienstverrichtung
einheitlich sein. Die Sommerbekleidung wurde in
den Land- und Luftstreitkräften in der Zeit vom
1.April bis zum 30.September, in den Seestreitkräf.
ten vom 1.Mai bis zum 30.September getragen. Die
restlichen Monate blieben der Winterbekleidung
vorbehalten,es sei denn,die Standortältesten ver
kürzten oder verl¤ngerten aufgrund der Wetterver
haltnisse die Tragezeit der Sommer- bzw. Winter
uniformen.
Auf eine Besonderheit soll in diesem Zusammen-
hang verwiesen werden.Generale erhielten schon
seit geraumer Zeit eine zusätzliche Uniform - eine
Modifikation der Ausgangsuniform.Aufgrund
einer Anordnung des Ministers für Nationale Ver.
teidigung der DDR vom 5.Juli 1956 war bei beson
deren feierlichen Anl¤ssen in der heißen [ahreszeit
das Tragen einer weißen Uniformjacke mit Schul.
terstücken und Kragenspiegeln sowie einer weißen
Mitze mit breitem rotem Rand genehmigt. Dazu
wurde die steingraue Tuchhose getragen.
Ein Hauptbestandteil der «Vorl¤ufigen Beklei.
dungsordnung der Nationalen Volksarmee》 von
1957 sind Regelungen zur Trageweise der einzelnen
Uniformstücke.Eines ist allerdings allen Armeen
der Welt in Geschichte und Gegenwart eigen: Man
che Soldaten,Unteroffiziere und sogar Offiziere su
chen nach MÃglichkeiten, um ihre xpersönliche
Note» in die Uniformierung einzubringen. Dies trat
auch in der NVA in der Art und Weise zutage, die
Kopfbedeckung aufzusetzen,Gelegentlich wurde
auch der Mützenring der Schirm- oder der Matro-
senmütze entfernt,um diesen ein-nach Ansicht der Träger-gefalligeres Aussehen zu verleihen.
Auch ein Knicken der Schulterklappen nach der
Halfte der Dienstzeit ist bis heute cin häufiger
Brauch der Soldaten und Unteroffiziere geblieben.