

Der Öffentlichkeit in der DDR präsentierte sich
die junge NVA, d.h. erste aufgestellte Truppenteile
und Verb¤nde, bei zwei bedeutsamen milit¤rischen
Zeremoniellen im Frihjahr 1956.Am 30. April
dem Vorabend des Kampftages der internationalen
Arbeiterklasse,waren die Soldaten, Unteroffiziere
und Offiziere des l.mech. Regiments der NVA in
ihren neuen Paradeuniformen zu einem feierlichen
Appell in Oranienburg angetreten. Funktionäre der
Partei der Arbeiterklasse,des Staates und der as
senorganisationen,Arbeiterveteranen,antifaschisti
sche Widerstandskämpfer und Kämpfer der Inter
brigaden wohnten diesem Zeremoniell auf der
Ehrentribüne bei, Neben der Tribüne standen Ein
heiten der Kampfgruppen, Abordnungen der GSl
und der FD]. Nach der Meldung an den Minister
fr Nationale Verteidigung der DDR,General
oberst W.Stoph, sprach dieser zu den Soldaten. Er
verpflichtete die Armeeangehörigen, das Kampf
banner der bewaffneten Volksmacht als Zeichen
der Würde des Truppenteils stets in Ehren zu hal
ten, Dann übergab er dem Kommandeur des Regi.
ments die Truppenfahne. Aus den H¤nden seines
Kommandeurs nahm stolz ein Unteroffizier die
Fahne entgegen, Vier junge Soldaten traten aus der
Paradeformation hervor,schritten auf die Fahnen
gruppe zu und beriihrten symbolisch für alle Ange
h¶rigen des Regiments die feierlich gesenkte Fahne
Von Hunderten Soldaten erschallte der Schwur,
das sozialistische Vaterland auch unter Einsatz des
Lcbens gegen jeden Feind zu schützen.

Am folgenden Tag nahm das Regiment - aufge
sessen auf Lkw G 5 - an der ersten Truppenparade
der NVA in Berlin anlaBlich des 1. Mai teil. Ein
Blick auf die am 2. Mai in der Presse veröffentlich
ten Fotos von der Maiparade verrät eine Vielzahl
von Details zur Uniformierung und Bewaffnung
der paradierenden Einheiten. So trugen die Musik
korps der Land- und Luftstreitkräfte an beiden
Ober¤rmeln ihrer Uniformen die aus der Geschichte bekannten Schwalbennester - mit Längs-
und Querborten verzierte Achselwülste, die ur-
spr¼nglich wie die Schulterklappen das Herunter-
gleiten des Lederzeugs verhindern sollten. Auf
Schüitzenpanzerwagen vom Typ BTR152
neuen
die Soldaten und Unteroffiziere mit ihren
hatten
neuen Stahlhelmen und Schützenwaffen in Parade-
haltung Platz genommen.