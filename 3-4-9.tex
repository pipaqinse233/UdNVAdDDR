

Für alle Armeeangehörige ist es Pflicht, verliehene
Auszeichnungen an der Parade-,Ausgangs- und
Gesellschaftsuniform zu tragen, Berufssoldaten
auch an der Dienst- und Stabsdienstuniform.
Die 1986er Vorschrift begrenzt die Trageweise
der Interimsspange auf maximal vier Reihen mit
insgesamt 16 Auszeichnungen,Wenn die Anzahl
der verliehenen Auszeichnungen diese Zahl über-
steigt, dürfen nur die 16 höchsten Auszeichnungen
angelegt werden, es sei denn, der Minister für Na-
tionale Verteidigung erteilt eine Ausnahmegeneh-
migung. Für den GroBen Gesellschaftsanzug gilt
die Festlegung, daß nur die 4 höchsten Orden und
Medaillen am Band in einer Reihe zu tragen sind.
Für die Paradeuniform setzt die Vorschrift die
obere Grenze auf 8 Orden und Medaillen am Band
in zwei Reihen fest.
An der Hemdbluse wird die Interimsspange mit
den 4 h'chsten verliehenen Auszeichnungen in
einer Reihe über der Patte der linken Brusttasche
getragen.Weibliche Armeeangehörige bringen an
der Hemdbluse und am Uniformkleid keine Aus.
zeichnungen an.

Zus¤tzlich zu den genannten Auszeichnungen ist
es möglich, über der Interimsspange auf der linken
Seite der Uniformjacke den Karl-Marx-Orden, die
Medaille <Goldener Stern》 zum Ehrentitel <Held
der DDR», die Medaille zum Ehrentitel «Held der
Arbeit» und den Vaterl¤ndischen Verdienstorden
zu tragen, Darunter k¶nnen dann die 16 höchsten
Auszeichnungen von rechts nachlinks eingereiht
werden, Dafür ist folgende Reihenfolge vorgesehen:
Scharnhorst-Orden.
a
Kampforden «Für Verdienste um Volk und a-
terland».
Orden «Banner der Arbeit».
Orden anderer Staaten,
Medaille für Teilnahme an den bewaffneten
Kämpfen der deutschen Arbeiterklasse in den
ahren 1918-1923
Medaille für Kämpfer gegen den Faschismus
1933-1945.
Hans-Beimler-Medaille.
Clara-Zetkin-Medaille.
Verdienstmedaille der DDR.
Medaillen zum Ehrentitel der NVA, der Grenz-
truppen der DDR und der Zivilverteidigung so-
wie zu anderen Ehrentiteln der DDR.
Verdienstmedaille der NVA, der Grenztruppen
der DDR und der Zivilverteidigung,
Ehrenzeichen <Für Verdienste in der Reservisten.
ausbildung»,
andere Verdienstmedaillen der DDR.
Medaille zum Ehrentitel <Aktivist der sozialisti-
schen Arbeit».
Medaille zum Ehrentitel <Kollektiv der sozialisti
schen Arbeit».
Verdienstmedaillen anderer Staaten,
Medaille der Waffenbrüderschaft der NVA.
Medaillen der Waffenbrüderschaft und militäri.
sche Erinnerungsmedaillen anderer Staaten,
andere Medaillen der DDR.
ubiläumsmedaille «30 Tahre NVA».
Medaillen für treue Dienste der NVA, der Grenz-
truppen der DDR und der Zivilverteidigung,
Medaillen für treue Dienste anderer Bereiche.

Weiterhin besteht die Möglichkeit, auf der rech.
ten Brustseite,über allen anderen staatlichen Aus
zeichnungen,die Abzeichen zum Ehrentitel «Flie
gerkosmonaut der DDR>unddarunterzum
<Hervorragenden Wissenschaftler des Volkes》 so.
wie zum Nationalpreis der DDR,zum Friedrich
Engels-Preis,zum Theodor-Körner-Preis und zu
anderen Preisen der DDR zu tragen.
Nichtstaatliche Auszeichnungen finden unter
den staatlichen Auszeichnungen aufder rechten
Seite der Uniformjacke ihren Platz. Ûber allen Aus
zeichnungen wird das Absolventenabzeichen de
höchsten Bildungseinrichtung befestigt. Rechts ne
ben oder unter den obengenannten Auszeichnun
gen werden auf der rechten Brustseite das Lei
stungsabzeichen der NVA.das Bestenabzeichen de
NVA,Klassifizierungsabzeichen, Fallschirmsprung
abzeichen,Reservistenabzeichen und andere Aus
zeichnungen der Parteien und gesellschaftliche
Organe oder Organisationen der DDR eingeordnet
Besitzt der Areeangehörige mehrere Klassifizie
rungsabzeichen oder einAbzeichen inmehreren
Stufen,steckt er nur die höchste Stufe des Abzei
chens an.
Mitglieder der FD]sind am auf der rechten
Brusttasche oder bei Uniformjacken ohne Brustta.
sche auf gleicher Höhe befestigten FD]-Abzeichen
zu erkennen.Das ilitärsportabzeichen oder das
Sportabzeichen der DDR kommt auf die Falte der
rechten Brusttasche.
Staatliche und nichtstaatliche Auszeichnungen
anderer sozialistischer und befreundeter Staaten
die an Bürger der DDR für Verdienste im Kampf
gegen Faschismus, füir den Frieden und fir den
Aufbau des Sozialismus verliehen wurden, dürfen
auch an derniform getragen werden. Sie sind
dann ihrer Bedeutung gemäß nach den Auszeich.
nungen der DDR einzuordnen.
In den mehr als 30 |ahren des Bestehens der Natio
nalen Volksarmee hat sich das Ehrenkleid der Ar
meeangeh'rigen entwickelt und vervollkommnet
Vielfaltige Einflüsse lösten Veränderungen auf dem Gebiet der Bekleidung und Ausr¼stung aus. An er-
ster Stelle ist immer wieder die Anpassung der Be-
kleidung und Ausrüstung an die mannigfaltigen Er-
fordernisse des militärischen Dienstes zu nennen.
Von groBer Bedeutung waren dabei auch Fragen
der Verbesserung der Dienst- und Lebensbedingun-
gen der Armeeangehörigen, die Berücksichtigung
der veränderten Bedingungen im Milit¤rwesen und
nicht zuletzt auch modischer Aspekte.

Die Nationale Volksarmee entwickelte sich durch
die Leistungen mehrerer Generationen von Solda-
ten, Unteroffizieren, Fahnrichen, Offizieren, Gene-
ralen und Admiralen zu einer modern ausgeriiste-
ten und ausgebildeten sozialistischen Koalitionsar-
mee. Die Uniform der Nationalen Volksarmee war
und ist das Ehrenkleid der sozialistischen deutschen
Soldaten, der Verteidiger des Friedens und des So-
zialismus.