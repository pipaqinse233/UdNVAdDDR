

Als einen Monat später mit dem gesetzgeberischen
Akt der Volkskammer der DDR die Schaffung der
Nationalen Volksarmee beschlossen wurde, begann
eine angespannte Arbeit der Angehörigen der rick-
wärtigen Dienste der NVA und vieler in der Beklei-
dungs-, Schuh- und Lederwarenindustrie der DDR
beschäftigten Werkt¤tigen, Es war im Gründungs
jahr der NVA nicht sofort möglich, die Armeeange-
hörigen vollständig mit den vorgesehenen Uniformarten zu versorgen. Die ökonomische Lage und die
kurze Zeitspanne von der Gesetzgebung bis zur
Aufstellung der Verbände und Truppenteile der
NVA machten es erforderlich, die Uniformierung
etappenweise durchzuführen.Deshalb legte der Mi-
nister für Nationale Verteidigung der DDR, Gene
raloberst W.Stoph,in seinem ersten Befehl zur
«Bildung der Nationalen Volksarmee, des Ministe.
riums für Nationale Verteidigung und Einfihrung
der Uniformen der Nationalen Volksarmee》 vom
10.Februar 1956 fest, die Armeeangehörigen ent
sprechend dem Zeitplan der Aufstellung der Ein.
heiten so einzukleiden, «daß zunächst alle Angehö.
rigen der neu aufzustellenden Dienststellen mit ie
einer Uniform(Ausgehuniform)ausgestattet wer
den».Schon im Frühjahr folgte die Ausgabe der er
sten Dienstuniformen.

Um von Beginn an den täglichen Dienst und diá
militärische Ausbildung durchführen zu kónnen,
griff die Fihrung der NVA, wie es seit langem bei
Ãähnlichen Situationen in anderen L¤ndern geschah.
auf vorhandene Uniformbestände anderer bewaff.
neter Organezurick.So wiesGeneraloberst
W.Stoph im selben Befehl an, während des Dien
stes verfügbare Uniformen aus dem Bestand de
KVP,deren Auflösung bis zum 1.Dezember 1956
erfolgte,aufzutragen.

Viele Angehörige der KVP hatten sich bereit er.
klärt,in die zu bildenden Streitkräfte einzutreten,
Sie trugen zunächst ihre khakifarbenen Uniformen
weiter. Da sich auch noch groBe Vorräte an KVP.
Uniforen und -stoffen in den Lagern befanden
konnten diese Uniformen im t¤glichen militäri.
schen Dienst bis Ende der 50er Jahre genutzt wer.
den. Damit wurde die Volkswirtschaft der jungen
Republik beträchtlich entlastet.

Bei den Seestreitkräften war die Uniformierung
ihrer Angehörigen in allen Dienstgraden einfacher
Die Volkspolizei-See(VP-See)war schon in der
Gestaltung ihrer Uniformen nationalen Beispielen
wie auch einem international einheitlichen rend
in der Entwicklung der Marineuniformen gefolgt.
Anfangs trugen die Matrosen der Seestreitkrafte der NVA noch ein Mützenband mit der Aufschrift
<See》 an der Tellermütze. Bereits am 3. Februal
1956 ordnete der Minister für Nationale Verteidi.
gung der DDR an, auf den MMützenbändern die Be
zeichnung xSeestreitkräfte» zu fiihren.

Die Verwendung von khakifarbenen Uniformen
der KVP in den Land- und Luftstreitkraften der
NVA bis zur vollständigen Einführung steingrauer
Dienstuniformen regelte der Chef Rickwärtige
Dienste der NVA,Generalmajor W.Allenstein, in
einer speziellen Anordnung vom 18.April 1956. Sie
sah u.a. vor, wie die khakifarbenen Uniformen de
Soldaten und Unteroffiziere beider Teilstreitkraäfte
geringfügig verändert werden sollte. Die Unifor.
mäntel blieben ohne Kragenspiegel. Dagegen wur.
den auf die Kragen der Dienst- und der Drillich
uniformen Kragenspiegel aufgenäht und Schulter.
klappen der NVA getragen,Alle diese Uniformen
wurden mit silberfarbenen Knöpfen versehen. Dic
Koppelschlösser der Lederkoppel wurden schritt
weise gegen solche mit dem geprägten Staatsem
blem der DDR ausgewechselt, Vorhandene khaki.
farbene Ausgangsuniformen und -mäntel ersetzten,
eingezogen und gereinigt,bei Bedarf abgetragene
Dienstuniformen bzw.-mäntel der NVA.

Die Schirmmützen der ffiziere bekamen ein
dunkelgraues Mützenband und eine silberfarbené
Kordel, Bei den Landstreitkraften wurden sie au
Berdem mit einem Mützenemblem gleicher Farbe
bei den Luftstreitkraften mit Schwinge und Ko
karde sowie Propeller mit Ahrenkranz versehen
Diese nderungen wurden auch an den Winter
mützen der ffiziere vorgenommen, Wie bei den
Soldaten und Unteroffizieren wurden auch die Uni
formjacken der Offiziere mit Kragenspiegel und
der NVA und silberfarbenen
Schultersticken
KnÃöpfen(auch an den ¤nteln)versehen.Gene.
rale trugen ihre Schirmmützen ebenfalls mit den
Effekten der NVA.

Geplant war, bis zum Sommer 1958 die khakifar-
benen Uniformen restlos aufzutragen. Ab Herbst
des genannten ,Jahres wurden in der NVA die stein-
grauen Uniformen bestimmend. Die Werktätigen
der Bekleidungswerke und die Angehörigen der
rückwärtigen Dienste leisteten'in jenen Wochen
und Monaten eine angespannte Arbeit, Beispiels-
weise hatte die NVA mit dem VEB Burger Beklei-
dungswerke- heute Leitbetrieb für Dienstbeklei-
dungen in der DDR- vereinbart, bis zum 30.April
1956 für Soldaten und Unteroffiziere 55000 Uni-
formen, d,h.Uniformjacken und -hosen, auszulie-
fern, Dies erfolgte schrittweise nach einem festge-
legten Größenschlüssel, Fir die Landstreitkrafte
muBten dabei unterschiedliche Paspelierungen der Waffenfarben bei der Produktion beachtet werden.
Ein anderer Betrieb, die Halleschen Kleiderwerke,
lieferte zum selben Termin 9000 Offiziersunifor.
men.Sie bestanden aus Uniformjacke,-hose und
Stiefelhose. Der VEB Leipziger Bekleidungswerke
stattete die NVA bis zum 30.April 1956 mit fast
34000 Uniformm¤nteln für Soldaten und Unterof-
fiziere aus.

Als auBerordentlich kompliziert erwies sich die
Fertigung der Effekten für die Uniformen, insbe-
sondere die Herstellung der Kragenspiegel und Ãr.
melpatten. Sie wurden in noch gröBerer Anzahl,
nämlich auch für die khakifarbenen Uniformen, be-
nÃötigt, Dabei waren vor allem Werktätige mit aus-
geprägtem handwerklichem Geschick gefragt. Zu-
gleich galt es auch, Betriebe mit
geeigneten
Stickmaschinen und entsprechenden Kapazitäten
zu finden.