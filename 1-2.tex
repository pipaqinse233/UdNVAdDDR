\section{}

Die Aufstellung der Stäbe, Verbände und Truppenteile der NVA 1956 war eine gewaltige Leistung der verantwortlichen Generale, Admirale und Offiziere. Die große Arbeit der rückwärtigen Dienste, insbesondere der Angehörigen der damaligen Verwaltung Bekleidung und Ausrüstung sowie der für dieses Fachgebiet zuständigen Offiziere in den Teilstreitkräften, Militärbezirken, Verbänden und Truppenteilen, trug entscheidend dazu bei, die Probleme zu lösen, die mit der Uniformierung einer ganzen Armee verbunden waren. Die Versorgung aller Verbände und Truppenteile mit den Uniformen und mit Sonderbekleidung ging bis Ende der 50er Jahre vonstatten. Zugleich wurde mit der notwendigen Entwicklung zusätzlicher Uniformen und verbesserter Uniformstücke sowie von Sonderbekleidung für spezielle militärische Aufgaben begonnen.

Mit der Unterzeichnung des Warschauer Vertrages am 14. Mai 1955 auch durch die DDR wurde bereits sichtbar, daß die Arbeiterklasse des Landes und ihre Verbündeten erstmals in der deutschen Militärgeschichte eigene nationale Streitkräfte schaffen würden. Deshalb berieten Leitungsorgane der Kasernierten Volkspolizei darüber, wie sie die Aufstellung von Truppenteilen und Einheiten einer Armee dann unterstützen könnten, wenn die Volks
kammer der DDR dies durch Gesetz beschlieBen
wiirde, So fand am 19.Dezember 1955 eine Sitzung
des Kollegiums leitender Kader der KVP statt, au
der Muster der beabsichtigten neuen Uniformen
vorgestellt und beraten wurden, Den Vorsitz führte
der Stellvertreter des Vorsitzenden des Ministerra-
tes der DDR,Generaloberst W.Stoph. Neben dem
Aussehen der Uniformen standen hier wie bei ähn.
lichen Gelegenheiten Normen und Tragezeiten der
Bekleidung und Ausrüstung im Mittelpunkt der
Ãberlegungen.



Als einen Monat später mit dem gesetzgeberischen
Akt der Volkskammer der DDR die Schaffung der
Nationalen Volksarmee beschlossen wurde, begann
eine angespannte Arbeit der Angehörigen der rick-
wärtigen Dienste der NVA und vieler in der Beklei-
dungs-, Schuh- und Lederwarenindustrie der DDR
beschäftigten Werkt¤tigen, Es war im Gründungs
jahr der NVA nicht sofort möglich, die Armeeange-
hörigen vollständig mit den vorgesehenen Uniformarten zu versorgen. Die ökonomische Lage und die
kurze Zeitspanne von der Gesetzgebung bis zur
Aufstellung der Verbände und Truppenteile der
NVA machten es erforderlich, die Uniformierung
etappenweise durchzuführen.Deshalb legte der Mi-
nister für Nationale Verteidigung der DDR, Gene
raloberst W.Stoph,in seinem ersten Befehl zur
«Bildung der Nationalen Volksarmee, des Ministe.
riums für Nationale Verteidigung und Einfihrung
der Uniformen der Nationalen Volksarmee》 vom
10.Februar 1956 fest, die Armeeangehörigen ent
sprechend dem Zeitplan der Aufstellung der Ein.
heiten so einzukleiden, «daß zunächst alle Angehö.
rigen der neu aufzustellenden Dienststellen mit ie
einer Uniform(Ausgehuniform)ausgestattet wer
den».Schon im Frühjahr folgte die Ausgabe der er
sten Dienstuniformen.

Um von Beginn an den täglichen Dienst und diá
militärische Ausbildung durchführen zu kónnen,
griff die Fihrung der NVA, wie es seit langem bei
Ãähnlichen Situationen in anderen L¤ndern geschah.
auf vorhandene Uniformbestände anderer bewaff.
neter Organezurick.So wiesGeneraloberst
W.Stoph im selben Befehl an, während des Dien
stes verfügbare Uniformen aus dem Bestand de
KVP,deren Auflösung bis zum 1.Dezember 1956
erfolgte,aufzutragen.

Viele Angehörige der KVP hatten sich bereit er.
klärt,in die zu bildenden Streitkräfte einzutreten,
Sie trugen zunächst ihre khakifarbenen Uniformen
weiter. Da sich auch noch groBe Vorräte an KVP.
Uniforen und -stoffen in den Lagern befanden
konnten diese Uniformen im t¤glichen militäri.
schen Dienst bis Ende der 50er Jahre genutzt wer.
den. Damit wurde die Volkswirtschaft der jungen
Republik beträchtlich entlastet.

Bei den Seestreitkräften war die Uniformierung
ihrer Angehörigen in allen Dienstgraden einfacher
Die Volkspolizei-See(VP-See)war schon in der
Gestaltung ihrer Uniformen nationalen Beispielen
wie auch einem international einheitlichen rend
in der Entwicklung der Marineuniformen gefolgt.
Anfangs trugen die Matrosen der Seestreitkrafte der NVA noch ein Mützenband mit der Aufschrift
<See》 an der Tellermütze. Bereits am 3. Februal
1956 ordnete der Minister für Nationale Verteidi.
gung der DDR an, auf den MMützenbändern die Be
zeichnung xSeestreitkräfte» zu fiihren.

Die Verwendung von khakifarbenen Uniformen
der KVP in den Land- und Luftstreitkraften der
NVA bis zur vollständigen Einführung steingrauer
Dienstuniformen regelte der Chef Rickwärtige
Dienste der NVA,Generalmajor W.Allenstein, in
einer speziellen Anordnung vom 18.April 1956. Sie
sah u.a. vor, wie die khakifarbenen Uniformen de
Soldaten und Unteroffiziere beider Teilstreitkraäfte
geringfügig verändert werden sollte. Die Unifor.
mäntel blieben ohne Kragenspiegel. Dagegen wur.
den auf die Kragen der Dienst- und der Drillich
uniformen Kragenspiegel aufgenäht und Schulter.
klappen der NVA getragen,Alle diese Uniformen
wurden mit silberfarbenen Knöpfen versehen. Dic
Koppelschlösser der Lederkoppel wurden schritt
weise gegen solche mit dem geprägten Staatsem
blem der DDR ausgewechselt, Vorhandene khaki.
farbene Ausgangsuniformen und -mäntel ersetzten,
eingezogen und gereinigt,bei Bedarf abgetragene
Dienstuniformen bzw.-mäntel der NVA.

Die Schirmmützen der ffiziere bekamen ein
dunkelgraues Mützenband und eine silberfarbené
Kordel, Bei den Landstreitkraften wurden sie au
Berdem mit einem Mützenemblem gleicher Farbe
bei den Luftstreitkraften mit Schwinge und Ko
karde sowie Propeller mit Ahrenkranz versehen
Diese nderungen wurden auch an den Winter
mützen der ffiziere vorgenommen, Wie bei den
Soldaten und Unteroffizieren wurden auch die Uni
formjacken der Offiziere mit Kragenspiegel und
der NVA und silberfarbenen
Schultersticken
KnÃöpfen(auch an den ¤nteln)versehen.Gene.
rale trugen ihre Schirmmützen ebenfalls mit den
Effekten der NVA.

Geplant war, bis zum Sommer 1958 die khakifar-
benen Uniformen restlos aufzutragen. Ab Herbst
des genannten ,Jahres wurden in der NVA die stein-
grauen Uniformen bestimmend. Die Werktätigen
der Bekleidungswerke und die Angehörigen der
rückwärtigen Dienste leisteten'in jenen Wochen
und Monaten eine angespannte Arbeit, Beispiels-
weise hatte die NVA mit dem VEB Burger Beklei-
dungswerke- heute Leitbetrieb für Dienstbeklei-
dungen in der DDR- vereinbart, bis zum 30.April
1956 für Soldaten und Unteroffiziere 55000 Uni-
formen, d,h.Uniformjacken und -hosen, auszulie-
fern, Dies erfolgte schrittweise nach einem festge-
legten Größenschlüssel, Fir die Landstreitkrafte
muBten dabei unterschiedliche Paspelierungen der Waffenfarben bei der Produktion beachtet werden.
Ein anderer Betrieb, die Halleschen Kleiderwerke,
lieferte zum selben Termin 9000 Offiziersunifor.
men.Sie bestanden aus Uniformjacke,-hose und
Stiefelhose. Der VEB Leipziger Bekleidungswerke
stattete die NVA bis zum 30.April 1956 mit fast
34000 Uniformm¤nteln für Soldaten und Unterof-
fiziere aus.

Als auBerordentlich kompliziert erwies sich die
Fertigung der Effekten für die Uniformen, insbe-
sondere die Herstellung der Kragenspiegel und Ãr.
melpatten. Sie wurden in noch gröBerer Anzahl,
nämlich auch für die khakifarbenen Uniformen, be-
nÃötigt, Dabei waren vor allem Werktätige mit aus-
geprägtem handwerklichem Geschick gefragt. Zu-
gleich galt es auch, Betriebe mit
geeigneten
Stickmaschinen und entsprechenden Kapazitäten
zu finden.


Der Öffentlichkeit in der DDR präsentierte sich
die junge NVA, d.h. erste aufgestellte Truppenteile
und Verb¤nde, bei zwei bedeutsamen milit¤rischen
Zeremoniellen im Frihjahr 1956.Am 30. April
dem Vorabend des Kampftages der internationalen
Arbeiterklasse,waren die Soldaten, Unteroffiziere
und Offiziere des l.mech. Regiments der NVA in
ihren neuen Paradeuniformen zu einem feierlichen
Appell in Oranienburg angetreten. Funktionäre der
Partei der Arbeiterklasse,des Staates und der as
senorganisationen,Arbeiterveteranen,antifaschisti
sche Widerstandskämpfer und Kämpfer der Inter
brigaden wohnten diesem Zeremoniell auf der
Ehrentribüne bei, Neben der Tribüne standen Ein
heiten der Kampfgruppen, Abordnungen der GSl
und der FD]. Nach der Meldung an den Minister
fr Nationale Verteidigung der DDR,General
oberst W.Stoph, sprach dieser zu den Soldaten. Er
verpflichtete die Armeeangehörigen, das Kampf
banner der bewaffneten Volksmacht als Zeichen
der Würde des Truppenteils stets in Ehren zu hal
ten, Dann übergab er dem Kommandeur des Regi.
ments die Truppenfahne. Aus den H¤nden seines
Kommandeurs nahm stolz ein Unteroffizier die
Fahne entgegen, Vier junge Soldaten traten aus der
Paradeformation hervor,schritten auf die Fahnen
gruppe zu und beriihrten symbolisch für alle Ange
h¶rigen des Regiments die feierlich gesenkte Fahne
Von Hunderten Soldaten erschallte der Schwur,
das sozialistische Vaterland auch unter Einsatz des
Lcbens gegen jeden Feind zu schützen.

Am folgenden Tag nahm das Regiment - aufge
sessen auf Lkw G 5 - an der ersten Truppenparade
der NVA in Berlin anlaBlich des 1. Mai teil. Ein
Blick auf die am 2. Mai in der Presse veröffentlich
ten Fotos von der Maiparade verrät eine Vielzahl
von Details zur Uniformierung und Bewaffnung
der paradierenden Einheiten. So trugen die Musik
korps der Land- und Luftstreitkräfte an beiden
Ober¤rmeln ihrer Uniformen die aus der Geschichte bekannten Schwalbennester - mit Längs-
und Querborten verzierte Achselwülste, die ur-
spr¼nglich wie die Schulterklappen das Herunter-
gleiten des Lederzeugs verhindern sollten. Auf
Schüitzenpanzerwagen vom Typ BTR152
neuen
die Soldaten und Unteroffiziere mit ihren
hatten
neuen Stahlhelmen und Schützenwaffen in Parade-
haltung Platz genommen.


Zur persönlichen Ausr¼stung des Soldaten und zur
Vervollständigung der Uniform gehört in der NVA
wie auch in anderen modernen Armeen der Stahl.
helm. Er wird insbesondere bei Handlungen in der
Gefechtsausbildung, aber auch beim Wachdienst getragen,Seine ZweckmäBigkeit hatte sich schon
wahrend der äuBerst verlustreichen K¤mpfe des er.
sten Weltkrieges sehr rasch als unverzichtbarer
Schutz des Kopfes vor Geschossen, Splittern und
Schlageinwirkung erwiesen.
Auch die bewaffneten Organe der jungen DDR
die Bereitschaften der Volkspolizei, waren schon
teilweise mit einem Stahlhelm ausgerüstet worden.
Er war in seiner Form Sturzhelmen nachempfun.
den und blieb deshalb für die Belange von Streit-
kräften unzureichend, Aus diesem Grunde wurden
bereits durch die Führungsorgane der KVP Ãberle-
gungen angestellt,einen neuen Stahlhelm zu ent
wickeln, Vereinzelt verwendeten Einheiten der KVP
und spÃäter der NVA - beispielsweise in Übungen
auch den Stahlhelm der Sowjetarmee. SchlieBlich
entschloB sich die Führung der NVA aber, einen
auf die Uniform der Volksarmee abgestimmten
Helm herstellen zu lassen.Demzufolge vereinbar
ten die rückwärtigen Dienste der NVA und das
Amt für echnik gemeinsam mit dem VEB Eisen
Hüttenwerk Thale und dem VEB Sattler- und Le
derwarenfabrik Taucha Ende |anuar 1956 Maßnah-
men,um die Entwicklung eines für die NVA
geeigneten Stahlhelms abzuschlieBen und diesen
unverzüglich in die Verb¤nde und Truppenteile
einzufihren.
In diesem Prozeß griffen die Konstrukteure, ver
antwortlicher Ingenieur war Erich Kiesan,auf eine
der letzten Entwicklungen des Stahlhelms der fa.
schistischen deutschen Wehrmacht zurick,die bis
1943 vorangetrieben worden war. Dieser Helmtyp
wurde aber nicht mehr hergestellt und eingesetzt
Bei den bekannten Stahlhelmen der Wehrmacht
Modell 1935 und Modell 1942 traten insbesondere
an den Knickstellen des felms an Stirn und Nak
ken infolge von Durchschlägen häufig Kopfschüsse
auf. Deshalb wählten die Konstrukteure nun eine
überschräge Form, die Geschosse und Splitter im
wesentlichen abgleiten ließ, Auch die Innenausstat
tung des Helms und die Metallegierung wurden
weiter verbessert.
Die Arbeiten an dem Stahlhelm der NVA gingen rasch und erfolgreich voran.Die Erprobung des
neuen Helms hatte gerade erst begonnen, als anl¤ß.
lich der ersten Parade der NVA am 1.Mai 1956
Teile der über den Marx-Engels-Platz paradieren
den Einheiten schon mit diesem Stahlhelm an die
Öffentlichkeit traten.
Die systematischen Erprobungen des Stahlhelms
setzten erst Mitte Mai 1956 ein, Insbesondere die
Beschußproben zogen sich- äuBerste Sorgfalt war
geboten - bis Ende des |ahres hin., In seiner Anord.
nung Nr.15/56 vom 14. Mai 1956 regelte der Mini.
ster für Nationale Verteidigung der DDR, General.
oberst W.Stoph,die Erprobung des Stahlhelms
VM 1/56 (Versuchsmodell 1/56).Zwischen dem 16.
und 19,Mai sollten mittels BeschuB- und Festig
keitsproben die Formgebung und die Materialhalt.
barkeit getestet und im ,Juni die Tragemöglichkei-
ten über längere Zeit festgestellt werden.
Beschuß-und Festigkeitsproben erfolgten durch
direkten Beschuß mit der Pistole TT33(10 m bis
25 m), mit der MPi PPSch 41 (25 m bis 100 m), mit
dem Scharfschützengewehr D(300 m bis 600 m)
und mit dem sMG(600 m)sowie mit Handgrana-
ten am 16.und 17.Mai. Weitere Beschußproben
fanden am 17.und 18.Juli sowie am 27.Dezember
1956 statt, W¤hrend der Erprobung im Juli wurde
der Stahlhelm auch der Wirkung von Artillerie-
munition-des 82-mm-Granatwerfers,der 76-mm-
Kanone und der 122-mm-Haubitze- ausgesetzt.
Alle diese Versuche zeigten eindeutig: Der Stahl-
helm bot seinem Träger mit absoluter Sicherheit
Schutz vor der Schußeinwirkung durch Pistolen ab
10m Entfernung und vor der von aschinenpisto
len ab 50m.Noch 1m von der Detonationsstelle
einer Handgranate entfernt, hielt der Stahlhelm der
Splittereinwirkung stand, Auch beim Detonieren
von Artilleriemunition konnte der Soldat mit dem
Stahlhelm vor Kopfverletzungen geschützt werden.
So bestand beim genannten Granatwerfer ab 10 m,
bei der Kanone ab 20 m und bei der Haubitze ab
25 m von der Detonationsstelle entfernt Sicherheit
Ebenfalls erfolgreich verliefen die Versuche zur
Feststellung der Druck- und Schlagfestigkeit.

Um die Zweckmäßigkeit des Stahlhelms hinsicht.
lich der Tragfahigkeit bei den verschiedensten mili-
tarischen Tatigkeiten zu erproben, fuhrten ein
Schützenzug,ein Aufklärungszug, ein Granatwer-
ferzug (82-mm-Granatwerfer), eine Geschützbedie-
nung (122-mm-Haubitze), ein Zug des Wachregi-
ments, ein Nachrichtenzug, cin Pionierzug und ein
Zug der Truppen der chemischen Abwehr Trage-
versuche durch, Sie fanden bei der Grundausbil-
dung, bei Marschübungen und bei der SchieBaus
bildung sowie während der Fahrten mit dem SPW
mit Kfz und Krad statt. Es galt festzustellen, ob die
Innenausstattung des Stahlhelms auch bei Wen-
dungen, beim Hinlegen oder Aufstehen und beim
Exerzier- oder Laufschritt stets einen einwandfreien
Sitz gewährleistete. Der Helm durfte keine Druck-
schmerzen hervorrufen, Meteorologische Bedingun-
gen wie Sonne, Regen und Wind sollten sich nicht störend auswirken,z.B,Regen und Wind nicht das
Wahrnehmen von Geräuschen beeinträchtigen.
Insgesamt erbrachten die Materialerprobungen
und die Trageversuche sowie vorgenommene Ver.
besserungen am Stahlhelm hinsichtlich der Legie.
rung des Stahlblechs sowie bei der Innenausstat.
tung des Helms bis zum Ende des lahres 1956 sehr
gute Ergebnisse, Sie erlaubten es, die Massenferti-
gung des Stahlhelms M56 aufzunehmen und die
Teilstreitkräfte mit ihnen auszustatten. Bereits vor
den Erprobungen des Stahlhelms rechtfertigten es
Foringestaltung undMaterialzusammensetzung.
Helme dieses Typs in drei Größen bis Mitte April
1956 in sehr kleiner Stückzahl an die Paradetrup
pen auszuliefern, Sie waren farblich steingrau-matt
gehalten, an der linken Seite mit einem schwarz
rot-goldenen Wappen als Abzichbild versehen und
von innen mit einem Stempel <S 1/56» als beson-
dere Serie gekennzeichnet.
Anfang 1957 wurden die Herstellungs- und Ab
nahmevorschriften bestätigt und die Produktion
der Stahlhelme in den drei GröBen 60 cm, 64 cm
und 68cm Bezugsmaß aufgenommen. Bis Ende
September 1957lieferte die Industrie ungefahr
50 000 Stahlhelme an die Truppe aus. Etwa ein .Jahr
später gewährlcistete sie die restlose Versorgung der
Verbände und Truppenteile der NVA mit diesem
Stahlhelm.
Den Hauptanteil an der Entwicklung des Stahl-
helms in weniger als einem Jahr, den der Ingenieui
Erich Kiesan geleistet hatte, anerkannte die Füh
rung der NVA zum 1.Mai 1957 mit der Verdienst
medaille der NVA in Bronze.
Ergänzend zur Entwicklung und Einführung des
Stahlhelms M 56 sei noch angemerkt, daß es diesen
Helm bis heute in zwei Formen gibt. Bei der ersten
Form ist das Helmfutter durch drei auBen sichtbare
Nieten befestigt, wihrend bei der zweiten Form das
Futter innen durch sechs Metallknöpfe aufgehängt
ist. Die Ehrenformationen der NVA verfügen, wie
international vielfach üblich, über einen am 17.Tuni
1957 bestätigten Kunststoffhelm mit cinem Ge
wicht von 500 Gramm.


Eine Geschichte der Uniformierung der NVA w¤re
nicht vollständig, ohne die vielfaltige Sonderbeklei-
dung wenigstens zu erwähnen. Nicht immer ist sie
auch eindeutig als Uniform zu bestimmen, d.h., die
Zugehörigkeit des Trägers von Sonderbekleidung
zu einem Dienstgrad bzw,zu ciner Dienstgrad-
gruppe l¤¡ßt sich nicht erkennen. Moderne Armeen
benötigen seit Mitte des 20..Jahrhunderts derartige
spezielle Bekleidungen für extreme klimatische Be-
dingungen, für verschiedene Tatigkeiten im milit¤-
rischen Dienst und für Spezialisten.
In allen Teilstreitkräften der NVA erhielten die
Kraft- und auch die Kradfahrer zus¤tzliche Beklei.
dungsst¼cke: die Kraftfahrer zum Parkdienst und
bei Reparaturen die Arbeitskombination, für Fahr.
ten im Winter ebenso wie die Kradfahrer einen
Watteanzug und Filzschaftstiefel. Zur Ausstattung
der Kradfahrer gehörten eine Kradhose, ein Krad.
regenmantel, Kradhandschuhe,Schutzbrille und
eine graue Kombination.

Die Panzerbesatzungen der Landstreitkräfte der
NVA verfügten zum Parkdienst iiber eine blaue Ar-
beitskombination, hinzu kam zur Ausbildung und
zur Fahrausbildung mit dem Panzer T-34 jene
Kopfhaube, die auch die sowjetischen Waffenbrii.
der besaßen. Des weiteren konnten die Panzerfah-
rer zu besonderen Anlässen cine graue Kombina-
tion und zur Fahrausbildung im Winter eine
Wattekombination anzichen.

Das Werkstattpersonal war generell mit blauen
Arbeitskombinationen ausgestattet. Posten erhiel-
ten im Winter bei Temperaturen unter minus 6 °C
Pelz-bzw.Ãbermäntel und Filzschaftstiefel.

In den Luftstreitkräften gab es differenzierte
Sonder- bzw. Spezialbekleidung für das fliegende
Personal,für Fallschirmspringer und das flieger-
technische Personal sowie sonstiges Flugpersonal.
Die Flugzeugführer trugen im Sommer die Flieger-
kopfhaube mit FT-Teil (darunter eine Leinenkopf.
haube), Fliegerlederjacke, Oberhemd und Binder,
Stiefelhose, Chromlederstiefel, ungefütterte Leder-
handschuhe, einen gelben Fliegerschal, Flieger-
brille, Sauerstoffmaske, ein Kappmesser und eine
Spezialkartentasche, Im Winter traten die Flieger-
kopfhaube(Winter), eine zweiteilige, wattierte
blaue Fliegerkombination, Fliegerpelzstiefel, lammfellgefütterte Stulpenhandschuhe, Fliegerpullover
und Fliegerwollschal an die Stelle entsprechender
Sommerbekleidungsstücke. Die Flugschüler zogen
im Sommer statt der Fliegerlederjacke eine ungefüt-
terte einteilige Fliegerkombination an.
Fallschirmspringer erschienen im Sommer mit
einer ledernen Fliegerkopfhaube ohne FT-Teil,
einer einteiligen Fliegerkombination, Sprungschu-
hen, Fliegerbrille und Kappmesser zum Sprung-
dienst, Im Winter waren sie mit Fliegerkopfhaube
ohne FT-Teil, einer zweiteiligen, wattierten Flieger
kombination,lammfellgefitterten Stulpenhand.
schuhen und Fliegerpullover mit Rollkragen verse-
hen.
Eine besondere Ausstattung erhielt das flieger-
technische Personal für beide ,Jahreszeiten. Flieger,
Unteroffiziere und Offiziere, die als Techniker
ihren Dienst versahen, trugen im Sommer eine un-
gefütterte einteilige schwarze Kombination, Die Offiziere setzten dazu eine Schirmmütze auf. Im Win-
ter arbeiteten sie in einer gefitterten schwarzen
Kombination,
lammfellgefütterten Handschuhen
aus Segeltuch und Filzschaftstiefeln. Start- und
Sonderposten trugen im Winter Pelz- bzw, Ãber-
mantel.
Bei den Seestreitkräften erhielten Boots- und
Schiffsbesatzungen zunächst das Recht, während
des Borddienstes folgende Bekleidungsstücke zu
tragen:Ledermantel oder lange Lederjacke, kurze
Lederjacke und Lederhose für das Maschinen- und
Sperrpersonal beim Dicnst an der Maschine oder
am Sperrgerät, Ölzeug für das Oberdeckpersonal
sowie für diesen Personenkreis auch Pelzmütze,
Watteanzug, Pelz- oder Wachmantel und Filzstiefel je nach Witterung. Unter Berücksichtigung streng
ster Sparsamkeit ordnete der Chef der Seestreit-
kräfte am 7.Juni 1957 an, Lederbekleidung als
Schutzbekleidung nur an das gesamte Personal von
Torpedoschnellbooten und an das Maschinenperso-
nal der Hochdruck-Heißdampfmaschinen auszuge-
ben, Matrosen und Maate des Maschinenpersonals
anderer Schiffe und Boote erhielten zusätzlich zur
bisherigen Ausstattung einen weiteren blauen Bord-
anzug,so daß sie während der Bordzeit über drei
derartige Uniformen verfügten, Nur die Komman-
danten und Flottillenchefs behielten Ledermantel
und -jacke.


In der NVA wurde von Beginn an der Sport groß
geschrieben,Die meisten Armeeangehörigen trie-
ben auch schon damals über den Dienstsport hinaus aktiv Sport. Dazu bot die am 1.Oktober 1956
gegründete Armeesportvereinigung
«Vorwärts»
vielfaltige Möglichkeiten.
Im Verlaufe der Jahre 1956 und 1957 gab der
B/A-Dienst der riückwärtigen Dienste der NVA an
alle Soldaten, Unteroffiziere und Offiziere einheitli.
che Sportbekleidung gegen Bezahlung aus. Die Tat
sache,daß das Sportzeug letztlich in den pers¶nli.
chen Besitz seines Trägers überging, rechtfertigte
diese kurzzeitige Regelung. Die Sportbekleidung
setzte sich bis Mitte der 6Oer ahre aus einem dun-
kelblauen Trainingsanzug,schwarzer Sporthose,
weiBem Sporthemd,dunkelblauer Schwimmhose
und schwarzen Ledersportschuhen zusammen.Der
entsprechende Befehl des Ministers für Nationale
Verteidigung der DDR vom Sommer 1956 schrieb
für den Dienstsport bei kalter Witterung die Dril-
lichuniform vor,solangeTrainingsanzüge noch
nicht ausreichend vorhanden waren. Bei strengem
Frost konnten die Armeeangehörigen die Dienst-
uniform und den Uniformmantel überziehen.
Der Minister forderte weiterhin, bis zum 1. De
zember 1956 eine Regelung zu treffen, nach der die
Unteroffiziere und Offiziere im Sportzeug ihrem
Dienstgrad nach erkennbar sind,Davon ausgehend
wurden für die Trainingsanzüge wie auch für Kom
binationen und Schutzbekleidung der Angehörigen
der Land- und Luftstreitkräfte der NVA Dienst
gradabzeichen in Tressenform eingef¼hrt. Sie wa
ren am linken Ober¤rmel befestigt, d. h., der obere
Rand des Abzeichens befand sich 14 cm unter der
Schulternaht, Die Dienstgrade von Unteroffizier bis
Oberst f¼hrten silbergraue, die Generale goldfar
bene Tressen, Ein Unteroffizier trug eine 9 mm
breite und l0 cm lange silbergraue Perlongespinst
tresse,ein Feldwebel zwei und ein Oberfeldwebel
drei derartige Tressen in einem Abstand von jeweils
5 mm zwischen den Tressen.Der Unterleutnant war
an einer 15mm breiten und cm langen Tresse
und dariiber einer 5 mm breiten und ebenso langen
Tresse erkennbar. Die Dienstgrade bis Hauptmann
fügten stets eine weitere 9-mm-Tresse hinzu. Der
Major trug zwei der beschriebenen Tressen von 15 mm und cine von 9 mm Breite, Für die Dienst-
grade Oberstleutnant und Oberst erhöhte sich die
Zahl der 9-mm-Tressen um je eine. Ein Generalma-
jor besaB die gleiche Tressenanordnung wie ein
Major - nur in goldfarbener Ausführung. Ebenfalls
weitere goldfarbene 9-mm-Tressen kamen für die
Dienstgrade Generalleutnant, Generaloberst und
Armeegeneral hinzu, so daß letzterer zwei goldfar-
bene 15-mm- und dariber vier 9-mm-Tressen auf
wies.
Nicht unerwähnt soll bleiben, daß die Unteroffi-
ziere der Land- und Luftstreitkräfte der NVA auf
den weißen Sporthemden am Halsausschnitt eine
9 mm breite schwarze Perlongespinsttresse und die
Offiziere zwei derartige Tressen, im Abstand von
3 mm eingefaBt,trugen. Maate und Meister der
Seestreitkräfte befestigten an der Sportbekleidung
einen gewebten goldfarbenen klaren Anker auf ova-
ler Tuchunterlage aus blauem Stoff. Für Offiziere
galt der gleiche Anker, jedoch mit einer goldfarbe-
nen Umrandung versehen.


Die Uniformierung einer Armee vollzieht sich nicht
einfach dadurch, daß Uniformen hergestellt und an
die Soldaten ausgegeben werden. Es galt, die ver-
schiedenen Uniformen militärischen Tätigkeiten
zuzuordnen, die Trageweise zu bestimmen und
Normen für die Tragezeiten der einzelnen Uniform-
stücke festzulegen, Dazu bedurfte es umfangreicher
und sorgsam abgewogener ÃJberlegungen von Ange-
hörigen vieler Dienstbereiche. Der Minister für Na-
tionale Verteidigung der DDR wies in einer Anord-
nung über Vorbereitungsmaßnahmen für die
Einführung der Uniform der Land-, Luft- und See-
streitkräfte der Nationalen Volksarmee》 bereits am
4.Februar 1956 an, daß durch den Chef des Haupt-
stabes der NVA bis Ende des Monats der Entwurf
einer Dienstvorschrift über die Bekleidungsord-
nung und durch den Chef Rückwärtige Dienste die Normen und Tragezeiten für alle in die NVA einge-
führten Bekleidungs- und Ausrüstungsgegenst¤nde
auszuarbeiten sind.Des weiteren sollte bis Ende
M¤rz der Entwurf einer Instruktion fǔr die Rege
lung der B/A-Wirtschaft erarbeitet werden.
Viele Vorschläge sachkundiger Offiziere aus na-
hezu allen Dienstbereichen gingen im Ministerium
ein, nachdem Ende August 1956 der Entwurf der
Bekleidungsvorschrift zur Stellungnahme in die
Truppe versandt worden war.Am 15.Juni 1957
setzte der Minister für Nationale Verteidigung der
DDR die «Vorläufige Bekleidungsordnung der Na-
tionalen Volksarmee» in Kraft, Er hatte schon vier
Monate zuvor die Chefs der Luftstreitkräfte, der
Luftverteidigung, der Seestreitkräfte sowie die der beiden Militärbezirke der Landstreitkräfte angewie.
sen, alle Angehörigen ihrer Dienstbereiche mit die-
ser Ordnung vertraut zu machen.Die ständige
Pflege und Instandhaltung der Bekleidung und
Ausr¼stung sowie die Einhaltung der Bekleidungs.
ordnung galt es durchzusetzen, um das diszipli-
nierte Auftreten und das ¤uBere Bild der NVA-An-
gehörigen in der Öffentlichkeit zu verbessern.
Im ersten Punkt der <Allgemeinen Grundsätze》
dieser Bekleidungsordnung unterstrich die Führung
der NVA den Charakter der Uniform der NVA. Sie
ist, hieß es, «das Ehrenkleid aller Angehörigen der
Nationalen Volksarmee der Deutschen Demokrati-
schen Republik. Jeder Angehörige der bewaffneten
Kräfte der Deutschen Demokratischen Republik ist
verpflichtet, die Ehre und Würde der Uniform zu
wahren».
Weitere Punkte jener Grunds¤tze regelten das
Tragen der Uniformen der NVA für Soldaten, Matrosen,Unteroffiziere,Maate,Offiziere,Generale
und Admirale, die im aktiven Dienst der NVA ste.
hen bzw, für diejenigen, die aus dem aktiven Dienst
entsprechend denDienstlaufbahnbestimmungen
ausgeschieden sind.In mehreren Kapiteln
be
stimmte die Vorschrift die Uniformarten, die Art
und die Trageweise der Uniformen sowie die Trage.
weise der Orden, Medaillen und Abzeichen. Festge-
legt wurde,welche Uniformarten für welche Dien.
ste galten (s. Tabelle S.14); innerhalb der Einheiten
muBte die Uniform bei gleicher Dienstverrichtung
einheitlich sein. Die Sommerbekleidung wurde in
den Land- und Luftstreitkräften in der Zeit vom
1.April bis zum 30.September, in den Seestreitkräf.
ten vom 1.Mai bis zum 30.September getragen. Die
restlichen Monate blieben der Winterbekleidung
vorbehalten,es sei denn,die Standortältesten ver
kürzten oder verl¤ngerten aufgrund der Wetterver
haltnisse die Tragezeit der Sommer- bzw. Winter
uniformen.
Auf eine Besonderheit soll in diesem Zusammen-
hang verwiesen werden.Generale erhielten schon
seit geraumer Zeit eine zusätzliche Uniform - eine
Modifikation der Ausgangsuniform.Aufgrund
einer Anordnung des Ministers für Nationale Ver.
teidigung der DDR vom 5.Juli 1956 war bei beson
deren feierlichen Anl¤ssen in der heißen [ahreszeit
das Tragen einer weißen Uniformjacke mit Schul.
terstücken und Kragenspiegeln sowie einer weißen
Mitze mit breitem rotem Rand genehmigt. Dazu
wurde die steingraue Tuchhose getragen.
Ein Hauptbestandteil der «Vorl¤ufigen Beklei.
dungsordnung der Nationalen Volksarmee》 von
1957 sind Regelungen zur Trageweise der einzelnen
Uniformstücke.Eines ist allerdings allen Armeen
der Welt in Geschichte und Gegenwart eigen: Man
che Soldaten,Unteroffiziere und sogar Offiziere su
chen nach MÃglichkeiten, um ihre xpersönliche
Note» in die Uniformierung einzubringen. Dies trat
auch in der NVA in der Art und Weise zutage, die
Kopfbedeckung aufzusetzen,Gelegentlich wurde
auch der Mützenring der Schirm- oder der Matro-
senmütze entfernt,um diesen ein-nach Ansicht der Träger-gefalligeres Aussehen zu verleihen.
Auch ein Knicken der Schulterklappen nach der
Halfte der Dienstzeit ist bis heute cin häufiger
Brauch der Soldaten und Unteroffiziere geblieben.


Im folgenden werden Festlegungen zur.Trageweise
der Uniformstücke in den Teilstreitkräften für die
Jahre 1957 bis 1960 wiedergegeben.Grundlegendc
Festlegungen haben sich bis in die Gegenwart nicht
geändert. Die einzelnen Mützenarten waren so auf.
zusetzen, daB sich die Kokarde immer in der Ver-
l¤ngerung der Mittellinie des Gesichts befand. Der
untere Schildrand der Schirm- und der Winter
mütze sollte mit den Augenbrauen abschlieBen, die
Feldmütze aber rechts einen Fingerbreit iber der
Augenbraue sitzen. Der waagerecht zu tragende
Stahlhelm muBte mit seinem vorderen Rand in
Höhe der Augenbrauen liegen.
Der Sitz der Uniformjacke ist bereits beschrieben
worden. Alle Angehörigen der Landstreitkräfte
kn'pften aus hygienischen Grinden eine weiBe
Kragenbinde so innen in den ]ackenkragen ein, daß
ein 2 mm breiter Rand gleichm¤ßig iberstand.
Die Soldaten und Unteroffiziere der Land- und
Luftstreitkräfte steckten bei Halbschaftstiefeln die
Uniformhbse zur Drillich-,zur Dienst- und zur Pa
radeuniform in die Stiefel, Dazu schlugen sie die
Hosenbeine von hinten nach vorn außen ein. Ahn.
lich verfuhren die atrosen und die Maate mit
ihren Bordhosen zu den Halbschaftstiefeln beim
Exerzieren und in der Schießausbildung.Dagegen
lieBen sie zum Wachdienst, bei der Stellung von
Ehrenkompanien und auf Befehl ihre Klapphosen
über die Stiefel auf den sogenannten halben Schlag
fallen, d.h., sie legten die Hosenbeine zweimal nach
außen zu einem 4cm breiten [mschlag um.
Komplizierte Bestimmungen regelten die Ver
wendung des Kieler Kragens und des seidenen
Halstuches durch die Matrosen und Maate. Die Dienst-,die Parade- und die Ausgangsuniformen
mit Kieler Hemd wurden durch den Kieler Kragen
und das seidene Halstuch ergänzt. Dagegen wurde
beim Exerzieren, bei der SchieBausbildung und
zum Unterricht zwar nicht auf den Kieler Kragen
zum blauen oder weißen Bordanzug,iedoch auf das
Halstuch verzichtet, Backschafter(das Küchenper-
sonal)erschienen im weißen Bordanzug mit Kieler
Kragen,atrosen zum Arbeitsdienst an Land oder
an Bord aber immer ohne. 【hre Freizeit verbrachten
die Matrosen und Maate im weißen Bordanzug mit
Kieler Kragen.Die Kommandeure konnten zu Kul.
turveranstaltungen sowie an Sonnabenden, Sonnta
gen und gesetzlichen Feiertagen zusätzlich das sei
dene Halstuch befehlen.
Feste Regeln gab es fir das Umschnallen von
Koppel und Feldbinde bzw.Schärpe. Schloß oder
Schnalle saBen stets in der itte der Knopfréihe
oder zwischen beiden Knopfreihen. Soldaten und
Unteroffiziere der Land- und Luftstreitkräfte sowie
Offiziere der Landstreitkräfte trugen das Koppel
zwischen dem vierten und fünften Knopf(von
oben)der Uniformjacke,Offiziere der Luftstreit.
kräfte so, daß der unterste Knopf verdeckt blieb,
Wurde das Koppel über dem Uniformmantel getra
gen, so befand es sich ebenfalls zwischen dem vier.
ten und fünften Knopf von oben. Diese Trageweise
war für alle Dienstgrade der drei Teilstreitkräfte
bindend,Offiziere und Generale der Land- und
Luftstreitkräfte schnallten zur Paradeuniform die
Feldbinde wie das Koppel um; Offiziere und Admi
rale der Seestreitkräfte legten ihre Sch¤rpe zwi
schen dem dritten und vierten Knopf der Parade.
iacke an.
Die Matrosen und Maate zogen das Koppel zur
Dienst-,zur Parade- und zur Ausgangsuniform
durch die Schlaufen der Klapphose bzw. beim
Exerzieren,bei der SchieBausbildung und im Un
terricht durch die Schlaufen der Bordhose. Zum
täglichen Dienst und in der Freizeit konnte das
Koppel weggelassen werden. Im Winterhalbjahr saß
das Koppel zwischen dem vierten und fiinften
Knopf des Uberziehers. Meister trugen es bei bestimmten Anlässen wie Exerzieren, SchieBausbil-
dung, Wachdienst und Paraden; Offiziere bei glei-
chen Gelegenheiten, jedoch nicht zu Paraden.
In den Uniformmantel oder in den berzieher
konnte der Schal glatt, d, h, nicht geknotet, einge.
legt werden, Einige weitere Einzelbestimmungen
seien noch erwähnt: Hauptfeldwebel und Unteroffi-
ziere der Land- und Luftstreitkräfte, die den Dienst
in Offiziersplanstellen versahen (z. B. als Zugführer)setzten zum Dienst, wenn nicht anders befoh-
len, die Schirmmütze auf, Offizieren war es gestat-
tet, zum Innendienst die Uniformhose zur Dienst-
uniform bzw. auch Stiefelhose und Dienstjacke
ohne Koppel zu tragen.


Ein besonderer Abschnitt der «Vorl¤ufigen Beklei-
dungsordnung der Nationalen Volksarmee》 von
1957 umfaßt Bestimmungen zur Trageweise der
Auszeichnungen an den Uniformen. Die Tabelle
gibt Auskunft über jene Orden, Medaillen und Abzeichen sowie Ordens- und Medaillenspangen, die
nach der Verleihung in jener Zeit an den Parade-
und Ausgangsuniformen aller Armeeangehöriger
sowie an den Dienstuniformen der Offiziere, Gene-
rale und Admirale getragen wurden.
Diese Regelung galt auch für ausländische Or-
den und Medaillen, die an Bürger der DDR für
Verdienste im Kampf gegen den Faschismus, fiür
den Frieden und den Aufbau des Sozialismus ver-
liehen wurden, Als Medaillen und Abzeichen ge-
sellschaftlicher Organisationen konnten das Ehren-
zeichen für Parteiveteranen der SED,die Ernst-
Thälmann-Medaille, die Ernst-Moritz-Arndt-Me.
daille,die Friedensmedaille des Deutschen Friedensrates, die Friedensmedaille der FD,das
Abzeichen «Für gutes Wissen» der FD] in Gold,
Silber und Bronze, das Sportabzeichen in Gold, Sil-
ber und Bronze,das AbzeichenPartisanen des
Weltbundes der Demokratischen .Jugend und das
FD]-Abzcichen getragen werden.
Komplizierte Bestimmungen, die sich an interna-
tional übliche Gepflogenheiten anlehnten, regelten
die Art und Weise sowie den Platz der jeweiligen
Auszeichnungen an den Uniformen. Grundsätzlich
befanden sich Orden und Medaillen mit Band links
und die ohne Band bzw, Abzeichen rechts auf der Uniformjacke, [hrer Bedeutung nach (siehe Tabelle
S,77)waren sie alle von rechts nach links auf der
Uniformjacke anzubringen. Ausländische Auszeich.
nungen ordneten sich nach Bedeutung und Zweck
bestimmung in diese Reihenfolge ein, folgten aber
immer den entsprechenden Auszeichnungen der
DDR, Für die Trageweise der Orden und Medail.
len am Band an der offenen Jacke der Offiziere galt
die Regelung, daß 6 bis 8 Auszeichnungen hinter-
einander in einer Reihe angelegt .werden durften,
Das Ordensband sollte jedoch keine gröBeren Aus.
maBe als 14 cm L¤nge aufweisen.