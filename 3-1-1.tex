

Die neue Uniformvorschrift, die DV 010/0/005, die
am 1.Juni 1972 in Kraft gesetzt wurde und damit
die Vorschrift von 1965 ablöste, offenbarte die Kontinuität der auf militärische ZweckmäBigkeit
und Tragefreundlichkeit ausgerichteten Uniform-
entwicklung.
Berufsunteroffizieren, Offizieren und Generalen
war es nunmehr m¶glich, zum Dienst in Stäben ab
Truppenteil aufwärts eine neue Uniformart, die
Stabsdienstuniform, zu tragen, Sie unterschied sich
von der Dienstuniform dieses Personenkreises vor
allem dadurch, daß anstelle der Stiefelhose die
lange Hose und statt der Stiefel schwarze Halb-
schuhe getragen werden konnten. Je nach .Jahreszeit
war es möglich, dazu die Hemdbluse oder die Uni.
formjacke zu kombinieren.
Die allgemeinen Bestimmungen der neuen Vor-
schrift legten fest, daß für Angeh¶rige des Ministe-
riums für Nationale Verteidigung, der dem Ministe-
rium direkt unterstellten zentralen Dienststellen,
der Wehrkommandos und der staatlichen Institutionen prinzipiell die Uniform der Landstreitkräfte
verbindlich war. Wurden Armeeangehörige aus den
Teilstreitkräften und Waffengattungen in die ge-
nannten Dienststellen versetzt oder kommandiert.
hatten sie das Recht, ihre Uniformen auch weiter-
hin zu tragen.
Neue Festlegungen regelten die Tragezeiten der
Kopfbedeckungen. Die Wintermütze durfte nur in
der Zeit vom 1.Dezember bis zum 28./29. Februar
aufgesetzt werden. Bereits am 1. M¤rz, dem Tag der
Nationalen Volksarmee, wurde nun die Schirm-
bzw,die Feldmütze getragen. Die Klappen der
Wintermütze durften bei Minusgraden und nicht
erst bei minus 1°G heruntergeklappt werden.
Dazu war das Gummiband unter dem Kinn zuzu
knöpfen.
Für Soldaten und Unteroffiziere der Landstreit.
kräfte und der SK/LV im Grundwehrdienst, auf
Zeit und im Reservistendienst sowie für Offiziers.
schüler der Offiziershochschulen im 1. und 2.Lehr-
jahr fiel die Arbeitsuniform weg. Der neueinge
führte Arbeitsanzug war zu allen Arbeitsdiensten
und zum Waffen- und Revierreinigen anzuziehen.
Die schwarzen Kombinationen waren mit Schulter.
klappen versehen und wurden mit Gurtkoppel ge.
tragen.Dazu wurde die Feldmütze und im Winter
zus¤tzlich der Kopfschützer, auf Befehl auch die
Wintermütze aufgesetzt.