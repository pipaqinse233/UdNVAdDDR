\subsection{国家人民军的制服}

1956 年,国家人民军陆军和空军选择石灰色制服,海军选择深蓝或白色制服,这主要是一个政治决定。
% Die Wahl der steingrauen Uniformen für die Land- und Luftstreitkrafte der NVA und der dunkelblauen bzw. weißen Uniformen für die Seestreitkräfte im Jahre 1956 war vor allem eine politische Entscheidung.

毫无疑问,对国家人民军制服的评价必须基于军队创建时对服装的要求和需要。为了让士兵在公共场合亮相,需要一种既能体现东德国家武装部队成员特征,同时又能将他们与其他穿制服的人员(包括铁路工人或邮政工人等平民以及其他武装机构人员)充分区分开来的制服。
% Eine Bewertung der Uniformen der NVA hat zweifelsohne von den Voraussetzungen und Bedürfnissen auszugehen, die bei der Schaffung der Armee bezüglich deren Bekleidung bestanden. Es wurde für das Auftreten der Soldaten in der Öffentlichkeit eine Uniform benötigt, die ihre Träger als Angehörige nationaler Streitkräfte der DDR charakterisierte und sie damit gleichzeitig von anderen Uniformierten hinreichend unterschied —— sowohl von den zivilen, beispielsweise den Eisenbahnern oder den Postangestellten, als auch von denen anderer bewaffneter Organe.

同时,制服的设计必须考虑军事要求及功能,必须根据中欧的自然环境为训练和可能的野战行动提供良好伪装。因此,从19世纪末开始,国际上用不显眼的灰色制服取代了「彩衣」,防止士兵在战场上暴露。在德国军队中,原野灰色(即灰绿色)制服一直沿用到1945年。在德国和几乎所有其他国家的海军部队中,蓝色和夏季白色制服一直是典型的制服。
% Gleichermaßen sollte die Uniform militärischen Erfordernissen Rechnung tragen und zweckmäßig gestaltet sein, Sie mußte entsprechend den natürlichen Bedingungen in Mitteleuropa für die Ausbildung und mögliche Handlungen im Gelände eine gute Tarnung gewähren. Vor allem deshalb wurde auf einen Grauton zurückgegriffen, der, seit dem international zu verzeichnenden Ersetzen des «bunten Rockes» ab Ende des 19. Jahrhunderts durch eine unauffällige Uniform, das Erkennen des Soldaten auf dem Gefechtsfeld erschwerte. In den deutschen Armeen war bis 1945 eine feldgraue, d. h. graugrüne Tönung der Uniform üblich. In den Seestreitkräften Deutschlands wie nahezu aller anderen Staaten galten blaue und im Sommer weiße Uniformen von jeher als typisch.

\begin{figure}
\includegraphics[width = \columnwidth]{./media/page (24).jpg}
\end{figure}

国家军事遗产中的其他元素也被添加到了石灰色中,石灰色作为陆军和空军的灰色变体得到了保留。这些元素包括上衣的特色剪裁(通常有四个贴袋)、大檐帽、船形帽和冬季帽的形状、兵种色的牙线、坚固的半靴以及保留两种肩章作为军衔徽章。不过,制服设计略有修改。为了促进《华沙条约》缔约国联合武装部队内部的合作,增加了军衔名称和军衔徽章,肩章上的图案使士兵和水兵的特殊军衔更容易辨认。军官军衔中增加了少尉和大将军衔。与大多数社会主义兄弟军队一样,军官肩章上也有数量相同的军衔星,以便统一识别军衔。
% Zum Steingrau als erhalten gebliebene Grauvariante der Land- und Luftstreitkräfte kamen weitere Elemente des nationalen militärischen Erbes hinzu. Dazu gehörten der charakteristische Schnitt der Uniformjacke mit den meist vier aufgesetzten Taschen, die Form der Schirm-, der Feld- und der Wintermütze, die Paspelierung in Waffenfarben, feste Halbschaftstiefel und die Beibehaltung von Schulterklappen und -stücken als Dienstgradabzeichen. Es gab jedoch einige Modifizierungen in der Uniformgestaltung. Um das Zusammenwirken in nerhalb der Vereinten Streitkräfte der Teilnehmerstaaten des Warschauer Vertrages zu erleichtern, wurden Dienstgradbezeichnungen und Dienstgradabzeichen ergänzt, Auf den Schulterklappen angebrachte Tressenstreifen lieBen die speziellen Soldaten- und Matrosendienstgrade besser erkennen. Bei den Offiziersdienstgraden kam der des Unterleutnants und des Armeegenerals hinzu. Auf den Schulterstücken der Offiziere ordnete man die gleiche Anzahl von Dienstgradsternen zur einheitlichen Kennzeichnung des Dienstgrades an wie in den meisten sozialistischen Bruderarmeen.

国家人民军的军服清晰地展现了清算军事历史遗产的意识。在东德,已经与普鲁士-德意志军事史上的军国主义遗产明显决裂。国家人民军的阶级性质和使命与以往所有德国军队有着本质区别,因为在首个德意志工农国家从军的目的,从一开始便完全是为了与兄弟军队一同保卫社会主义和国家公民的和平生活。另一方面,与反动军事史遗产的明确决裂并不意味着对整个德国军事史持虚无主义态度。虽然国家人民军在其传统中提到了群众斗争,特别是工人革命运动,以及1813至1814年进步力量在反抗拿破仑枷锁的斗争中所带来的部分遗产,但其制服却是以早期的德国军队为基础的。
% Mit der Uniformierung der NVA wurde der Sinn für das militärgeschichtliche Erbe der Vergangenheit deutlich bekundet. In der DDR war ein eindeutiger Bruch mit dem militaristischen Erbe preußisch-deutscher Militärgeschichte vollzogen worden. Klassencharakter und Auftrag der Nationalen Volksarmee unterscheiden sich prinzipiell von dem aller früheren deutschen Armeen, besteht doch der Sinn des Soldatseins im ersten deutschen Arbeiter-und-Bauern-Staat von Anfang an ausschließlich darin, an der Seite der Bruderarmeen den Sozialismus und das friedliche Leben der Bürger des Landes zu schützen. Eroberungsabsichten und Bedrohung anderer Völker und Staaten sind den neuen Streitkräften wesensfremd, Eindeutiger Bruch mit dem reaktionären militärgeschichtlichen Erbe bedeutet andererseits kein nihilistisches Verhalten zur deutschen Militärgeschichte in ihrer Gesamtheit. Während sich die Nationale Volksarmee hinsichtlich ihrer Traditionen auf den von den Kämpfen der Volksmassen, besonders der revolutionären Arbeiterbewegung, sowie auch auf den von den progressiven Kräften im Kampf gegen das napoleonische Joch 1813/14 hervorgebrachten Teil des Erbes beruft, lehnt sie sich in ihrer Uniformierung hingegen an frühere deutsche Armeen an.

这也揭示了东德国家人民军与西德联邦国防军的重大区别。联邦国防军的建立者选择了以美军为蓝本的制服样式,以此来吸引北约的合作伙伴,同时也希望转移联邦国防军与德国过去的帝国主义军队在政治上的类同。
% Hierin zeigte sich auch ein wesentlicher Unterschied zur Bundeswehr der BRD. Deren Schöpfer wählten einen an die USA-Armee angelehnten Uniformstil, hofierten so die NATO-Partner und wollten zugleich von der politischen Wesensverwandtschaft der Bundeswehr mit den imperialistischen Armeen der deutschen Vergangenheit ablenken.

同样与国家人民军制服的确定有关的是,东德的社会经济动荡与和平政策使人们有理由相信,东德、社会主义兄弟国家和其他国家的公民会承认德国土地上第一支社会主义军队的真正阶级性质。1956年1月18日上午,东德部长会议副主席在为创建国家人民军的法律草案中有关军服的规定辩护时说:「东德在国家人民军中,德国军服将具有真正的爱国意义,它表达了捍卫我们的民主成就的坚决准备。」历史最终证实了当时决定的正确性。今天,国家人民军的石灰色制服是公认的社会主义军队的象征,其宪法任务是捍卫和平并保护劳动人民的成就。
% 东德部长会议副主席W. Stoph
% Auch im Zusammenhang mit der Festlegung der Uniformen für die NVA ließen die sozialökonomischen Umwälzungen und die Friedenspolitik der DDR zurecht darauf vertrauen, daß die Birger der DDR, der sozialistischen Bruderländer und andere Staaten das wahre Klassenwesen der ersten sozialistischen Armee aufdeutschem Boden erkennen würden. «In der Nationalen Volksarmee wird die deutsche Uniform als Ausdruck der entschlossenen Verteidigungsbereitschaft unserer demokratischer Errungenschaften einen wirklichen patriotischen Sinn erhalten», hatte W. Stoph, Stellvertreter des Vorsitzenden des Ministerrates der DDR, am Vormittag des 18. Januar 1956 zur Begründung der Gesetzesvorlage für die Schaffung der NVA in bezug auf deren Uniformen ausgefihrt. Die Geschichte bestätigte schließlich die Richtigkeit der damaligen Entscheidung. Heute sind die steingrauen Uniformen der NVA anerkanntes Symbol einer sozialistischen Armee, deren Verfassungsauftrag auf die Sicherung des Friedens und des Schutzes der Errungenschaften der Werktätigen gerichtet ist.