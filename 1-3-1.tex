

Die militärtechnische Entwicklung forderte bereits
in den 50er Jahren eine immer höhere Qualifika.
tion und Spezialisierung der Armeeangehörigen. Im
äuBeren Bild der NVA zeigte sich das in der Ver.
wendung von Dienstlaufbahnabzeichen ab Mitte
Dezember 1957.In einem Befehl vom 22.Juni 1957
ordnete der Minister für Nationale Verteidigung
der DDR an, derartige Abzeichen und solche für
Sonderausbildung der Matrosen und Maate einzu-
führen. In allen Teilstreitkräften der NVA gab es
schon seit Schaffung der Armee 1956 Dienstlauf-
bahnabzeichen für Offiziere des Sanit¤ts- und An-
gehörige des Musikdienstes. Sie waren auf den
Schulterstücken oder -klappen befestigt.
In den Teilstreitkräften, Waffengattungen, Spe.
zialtruppen und Diensten kam es für alle Armeean-
gehörigen verst¤rkt darauf an, die Einsatzbereit-
schaft in schnellstm¶glicher Zeit herzustellen. Dies
erforderte viel Ãbung und ständiges Lernen. Inso-
fern stellten die Dienstlaufbahnabzeichen einen Ansporn dar, hohe Fachkenntnisse und Fertigkeiten
auf den unterschiedlichsten Gebieten zu erlangen.
Vor allem Soldaten, Flieger, Matrosen, Unteroffi-
ziere, Maate und Meister, die über cine abgeschlos-
sene Spezialausbildung verfügten und in dieser
Richtung eingesetzt wurden, waren berechtigt, diese
Dienstlaufbahnabzeichen an der niform zu tra-
gen. In den Landstreitkräften der NVA legten bei-
spielsweise Soldaten und Unteroffiziere, die in mot.
Schiitzeneinheiten ihren Dienst als Kraftfahrer ver-
sahen, Dienstlaufbahnabzeichen an.
In dieser Teilstreitkraft bestanden die Dienst-
laufbahnabzeichen aus einer farbigen Kunstsei-
denstickerei auf einer grauen Tuchunterlage. Diese
Unterlage war oval und maß in der L¤nge 6 cm und
in der Breite 5 cm. W¤hrend die Stickerei des Ab.
zeichens für den Steuermann (Soldaten und Unteroffiziere von Pioniereinheiten nach bestandener
Prüfung zum Führen von Wasserfahrzeugen) und
fiür das waffentechnische Personal (Soldaten und
Unteroffiziere mit abgeschlossener Spezialausbil-
dung und bestandenem Examen,eingesetzt als
Waffen-,Geschütz-und Optikermeister bzw. als
Feuerwerker)in Gelb festgelegt war, entsprach sie
bei den anderen Trägern der Waffenfarbe. So führ.
ten Soldaten und Unteroffiziere der Truppennach-
richteneinheiten nach abgeschlossener Ausbildung
in einem Artillerietruppenteil ihr Laufbahnabzei-
chen mit ziegelroter Stickerei. Die beiden Abzei-
chen des waffentechnischen Personals waren übri.
gens nicht oval,sondern rund mit einem Durch.
messer von 4,5 cm. Die Dienstlaufbahnabzeichen
wurden an der Dienst-,der Parade- und der Aus
gangsuniform in der Mitte des linken Unter¤rmels
der Uniformjacke 12 cm vom unteren Armelrand
entfernt angebracht.
In den Luftstreitkräften der NVA gab es den
Landstreitkräften analoge Dienstlaufbahnabzeichen
für Flieger und Unteroffiziere mit den Bezeichnungen Kraftfahrer und Traktorist,Sanitätsdienst
nachrichtentechnisches Personal,Nachrichtenper
sonal,Funkortungsdienst,Waffenmeister und Opti
kermeister sowie Feuerwerker,Die Gestaltung die
ser Abzeichen entsprach denenderLandstreit
kräfte, nur war die Stickerei weiß, In dieser weiBer
Stickerei kam das Laufbahnabzeichen Fallschirm
dienst für Flieger und Unteroffiziere des Fall.
schirmdienstes nach Erreichen derfestgelegten
Qualifikationsstufe hinzu.Weiter gab es ein Abzei
chen für diese Dienstgrade mit einer abgeschlosse.
nen Ausbildung als Flugzeugmechaniker. Die Offi
ziersschüler, die sich als Flugschüler zur Ausbil
dung auf Offiziersschulen befanden, nǎhten eine
vierblättrige Luftschraube in ovaler Form als
Dienstlaufbahnabzeichen auf.
In den Luftstreitkräften erhielten damals auch
Offiziere Dienstlaufbahnabzeichen. An Flugzeug
führer aller Flugzeugtypen, Flugzeugtechniker und
Leiter des Fallschirmdienstes sollten diese Abzei
chen entsprechend der erreichten Oualifikation in
den Stufen Bronze, Silber und Gold verliehen wer.
den.Doch dazu kam es nicht. Es gab schlieBlich
nur silbergestickte Dienstlaufbahnabzeichen,die
5cm hoch und 4,5cm breit waren.Wahrend die
Abzeichen der Flieger und Unteroffiziere wie bei
den Landstreitkräften am linken Unter¤rmel befe
stigt wurden, trugen die Offiziere ihre Laufbahnab.
zeichen in der Mitte der linken Brusttasche der
Dienst- und der Parade-/Ausgangsjacke.
Im Jahre 1960 wurden für ffiziere der Luft.
streitkräfte im Zusammenhang mit der Einf¼hrung
von Klassifizierungsabzeichen auch neue Dienst
laufbahnabzeichen geschaffen, die einander in der
äuBeren Form glichen. Es handelte sich um Span
gen, d.h.aus Metall geprägte Abzeichen mit einer
Höhe von 2,6 cm und einer Breite von 9,1 cm. Ein
emailliertes Hoheitszeichen und die spezifische
Symbolik waren eingearbeitet. Dienstlaufbahnab
zeichen erhielten Flugzeugführer, die Steuerleute
der Luftstreitkräfte, die zum fliegenden Begleitper
sonal gehörten,und alle Offiziere des ingenieur
und fliegertechnischén Personals sowie die Offiziere des Fallschirmdienstes mit abgeschlossener
Spezialausbildung. Diese Dienstlaufbahnabzeichen
waren in der Mitte 5 mm üiber der rechten Brustta-
sche der Uniformjacke bzw, bei Uniformjacken
ohne Brusttasche in gleicher Höhe zu tragen.
Bedingt durch einen hohen Grad der Spezialisie-
rung bei den Seestreitkräften, zudem auch interna-
tional traditionell stark ausgeprägt, war das System
der Dienstlaufbahnabzeichen in dieser Teilstreit-
kraft von Beginn an sehr vielfaltig. AuBerdem wa-
ren die Dienstlaufbahnabzeichen der Maate, Mei.
ster und Offiziersschüler oft direkt mit den
Dienstgradabzeichen verbunden, Hinzu kam noch
eine groBe Anzahl von Abzeichen für die Sonder.
ausbildung der Matrosen und Maate. Auf der
Grundlage einer Anordnung des Chefs der See
streitkräfte vom 17.0ktober 1957 legten die Chefs der Flottillen,die Kommandeure von Schulen und
die Leiter selbständiger Dienststellen die Verlei
hung der Dienstlaufbahnabzeichen fest. Die Bestä.
igung dieser Abzeichen fir die Offiziere behiel
sich der Chef der Seestreitkräfte selbst vor. Die
Dienstlaufbahnabzeichen bei den Seestreitkräften
bestanden auBer bei den Offizieren für die blaue
Bekleidung aus einer goldfarbenen Kunstseiden
stickerei auf blauer Tuchunterlage und fúr die weife
Uniform aus blauer Kunstseidenstickerei auf wei.
Ber Unterlage; die Stickerei für die Sonderabzei-
chen war stets aus roter Kunstseide.Eine Aus
nahme bildeten die Meister,deren Laufbahnabzei
chen aus goldfarbenem Metall auf den Schulter
clappen angebracht war.
Die runden Dienstlaufbahnabzeichen der atro.
sen hatten 6cmDurchmesser und wurden auf der
Mitte des linken Oberärmels,14cm vom oberen
cingenähten Rand des Armels entfernt, befestigt
Ebenso sah die Regelung für die Maate und Ober.
maate aus.Obermaate kennzcichnete iedoch noch
ein 1c unter dem Dienstlaufbahnabzeichen be
findlicher, nach oben offener stumpfer Winkel. Für
Offiziersschüler gab es ovale Abzeichen von 5,8 cm
L¤nge und 4,7 cm Breite, unter denen je nach Lehr
jahr ein bis vier Winkel angeordnet wurden.
Die Dienstlaufbahnabzeichen der Offiziere wie
sen unterschiedliche Abmessungen-zwischen
2cm und 3cm Durchmesser - auf und bestanden
aus goldfarbener Stickerei auf blauem Grundtuch
oder weißem Wollstoff. Männliche Offiziere trugen
das Laufbahnabzeichen auf beiden Unterärmeln
der Dienst- und der Parade-/Ausgangsjacke 2 cm
über dem Armelstreifen; weibliche Offiziere 8 cm
von der Unterkante des Armels entfernt.
Nur in den Seestreitkräften der NVA gab und
gibt es Abzeichen für Sonderausbildung für die
Matrosen und Maate.Sie waren in ihren Abmes
sungen je nach Höhe und Breite der Stickerei un-
terschiedlich groß. Man trug diese Abzeichen 2 cm
unterhalb der Dienstlaufbahnabzeichen in einem
Abstand von 2cm.Mehr als zwei Abzeichen fǔ
Sonderausbildung durften nicht getragen werden Generell wurden die Dienstlaufbahnabzeichen
und die Abzeichen für Sonderausbildung am Ende
einer Ausbildungsperiode bzw,am Ende des Aus.
bildungsjahres in würdiger Form verliehen und tru-
gen damit in allen Teilstreitkräften der NVA dazu
bei, Kampfkraft und Gefechtsbereitschaft zu erhö-
hen.