

Die Waffenfarben sind an den Schulterklappen
und -stiicken sowie z.T, an den Kragenspiegeln
sichtbar und erleichtern die Zuordnung der Armee-
angehörigen zu den Teilstreitkräften, Waffengat-
tungen, Spezialtruppen und Diensten. In begrenz-
tem mfang können dabei auch die Biesenfarben
helfen, Sie sind an den Uniformjacken und -hosen
aller m¤nnlichen Soldaten- Bekleidungsstücke
weiblicher Armeeangehöriger werden generell ohne
Biesen gearbeitet - sowie an den Schirmmützen bei
den Landstreitkräften Weiß und in den LSK/LV
Hellblau. Seit 1986 gelten die in der nebenstehen-
den Tabelle aufgeführten Waffenfarben.
Soldaten,Matrosen,Unteroffiziersschüler,Un-
teroffiziere, Fahnrichschüler und ffiziersschüler
tragen auf
Uniformmänteln,Sommermänteln,
Ãberziehern, Uniformjacken,Bordjacken, Hemd.
blusen und grauen Oberhemden sowie Uniform-
kleidern und -westen Schulterklappen aus Uniform-
gewebe mit einer Biesenumrandung in der jeweili-
gen Waffenfarbe. Unteroffiziersschüler sind an
Schulterklappen aus Uniformgewebe mit einer Bie-
senumrandung und einem 9 mm breiten Ouerstrei.
fen in der jeweiligen Waffenfarbe zu erkennen.
Zusätzlich zu den Schulterklappen nähen die
Unteroffiziere am Kragen der Uniformjacke, Ange-
h¶rige dieser Dienstgradgruppe der Volksmarine
am Kragen des Ãberziehers, 6 mm vom äußeren
Rand entfernt,eine 9mm breite Litze auf. Die
Litze und die Buchstaben auf den Schulterklappen sind silberfarben, bei der Volksmarine goldfarben.
Die Sterne aller Unteroffiziersdienstgrade sind sil-
berfarben und vierzackig mit einer Kantenlänge
von 12 mm.
Die SchulterstÃücke der Fähnriche und Offiziere
bestehen aus Silberplattschnüren auf einer Tuchun-
terlage in der Waffenfarbe. Die Sterne sind goldfar-
ben und vierzackig mit einer Kantenlänge von
ebenfalls 12 mm.Generale und Admirale tragen
fünfschlaufige silber-gold-farbene dickgeflochtene
Schulterstücke mit silberfarbenen fünfzackigen Sternen in einer Reihe. Attribute eines Marschalls
der DDR sind vierschlaufige, ebenfalls silber-gold-
farbene Schulterstücke mit einem groBen fünfzacki-
gen vergoldeten Stern mit eingelassenem Rubin.
Auf Felddienst- und Arbeitsuniformen sowie auf
Kradanzügen befinden sich bei den Angehörigen
der Landstreitkräfte und der LSK/LV Schulterklap-
pen und -stücke auf steingrauer Tuchunterlage mit
mattgrauen Litzen bzw, Plattschnüren und matt-
grauen Sternen.
Angehörige der Volksmarine führen auf Feld-
dienst- und Arbeitsuniformen Schulterklappen und
Schulterstücke gleicher Ausführung wie bei den an-
deren Uniformarten, Matrosen sowie Unteroffiziere
auf Zeit und im Reservistenwehrdienst befestigen
an der Arbeitsuniform alter Form, außer am Ar.
beitsanzug (Winter), keine Schulterklappen.
Die Knöpfe zu den farbigen Schulterklappen
und -stücken sind bei den Angehörigen der Land-
streitkräfte und der LSK/LY bis einschlieBlich
Oberst silberfarben, bei Generalen goldfarben. Zu
mattgrauen Schulterklappen und Schulterstücken
gehören mattgraue Kn¶pfe. Die Angehörigen der
Volksmarine, einschlieBlich der Admirale, haben goldfarbene Ankerkn'pfe.Der Durchmesser aller
Knöpfe beträgt 16 mm.
Unverändert führen Obermatrosen, Unteroffi-
ziersschüler und Maate auf Zeit an den Uberzie.
hern,Kieler Hemden,den Blusen des Arbeitsanzu
ges alter Form und des weiBen Bordanzuges
Armelabzeichen, die auf blauem Uniformstoff gold.
farben und auf weißem Uniformstoff blau sind.
Obermatrosen sind an einer mm langen und
7mm breiten goldfarbenen oder blauen Tresse
Stabsmatrosen an zwei der genannten Tressen zu
erkennen.Unteroffiziersschiler der Wolksmarinc
tragen einen nach oben offenen Winkel von 140
aus 7mm breiter goldfarbener oder blauer Tresse
Maate einen goldfarbenen oder blauen Anker mi
Dienstlaufbahnabzeichen als Symbol.Obermaate
tragen zus¤tzlich zur Kennzeichnung der Maate un
ter dem Anker einen nach oben offenen Winkel
Die Länge der Winkelschenkel beträgt 5 cm. Die
Dienstlaufbahnabzeichenbefindensichinder
Mitte des linken Armels des berziehers und des
Kieler Hemdes, der obere Rand des Abzeichens ist
14 cm von der Schulternaht entfernt. An den Uni
formen der Berufsunteroffiziersschüler,Berufsun
teroffiziere,Fähnrich-und Offiziersschüler der
Volksmarine befinden sich keine Armelabzeichen
und Armelstreifen.
Fähnriche, Offiziere und Admirale der Volksma.
rine besitzen zur Dienstgradkennzeichnung zusatz
lich nach wie vor Armelstreifen an ihren Uniform
iacken,Diese Armelstreifen weisen eine 【änge von
10 cm auf und befinden sich 9 cm über dem Armel.
saum.An der Gesellschaftsjacke gehen die rmel.
streifen von Naht zu Naht. Weibliche Offiziere der
Volksmarine tragen keine Armelstreifen.(Siehe Ta-
belle S.268)
Völlig neu und einheitlich gestaltet wurden die
Dienstgradabzeichen,die Angehörige der Flieger
kräfte der drei Teilstreitkräfte an Flieger- und
Technikeranzügen seit 1986 anbringen.Diese
neuen Dienstgradabzeichen befinden sich auf der
Mitte der linken Brusttasche: an der Tacke des Flie
geranzuges 5 cm oberhalb der Seitentasche, an der Jacke des Technikeranzuges 1 cm oberhalb der Ta
schenklappe und an der Latzhose des Technikeran
zuges 2 cm unterhalb der oberen Kante des Latzes
Die Dienstgrade befinden sich auf einer 9 cm lan
gen und 6cm breiten nterlage aus steingrauem
Uniformgewebe.(Siehe Tabelle S. 268 f.)
Das Tragen von Dienstlaufbahnabzeichen wurdc
mit der 1986er Vorschrift wesentlich eingeschränkt
Sie werden nun nur noch von Angehörigen der
Volksmarine,von Angehörigen der Musikkorps
Spielleuten und Militärmusikschülern sowie Ar
meeangehörigen der ilitäriustizorgane und des
medizinischen Dienstes angelegt. An Felddienst-
Bord- und Arbeitsuniformen gibt es keine Dienst.
laufbahnabzeichen mehr.Admirale fiihren auf bei
den Armeln der Uniform- und Gesellschaftsiacke
weiterhin den Seestern.
Die Dienstlaufbahnabzeichen der Volksmarine
bestehen für Matrosen aus gelber Stickerei au
blauer oder blauer Stickerei aufweier runder
Tuchunterlage mit einem Durchmesser von 6cm:
fir Berufsunteroffiziersschüler und Berufsunteroffi
ziere sind sie aus goldfarbenem Metall, 2 cm hoch
und 2 cm breit gepragt.
Fähnrichschüler,Offiziersschüler, Fahnriche, Of
fiziere und Admirale tragen Dienstlaufbahnabzei
chen aus goldfarbener Stickerei auf blauer oder
cremefarbener runderuchunterlage miteinem
Durchmesser von 4cm.Die Dienstlaufbahnabzei.
chen der Militarjustizorgane,des medizinischen
Dienstes,der ilit¤rmusiker und Militarmusik
schiüler sind aus silbergrauer Stickerei auf ovaler
Tuchunterlage gefertigt. Askulapstab und Lyra blei.
ben für Offiziere aus goldfarbenem, füir Soldaten
und Unteroffiziere aus silberfarbenem Material
Auf den Schulterstücken der Generale des medizi
nischen Dienstes befindet sich ein silberfarbenei
geprägter Askulapstab.
Die Schwalbennester der Angehörigen der Mu-
sikkorps und der Spielleute an der Paradeuniform
bestehen aus Aluminiumgespinsttresse auf halbrun
der steingrauer oder blauer Tuchunterlage.Ihi
Durchmesser beträgt 20cm.Die Schwalbennester der Mitglieder der Stabsmusikkorps weisen zudem
5 cm lange silberfarbene Fransen auf. Soldaten.Un.
teroffiziersschüler,Unteroffiziere,Fahnrichschüler
und Fahnriche der Landstreitkrafte und der
LSK/LV bringen die Dienstlaufbahnabzeichen in
der Mitte des linken Armels der Uniformjacke,
12cm vom Armelsaum entfernt. an.
Offiziere des medizinischen Dienstes sind an
Dienstlaufbahnabzeichen auf den Schulterstiüicken
zu erkennen.Auch Berufsunteroffiziersschüler und
Berufsunteroffiziere der Volksmarine tragen ihre
Dienstlaufbahnabzeichen auf den Schulterklappen.
Fähnrichschüler,ffiziersschüler,Fahnriche und
Offiziere dieser Teilstreitkraft führen ihre Dienst-
laufbahnabzeichen auf beiden Armeln der Uni.
formjacke,2cm über den Armelstreifen,in der
Mitte der rmel. Weibliche Offiziere, Offiziers
schüler und Fähnrichschüler, die an den Uniform
iacken keine Armelstreifen haben, befestigen ihre
Dienstlaufbahnabzeichen8cmvomArmelsaum
entfernt.
Militärmusikschüler tragen ihr Dienstlaufbahn.
abzeichen an der Dienstuniform am linken Unter.
arm, Die Lyren befinden sich auch auf den Schul
terklappen desniformmantels und der Hemd.
bluse; an der
Uniformjacke ersetzen sie die
Kragenspiegel.
Nach Sonderlehrgängen mitabgeschlossener
Prüfung haben Matrosen, Unteroffiziersschüler
und Maate auf Zeit das Recht, bis zu 2 Abzeichen
fiir Sonderausbildungen zu tragen. Sie bestehen aus
roter Stickerei auf blauer oder weiSer Tuchunter
lage mit einem Durchmesser von 7cm und sind in
der itte des linken Armels des berziehers und
des Kieler Hemdes,2cm unter dem Dienstlauf
bahnabzeichen anzubringen.
Zur Kennzeichnung ihres Dienstverhaltnisses
tragen Soldaten, Matrosen,Unteroffiziere und
Maate auf Zeit auf dem rechten Unterärmel der
Uniformjacke oder des Kieler Hemdes sowie des
Uniformmantels oder des Uberziehers einen nach
oben offenen stumpfen Winkel, Die Winkel der
Landstreitkrafte und der LSK/Ysind aus 15 mm breiter Aluminiumgespinsttresse aufsteingrauer
Tuchunterlage gefertigt, die der Volksmarine für
blaue Bekleidung aus goldfarbenem Gespinst auf
blauem Tuch und für weiße Bekleidung aus blauer
Wollstickerei auf weiBem Nessel.
Die Fähnriche tragen auf dem linken Oberärmel
der Uniformjacke und des Uniformmantels 12 cm
bis 14 cm von der oberen Schulternaht entfernt ihr
Armelabzeichen. Hauptfeldwebel sind weiterhin an
Ãrmelstreifen zu erkennen.
Außer den Hauptfeldwebeln führen die Angehö-
rigen des Wachregiments <Friedrich Engels», des
NVA-Wachregiments,des Erich-Weinert-Ensem-
bles und die Militärmusikschüler spezielle Ãrmel-
streifen mit entsprechend gestickter Aufschrift.