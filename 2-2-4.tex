

Bis 1964 unterschieden sich die Armeeangehörigen,
die die Unteroffizierslaufbahn als Beruf eingeschla-
gen hatten, also 10 Jahre und l¤nger dienten, ¤uBer-
lich nur wenig von den Unteroffizieren, die auf
Zeit, d,h, für 3 Jahre, aktiven Wehrdienst leisteten.
Nur der erreichte Dienstgrad eines Feldwebels bzw.
Wachtmeisters war deutliches Unterscheidungs
merkmal. Die ab 1.]anuar 1965 verwendeten
stumpfen Winkel und Doppelwinkel kennzeichne-
ten das Dienstverhältnis ebenfalls. Die erheblich
l¤ngere Dienstzeit und die höheren Forderungen,
die an diese Kader gestellt wurden, rechtfertigten
es, sie in der zweiten Hälfte der 60er Jahre, als es
die ökonomischen Möglichkeiten der Volkswirt-
schaft der Republik zulieBen, mit Uniformen der
Berufsoffiziere auszustatten, Die Meister der Volks-
marine verfiigten, wie schon dargestellt, seit Beginn
der 60er ,Jahre über Uniformen aus dem Stoff für
Offiziersuniformen.
Angemerkt werden soll hier, daß mit dem Erlaß
des Staatsrates der DDR vom 24.Januar 1962 über
den aktiven Wehrdienst in der NVA und der Dienstlaufbahnordnung in der Fassung vom 14. [a-
nuar 1966 die Unteroffiziersschüler und die Off
ziersschüler als Dienstgrade nach den Soldaten
bzw,Unteroffizieren eingeordnet wurden.
In der Neufassung dieses Erlasses vom 10.De
zember 1970 präzisierte der Staatsrat der DDR dic
Einordnung der Unteroffiziers- und der Offiziers.
schüler folgendermaBen: Die Unteroffiziersschüler
waren seitdem im Dienstgrad den Gefreiten bzw
Obermatrosen und die Offiziersschüler bis 1983 im
1.Lehrjahr dem Dienstgrad Unteroffizier/Maat, im
2.Lehrjahr dem Dienstgrad Feldwebel/Meister, im
3.Lehriahr dem Dienstgrad Oberfeldwebel/Ober.
meister und im 4. Lehrjahr dem Dienstgrad Stabs.
feldwebel/Stabsobermeister gleichgestellt. Diesen
Vorgriff abschlieBend, sei an dieser Stelle ange
merkt, daß der Staatsrat der DDR 1970 auBerdem
beschloß, für den ersten Dienstgrad der Landstreit.
kräfte und der LSK/LV einheitlich die Bezeich.
nung Soldat zu führen und auch parallel die zu den
Feldwebeln verwendeten Wachtmeisterbezeichnun
gen entfallen zu lassen.
Erste Festlegungen, um in den Landstreitkräften
und den LSK/LY die Berufsunteroffiziere und die
Offiziersschüler ab 3.Lehriahr mit Offiziersunifor
men auszustatten, traf der Stellvertreter des Mini.
sters für Nationale Verteidigung der DDR und
Chef Rückwärtige Dienste der NVA mit seiner An-
ordnung Nr.6/64 vom 5.Juni 1964.
Ab 1.]anuar 1965 trugen zuerst die Stabsfeldwe
bel bzw.Stabswachtmeister diese neuen [niformen,
Die Wintermütze wurde bis 30,November des [ah.
res an sie ausgegeben,Eine weitere Anordnung Ge
neralmajor W,Allensteins- Nr.6/65 vom 12. Sep
tember 1965-regelte die Ausstattung für alle
Oberfeldwebel und Oberwachtmeister,die noch
mindestens 2 Dienstjahre vor sich hatten, mit Offi-
ziersuniformen, Als Tragebeginn war dafür der
1. März 1966 festgelegt worden. Im Verlaufe des
ahres 1966 erhielten die Feldwebel und Wachtmei.
ster diese Ausstattung, und bis Ende des ]ahrzehnts
bekamen die restlichen Berufsuntcroffiziere die ent.
sprechenden Uniformen.

Im einzelnen erhielten die Berufsunteroffiziere
zunächst 1 Schirmmütze,1 Feldmütze,1 Winter-
mütze,1 Uniformmantel, 1 Paradejacke(Kragen
und Armel mit Biese),1 Dienstjacke,2 Stiefelho-
sen,1 lange Hose(mit Biese),1 Paar glatte und
1 Paar genarbte Schaftstiefel sowie 1 Koppel mit
Schnalle. Für die Berufsunteroffiziere in den
LSK/LV gab es aufgrund des offenen Fassonschnit.
tes ihrer Uniformen 2 silbergraue Uniformhemden,
2 dunkelgraue Binder und 1 grauen Schal.
MMit diesen Grundausstattungsnormen verfügten
die Berufssoldaten der Landstreitkräfte und der
LSK/LV der NVA über ausreichende Bekleidung.
um die vielfältigen, neu gestellten Aufgaben zu er.
füllen. Die Tabelle auf S.139 gibt Aufschluß über
die entsprechenden Grund- und Ergänzungsnor
men für ihre Uniformen aus der DV-98/4. Beklei.
dungs- und Ausrüstungsnormen der Nationalen
Volksarmee 1965.
Trotz festgelegter Normen war es m¶glich, Be.
kleidung entsprechend der Tätigkeit der Berufssol.
daten wahlweise zu empfangen.Beispielsweise
konnte die Stiefelhose gegen eine Uniformhose
oder das genarbte Paar Schaftstiefel gegen 1 Paar
Halbschuhe von Angehörigen der Stäbe und ande-
rer Einrichtungen ausgewechseltwerden.Be
stimmte Ausrüstungsgegenstände wie Stahlhelm,
Kopfschützer, Felddienstanzug, Gurtkoppel, Trage
gestell,Sturmgepäck, Feldflasche, Kochgeschirr
Zeltbahn und Decke wurden nur bei Bedarf an die
Berufssoldaten ausgegeben. Die Uniformstiicke ver
blieben nach Ablauf der Tragezeit im Besitz der Be-
rufssoldaten.