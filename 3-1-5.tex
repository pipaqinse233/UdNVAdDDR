

Am 10.Juli 1972, fünf Tage nach dem Beschluß des
Pr¤sidiums des Ministerrats über weitere MaBnah.
men zur Verbesserung der Lage der Berufssoldaten.
hatte der Minister für Nationale Verteidigung der
DDR, Armeegeneral H.Hoffmann, seinem Stellver.
treter und Chef Rückwärtige Dienste, Generalleut-
nant H, Poppe, den Auftrag erteilt, unter Ausnut-
zung der produktionstechnischen und ökonomi-
schen Möglichkeiten der DDR weitere Schritte zur
Verbesserung der Uniformen der Berufssoldaten
einzuleiten,Ziel war es, die Berufssoldaten attrakti.
ver und modischer einzukleiden, ohne Abstriche an
der militärischen Zweckmäßigkeit zuzulassen.
Bereits am 1.November desselben Jahres stellte
die Hauptabteilung Bekleidung/Ausrüstung im Mi-
nisterium für Nationale Verteidigung den Stellver-
tretern des Ministers für Nationale Verteidigung
entsprechend veränderte Uniformmuster vor. Am
22.Januar 1973 wurde die Kollektion der neuen
Uniformen offener Fasson vom Minister für Natio-
nale Verteidigung gebilligt. Auf entsprechenden
Antrag stimmte der Vorsitzende des Nationalen
Verteidigungsrates der DDR am 3. Mai 1973 der Einführung veränderter Uniformen der Nationalen
Volksarmee zu. Die Produktion der neuen Uni-
formteile begann im IIl.Quartel 1973.
Die Teilausstattung der Berufssoldaten in Etap-
pen regelte der von Generaloberst H.Keßler, Stell-
vertreter des Ministers und Chef des Hauptstabes
der NVA, unterzeichnete Befehl Nr. 36/74 des i-
nisters für Nationale Verteidigung üiber die Verän-
derung der Uniformen für Berufsunteroffiziere,
F¤hnriche und Offziere der Landstreitkräfte und
der LSK/LV der NVA vom 11.April 1974.
In diesem Befehl wurde festgelegt, daß die Be-
rufssoldaten der Landstreitkräfte Paradejacken of-
fener Fasson, dazu weiße Oberhemden und dunkel.
graue Regattes (Binder) erhalten, Als Tragebeginn
wurde der 1.Oktober 1974 bestimmt. Mit demsel-
ben Befehl erging die Weisung, den obengenannten
Personenkreis mit einer Dienstjacke offener Fasson,
silbergrauem Oberhemd, dunkelgrauem Binder und einem grauen Schal auszustatten. Diese neuen
Uniformstiüicke durften allerdings erst ab 1.Oktober
1975 getragen werden, Zur Vervollkommnung ihrer
Parade- und Ausgangsuniform erhielten Berufssol-
daten der LSK/LV bis zum 1.Oktober 1974 weiße
Oberhemden.
Die Ausstattung der Berufssoldaten der Land-
streitkräfte mit Uniformen offener Fasson war von
groBer volkswirtschaftlicher Tragweite, da zugleich
festgelegt worden war, daß nicht nur Berufssolda-
ten, also Berufsunteroffiziere, Fahnriche und Offi-
ziere, sondern auch Offiziere auf Zeit, Offiziers-
schüler, Unteroffiziere im Reservistenwehrdienst,
die aktiv als Berufsunteroffiziere gedient hatten, Offiziere im Reservistenwehrdienst, Armeeangehö-
rige der Musikkorps, m¤nnliche Armeeangehörige
des Erich-Weinert-Ensembles, Sportinstrukteure
der Armeesportklubs und die Angehörigen von Eh-
renkompanien mit diesen Uniformen auszustatten
waren.
Die hochgeschlossenen Uniformjacken waren
weiterhin zur Felddienstuniform, zur Dienst- und
Stabsdienstuniform bis zum 30.$eptember 1975 so-
wie ab Winterhalbjahr 1975/76 in der Zeit vom
1.Dezember bis zum 28./29.Februar aufzutragen.
Noch bis zum 30.$eptember 1976 war es gestattet,
im Ausgang die zweireihige Maßuniform anzuzie-
hen.
Erstmals stellten sich die Paradeeinheiten zur
Ehrenparade der NVA auf der Berliner Karl-Marx
Allee anläBlich des 25..Jahrestages der DDR am
7.Oktober 1974 in veränderten Uniformen vor.
Zur Paradeuniform gehörte jetzt bei den Berufssoldaten der Landstreitkräfte und der LSK/LV ein
weißes Oberhemd mit dunkelgrauem Regattes (Bin-
der). Zu Stiefeln und Stiefelhosen trugen die Be-
rufssoldaten der Landstreitkräfte die neue Parade-
jacke, die im Schnitt der der LSK/LV entsprach.
Sie war in offener Fasson einreihig mit vier Knöp-
fen zu schließen. Der Kragen war wie die gesamte
Uniformjacke steingrau und mit weiBen Biesen,
Kragenspiegeln auf dunkelgrauem Untergrund und
weißer Kantillenfüllung versehen. Die .Jacke wies
vier aufgesetzte Taschen zum Kn¶pfen und ge-
schweifte Patten sowie Ãrmelaufschläge mit Biesen
und Ãrmelpatten auf. Die Paradejacke für Generale
war ebenfalls einreihig mit offener Fasson geschnitten. Sie konnte mit vier Knöpfen geschlossen wer-
den und hatte einen steingrauen Kragen mit Biese,
arabeskengeschmückte Kragenspiegel, vier aufge-
setzte Taschen mit geschweiften Patten zum Knöp-
fen sowie Ãrmelaufschläge mit Biesen und Arabes-
ken. Die Biesen- und Lampassenfarbe für Generale
der Landstreitkräfte blieb Hochrot, für Generale
der LSK/LV Hellblau. Wie im Befehl Nr.36/74 des
Ministers für Nationale Verteidigung festgelegt.
trugen Berufssoldaten der Landstreitkräfte ab
1.Oktober 1975 zur Dienst- und Stabsdienstuni-
form die neue Dienstjacke offener Fasson. Sie un-
terschied sich von der Paradejacke lediglich durch
das Fehlen der Ãrmelpatten bei den Berufssoldaten
bis Dienstgrad Oberst und der Arabesken bei den
Generalen.
Seit dieser Zeit gehören zur Dienstjacke offener
Fasson ein silbergraues Oberhemd und ein dunkel
grauer Binder.