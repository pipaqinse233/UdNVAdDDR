\subsection{海军的制服}

\begin{figure}
\includegraphics[width = \columnwidth]{./media/page (16).jpg}
\end{figure}

在德国体育馆的展览中,参观者发现海军制服种类比陆军和空军的更多。在海军制服前人头攒动。海军制服与陆军和空军制服的最大区别在于配色和剪裁。此外,大多数类型的制服除了蓝色款式外,还允许在夏季(5月1日至9月30日)穿着白色款式。
% Eine größere Vielfalt der Uniformierung als bei den Land- und Luftstreitkräften fanden die Besucher der Ausstellung in der Deutschen Sporthalle bei den Seestreitkräften vor. Der Andrang vor den Uniformen dieser Teilstreitkraft war außerordentlich groß. Die gravierendsten Unterschiede zwischen den Uniformen der Seestreitkräfte und denen der Land- und Luftstreitkräfte bestanden in der Farbgebung und der Schnittgestaltung. Weiterhin ließen die meisten Uniformarten neben der blauen noch eine weiBe Ausführung für die Sommertrageperiode, die bei den Seestreitkräften die Zeit vom 1. Mai bis zum 30. September umfaßte, zu.

水兵、海军下士和海军中士穿白色或蓝色的水兵服。因此,水兵帽、上衣和裤子以及半靴、系带鞋或水手鞋和带扣腰带都是必备的。上课和操练时也要佩戴基尔领。在冬季,水手和船员们会在蓝色水兵服的基础上再搭配一件蓝白条纹长袖背心\footnote{译者注:应指「海魂衫」。}和蓝色羊毛手套。在恶劣天气条件下,指挥官可以命令他们穿外套和高领毛衣。
% Für Matrosen und Maate war das Tragen des weißen oder des blauen Bordanzuges möglich. Vorgeschrieben waren dementsprechend Bordkäppi, -bluse und -hose sowie Halbschaftstiefel, Schnürschuhe oder Bordschuhe und Koppel mit Schloß. Im Unterricht und beim Exerzieren kam der Kieler Kragen hinzu. In der Wintertrageperiode ergänzten die Matrosen und Maate den blauen Bordanzug durch ein blau-weiß gestreiftes Unterhemd mit langen Ärmeln und blaue Wollhandschuhe. Bei schlechten Witterungsverhältnissen konnte der Kommandeur befehlen, Überzieher und Rollkragenpullover anzuziehen.

海军上士、海军尉官和海军校官只穿蓝色水兵服,包括水兵帽、上衣和长裤,以及半靴(只适用于海军上士)或系带鞋或水兵鞋。此外,海军上士还戴着蓝色羊毛手套,海军尉官和海军校官则戴着黑色皮手套,以抵御冬季的寒冷。
% Meister und Offiziere der Seestreitkräfte besaßen ausschließlich den blauen Bordanzug. Dieser bestand aus Bordkäppi, -jacke und -hose sowie Halbschaftstiefeln (nur bei den Meistern) oder Schnürschuhen bzw. Bordschuhen. Zusätzlich schützten sich die Meister mit blauen Wollhandschuhen und die Offiziere mit schwarzen Lederhandschuhen gegen die Winterkälte.

\begin{figure}
\includegraphics[width = \columnwidth]{./media/page (19).jpg}
\end{figure}

水兵、海军下士和海军中士的常服、礼服和外出服分为很多部分(见第16页表格)% TODO
% Die Dienst-, die Parade- und die Ausgangsuniformen der Matrosen und Maate setzten sich aus vielen Uniformstücken zusammen. (Siehe Tabelle S.16)

除了偶有不同设计外,海军上士、海军尉官、海军校官和海军将官的常服、礼服和外出服有许多共同之处。
% Von einer manchmal unterschiedlichen Ausführung abgesehen, wiesen die Dienst-, die Parade- und die Ausgangsuniformen der Meister, Offiziere und Admirale der Seestreitkräfte viele Gemeinsamkeiten auf.

海岸炮兵和高射炮兵、工兵连和保卫连、汽车兵以及训练和警卫部队的海军尉官和海军校官在某些场合穿着蓝色马裤和高筒靴。对于汽车兵来说,这只适用于驾驶任务和车辆训练;对于其他海军尉官和海军校官来说,这适用于警卫任务、操练、射击训练、行军、演习和视察。海军将官在演习和视察陆军部队时也穿马裤和高筒靴。
% Offziere der Küsten- und der Flakartillerie, der Pionier- und der Schutzkompanien, des Kfz-Dienstes sowie Kommandeure und Ausbildungsoffiziere in Ausbildungs- und Wacheinheiten trugen zu bestimmten Anlässen eine blaue Stiefelhose und Schaftstiefel. Dies traf bei den Offizieren des Kfz-Dienstes nur für den Fahrdienst und die Ausbildung an Fahrzeugen, bei den anderen Offizieren für den Wachdienst, das Exerzieren, die Schießausbildung, für Märsche, Übungen und Besichtigungen zu. Admirale zogen bei Übungen und Besichtigungen von Landeinheiten ebenfalls Stiefelhose und Schaftstiefel an.