

Ab Oktober 1975 erhielten auch Generale und Ad.
mirale einen veränderten Felddienstanzug aus Drei
fasermischgewebe imStricheldruck.ReiBver-
schliisse an den Felddienstuniformen hatten sich
nicht bewährt. Sie wurden wegen der besseren Halt.
barkeit wieder durch Knöpfe ersetzt. Der neue, für
Sommer und Winter bestimmte modifizierte Feld.
dienstanzug für Generale und Admirale wurde
nicht mit wattiertem Unteranzug getragen wie bei
den Fallschirmjägern, Im Winter wurde als zusätzli-
cher Kalteschutz ein isolierendes Steppfutter einge-
knöpft. Mit der Einführung des Felddienstanzuges
für Sommer und Winter fielen fǔr Generale die
Schirmmütze, der Uniformmantel und die Dienst-
jacke mit jeweils mattgrauen Effekten sowie die
Stiefelhose ohne Lampassen weg. Diese Beklei.
dungsstüicke konnten noch bis zum 30.September
1975 aufgetragen werden.
[m Zusammenhang mit der Einführung von Uni.
formen mit offener Fasson für Berufssoldaten wur-
den zunächst Festlegungen getroffen, die gewahrlei-
steten, daß bei der Felddienstuniform Oberhemd
und Binder nicht sichtbar waren. Nach einer Vor
führung im Ministerium für Nationale Verteidi.
gung am 20.September 1976 wurde verfügt, daB
Generale im
Sommer den Felddienstanzug ohne Einknöpffutter mit geöffnetem ackenkragen und
darunter breitgelegtem Hemdkragen zu tragen ha-
ben. Im Winter zogen die Generale den Felddienst.
anzug mit eingekn¶pftem Futter an und schlossen
den Kragen. Darunter wurde dann die Dienstuni.
form mit Oberhemd und Binder angezogen. In der
Zeit vom 1.Dezember bis zum 28./29.Februar
wurde der Webpelzkragen aufgeknöpft. Dazu kam
die Wintermütze.
Offiziere öffneten im Sommer ebenfalls den Kra-
gen des Felddienstanzuges und legten den Kragen
des Oberhemdes breit, Für den Winter galt, daß sie
unter dem Watteanzug die Dienstuniform ein-
schlieBlich Pullover mit V-Ausschnitt und stein-
grauem Oberhemd anzuziehen hatten, jedoch kei-
nen Binder.Vom 1.Dezember bis zum 28./29.Fe
bruar trugen auch sie die Wintermütze und hatten
den Webpelzkragen aufgeknöpft.
Der Webpelzkragen für Berufssoldaten (Tragebe
ginn 1.Dezember 1976),eine Neuentwicklung,
durfte nur in Verbindung mit der Wintermütze auf-
gekn¶pft werden. Er war einteilig mit Steg gearbei-
tet. Im Steg befanden sich zwei Knopflöcher und
an den Kragenecken je eine Tasche zum Ankn¶p
fen und Aufstecken des Webpelzkragens auf den
Kragen des Watteanzuges. Der Pelzkragen wurde in
fünf GröBen gefertigt, Der Oberkragen bestand aus
steingrauem, für die Volksmarine aus schwarzem
Webpelz. Für den Unterkragen mit Steg wurde das
steingraue bzw, das schwarze Gewebe des Wattean-
zuges oder das auch für die Felddienstanzüge ver-
wendete Dreifasermischgewebe mit Stricheldruck
eingesetzt.