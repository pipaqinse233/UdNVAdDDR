

In gleicher Weise und mit demselben Befehl des
Ministers für Nationale Verteidigung der DDR
über die Dienstlaufbahnabzeichen wurde ab Ende
1957 die Schützenschnur als Auszeichnung ge-
schaffen. Entsprechend dem Ziel, die Entwicklung sozialistischer Soldatenpers硕nlichkeiten umfassend
zu fördern,beschränkten sich die Verleihungsbe.
dingungen nicht allein auf die Zahl der erzielten
Treffer bei den Schießübungen. Die Schützen-
schnur erhielten nur die Soldaten, Unteroffiziere
und Offiziersschüler,die in der gesamten Dienst
durchführung vorbildlich handelten und gleichblei.
bend gute und sehr gute SchieBergebnisse erzielten.
Fir den Erhalt der Schitzenschnur wurden keine
besonderen,sondern die Ubungendes Ausbil
dungsprogramms geschossen.In der zweiten Halft(
der 50er |ahre k¤mpften die Angehörigen der mot.
Schiitzen-und Aufklarungseinheiten mit Schitzen.
waffen,d.h.Karabinern,Maschinenpistolen und
Maschinengewehren,um die 4 Stufen der Schüt
zenschnur inn SchieBen von Schul- und Einzelge
fechtsüibungen; die Angehörigen der anderen Waf.
fengattungen,Spezialtruppen und Dienste in den
Schulübungen um 2 Stufen. Die Schützenschnur
für das SchieBen mit Spezialwaffen konnte von den
Angehörigen der Artillerie- und Panzereinheiten
sowie von denen der Seestreitkräfte beim Artillerie.
und TorpedoschieBen in 2 Stufen errungen werden
Immer begann die Verleihung der Schützen
schnur mit der Stufe I- der eigentlichen Schützen.
schnur ,die höheren Stufen wurden durch Ei
cheln kenntlich gemacht und setzten den Erwerb
der davorliegenden Stufe voraus. Jeder Tr¤ger der
SchieBauszeichnung wuBte, daß er nur durch inten
sives Training zu besseren Ergebnissen und damit
von Stufe zu Stufe gelangen konnte. Da zudem dic
Schützenschnur nur am Ende einer Ausbildungspe
riode und in begrenzter Anzahl verliehen wurde
trugen ausschlie&lich die Besten die begehrte Aus.
zeichnung.Anfang Oktober 1958 erhielten erstmals
305 Soldaten und Unteroffiziere die Schiitzen
schnur.Sie dokumentierten damit in der ffent
lichkeit,daß sie an ihrem Kampfabschnitt den
Werktätigen im Ringen um hóhere Leistungen
nicht nachstanden und daß der Schutz des Friedens
und des Sozialismus bei ihnen in zuverlässigen
Händen liegt.

Die ungefihr 35cm lange Schiützenschnur besteht auch heute noch in ihrer allgemeinen Form
aus einer geflochtenen silberfarbenen Aluminium.
schnur, Ein 50mm hohes Alminiumabzeichen
zeigt auf einem silberfarbenen Eichenlaubkranz ge.
kreuzte Schützenwaffen, An Artilleristen wurde die
Auszeichnung mit einer stilisierten Granate und an
Panzerschützen mit einem stilisierten Panzer im Ei-
chenlaubkranzoval vergeben, Die Matrosen, Maate,
Meister und Offiziersschüler der Seestreitkräfte tru-
gen eine kornblumenblaue Schnur mit goldfarbe.
nen Abzeichen, Für das Schießen mit Torpedos gab
es die besondere Schnur mit einem stilisierten Tor.
pedo im Eichenlaubkranz.
Die Schützenschnur wurde an der rechten
Schulterklappe und an der Knopfleiste der Parade.
und Ausgangsuniform befestigt. Matrosen und
Maate legten die Schützenschnur, von der rechten
Schulter fallend,unter dem Kieler Kragen zum
Knoten des seidenen Halstuches an. Die Eicheln
wurden jeweils am unteren Drittel der Schnur ange
bracht.