

Im folgenden soll eine Vielzahl von weiteren Ergän-
zungen der Uniformierung der NVA und Verände.
rungen bei den Effekten und anderen Details er
wähnt werden,Nach einer vierteliährlichen erfolg
reichen Erprobung-vom 1.Dezember 1964 bis
31.März 1965 im Milit¤rbezirk Leipzig und in der
Volksmarine - erhielten von 1966 an die Meister
später auch die Berufsunteroffiziere,Offiziere, Ge
nerale und Admirale verbesserte niformmäntel
Sie waren aus Kammgarngewebe gearbeitet und mit
Schaumstoff beschichtet. Diese Mäntel zeichneten
sich durch ein geringeres Gewicht, erhöhte W¤rme.
haltung, Knitterarmut und gute Tragbarkeit aus. Es
gab keine ausgearbeitete Rückenfalte mehr.
Noch im selben Jahr wurden für die Komplettie
rung der Offiziersuniformen nur noch aus Metal
geprägte Effekten verwendet, d.h., die Embleme der
Schirm-und Wintermützen, die Kragenspiegel und
die Armelpatten waren in den Landstreitkräften
und den LSK/LV aus Neusilber und in der Volks
marine aus vergoldetem Tombak. Bei den Genera
len und Admiralen ver¤nderte sich das Material der
Effekten ebenfalls:Schulterst¼cke und die Korde)
der Schirmmützen waren aus Dederonseide.
1967 erhielten die Offiziere,Generale und Admi.
rale, die ilitärdelegationen der DDR in sozialisti
sche 【änder und in junge Nationalstaaten tropi
scher Regionen angehörten, dem dortigen heiBen
Klima angepaBte einreihige ,ackenblusen mit kur.
zen Armeln und offener Fasson. Die bisher verwen
dete weiBe Uniformjacke sollte wegfallen, wurde
aber offenbar weiter angezogen. Seitdem werden
auch die weißBe Dienst- und die weiBe Ausgangs
iacke der Offiziere und Admirale der Volksmarine
nicht mehr getragen.
Ebenfalls 1967 wurden durch Rationalisierungs
maßnahmen die Abmessungen der Dienstgrad
sterne für die Schulterklappen und -stücke verein
heitlicht. Es gab sie sowohl für die Stabsoffiziere
und die Offiziere der Dienstgrade Unterleutnant bis Hauptmann als auch für die Unteroffiziere nur
noch in einer Größe von 12 cm Länge einer Seiten
kante des Sterns. Die Unterscheidung in silber- und
goldfarbene Dienstgradsterne blieb erhalten. Aller
dings sei erwähnt, daB noch bis in die Gegenwart
besonders bei Unteroffizieren Dienstgradsterne der
alten,gröBeren Abmessungen auftauchen.
Im Zusammenhang mit der Einführung eines
neuen Druckanzuges sowietischen Fabrikats fiir die
Flugzeugführer von berschalliagdflugzeugen in
die Luftstreitkräfte der NVA erhielten diese Flug
zeugführer 1968 auch eine neue Fliegerkombina.
tion, Sie wies Verbesserungen in der Gewebezusam
mensetzung auf,spezielle Halterungen
WVaTeT
angebracht,und ein neues GröBensystem erleich-
terte die paBgerechte Auswahl der Kombination.
Der Druckanzug selbst bestand aus festem Perlon.
gewebe mit Reiß-und Schnürverschlüssen sowie
eingearbeiteten Druckschläuchen.
Gleichfalls 1968 bekamen die Panzerbesatzun
gen eine neue weiteilige steingraue Kombination
die in ihrer verbesserten Form und Ausführung den
Felddienstanzügen entsprach.Auch für diese Son.
derbekleidung galten die Maße des neuen Größen
Die Spezialkombination mit PC-Be
systems.
schichtung für Kradfahrer wurde in der PaBform
verbessert.Weiter erhielt diese Kombination einer
Kragen mit anknöpfbarem Webpelz. Die Techni.
kerkombination des fliegertechnischen Personals
der Luftstreitkr¤fte wurde in Form und Ausführung
dem Watteanzug angeglichen.
Geregelt durch die Anordnung Nr.2/69 des Stell
vertreters des Ministers füir Nationale Verteidigung
der DDR und Chefs Rückw¤rtige Dienste der NVA
vom 15.Februar 1969,wurde für Berufssoldaten
und weibliche Armeeangeh¶rige ein neuer Sommer
mantel aus synthetischem Mischgewebe eingefihrt
Berufssoldaten konnten ihn ab April 1969 gegen
Bezahlung über die B/A-Lager erwerben, Frauen
auBer Offiziere - erhielten ihn kostenlos. Der bis
herige Sommermantel ausgummiertemGewebe
wurde bis einschlieBlich 1972 noch aufgetragen.
Aufgrund des Befehls Nr.148/69 des Ministers für Nationale Verteidigung der DDR vom 5. De
zember 1969 erhielten Soldaten und Unteroffiziere
auf Zeit der Landstreitkräfte und der LSK/LV
1970 wieder das schwarze Lederkoppel mit SchloB
Es war etwas ver¤ndert worden, d.h., die Koppelha
ken befanden sich im Koppelschloß.
Ein neuer Kampfanzug verbesserte ab 1970 er
heblich die Ausstattung der Angehörigen der fah
renden Volksmarineeinheiten, ÃuBerlich im wesent
lichen gleichbleibend, seien doch die neue Gewebe
zusammensetzung(Polyamidseide)mit Spezialbe
schichtung und die nahtarme Gestaltung besonders
hervorgehoben.
Im einzelnen bestand der auch heute noch ver
wendete Kampfanzug verbesserter Ausführung nur
noch aus zwei Stücken - der Hose mit enganliegen-
den Beinmanschetten und Hosenträgern und der
acke mit innen eingearbeiteter Schwimmweste mit
Rettungsgurt und Sicherheitsleine. Die acke lie
oben in einer Kapuze aus,die mit einer Spezial
leuchtfarbe überzogen war und einen verstärkt ge
polsterten KinnverschluB hatte.Die Tacke konnte
mit Knebeln und Plastdruckknöpfen sicher ge
schlossen werden.Vervollständigt wurde sie durch
den Rettungsgurt mit einer kleinen versteiften Rük
kenstütze,der vorn ein etallring eingearbeitet
war. Hier konnte im Bedarfsfall das Rettungssei
eingehakt werden, mit dem über Bord gegangene
Besatzungsmitglieder geborgenwerdenkonnten
Neu war an der Gefechtsuniform der Volksmarine
ein Sichtfenster auf der linken Brustseite, hinter
dem die Rollennummer des Trägers, d.h. seine
Funktionsnummer an Bord,sichtbar war.Das war
eine notwendige Information, da am Kampfanzug
der Volksmarine sonst keine Dienstgrad- oder Lauf
bahnabzeichen befestigt wurden.Gummistiefel
Schutzmaskentasche mit Schutzmaske und Schutz.
handschuhe sowie der Stahlhelm vervollständigten
die Gefechtsuniform.
Auf kleinen Torpedoschnellbooten wurden die
Besatzungsmitglieder statt mit dem Stahlhelm mit
Spezialhelmen ausgerüstet,die den notwendigen
Schallschutz garantierten.Da die Besatzungsmitglieder dieser KTS-Boote untereinander keinen
Sichtkontakt hatten, waren in diesen Helmen auch
Kopfhörer für die Befehlsübermittlung eingebaut.
Die neue Gefechtsuniform erfullte optimal drei
Funktionen:
:Sie bot Schutz vor Massenvernich-
tungsmitteln, schirmte Kalte und Nässe bei Tatig-
keiten am Oberdeck ab und war zugleich persönli.
ches Rettungsmittel mit einem Auftrieb von
16 Kilopond, Sie bewährte sich unter den verschie-
densten Gefechts- und Einsatzbedingungen hervor-
ragend. Trotz ihrer Mehrzweckfunktion bot sie
ihrem' Träger volle Bewegungsfreiheit, Das beklei-
dungsphysiologische Wohlbefinden wurde unter-
stiüitzt durch das Vermögen, Luft durchzulassen,
W¤rme zu leiten und sie zu halten.
Die Angehörigen der fahrenden Einheiten der
Volksmarine fiihrten auch eine Tragetasche aus
dem Material des Kampfanzuges sowie Farbbeutel
und Schrillpfeife zur optischen und akustischen Si-
gnalgebung in Gefahrensituationen mit sich.