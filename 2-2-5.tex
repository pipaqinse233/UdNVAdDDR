

Nachdem etwa ein halbes Jahrzehnt von den Solda.
ten, Unteroffizieren und Offizieren der Landstreit-
kräfte(auBer den Panzerbesatzungen)und der
Truppen der Luftverteidigung der NVA der
Kampfanzug im Flchendruck als Felddienstuni.
form getragen worden war, begann Mitte der 60er Jahre die Einführung eines neuen Kampfanzuges.
Ihr ging wie zuvor bei anderen Uniformen und
Uniformstücken eine sorgfältige Erprobung- seit
dem Sommer 1963 und dieses Mal nicht in der
NVA, sondern in den Grenztruppen der DDR-
voraus, Mit der Bezeichnung <Kampfanzug leichter
Art» wurde schon auf einen der wesentlichen Vor-
züge der neuen Uniform verwiesen, auf das gerin-
gere Gewicht dieser Felddienstuniform.
Weitere Verbesserungen, die mit dem neuen
Kampfanzug, auch als «Kampfanzug 64» benannt, verwirklicht werden konnten, resultierten aus den
mehrjährigen Erfahrungen mit dem bisherigen
Kampfanzug im Flächendruck in den Truppentei-
len und Einheiten, Dabei gingen die mit der Ent-
wicklung betrauten Ingenieure der Bekleidungsin-
dustrie und die verantwortlichen Offiziere des
B/A-Dienstes der NVA von der Notwendigkeit aus,
bei diesem Kampfanzug st¤rker die Anforderungen
des Gefechts und die modernen bekleidungshygie-
nischen Ansprüche miteinander zu verknüpfen.
Der neue Felddienstanzug- so schließlich seine
Bezeichnung in der DV-10/5, Ausgabe 1965 - be-
stand aus einem nur gering schmutzempfindlichen,
relativ gut wasserabweisenden und hohen reiß- und
scheuerfesten Dreifasermischgewebe aus Dederon, Grisuten und Baumwolle in grünbräunlichem Farb
ton, Aufgedruckte kleine braune Striche verliehen
ihm einen zusätzlichen Tarneffekt. Deshalb brach-
ten sie ihm auch die von den Soldaten geprägte
scherzhafte Bezeichnung <Ein-Strich-kein-Strich-
Anzug» cin. Imprägnierte aufgesetzte Stoffteile ver-
stärkten den Felddienstanzug an den Ellenbogen,
Knien und am Ges¤ß. Kn¶pfe an den Saumbünd.
chen der rmel- und Hosenbeinenden dienten
dazu, die Öffnungen für die Arme und Beine zu
vergröBern oder auch zu verkleinern. Die .Jacke
schloß am unteren Bund mit einem Gummizug ab.
Die Hosen wurden iiber die Stiefel getragen und
unten gekn¶pft. Zum Felddienstanzug gehörten
weiterhin dunkelgraue Schulterklappen und -stücke mit mattgrauen itzen und Dienstgradsternen(au
Ber Volksmarine), eine grau-weiBe Kragenbinde,
¤ußerst strapazierfahige Spezialhosenträger und das
Gurtkoppel mit Schloß.Zur Felddienstuniform
wurde die Feldmütze bzw, das Bordkäppi genom-
men. Auf Befehl der Kommandeure muBte der
Stahlhelm getragen werden, Berufssoldaten setzten
damals noch die Schirmmütze auf.
Wenn zur Ausbildung zusätzliche Ausrüstung
befohlen wurde,kam wie schon vorher beim
Kampfanzug im Flächendruck das Tragegestell aus
Gurtgewebe hinzu, um an ihm die weiteren Ausrü-
stungsgegenstände befestigen zu k¶nnen, AuBer der
persönlichen Waffe (für Soldaten in der Regel die
MPi KM)und der Schutzmaske führte der Soldat
an Gurtkoppel und Tragegestell mit sich: vorn
rechts die Tasche mit drei Stangenmagazinen für
die MPi; vorn links das Seitengewehr - ein dolchar.
tiges Messer zur Verwendung als Bajonett nach Aufpflanzen auf die Laufmündung der MPi oder
auch als Drahtschere bzw, Säge; hinten rechts den
zusammenklappbaren Feldspaten; hinten links die
Feldflasche; auf dem Rücken oben den zusammen
gerollten Schutzanzug.Die Schutzmaske wurde in
einer gesonderten Tasche links in Höhe des Kop
pels getragen. Die neun Taschen des Felddienstan.
zuges erm¶glichten es dem K¤mpfer, ie nach Ein.
satzart die verschiedenstenmilitarischen
und
persónlichen Gegenstände unterzubringen.
bei
spielsweise
Waffenreinigungsgerät, Entgiftungs
päckchen,
Verbandmittelsatz.
medizinisches
Schutzpäckchen,Personendosimeter,KompaSund
Weitere Bekleidung,Ausrüstung
Taschenlampe.
und pers¶nliche Dinge wurden wie schon vordem
in einem zweiteiligen Sturmgepäck- jetzt auch im
Stricheldruck gefertigt - auf dem Gefechtsfahrzeug
mitgeführt.Die Armeeangeh¶rigen zogen den Feld
dienstanzug entweder nur über die nterwäsch
oder über die Sommerdienstuniform bzw. bei der
Volksmarine über den Bord- oder Arbeitsanzug an
Die schrittweise Einführung des neuen Feld
dienstanzuges regelte der Stellvertreter des ini
sters für Nationale Verteidigung der DDR und
Chef Rückwärtige Dienste der NVA in seinen An-
ordnungen Nr.2/65 vom 14.Mai 1965 und Nr.4/67
vom 14.0ktober 1967.Aufgrund erstgenannter An
ordnung bekamen zunächst bis Ende 1966 die Ver
bände, Truppenteile und Einheiten des Militärbe
zirkes Neubrandenburg sowie die ihm wirtschaftlich
unterstellten Truppenteile und Einheiten den Feld.
dienstanzug.Die Fallschirmjäger des Truppenteils
«Willi S¤nger》 und auch die Aufklärer der Trup.
penteile erhielten ebenfalls diesen Felddienstanzug
in der füir sie optimalen Schnittgestaltung. Als Tra.
gebeginn war der 1.M¤rz 1967 festgelegt worden.
Der weitere Prozeß der Einf¼hrung dieses Feld.
dienstanzuges vollzog sich in Etappen bis zum
Ende des ahrzehnts,d.h.bis Dezember 1967 für
den Militärbezirk Leipzig,bis Dezember 1968 für
selbständige Truppenteile, Einheiten, Dienststellen
und Einrichtungen des Ministeriums für Nationale
Verteidigung der DDR und bis Ende 1969 im Bereich des Kommandos der LSK/LV der NVA sowie
für die Landeinheiten der Volksmarine. In all die-
sen Fallen galt immer der 1. Mai des folgenden Jah-
res als Tragebeginn. Die bisher verwendeten
Kampfanzüge im Flächendruck wurden noch auf-
getragen, Die mit dem neuen Felddienstanzug aus-
gestatteten Armeeangehörigen trugen ihn bei allen
taktischen Ãbungen und bei der taktischen Einzel-
ausbildung einschließlich der SchieBausbildung.
Im Zusammenhang mit der Einführung der
neuen Felddienstbekleidung konnten verschiedene
Uniformen als Arbeitsanzüge genutzt werden. Dies
galt u. a. für alle Bestände nicht mehr für die Aus-
bildung verwendeter Dienstbekleidung wie Drillich-
uniformen,Sommerdienstuniformen der Kategorie Il und Kampfanzüge im Flächendruck. Als
Arbeitsbekleidung wurden sie schwarz umgefärbt.
In den Lagern der Truppenteile und selbständigen
Einheiten vorhandene Uniformmäntel der Kateg0
rie II- so legte es die Anordnung Nr,2/67 fest-
wurden zu bestimmten Diensten wie Außendienst
in der Übergangszeit, Ernteeinsätze, Arbeitsdienste
usw. ausgegeben.
Am 8. Oktober 1969 ordnete der Stellvertreter
des Ministers für Nationale Verteidigung der DDR
und Chef Rückwärtige Dienste der NVA zusätzlich
an, daß die Armeeangehörigen, die mit einem
Stahlhelm ausgerüstet sind, zur weiteren Vervoll-
kommnung ihrer pers¶nlichen Ausrüstung bis Ende 1970 ein dunkelgrünes Stahlhelmtarnnetz aus De
derongarn erhalten.Dieses Netz war bei Truppen
übungen, der Taktikausbildung und beim Gefechts
dienst sowie bei allen Ausbildungsmaßnahmen, die
Tarnung erforderten,zum Felddienstanzug mitzu
führen, Die Kommandeure, die die jeweilige Aus
bildung leiteten, konnten festlegen, ob das Stahl
helmtarnnetz oder die Kapuze des Felddienstanzu
ges zu tragen war.Der Stahlhelm selbst bekam
übrigens seit 1966 eine neue Helminneneinrich.
tung aus Polyäthylen, die auswechsel- und verstell.
bar war.
Nicht nur die Soldaten, Unteroffiziere und Off.
ziere der genannten Dienstbereiche erhielten neu
Felddienstanzüge, auch für Generale und Admirale
gab es seit 1965 einen speziellen Felddienstanzug
Er bestand aus einer einreihigen, durch ReiBver
schluß zu schließenden Jacke mit Steppfutter und
einer Hose,Er wurde 1967 zus¤tzlich mit eine
Webpelzf¼itterung und einem einknöpfbaren Web
pelzkragen versehen. Die DV-10/5,Ausgabe 1965
schrieb für die Generale und Admirale als Uniform
zum Felddienst,zu Truppenübungen und zur Ge
fechtsausbildung noch dasTragen von Schirm-
Feld- oder Wintermütze,Uniformmantel, bei der
entsprechenden Witterung Regenumhang,Dienst
iacke,Schal,Uniformhemdbluse, Stiefelhose(Ad
mirale die lange Hose), Stiefel, Stahlhelm und Kop
pel mit Schnalle vor. Bis Ende November des
ahres 1967 konnten die Generale und Admirale im
Winter generell den Ledermantel anziehen. Danach
galt die Festlegung, diesen Mantel nur noch zur
Felddienstuniform aufzutragen.
Die weiblichen Armeeangehörigen erhielten ab
1964 ebenfalls eine zweiteilige Felddienstbeklei
dung. Die DV-10/5 von 1965 schrieb für die weibli
chen Armeeangeh¶rigen aller Teilstreitkräfte dic
Einzelheiten ihrer neuen,dritten Uniformart yor
Farblich in Steingrau bzw.Dunkelblau gehalten
setzte sich die Felddienstuniform der Frauen au!
Schiffchen,Wintermütze,Uniformmantel, Uni
formjacke,Uniformrock,Feldbluse in Anorakform
mit einknöpfbarem Futter,Uniformhose,Bluse Binder, Schal, Stiefeln, Halbschuhen (Sporta)und
Handschuhen zusammen, die nach Notwendigkeit
und Witterung von den Frauen kombiniert werden
konnten.
Die Jacke der Felddienstuniform war einreihig,
hatte eine verdeckte Knopfleiste und einkn¶pfbares
Futter. Die Hose war gering keilförmig geschnitten,
wurde über den Stiefeln getragen und jeweils an der
Seite mit einem ReiBverschluß geschlossen. Erst
1967 kam eine andere lange Hose hinzu, die ähn-
lich einer Skihose stark keilf¶rmig geschnitten war.
Insgesamt wurde die Felddienstbekleidung der
NVA in der zweiten Hälfte der 60er Jahre erheblich
vervollkommnet, Mit diesen Weiter- und Neuent
wicklungen im Verantwortungsbereich des B/A
Dienstes verbesserten sich für alle Armeeangehöri-
gen die Bedingungen füir die Dienstdurchführung,
für die Erfüllung der gestellten höheren Aufgaben
zur Gewährleistung von Kampfkraft und Gefechts-
bereitschaft der Verbände, Truppenteile und Ein-
heiten, Aber nicht nur auf diesem sehr wichtigen
Gebiet der Ausstattung mit Bekleidung und Ausrii.
stung vollzogen sich im Rahmen der volkswirt
schaftlichen Möglichkeiten weitere Vervollkomm-
nungen. Auch auf anderen Gebieten der Uniform-
entwicklung jener Zeit sind gr¶ßere und kleinere
Veränderungen zu vermerken.