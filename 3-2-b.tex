

1974 hatte der Minister für Nationale Verteidigung
die Aufgabe gestellt, bis 1979 die Ausstattung der
Soldaten und Unteroffiziere auf Zeit mit Paradeund Ausgangsuniformen offener Fasson vorzuneh.
men,Die Einführung der veränderten Parade- und
Ausgangsuniformen für den obengenannten Perso
nenkreis regelte sein Befehl Nr.29/79 vom 12.April
1979,Danach warenSoldaten imGrundwehr
dienst, Soldaten und Unteroffiziere auf Zeit und im
Reservistenwehrdienst, die in den Landstreitkräften
und den LSK/V dienten, mit einer Parade-/Aus
gangsjacke mit steingrauem Kragen, einer langen
Hose mit Schlaufen, silbergrauen Oberhemden,
einem dunkelgrauen Binder undeinemgrauen
Schal auszustatten.Ausgenommen von dieser Maß
nahme waren die Fallschirmjäger, die bereits seit
1969 Uniformen offener Fasson trugen, Für die An.
gehörigen der Landstreitkräfte wurde bestimmt
daß sie an der Parade-und Ausgangsuniform ein
heitlich Kragenspiegel und Armelpatten mit weiBer
Kantillenfüllung anzubringen hatten.
Auch die neue Parade-/Ausgangsjacke fiür Solda
ten und Unteroffiziere folgte in Schnitt und Ta
schengestaltung den bisher getragenen Uniformjak
ken.Sie war einreihig mit steingrauem Kragen und
Kragenspiegeln und auf vier Knópfen zu schlieBen
Weiterhin war sie mit vier aufgesetzten Taschen mit
Falten,geschweiften Patten undsilberfarbenen
Kn¶pfen versehen. Die Armel waren mit Aufschl¤
gen,Biesen und Armelpattengearbeitet.Inde
Schulternaht befanden sich Schnürlöcher und
Schlaufen zur Befestigung der Schulterklappen.
Die lange steingraue Hose mit Rundbund und
Koppelschlaufen verfügte üiber zwei Seitentaschen
und zwei Gesäßtaschen mit Patten.
Die silbergrauen Oberhemden waren mit zwei
aufgesetzten Taschen mit Falten und geschweiften
Patten zum Kn'pfen versehen.Mittels Schnürlö
chern und Schlaufen konnten auch an den ni-
formhemden Schulterklappen befestigt werden.
Tragebeginn für die neue Parade- und Ausgangs
uniform war der 1.Oktober 1979.
Entsprechend der 1.Durchführungsanordnung
des Stellvertreters des Ministers für Nationale Ver
teidigung und Chefs Rückw¤rtige Dienste zum Be.
fehl Nr.29/79 wurde die Ausstattung der Soldaten und Unteroffiziere bis zum 20. September 1979 ab-
geschlossen. Die alten Parade-/Ausgangsjacken mit
dem hochgeschlossenen dunklen Kragen wurden
bis zum AbschluB des Ausbildungsjahres 1982/83
zur Dienstuniform aufgetragen.
Mit der Einführung der neuen Uniformstücke
gab es nun zwei Arten der Paradeuniform, drei Ver-
sionen der Ausgangsuniform und zwei Varianten
der Dienstuniform, Die Paradeuniform für die
Winterperiode setzte sich aus Stahlhelm, Uniformmantel, neuer Parade-/Ausgangsjacke, silbergrauem
Oberhemd, dunkelgrauem Binder,gestrickten
Handschuhen, Halbschaftstiefeln und Lederkoppel
mit Schloß zusammen. Bei der Paradeuniform für
die Sommerperiode fielen Mantel und $trickhand-
schuhe weg.Zur Ausgangsuniform (Winter) gehör-
ten $chirm- bzw, Wintermütze, Uniformmantel, Pa-
rade-/Ausgangsjacke, lange Hose, silbergraues
Oberhemd mit dunkelgrauem Binder, grauer Schal,
gestrickte Handschuhe, schwarze Halbschuhe und
Lederkoppel mit Schloß, In der Sommerperiode fie-
len Mantel, Schal und Strickhandschuhe weg. Das
Lederkoppel mit Schloß wurde dann über der Jacke getragen. Die dritte Möglichkeit der Ausgangsuni-
form war die nur zum Ausgang, im Urlaub und zu
Kulturveranstaltungen zu tragende Ausfihrung
ohne Parade-/Ausgangsjacke.Dazugehörten
Schirmmütze, silbergraues Oberhemd mit ge¶ffne-
tem Kragen, lange Hose, durch deren Schlaufen am
Bund das Lederkoppel mit Schloß zu ziehen war,
und schwarze Halbschuhe. In dieser dritten Version
der Ausgangsuniform trugen die Soldaten und Un-
teroffiziere das silbergraue Oberhemd mit Schulter.
klappen.
Die Dienstuniform (Winter) setzte sich aus Feld-
bzw,Wintermütze,
Uniformmantel, Dienstjacke mit geschlossenem Kragen, langer Hose, gestrickten
Handschuhen, Halbschaftstiefeln und Gurtkoppel
zusammen. In der Sommertrageperiode wurde der
Uniformmantel jedoch nicht zur Dienstuniform ge-
tragen.
Zur Ehrenparade der NVA auf der Karl-Marx-
Allee in Berlin, Hauptstadt der DDR, zum 30. Jah-
restag der DDR stellten die Paradeeinheiten nicht
nur neue Waffensysteme wie den mittleren Panzer
T-72, die 152-mm-SFL-Haubitze und moderne Fla-
Raketenkomplexe der Truppenluftabwehr, sondern
auch ihre neue Paradeuniform der ffentlichkeit
VOr.