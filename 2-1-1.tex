

Mit Beginn der 60er ,Jahre muBten die vielen Ver-
besserungen und Anderungen an den Uniformen
und ihrer Trageweise in neue Dienstvorschriften
eingearbeitet werden.Es traten die DV-98/4. Be-
kleidungs- und Ausrüstungsnormen der Nationalen
Volksarmee mit Wirkung vom 1.September 1960
und die DV-10/5.Bekleidungsvorschrift der Natio-
nalen Volksarmee mit Wirkung vom 1. Dezember
desselben Jahres in Kraft. Es genügte nicht mehr,
die Bekleidungsvorschrift aus dem ,Jahre 1957, die
zudem noch eine <vorläufige》 Vorschrift war, wei-
ter durch Befehle und Anordnungen zu ergänzen.
In der Bekleidungsvorschrift von 1960 wurden
nun für alle Teilstreitkräfte die Trageperioden fir
Sommer- und Winteruniformen einheitlich festge-
legt: vom 1. Mai bis zum 30.September die Som-
mer-und vom 1.ktober bis zum 30.April die
Winterperiode. Die Wintermütze war in der Zeit vom 1.Dezember bis zum 28./29,Februar zu tragen.
Die Vorschrift wies weiterhin die neugeschaffe
nen Dienstgrade Unterfeldwebel bzw. Unterwacht-
meister,Stabsfeldwebel bzw.Stabswachtmeistei
und Stabsobermeister in Wort und Bild aus. Die
Entscheidung dariiber ist im Befehl Nr.34/60 des
Ministers für Nationale Verteidigung der DDR
vom April 1960 enthalten.
Die Unterfeldwebel bzw,Unterwachtmeister be
kamen ähnliche Schulterklappen wie die Unteroffi
ziere,nur war ihr Tressenbesatz, unten geschlossen
Stabsfeldwebel,Stabswachtmeister und Stabsober
meister fuhrten drei Gradsterne auf der Schulter
klappe: in der unteren Halfte zwei nebeneinander
und in der Mitte dariber einen dritten. Die Dienst
gradabzeichen in Tressenfor an Kombinationen
rainingsanzugen usw,bestandenjeweils ausde
gleichen Anzahl silberfarbener Tressen wie bei den
Unteroffizieren und Oberfeldwebeln. Angemerkt
sei, daß diese Tressen insgesamt in der neuen DV
10/5 in der Zahl je Dienstgrad gleich blicben, aber
in der Breite mit 7mm bzw,12 mm schmaler wur
den.
Die Bekleidungsvorschrift von 1960 berücksich.
tigte bereits eine Reihe in Vorbereitung befindli
cher Anderungen der Uniformen in der Armee, die
aber erst in den folgenden ,ahren mit spezifischen
Befehlen vorgeschrieben und schrittweise verwirk
licht wurden. Dazu rechneten beispielsweise dic
Vereinheitlichung der Paspelierung an den Unifor.
men der Landstreitkräfte oder die Ver¤nderung der
Kokarde an den Mützenemblemen.
Mit Befehl Nr.51/61 des Ministers für Nationale
Verteidigung der DDR vom 9,August 196l felen
die speziellen Waffenfarben der Waffengattungen,
Spezialtruppen und Dienste der Landstreitkrafte
als Paspelierungen an den Uniformjacken und -ho
sen sowie Kopfbedeckungen fort. Jetzt gab es in
den Landstreitkräften eine einheitliche weiße Pas.
pelierung.Die Waffenfarben blieben nur noch als
farbige Tuchunterlage der Schulterklappen und
stücke sowie an Kragenspiegeln und Armelpatten
erhalten, Diese Veränderungen vollzogen sich etappenweise.In den Ausbildungsjahren 1961/1962
und 1962/1963 wurden die Neueingestellten, dic
L¤ngerdienenden und die zum Unteroffizier bzw.
Offizier Ernannten mit weiß paspelierten Uniform
iacken ausgestattet bzw, ergänzt. Die bei den Trup
pen und in den Stäben vorhandenen Wniformen in
verschiedenfarbigen Paspelierungen konnten und
sollten aus konomischen Gründen aufgetragen
werden,Im Januar 1964 endete die gesamte Aktion
in den Landstreitkräften der NYA..
Diese Maßnahme zeugte von dem Bestreben, die
Unifornen weiter zuvereinheitlichen und dami
auch gleichzeitig Vereinfachungen in der Planung
der Bekleidungswirtschaft zu erreichen.
Weitere Modifizierungen an Effekten der NVA
Uniformen seien noch erwahnt. Ab 1.[uli 1962 er
hielten die Schirmmützen der Meister und Offi
ziere derolksmarine einen
Mützenkranz,de
gleich gearbeitet war wie der der Offiziere der
Landstreitkräfte,allerdings in Goldstickerei aul
dem Grundtuch in der Farbe Dunkelmarine. Fi
die Admirale gab es seit dieser Zeit Mützenkränze
in der Ausführung der Mützenkränze der Generale
der Landstreitkräfte, Wahrend die Mützenkränze
der Schirmmützen der Meister maschinengestickt
waren,bestanden die der Offiziere und Admirale
aus Handstickerei.
Um eine einheitliche Fertigung der aus Metall
geprägten Mitzenkränze sowohl in der NVA als
auch in den anderen bewaffneten Organen de
DDR zu erreichen, wurde die St¤rke der Mützen
kränze von 5 mm auf 3 mm verringert. Dadurch
konnten Effekten rationeller und damit sparsamer
produziert werden.
Noch vor Inkraftsetzen der DV-10/5 wurde ein
neues Uniformstück,das sich sehr schnell groBer
Beliebtheit erfreute,in die NVA eingeführt- dic
Uniformhemdbluse.Diese Uniforrnhemdblusen er
fuhren - wie in nachfolgenden Abschnitten noch
gezeigt werden wird-in spaterenahren cinc
Reihe von Veränderungen.AuBerdem wurde der
Kreis der Trageberechtigten im Laufe der Zeit im
mer iehr erweitert.