

Der Tag der Einberufung der ersten Wehrpflichti-
gen, der 4.April 1962, wird in dem Buch «Armee
für Frieden und Sozialismus, Geschichte der Natio-
nalen Volksarmee der DDR» folgendermaBen ge-
schildert:«Wie im Truppenteil <Anton Saefkow
verlief für die Wehrpflichtigen überall in der NVA der erste Tag, <Genosse>, lautete die füir die meisten
ungewohnte Anrede. Die Vorgesetzten stellten sich
vor. Die Wehrpflichtigen empfingen Uniform und
Was später selbstver
Ausrüstungsgegenstände.
ständlich wurde,schien anfangs unmöglich:Wic
sollte der ganze Berg von Uniformteilen, die Stiefel,
das Sportzeug, die Unterwäsche, das Kochgeschirr
und anderes mehr in diesen engen Spind passen?
Doch beim Einräumen wie beim erstenBetten
bauen> waren ihnen die Gruppenführer behilflich.
Einige Wehrpflichtige muBten noch den Gang zum
Friseur antreten, denn nicht jeder Haarschnitt, der
gerade ‹in Mode> war, erwies sich für den Armee.
dienst als zweckmäBig,Auch die Uneinsichtigen
verstanden spätestens nach der ersten Ausbildungs
stunde mit der Schutzmaske den militarischen Sinn
dieser Festlegung.»
Die Einberufung der jungen Männer zu ihrem
18monatigen Ehrendienst zum Schutz des Friedens
und des Sozialismus erfolgte auf der Grundlage des
Gesetzes über die allgemeine Wehrpflicht. Es war
von den Abgeordneten der obersten olksvertre.
tung der DDR am 24.[anuar 1962 einmütig verab.
schiedet worden,Gewichtige Grüinde sprachen für
diese einschneidende Maßnahme. Seit Einführung
des Freiwilligenprinzips im Jahre 1956 war zwar
eine zuverlässige und kampfstarke Armee aufgebaut
worden, doch hatten sich Ende der 50er/Anfang
der 6Oer Jahre die Bedingungen für den bewaffne.
ten Schutz des Sozialismus grundlegend verändert
Die USA hatten die NATO zum Hauptinstrument
ihrer Globalstrategie in Europa und mit Hilfe ultra.
reaktionärer Politiker der BRD die Bundeswehr zur
Speerspitze dieser Strategie gemacht. Die Siche.
rungsmaBnahmen der DDR vom 13.August 1961
hatten - abgestimmt mit den Teilnehmerstaaten
der sozialistischen ilit¤rkoalition - dem Imperia-
lismus der BRD den Weg versperrt, seinen Herr
schaftsbereich nachOsten auszudehnen.Doch der
NATO-Pakt steigerte seine Bereitschaft und Fahig
keit,seine antisozialistische Politik mit militäri
schen Mitteln fortzusetzen.Die vom Imperialismus
ernsthaft bedrohtenErrungenschaften mehr als zehnjähriger Aufbauarbeit der Bürger zu schützen
wurde zur Pflicht aller Wehrdienstfähigen.
Weitere wichtige Gründe für die allgemeine
Wehrpflicht ergaben sich aus den Veränderungen
im Militärwesen selbst. Der von der NATO gegen
die Staaten des Warschauer Vertrages geplante
Krieg hätte zum massenhaften Einsatz moderner
und komplizierter Technik und Ausristung ge-
fihrt, Zu ihrer Beherrschung mußte die Armee auf
eine groBe Anzahl militärisch gut ausgebildeter
Bürger zurückgrcifen können. Nur durch die allge
meine Wehrpflicht konnte die NVA entsprechend diesen neuen Bedingungen kontinuierlich und qua-
lifikationsgerecht aufgefüllt werden.
Mit dem Wehrpflichtgesetz waren eine Reihe er-
gänzender gesetzlicher Bestimmungen verabschie-
det worden: die Dienstlaufbahnordnung mit dem
Fahneneid durch Erlaß des Staatsrates der DDR;
die Erfassungs-, die Musterungs- und die Reservi-
stenordnung durch Anordnung des am 10. Februar
1960 geschaffenen Nationalen Verteidigungsrates
der DDR; die Besoldungs-, Unterhalts- und Förde-
rungsverordnung durch Verordnung des Minister-
rates der DDR.
Vielfaltige Vorbereitungen auf das Eintreffen der
ersten Wehrpflichtigen und ihre bevorstehende
Ausbildung bestimmten im Frihjahr 1962 die Tä-
tigkeiten aller Vorgesetzten in den Stäben, Verbänden, Truppenteilen und Einheiten, Die Einführung
der allgemeinen Wehrpflicht erforderte auch von
den AngehÃörigen des B/A-Dienstes, ihre Aufgaben
hinsichtlich der Sicherstellung und der Versorgung
der Truppen mit Uniformen und Ausrüstung noch
exakter und straffer zu lösen, denn die Einkleidung
zum Wehrdienst einberufener junger änner ver-
mittelte die ersten Eindrücke vom künftigen Dienst
und Leben in der Armee, Aus diesem Grund war
die Vorbereitung, Durchführung und Nachberei-
tung der Einkleidung eine wichtige politische, öko-
nomische und organisatorische Aufgabe des Ver.
antwortlichen des B/A-Dienstes und der Vorgesetz-
ten aller Stufen in den Truppenteilen. Dies gilt
auch heute noch. So können sich die Organisation
des Ablaufes sowie das Anpassen der Bekleidung,
insbesondere des Schuhwerks, positiv oder auch negativ auf die Bereitschaft zum Ehrendienst in der
Armee auswirken. Auch wird bereits bei der Ein.
kleidung die Einstellung des Armeeangehörigen zu
der ihm übergebenen Bekleidung und Ausrüstung
geprägt.
Notwendige Festlegungen der Einkleidung der
Wehrpflichtigen traf der Stellvertreter des Mini-
sters für Nationale Verteidigung der DDR und
Chef Rückwärtige Dienste der NVA in seiner An-
ordnung Nr.7/62 vom 31.März 1962. Danach gal.
ten die Grunds¤tze der DV-98/4. Bekleidungs- und
Ausr¼stungsnorm der Nationalen Volksarmee - ge
ring erweitert - auch füir die zum Grundwehrdienst
einberufenen jungen Männer, So erhielten die Sol-
daten und Flieger für die Dauer ihres gesamten Grundwehrdienstes die in der obigen Tabelle erfaß-
ten Bekleidungsstücke.
Auch Kampfanzug, Sturmgepäck und Sonderbe-
kleidung (nach Notwendigkeit) wurden gemäß der
genannten Dienstvorschrift ausgegeben.Fiir die
Matrosen der Volksmarine, die für den Grundwehr-
dienst in die Landeinheiten einberufen wurden, gal-
ten analoge Festlegungen für ihre speziellen Uni-
formstücke. Auch die Bestimmungen für die
Tragezeiten der Uniformen blieben insgesamt un-
verändert, Soldaten auf Zeit erhielten nach Ablei-
stung ihres Grundwehrdienstes, dann zum [nterof-
fizier bzw,zum Maat befördert, neue Uniform-
stücke: eine Schirm- oder Tellermütze, eine Parade-
und Ausgangs-bzw,Dienstuniform und ein Paar
Halbschaftstiefel.
In jenen ,ahren erfolgte der Dienstsport noch in
der Dienstbekleidung. Deshalb muBten die Wehr
pflichtigen ihre Sportbekleidung für den Massen-
und Freizeitsport zunächst noch selbst mitbringen.
Dies traf auch auf Schlafanzüge oder Nachthem-
den, Taschentücher sowie Toilettenartikel und
Schuhputzzeug zu.

Die Anordnung Generalmajor W.Allensteins sah
ferner eine Reihe von Maßnahmen zur Pflege und
Erhaltung der Uniformen im Interesse der sparsa-
men Verwendung volkswirtschaftlicher Mittel vor.
Dementsprechend richteten die Kommandeure und
insbesondere die Hauptfeldwebel der Einheiten
ihre Anstrengungen auf die konsequente Durchset-
zung der Anzugsordnung, wie sie die DV-10/5 vom
Dezember 1960 vorschricb, und auf die unbedingte
Einhaltung der Putz- und Flickstunde im Tages-
dienstablaufplan sowie auf periodische Kontrollen
und Vollz¤hligkeitsappelle der Bekleidung und
Ausrüstung.
Weitere Festlegungen betrafen Maßnahmen, alle
nach der Entlassung der Wehrpflichtigen anfallen-
den Bekleidungsstücke sofort zu reinigen, zu wa-
schen und instand zu setzen,um sie als Dienst-
bzw. Arbeitsbekleidung schnell in die Versorgung
der Neueinberufenen einbeziehen zu k¶nnen,
Andere Regelungen richteten sich auf die paßge-
rechte Einkleidung der wehrpflichtigen Soldaten,
Flieger und Matrosen. Sie sahen u,a. vor, den Stahl-
helm nur in Verbindung mit der aufgesetzten Schutzmaske anzupassen, die Uniformjacken über
die Unterbekleidung und mit dem Pullover anzu-
probieren und das Schuhwerk nach festgelegten
Kriterien sorgfaltig auszuwählen. Zur Durchfüh-
rung dieser MaBnahmen wurden - und werden
auch heute noch- die Schneider und Schuhma-
cher der Truppenteile und Verbände eingesetzt.