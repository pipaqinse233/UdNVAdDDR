

In den ersten Jahren des Bestehens der NVA wur-
den durch die Staatsführung der DDR auch solche
politischen Entscheidungen getroffen, die sich auf
die Uniformierung auswirkten, So war das Gesetz
zur Ãnderung des Gesetzes iiber das Staatswappen
und die Staatsflagge der DDR, das Wilhelm Pieck
am 1.Oktober 1959 unterzeichnete, sehr bedeut-
sam, Bis dahin bestand die Staatsflagge nur aus den
Farben Schwarz,Rot,Gold.Aber von nun an
fihrte sie auf beiden Seiten in der Mitte das Staats-
wappen der DDR - Hammer und Zirkel, umgeben
von einem Ãhrenkranz,der im unteren Teil von
einem schwarz-rot-goldenen Band umschlungen ist.
Dieses Symbol brachte den Charakter der DDR
als sozialistischen Staat der verbündeten Arbeiter und Bauern äuSerlich eindrucksvoll zum Ausdruck,
In der NVA wie auch in den anderen bewaffne.
ten Organen der Republik wurde sodanndic
schwarz-rot-goldene Kokarde im Mützenemblem
gegen die mit dem Staatsemblem ausgewechselt.
Unter Beibehaltung des Mützenkranzes - aus Spar
samkeitsgr¼nden -entstand somit ein fǔr alle be
waffneten Kräfte der DDR einheitliches Emblem.
In der NVA wurden die Kokarden auf der Grund.
lage des Befehls Nr.51/61 des Ministers für Natio.
nale Verteidigung der DDR, Armeegeneral H.Hoff-
mann,vom 9,August 1961 bis Mitte des ahres
1962 ausgetauscht.
Das auf der linken Seite des Stahlhelms befindli-
che schwarz-rot-goldene Emblem in Gestalt eines
Wappens wurde bereits Mitte April 1960 entfernt.
Vor allem aus Gründen der Tarnung wurde auf ein
neues Wappen am Helm verzichtet.
Am 3.November 1960 verlieh der Minister für
Nationale Verteidigung der DDR-seit dem
14.[uli 1960 übte Generaloberst H.Hoffmann diese
Funktion aus - den Seestreitkräften der NVA auf
BeschluB des Nationalen Verteidigungsrates der
DDR vom 19.Oktober des Jahres anläBlich des
42.|ahrestages des Aufstandes der Kieler Matrosen
den ehrenvollen undverpflichtendenNamen
«Volksmarine».
Alle Schiffe und Boote setzten als Repräsentan
ten des Arbeiter-und-Bauern-Staates die neue rote
Dienstflagge mit dem schwarz-rot-goldenen Strei
fen und dem Staatsemblem der DDR. Zur Namens
verleihung fanden auf dem Greifswalder Bodden
in Anwesenheit ehemaliger Angehöriger der Volks
marinedivision und von Teilnehmern an revolutio.
n¤ren Kämpfen der deutschen Arbeiterklasse - ein
milit¤risches Zeremoniell und eine Flottenparade
von Kampfschiffen und -booten statt. General
oberst H.Hoffmann forderte von den Angehörigen der Volksmarine: «Erweisen Sie sich der großen Eh-
rung würdig, die lhnen durch die Auszeichnung
mit dem Ehrennamen <Volksmarine> und der Ver-
leihung der roten Dienstflagge zuteil wurde. Schüt-
zen und hÃüten Sie 【hren Ehrennamen und die Ehre
[hrer Flagge.»
Die Namensverleihung an die Seestreitkräfte der
NVA spiegelte sich auch in einem Detail der Uni-
form wider: Die Matrosen und Maate wechselten
ihre Mitzenb¤nder, Sie führten nunmehr solche
mit der gestickten Aufschrift «Volksmarine》.

Mit diesen Veränderungen endete ein erster gro-
Ber Abschnitt der Uniformentwicklung der NVA.
Teilweise reichten diese Veränderungen und Neu-
einführungen noch bis in die 60er Jahre hinein, Das
Ehrenkleid der Nationalen Volksarmee war in den
50er Jahren sehr rasch zu einem gewohnten An-
blick in der Öffentlichkeit geworden, auch bei
Ãbungen, im Ausgang und im Urlaub sowie auf Pa-
raden, die damals am 1. Mai in nahezu allen Be-
zirksstädten der DDR durchgefuhrt wurden.