

Auch in der ersten H¤lfte der siebziger ,Jahre ver-
spürten die Armeeangeh¶rigen eine kontinuierliche
Verbesserung ihrer Bekleidung und Ausrüstung
und eine bedeutende Erhöhung der Ausstattungs.
normen.
Ab Juli 1973 erhielten Berufssoldaten statt einer
zwei Wintermützen und analog zwei Feldmiützen.
Die Normausstattung bei Stiefelhosen wurde von
zwei 'auf drei, bei genarbten Schaftstiefeln von
einem auf zwei Paar erhöht. Die Bekleidung der
weiblichen Armeeangeh¶rigen wurde mit der Ein-
führung eines weißen Pullovers, einer langen Hose
einer Weste, Schaftstiefeln mit Reißverschluß und
Slingpumps (Tragebeginn .Januar 1974) verbessert.
Die Berufssoldaten wurden mit einer Uniform-
jacke offener Fasson, einem weißen und einem sil.
bergrauen Oberhemd, zwei dunkelgrauen Bindern
und einem hellgrauen Schal ausgestattet.
Seit der Neueinberufung im Mai 1975 brauchen
Soldaten im Grundwehrdienst sowie Soldaten und
Unteroffiziere auf Zeit keine eigenen schwarzen Halbschuhe für den Ausgang mehr mitzubringen.
Schwarze Halbschuhe wurden in die Grundnorm
aufgenommen, bei der Ersteinkleidung ausgegeben
und nach Ablauf von 18 Monaten bei Soldaten und
Unteroffizieren auf Zeit kostenlos ergänzt. Diese
Halbschuhe wurden bei der Entlassung aus dem ak-
tiven Wehrdienst bzw. bei der Erg¤nzung dem Trä-
ger übereignet. Durch diese Anordnung wurde
auch festgelegt, daB die Halbschuhe auch immer
dann zu tragen waren, wenn zur Felddienstuniform
bzw, zur Dienstuniform Schnürschuhe befohlen wa-
ren. Schnürschuhe geh¶rten nicht mehr zur Grund-
ausstattungsnorm.
Ab Mai 1975 erhielten Unteroffiziere auf Zeit und
Berufssoldaten Trainingsanzug, Sporthose, Sport-
hemd sowie Lederturnschuhe kostenlos ergänzt.
All diese - nicht immer auf den ersten Blick
sichtbaren -Veränderungen an den Uniformen
und bei der Ausstattung der Berufssoldaten, der
Soldaten und Unteroffiziere auf Zeit sowie der Sol-
daten im Grundwehrdienst waren auf die Möglich-
keit einer zweckmÃäßigeren Ausübung ihres militäri-
schen Berufs sowie auf eine Verbesserung der
Dienst- und Lebensbedingungen gerichtet.