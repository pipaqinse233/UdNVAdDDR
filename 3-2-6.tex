

Am 26.Mai 1977 hatte der Minister für Nationale
Verteidigung in seinem Befehl Nr. 93/77 die Wei-
sung erteilt, zur Durchsetzung der neuen Straßenverkehrsordnung eine neue StraBenkommandan
tendienstordnung zu erarbeiten, Bestandteil dieser
neuen Straßenkommandantendienstordnung waren
Festlegungen zur Ausrüstung und Kennzeichnung
der Regulierer. In Abänderung der DV 010/0/005
wurde bestimmt, daß Regulierer den weißen Fall.
schirmjägerhelm mit einem vorn unterbrochenen
und an den Enden abgerundeten roten Streifen auf
setzen.Vorn in der Mitte des Helms befand sich
nicht mehr das «KD-Kennzeichen》, sondern ein
rundes Hoheitsabzeichen der DDR. Die Breite des
roten Streifens betrug 40mm, der Abstand des
Streifens vom Helmrand 15 mm und vom Hoheits-
abzeichen ebenfalls 15 mm. Der Durchmesser der
unterlegten Farben Schwarz, Rot, Gold war mit
60 mm und der des Staatswappens der DDR mit
30 mm bemessen. NichtstrukturmäBige Regulierer,
so legte es die neue Straßenkommandantendienst-
ordnung fest, zogen üiber den Stahlhelm einen wei.
ßen Helmüberzug mit gleicher Kennzeichnung wie
die strukturmäßigen Regulierer am Fallschirmj¤ger-
helm. Sie waren außerdem mit weißen Ärmelstul.
pen,einem Signalstab und Brust- und Rückenre.
flektoren ausgestattet, Wahrend ihres Einsatzes
trugen sie je nach Jahreszeit den Felddienstanzug
Sommer oder Winter. Strukturmäßige Regulierer
waren mit zweiteiligen Anzigen ausschwarzem
Lederol, die wie Felddienstanzüge geschnitten wa
ren und nach den gleichen Bestimmungen getragen
wurden, ausgestattet.Vervollständigt wurde ihre
Ausrüstung durch einen weißen Schulterriemen
und einen Blinkgürtel.