

Die vielfaltigen Verbesserungen und Ãnderungen
der NVA-Uniformen in den erstenJahren des
neuen Jahrzehnts bezogen auch die Bekleidung der
weiblichen Armeeangeh¶rigen ein, Bis zum 1. März
1962 sollten die Frauen mit einer neuen Ausgangs-
uniform und ab 1.Juli desselben Jahres mit Dienst-
uniformen in neuer Schnittgestaltung versorgt wer-
den, Diese terminlichen Regelungen traf der Chef
Rückwärtige Dienste der NVA, Generalmajor
W.Allenstein, seit 25. September 1961 auch Stell-
vertreter des Ministers für Nationale Verteidigung
der DDR,in seiner Anordnung Nr.7/61 vom
27.Oktober 1961. Die neuen Uniformen wurden
1962 ausgegeben.Bis zu dieser Zeit trugen die
Frauen die vorhandenen Uniformen noch auf.
Die genannte Anordnung regelte gleichzeitig die
Beschaffenheit und die Schnittgestaltung der neuen
Uniformen.Nunmehr wurden auch die Uniform-
stücke der weiblichen Soldaten, Flieger und Matro-
sen sowie der Unteroffiziere und Maate aus Kamm-
garnstoff mageschneidert, Anstelle der bisher getragenen Baskenmützen geh¶rte jetzt ein Schiff-
chen zur Dienstunifor und cine Kappe zur Aus-
gangsuniform -in Steingrau bei den Landstreit-
kräften und den LSK/LV bzw, in Wei8 oder Blau
bei der Volksmarine gehalten. Die Uniformjacke
für die Dienst- und die Ausgangsuniform war nur
rechts mit einer Innentasche versehen, Die beim al-
ten Uniformschnitt vorher außen befindlichen zwei
Taschen entfielen. Die Ãrmel der Uniformjacke wa-
ren ohne Aufschläge gearbeitet, Auf der Ausgangs-
uniform der Landstreitkräfte und der der LSK/LV
befanden sich die Ãrmelpatten in entsprechenden
Waffenfarben. In der Sommertrageperiode war es
gestattet, den Kragen der Uniformhemdbluse bei beiden Uniformarten über den Jackenkragen zu le-
gen.Die Hemdbluse konnte auch- dann mit
Schulterklappen oder -stücken- ohne die Uni-
formjacke getragen werden, In allen drei Teilstreit-
kräften zogen die Frauen zu beiden Uniformarten
schwarze Halbschuhe an.
Die obige Tabelle gibt Aufschluß über die verän-
derten Bekleidungs- und Ausrüstungsstüicke, die
die Frauen ab 1962 für einige Jahre laut der Aus-
stattungsnorm der DV-98/4 besaBen.
Alle übrigen Uniformstücke, z. B. der Uniform-
mantel und die Schaftstiefel, blieben in Schnitt und
Form unver¤ndert und in gleicher Anzahl im Besitz
der weiblichen Armeeangehörigen.