

Die Uniformvorschriften von 1960 und 1965 wie
sen gegeniiber der <Vorl¤ufigen Bekleidungsvor
schrift der Nationalen Volksarinee》 aus dem Jahre
1957 einige bedeutsame Änderungen auf. Es ging
vor allem darum, die in den |ahren jeweils neu hin.
zukommenden Abzeichen und Auszeichnungen in
das bestehende System der Trageweisen an der Uni
form einzuordnen.
Anfang der 60er Jahre betraf dies u.a. die Absol.
ventenabzeichen derilitarakademie xFriedrich
Engels» und der Milit¤rmedizinischen Sektion. Sie
muBten laut der DV-10/5 vom 1.Dezember 1960
genau 1 cm über der Mitte der rechten Brusttasche
der Uniformjacke angebracht werden. Die Klassifi
zierungsabzeichen, die es seit 1959 gab, sollten
ebenfalls an dieser Stelle,aber 5mm über der
Brusttasche, befestigt werden, In diesem Fall war
das Abzeichen der Militärakademie, wenn der Tr¤
ger cin Klassifizierungsabzeichen erworben hatte
über diesem zu tragen.
Umfassend wurde erst in der Vorschrift von 1965
die Trageweise des Bestenabzeichens der NVA be.
rücksichtigt, Die Soldaten, Unteroffiziere und Off.
ziersschüler trugen es über der rechten Brusttasche
bzw.an Uniformijacken ohne diese Taschen auf
gleicher Höhe,Wenn notwendig, wurde diese Aus
zeichnung auch iber demKlassifizierungsabzei
chen an der Uniform befestigt. Mehr als 3 Besten
abzeichen durften in dieser Zeit,als es noch keine
Anhänger fiir mehrmalige Verleihung gab, nicht
angesteckt werden.
Diese Auszeichnung wurde auf der Grundlage
der Bestenordnung des Ministers für Nationale Verteidigung der DDR vom 23.Marz 1964 am
7.Oktober des Jahres erstmalig an Armeeangehö.
rige iiberreicht. Sie erhielten das Bestenabzeichen
für gute Leistungen in der politischen Schulung
und in der militärischen Ausbildung. Sie muBten
ihre SchieBübungen und die Normen der milit¤ri-
schen Körperertüchtigung,der Pionier-und der
Schutzausbildung mindestens mit der Note xgutx
erfiillen, die ihnen anvertraute Waffe, Technik, Be-
kleidung und Ausrüstung sorgsam pflegen und
st¤ndig einsatzbereit halten sowie alle Forderungen
der Dienstvorschriften vorbildlich erfillen.
Auch bei den Orden und Medaillen galt es, die
Vorschrift von 1965 erg¤nzend,neue Auszeichnun.
gen einzuordnen.Am 17.Februar 1966 stiftete der
Ministerrat der DDR den Scharnhorst-Orden als
höchste militärische Auszeichnung der Republik
den Kampforden «Für Verdienste um Volk und Va.
terland》 und die Medaille der Waffenbrider
schaft - die beiden letztgenannten jeweils in
Bronze.Silber und Gold.
Der Scharnhorst-Orden wurde in der Uniform-
vorschrift in der Rangfolge nach dem Karl-Marx-
Orden, dem Ehrentitel «Held der Arbeit» und dem
Vaterländischen Verdienstorden eingereiht. Da die
3 höchsten Auszeichnungen stets über allen Orden,
Ehrentiteln und Medaillen an der Uniform getra-
gen wurden,kam der Scharnhorst-Orden an die er
ste Stelle einer Auszeichnungsreihe auf die linke
Seite der Uniformiacke. Ihm folgte der Kampfor.
den. Die Medaille der Waffenbrüderschaft war
nach der Verdienstmedaille der NVA einzuordnen,
Wurden mehr als 2 rden oder Medaillen am
Band angebracht, durften sie seit 196l die Breite
von 12cm nicht mehr überschreiten, vorher waren
es 14cm.Nach wie vor konnten Auszeichnungen.
ob Orden und Medaillen am Band oder als Inte
rimsspange,nur an der geschlossenen,an der offe.
nen einreihigen oder der zweireihigen niform
jacke sowie der Uniformjacke der weiblichen
Armeeangeh¶rigen getragen werden. Nicht gestattet
war es,an Uniformmänteln und an den Uniform.
hemdblusen Auszeichnungen zu fiihren.