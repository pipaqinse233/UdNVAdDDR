

Die Frauen erhielten 1983 neue Felddienstanzüge
für den Sommer und für das Winterhalbjahr, eine
neue Wintermütze und Uniformkleider. Anstelle
des steingrauen Schiffchens trugen weibliche Ar-
meeangehörige nun zum Felddienstanzug eine dun-
kelgraue, Angeh¶rige der Volksmarine eine dunkel-
blaue Baskenmütze. Die Felddienstuniform der
weiblichen Armeeangehörigen für den Sommer
setzte sich aus dem Felddienstanzug im Strichel-
druck, wie ihn auch die männlichen Armeeangehö-
rigen trugen, Schaftstiefeln und Gurtkoppel zusam-
men. In den bergangsperioden war es nun
m¶glich, entweder Uniformjacke und hose unter-
zuziehen oder auf Befehl den wattierten Felddienst-
anzug ohne Webpelzkragen und die Lederhand
schuhe zu tragen. Auch der wattierte Felddienstan-
zug entsprach in Formgebung und Wattierung den
Felddienstanzügen der männlichen Armeeangehöri-
gen.Weibliche Armeeangehörige trugen ab sofort
die Hose des Felddienstanzuges über den Stiefeln
und stellten den Saumbund mittels der angebrach-
ten Knöpfe so ein, daß die Hosenbeine eng an den
Stiefeln anlagen, Die neue Wintermütze war aus
steingrauem, bei der Volksmarine dunkelblauem
Uniformtuch mit $chirm und Webpelzbesatz in Sil-
bergrau, bei der Volksmarine in Schwarz gearbeitet.

Zur Dienstuniform zogen die weiblichen Armee-
angehörigen einen modisch aktuellen, leicht ausge-
stellten Rock an, bei dem die zwei vorderen Tei-
lungsnähte in Falten ausliefen. Dazu konnte
entweder die einreihige, auf drei Kn¶pfen zu schlie-
Bende Uniformjacke steigender Fasson mit schrä-
gen Seitentaschen und Patten oder die neue silber-
graue Hemdbluse mit oder ohne Bund, durchgehender Knopfleiste, zwei aufgesetzten Brusttaschen
mit Patten und je zwei Kn¶pfen, langen Ärmeln mit
Schlaufen und Knöpfen zum Hochschlagen getra-
gen werden. Die Uniformart Stabsdienstuniform
war am mannigfaltigsten, Hier konnten entspre-
chend der jeweiligen Trageperiode 18 verschiedene,
zur Uniform gehörige Bekleidungsstücke kombi-
niert werden, Es gab drei Uniformierungsvarianten
für den Sommer, zwei für die Ãbergangsperiode
und zwei für die Winterperiode.
Zur Stabsdienstuniform(Sommer)und in den
Ãbergangsperioden gehörte die graue bzw. blaue
Kappe, im Winter die Wintermütze aus Webpelz. Für hochsommerliche Temperaturen erwies sich
das neue hellgraue Uniformkleid aus Seidengestrick
als sehr zweckmäßig.Es bestand aus einem Oberteil
mit durchgehender Knopfleiste,zwei aufgesetzten
Taschen ohne Falte,aber mit Patten und je zwei
Kn¶pfen,einem leicht ausgestellten Rockteil mit
vorderer Kellerfalte,verschließbarem Kragen ohne
Kragenspiegel,einemGirtel aus dem Obermate
rial, kurzen Armeln und imitierten Aufschlägen mit
Knopfschlaufen.Hinzu kamen Schulterklappen
bzw.-st¼cke.Das Kleid konnte durch ein silber
graues oder dunkelblaues Halstuch erg¤nzt werden.
Sollte derSommermantelangezogenwerden,
muBte der Kragen geschlossen und ein Binder ge-
nommen werden.
An kihleren Tagen konnten Uniformjacke uná
Uniformrock, an besonders kühlen Tagen der Pul
lover über der silbergrauen Hemdbluse mit Binder
getragen werden,' Die dritte Möglichkeit für den
Sommer bestand aus dem Uniformrock und der sil.
bergrauen Hemdbluse mit offenem Kragen, Schul.
terstücken oder Schulterklappen.
In den bergangsperioden konnte zur Stabs
dienstuniform wahlweise der Sommer-oderder
Uniformmantel angezogen werden. Es bestand auch
die öglichkeit,darunter ein ebenfalls neues Uni-
formkleid zu tragen. Dieses für die Übergangs- und
Winterperiode vorgesehene Kleid war aus stein
grauem oder dunkelblauemSeidengestrick.In
Form und Ausführung glich es dem silbergrauen
Uniformkleid, die rmel waren jedoch lang und
mit Schlaufen und Knöpfen zumHochschlagen
versehen.Ein silbergraues Halstuch oder ein Bin
der komplettierte das Uniformkleid. Dazu kamen
wahlweise die schwarzen Halbschuhe oder die
Schaftstiefel mit ReiBverschluB.
Eine zweite Variante der Stabsdienstuniform in
der Übergangsperiode setzte sich aus Uniformrock,
Uniformjacke, silbergrauer Hemdbluse, Binder, bei
Bedarf Pullover und ie nach Temperatur Halbschu.
hen oder Schaftstiefeln mit Reißverschluß zusam
men, An die Stelle des Uniformrockes konnte auch
die Uniformhose treten,Zur Winteruniform geh'rten in jedem Fall Wintermütze und Uniformmantel
sowie Lederhandschuhe. Die Frauen konnten zwi-
schen Uniformkleid, -hose oder -rock w¤hlen. An-
stelle der Uniformjacke war es möglich, innerhalb
von Geb¤uden auch die Uniformweste zu tragen.
Mit der 1983er Bekleidungsvorschrift wurde für
alle Armeeangehörigen die Arbeitsuniform einge.
führt. Weibliche Armeeangehörige trugen dazu
stets die Baskenmütze, auch wenn der Arbeitsanzug wattiert befohlen war,das Arbeitshemd, in der
Übergangsperiode bei Notwendigkeit, im Winter
immer Lederhandschuhe und Schaftstiefel
Zur Ausgangsuniform waren den Frauen Uni
formrock und -jacke, nicht aber Uniformkleider
oder -hose gestattet.Im Sommer war es moglich
entweder Uniformrock und -jacke oder niform
rock und weiße Hemdbluse anzuziehen. Im ersten
Fall wurde der Blusenkragen geschlossen und der
Binder umgebunden,im zweiten der Kragen geöff
net, aber Schulterklappen oder -stücke aufgekn¶pft
Dazu durfte der Sommermantel genommen wer
den. Füir den Sommer und die Ubergangszeiten war
die Kappe obligatorisch.In der Ubergangsperiode
konnte entweder der Uniform-oder der Sommer.
mantel angezogen werden,Im Winter sorgten zu
sätzlich Wintermütze,Uniformmantel und Leder
handschuhe für ausreichenden Wetterschutz.
Weibliche Offziere wählten in Fällen, wo die
Gesellschaftsuniform getragen wurde,im Sommer
und in der Ubergangszeit zwischen je zwei Varian-
ten, im Winter gab es nur eine. Je nach Anlaß und
Außentemperatur war die Gesellschaftsuniform der
weiblichen Offiziere im Sommer entweder mit Ge
sellschaftsjacke als GroBer oder nur mit weißer
Hemdbluse und Schulterstiücken als Kleiner Gesell.
schaftsanzug zu tragen.In denbergangszeiten
konnte entweder der Sommer-oder der Uniform
mantel angezogen werden.Darunter wurden immei
Uniformrock,Gesellschaftsjacke sowie weiße Uni.
formbluse mit Binder getragen. Beim Großen Ge-
sellschaftsanzug wurden die Orden am Band, sonst
nur die Interimsspangen angelegt. Im Sommer und
in den Übergangszeiten geh¶rten zur Gesellschafts
uniform die Kappe,im Winter die Wintermütze
und dazu auch der Uniformmantel und Lederhand.
schuhe.Alle Varianten der Gesellschaftsuniform
fiir weibliche Offiziere wurden durch schwarze
Halbschuhe vervollständigt.
Insgesamt war den weiblichen Angehörigen der
NVA eine auBerordentlich groBe Vielfalt an Uni.
formarten und vor allem Kombinationsmöglichkei.
ten innerhalb dieser niformarten geboten.