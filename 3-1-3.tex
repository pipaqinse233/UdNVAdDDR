

Frauen mit Soldaten-, Unteroffiziers- und Offiziers.
dienstgraden versahen in allen Teilstreitkräften der
NVA ihren verantwortungsvollen Dienst, Die breite
Palette ihrer Einsatzm¶glichkeiten, z. B. als Funker,
Zeichner oder im Wetterdienst, erforderten gerade
für sie eine Uniformierung, in der sich militärische
Zweckmäßigkeit, groBe Kombinierfahigkeit, Varia-
bilität und Modernität vereinten, Im Rahmen der
schrittweisen Verbesserung der Dienst- und Le-
bensbedingungen für alle Armeeangehörigen, aber auch zur Erhöhung der Attraktivität eines militäri.
schen Berufes für interessierte Madchen und
Frauen leitete die Anordnung Nr. 21/73 des Stell-
vertreters des Ministers für Nationale Verteidigung
und Chefs Rückwärtige Dienste vom 4. September
1973 weitere Maßnahmen zur besseren Versorgung
der weiblichen Armeeangehörigen mit Bekleidung
und Ausrüstung ein, Diese gravierenden Verände-
rungen in der Uniformierung der weiblichen Ar
meeangehörigen wurden mit der Berichtigung Nr. 1
zur DV010/0/005,die am 1.Dezember 1973 in
Kraft trat, Bestandteil der gültigen Uniformvor-
schrift der NVA. Neu für weibliche Armeeangehö-
rige war die Stabsdienstuniform, die sehr unter-
schiedlichzusammengestelltg
getragen
werden
konnte.
Mit der Einführung einer Weste, einer langen
Hose, weißen Pullovern mit Roll- bzw. R¶merkragen,von Schaftstiefeln mit ReiBverschluB und
schwarzen Slingpumps sowie der Modernisierung
des Schnitts der Uniformmäntel und -jacken, der
weiBen Hemdblusen,der Felddienstjacke,der
Strickiacke sowie des Schuhwerks ¤nderte sich das
äuBere Erscheinungsbild der Frauen in Uniform
sehr zu ihrem Vorteil. Der Uniformrock war nun
modisch aktuell kniefrei.Der Uniformmanteler
hielt durch die Einarbeitung von Teilungsnähten in
die Vorderteile, von der Schulternaht beginnend bis
zum Tascheneingriff,eine verbesserte anatomiege
rechte Paßform.Er wurde kniebedeckend getragen,
Die Uniformjacke war einreihig, mit vier Schließ.
kn'pfen,in offener Fasson gearbeitet. Die Vorder.
teile waren mit Teilungsnähten von der Schulter
naht bis zum Saum und ohne AuBentaschen
gestaltet.
Besondere Zustimmung fanden die neuen We
sten.Einreihig mit vier Knöpfen zum Schließen
und ohne Kragen gearbeitet,erhielten sie durch
zwei in die vorderen Teilungsnähte eingearbeitete
Taschen modischen Chic.Auf die Westen wurden
Schulterklappen bzw.-stüicke aufgekn¶pft.
Zur Stabsdienstuniform(Winter)gehörte die
neue Rundbundhose mit Bundverl¤ngerung, Reiß
verschluß und eingearbeiteter Tasche an der linken
Seite sowie modisch ausgestelltem Hosenbein.
Die Felddienstjacke war in der neuen Ausfüh
rung ohne Reißverschlüsse,mit verdeckter Knopf
leiste, zwei Seitentaschen, aufknöpfbarem Webpelz.
kragen, einknöpfbarem Webpelzfutter und Strick
¼ndchen gearbeitet. Die weiBen Hemdblusen
erhielten nunmehr silberfarbene Knöpfe,bei der
Volksmarine wurden auf der weißen und der sil.
bergrauen Hemdbluse goldfarbene Knöpfe getra.
gen.
Die neuen weißen Pullover in den Ausführungen
mit langem Armel und Rollkragen sowie kurzem
Armel und Römerkragen bestanden zu 100% aus
tragefreundlichen texturierten Polyesterfasern. Die
einreihige, mit f¼nf Knöpfen zu schlieBende Strick
acke mit spitzem Ausschnitt,langem Armel und
breitem Bund war aus 100% Wolpryla gefertigt.

Militärische Zweckmäßigkeit und dem unterge-
ordnet auch modische Attribute bestimmten Ausse-
hen und Verarbeitung des Schuhwerks der weibli-
chen Armeeangehörigen, Die Schaftstiefel mit
seitlich verdecktem ReiBverschluß und geklebter
Formgummisohle entsprachen dem Zeitgeschmack
durch einen 4 cm hohen Blockabsatz. Halbschuhe,
Pumps und Sporta durften nur noch in der Farbe
Schwarz getragen werden. Die neuen schwarzen
Slingpumps waren ohne Zierelemente, fersenfrei
mit VerschluBriemen und ebenfalls 4cm hohem
Blockabsatz gearbeitet.
Die vorschriftsmäBige Kombination der Uniformstücke bei den Uniformarten Felddienstuni-
form, Dienstuniform, Stabsdienstuniform und Aus-
gangsuniform für weibliche Armeeangehörige der
Landstreitkräfte und der LSK/LV gestattete es den
Frauen, sich abwechslungsreich, witterungsgerecht
und aufgabenbezogen so zu kleiden, daß spezifisch
weibliche Ansprüche in der veränderten Uniform
ausreichend berüicksichtigt werden konnten. Es war
zu beachten, daß die Weste ohne Uniformjacke nur
innerhalb der militärischen Objekte sowie in ge-
schlossenen Riumen angezogenwerdendurfte.
Hemdblusen konnten mit Binder kombiniert wer-
den, wenn der Blusenkragen geschlossen war bzw. wenn der Sommermantel darübergezogen wurde.
Die Kombination des weißen Pullovers mit dem
Rock bzw, der langen Hose war nur in Verbindung
mit dem Tragen der Uniformjacke bzw. der Weste
möglich. Die Tragezeit der langen Hose als Be-
standteil der Stabsdienstuniform für weibliche Ar-
meeangehörige aller Teilstreitkräfte beschränkte
sich auf den Zeitraum vom 1.November bis zum
15.April.
Weibliche Angehörige der Volksmarine trugen
die in der obigen Tabelle zusammengestellten Uni-
formarten.