

In der NVA wurde von Beginn an der Sport groß
geschrieben,Die meisten Armeeangehörigen trie-
ben auch schon damals über den Dienstsport hinaus aktiv Sport. Dazu bot die am 1.Oktober 1956
gegründete Armeesportvereinigung
«Vorwärts»
vielfaltige Möglichkeiten.
Im Verlaufe der Jahre 1956 und 1957 gab der
B/A-Dienst der riückwärtigen Dienste der NVA an
alle Soldaten, Unteroffiziere und Offiziere einheitli.
che Sportbekleidung gegen Bezahlung aus. Die Tat
sache,daß das Sportzeug letztlich in den pers¶nli.
chen Besitz seines Trägers überging, rechtfertigte
diese kurzzeitige Regelung. Die Sportbekleidung
setzte sich bis Mitte der 6Oer ahre aus einem dun-
kelblauen Trainingsanzug,schwarzer Sporthose,
weiBem Sporthemd,dunkelblauer Schwimmhose
und schwarzen Ledersportschuhen zusammen.Der
entsprechende Befehl des Ministers für Nationale
Verteidigung der DDR vom Sommer 1956 schrieb
für den Dienstsport bei kalter Witterung die Dril-
lichuniform vor,solangeTrainingsanzüge noch
nicht ausreichend vorhanden waren. Bei strengem
Frost konnten die Armeeangehörigen die Dienst-
uniform und den Uniformmantel überziehen.
Der Minister forderte weiterhin, bis zum 1. De
zember 1956 eine Regelung zu treffen, nach der die
Unteroffiziere und Offiziere im Sportzeug ihrem
Dienstgrad nach erkennbar sind,Davon ausgehend
wurden für die Trainingsanzüge wie auch für Kom
binationen und Schutzbekleidung der Angehörigen
der Land- und Luftstreitkräfte der NVA Dienst
gradabzeichen in Tressenform eingef¼hrt. Sie wa
ren am linken Ober¤rmel befestigt, d. h., der obere
Rand des Abzeichens befand sich 14 cm unter der
Schulternaht, Die Dienstgrade von Unteroffizier bis
Oberst f¼hrten silbergraue, die Generale goldfar
bene Tressen, Ein Unteroffizier trug eine 9 mm
breite und l0 cm lange silbergraue Perlongespinst
tresse,ein Feldwebel zwei und ein Oberfeldwebel
drei derartige Tressen in einem Abstand von jeweils
5 mm zwischen den Tressen.Der Unterleutnant war
an einer 15mm breiten und cm langen Tresse
und dariiber einer 5 mm breiten und ebenso langen
Tresse erkennbar. Die Dienstgrade bis Hauptmann
fügten stets eine weitere 9-mm-Tresse hinzu. Der
Major trug zwei der beschriebenen Tressen von 15 mm und cine von 9 mm Breite, Für die Dienst-
grade Oberstleutnant und Oberst erhöhte sich die
Zahl der 9-mm-Tressen um je eine. Ein Generalma-
jor besaB die gleiche Tressenanordnung wie ein
Major - nur in goldfarbener Ausführung. Ebenfalls
weitere goldfarbene 9-mm-Tressen kamen für die
Dienstgrade Generalleutnant, Generaloberst und
Armeegeneral hinzu, so daß letzterer zwei goldfar-
bene 15-mm- und dariber vier 9-mm-Tressen auf
wies.
Nicht unerwähnt soll bleiben, daß die Unteroffi-
ziere der Land- und Luftstreitkräfte der NVA auf
den weißen Sporthemden am Halsausschnitt eine
9 mm breite schwarze Perlongespinsttresse und die
Offiziere zwei derartige Tressen, im Abstand von
3 mm eingefaBt,trugen. Maate und Meister der
Seestreitkräfte befestigten an der Sportbekleidung
einen gewebten goldfarbenen klaren Anker auf ova-
ler Tuchunterlage aus blauem Stoff. Für Offiziere
galt der gleiche Anker, jedoch mit einer goldfarbe-
nen Umrandung versehen.