

Am 1.September 1975 begann an der Technischen
Unteroffiziersschule <Erich Habersaath》in Prora
die dreijährige Ausbildung von Militärmusikern.
Für die Militärmusikschüler mußten genaue Festle.
gungen zur Uniformierung getroffen werden. Prin-
zipiell galt die Festlegung der DV010/0/005, daB
Soldaten im Grundwehrdienst und auf Zeit in u-
sikkorps und im Erich-Weinert-Ensemble mit den
Uniformarten der Berufssoldaten auszustatten sind.
Militärmusikschüler, die sich noch nicht im aktiven
Wehrdienstverhältnis befanden, trugen Schulter-
klappen ohne Paspelierung.
Die Dienstlaufbahnabzeichen für Militärmusik-
schüler bestanden aus einer silberfarbenen Lyra mit
Winkeln. Ein Winkel zeigte, daß sich der Träger im
1. Lehrjahr, zwei Winkel im 2. Lehrjahr und drei
Winkel im 3, Lehrjahr befand. Die Dienstlaufbahn-
abzeichen wurden von den ilitärmusikschülern zur Dienstbekleidung auf dem linken Ãrmel aufge-
näht,Zusätzlich zu dieser besonderen Kennzeich-
nung trugen die Schüler der Fachrichtung Milit¤r-
musik einen rmelstreifen mit der Aufschrift
<Milit¤ärmusikschüler》 links an der Uniformjacke
und am Uniformmantel sowie metallgeprägte Lyren
auf den Schulterklappen am Mantel und an der
Uniformhemdbluse. Anstelle der Kragenspiegel wa-
ren bei den Militärmusikschülern auf den Kragen-
revers der Uniformjacken ebenfalls Lyren befestigt.
An der Wintermütze und am Koppel befanden sich
Embleme ohne Eichenlaub.