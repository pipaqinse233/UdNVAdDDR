

In dieser Vorschrift werden für die Angehörigen
der Landstreitkräfte und der LSK/LV die Uniformarten Felddienstuniform,Dienstuniform,Stabs
dienstuniform,Ausgangsuniform,Paradeuniform,
Gesellschaftsuniform und Arbeitsuniform in ihrer
genauen Zusammenstellung vorgeschrieben.Fir
Angehörige der Volksmarine legt die neue Vor
schrift die Uniformarten Gefechtsuniform, Feld
dienstuniform,Dienstuniform,Borduniform,Aus
gangsuniform,Paradeuniform,Gesellschaftsuni
form und Arbeitsuniform fest. Diese Uniformarten
k¶nnen jahreszeitlich bedingt als Sommeruniform
in der Zeit vom 16.April bis zum 31.Oktober,als
Übergangsuniform in der Zeit vom l. März bis zum
15.April und vom 1.November bis zum 30.Novem
ber und alsWinteruniform vom 1.Dezember bis
zum 28./29.Februar getragen werden. Bei Einhal
tung des Grundsatzes,da Armeeangehörige be
gleichem Anlaß und bei gleicher Dienstdurchfüh
rung die gleiche Uniformart tragen, bietet die neu
Vorschrift optimale Möglichkeiten, unter konkretei
Berücksichtigung der Witterungsbedingungen alle
Aufgaben des militärischen Dienstes in zweckmÃäßi-
ger Uniformierung zu erfüllen.
Eingearbeitet in die neue Vorschrift sind inzwi
schen eingefihrte oder geänderte Uniformstücke
und ihre Trageweise.
e nach Ausstattung setzen Armeeangehörige zur
Arbeitsuniform die neue schwarze Arbeitsmützc
mit Mützenschirm oder wie in der Vergangenheit
die Feldmütze auf, Angehörige der Volksmarine,
die nun ebenfalls einen zweiteiligen Arbeitsanzug
besitzen,tragen zurArbeitsuniform(Sommer)das
Bordk¤ppi oder ebenfalls die Arbeitsmitze mit
Schirm.
Die Fallschirmjäger aller Dienstgradgruppen be.
festigen nun auf den orangefarbenen und dunkel
grauen Baskenmützen links neben dem itzenem-
blem als weitere Symbolik einen weißen gestickten
Fallschirm mit darunter befindlicher Schwinge.
1986 wurde damit begonnen, alle Dienstgrad
gruppen der Landstreitkräfte und der LSK/LV mit Uniformmänteln ohne dunkel abgesetztem Kragen
zu versorgen. Mit der kontinuierlichen Vereinheitli.
chung aller Uniformstücke fielen ab 1986 auch die
Kragenspiegel auf den Mänteln der Angehörigen
der Luftstreitkräfte weg. Kragenspiegel aus hochro.
tem oder blauem Stoff mit darauf befindlicher Stik
kerei tragen nun nur noch Generale der Landstreit.
kräfte und der LSK/LV, Für Generale wurde mit
dem 1.Dezember 1986 die zweireihige steingraue
Uniformjacke aus der Ausstattung herausgenom
men.Dafür ist die einreihige Uniformjacke zur
Dienst-,Stabsdienst-,Ausgangs- und Paradeuni.
form anzúziehen,Auch an der Paradeuniform der
Generale werden von nun an keine Arabesken mehi
angebracht.
1988 erfolgte die Ausstattung der Berufssoldaten
mit einer neuen weißen Hemdbluse im gleichen
Schnitt wie die silbergrauen bzw. cremefarbenen
Hemdblusen.
Für weibliche Armeeangehörige wurde ab 1.De
zember 1986 wieder die Wintermütze mit Webpelz
und ohne Schirm eingeführt, Für alle veränderten
oder neuen Uniformstücke legt die Bekleidungsvor
schrift 1986 exakte Tragezeiten fest, so daß unter
Beachtung des ökonomischen mgangs mit den
Uniformstücken bis 1990 sowohl veränderte als
auch unveränderte Uniformstücke getragen werden
können, Bei den meisten Uniformarten ist es mÃg
lich, jahreszeitlich bedingt und auf der Grundlage
der Festlegungen der Bekleidungsvorschrift drei bis
fünf Varianten, die in der Vorschrift durchnume-
riert sind, zusammenzustellen.