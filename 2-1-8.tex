

AnläBlich des 6..Jahrestages der Nationalen Volks-
armee am 1.März 1962 fand erstmals in der Ge-
schichte der jungen Arbeiter-und-Bauern-Armee
der GroBe Zapfenstreich statt, Auch in der Folge
zeit fand er am Nationalfeiertag der DDR, bei Jubi-
läen der Armee sowie bei anderen besonderen An-
lässen eine rege Anteilnahme bei der Bevölkerung, aber auch bei ausländischen Gästen. Dieses Zere-
moniell war der NVA der DDR aus der deutschen
Militärgeschichte als Erbe überkommen. .Jedoch er
hielt der Große Zapfenstreich der NVA als Zeremo.
niell der sozialistischen Streitkräfte einen neuen In
halt, der tief in den fortschrittlichen und revolutio.
nären Traditionen des deutschen Volkes und der
internationalen Arbeiterbewegungverwurzelt ist.
Von 1962 bis 198l erg¤nzten der Fanfarenmarsch
«Für Frieden und Freundschaft) und das Arbeiter.
kampflied xBrüder,zur Sonne,zur Freiheit> das
Musikrepertoire des Zapfenstreiches und lieBen
ebenfalls den neuen,sozialistischen Charakter die-
ses Zeremoniclls deutlich werden.
Weitere, von vielen Menschen immer wieder in.
teressiert verfolgte Zeremonielle der NVA sind der
GroBe und der Kleine Wachaufzug am Mahnmal
fir die Opfer des Faschismus und Militarismus in
der Hauptstadt der DDR, Berlin,Unter den Lin
den. Am 1.Mai 1962 zog dort um 10.00 Uhr das er-
ste Postenpaar mit geschultertem Karabiner zum
Kleinen Wachaufzug auf. Auch der GroBe Wach.
aufzug der NVA hatte an jenem Tag an derselben
Stätte seine Premiere, Mit diesen militärischen
Zeremoniellen zu Ehren der antifaschistischen Wi
derstandskÃämpfer und zum Gedenken an die Opfer
des Faschismus und Militarismus bekundet die
DDR vor der Welt ihren festen Willen, alles zu tun,
damit nie wieder von deutschem Boden ein Krieg
ausgeht,Der Große Wachaufzug findet an jedem
Mittwoch, am Tag der NVA (1. Mǎrz),am Interna-
tionalen Kampf- und Feiertag der Werktätigen
(1.Mai),am Jahrestag der Befreiung vom Hitlerfa-
schismus(8.Mai)und am Nationalfeiertag der
DDR(7.0ktober)statt, Den Kleinen Wachaufzug
der von der Ehrenwache durchgeführt wird, gab
und gibt es täglich.Zu weiteren Zeremoniellen der
NVA gehören der Einsatz von Ehreneinheiten - be
stehend aus Zigen aller Teilstreitkräfte - und Eh
renwachen zur Ehrenbezeigung sowie bei besonde.
ren Anlässen,Einbezogen ist hiufig auchein
Musikkorps.

Die militärischen Zeremonielle in der Hauptstadt der DDR bestreitet auch heute noch das
Wachregiment der NVA. Für diesen Truppenteil
regelte der Befehl Nr.62/62 des Ministers fir Na-
tionale Verteidigung der DDR vom 28.Juni 1962
die Ausstattung der Soldaten und Unteroffiziere
mit Offiziersbekleidung, damit sie ihre repräsentati-
ven Aufgaben angemessener wahrnehmenkn
nen.Die Uniform war entsprechend der Ausfih-
rung für die Landstreitkräfte weiß paspeliert und
die Uniformjacke zus¤tzlich mit einem aufgenähten
Ãrmelband mit der Aufschrift <NVA-Wachregi-
ment» um den linken Unterärmel versehen. Beson-
ders wurde im Befehl auf die Notwendigkeit hinge-
wiesen, daß Uniformjacke und Stiefelhose im Farbton übereinstimmen müssen- eine damals
nicht immer leichte Aufgabe für die Tuchmacher in
der Bekleidungsindustrie der DDR, Bis Ende Sep
tember 1962 war die Umkleidung des Regiments
abgeschlossen.
Die DV-98/4. Bekleidungs- und Ausriistungsnor-
men der Nationalen Volksarmee von 1960be-
stimmte zusammenfassend die Uniformstücke, die
zus¤tzlich zur Ausstattungsnorm den Soldaten, Un-
teroffizieren und Offizieren der Orchester und
Standortmusikkorps auszugeben waren. Diese Uni-
formstücke waren aus dem Uniformstoff für Offi-
ziere angefertigt und wurden mit den Effekten der
Soldaten und Unteroffiziere getragen, So verfügten die Soldaten und Unteroffiziere des Zentralen Or-
chesters des inisteriums für Nationale Verteidi-
gung der DDR, des Standortmusikkorps Strausberg
und der Standortmusikkorps der Kommandos der
Militärbezirke der Landstreitkräfte sowie des Kom-
mandos der LSK/LVzus¤tzlich über je cine
Schirmmütze, einen Hel aus Kunststoff in Stahl-
helmform(nur Zentrales Orchester und Standort-
musikkorps Strausberg), einen Uniformmantel, eine
Uniformjacke(Ausgang/Parade),eine lange Uni-
formhose, eine Stiefelhose, ein Koppel mit
Schnalle, ein Paar glatte Schaftstiefel, ein Paar
glatte Halbschuhe und zwei Paar weiße Hand-
schuhe (nur Zentrales Orchester und Standortmu-
sikkorps Strausberg).
Das Standortmusikkorps des Kommandos der
Volksmarine wurde ahnlich zusätzlich ausgestattet.
Scine Angehörigen in den Dienstgraden Matrose
bis Meister erhielten folgende aus Uniformstoff für Offiziere hergestellte Uniformstücke: eine Schirm-
mütze, zwei weiße Mützenbezüge, einen Uniform-
mantel, eine Parade-/Ausgangsjacke, zwei weiBe
Ausgangsjacken, zwei Uniformhosen, drei weiBe
und dreisilbergraue Uniformhemden,
drei
schwarze Binder,ein Paar Halbschuhe sowie an-
stelle der genarbten Schnürschuhe cin Paar glatte
Schnürschuhe.
Die anderen Musikkorps der NVA bekamen
ebenfalls einige zusätzliche niformstücke. Ãber
die Grundnorm hinaus wurden das Erich-Weinert-
Ensemble, die Zentralen Kulturensembles der Mili-
t¤rbezirke und der Kommandos der LSK/LV sowie
der Volksmarine versorgt. Derartige Maßnahmen
verbesserten das äuBere Erscheinungsbild dieser
Orchester, Musikkorps und Ensembles bei ihren
Auftritten.