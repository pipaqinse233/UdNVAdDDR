

Die NVA war in den 50er ,Jahren auf dem Wege zu
einer modernen sozialistischen Armee erfolgreich
vorangeschritten, Sie hatte sich bei der Sicherung
der Staatsgrenzen der DDR 1961 und des Friedens
im Zentrum Europas im Zusammenwirken mit an-
deren bewaffneten Kräften der DDR und den
Truppen der GSSD bewährt.
Mit Beginn des ,Jahrzehnts trat die NVA in eine
neue Entwicklungsetappe ein, Der Ausbau der Ar.
mee stand im Zeichen der zunehmenden militärpo-
litischen und militärischen Zusammenarbeit der
Teilnehmerstaaten des Warschauer Vertrages und
der stürmischen Umwälzungen im ilitärwesen.
Das Gesetz zur Verteidigung der DDR vom 20.Sep
tember 1961 und das Gesetz über die allgemcine
Wehrpflicht vom 24.Januar 1962 enthielten dafür
die grundlegenden Orientierungen, Auf die mit
dem Sieg der sozialistischen Produktionsverhält-
nisse gewachsenen wirtschaftlichen, politischen und
ideologischen Potenzen des Landes konnte bei der
notwendigen Stärkung der Verteidigungsfahigkeit
zur¼ckgegriffen werden.
Die NATO setzte gegenüber den Staaten der
Warschauer Vertragsorganisation nach wie vor auf Gewalt und stützte sich dabei besonders auf ihr
Kernwaffenpotential, Der von ihr verstärkt vorbe
reitete Krieg erstreckte sich vom konventionellen
bis zum begrenzten Kernwaffeneinsatz. Dem durf.
ten die Staaten der sozialistischen Verteidigungs
koalition,darunter die DDR,nicht tatenlos zuse-
hen. Um die Friedensstrategie des sozialistischen
militärpolitischen Bündnisses zuverlässig zu si-
chern, wurden auch Kampfkraft und die Gefechts-
bereitschaft der NVA in gebotenem MaBe erhöht.
Während es noch im ersten Entwicklungsabschnitt
der NVA vornehmlich darauf ankam, Teilaufgaben
beim sicheren äuBeren Schutz der DDR zu über.
nehmen und die Verbände und Truppenteile auf
die Führung von Gefechtshandlungen unter moder.
nen Bedingungen vorzubereiten, ging es jetzt um
weit mehr. Es galt, alle Teilstreitkräfte der NVA zu
befähigen,im Zusammenwirken mit der Sowjetar-
mee und den anderen Bruderarmeen cinem plötzli-
chen Kernwaffenüberfall der NATO - die gefähr-
lichste Variante der Kriegsentfesselung-zu
begegnen, Die NVA muBte aber auch in der Lage
sein,in konventionellen Kampfhandlungen zu bestehen.

Im Rahmen der Vereinten Streitkr¤fte der War-
schauer Vertragsstaaten wurde die NVA schritt-
weise ausgebaut.Zuerst wurden die Krifte der
Luftverteidigung mit Raketenwaffen sowjetischer
Konstruktion ausgeriistet. Danach bekamen die
Landstreitkräfte taktische und operativ-taktische
Raketen und die Küstenartillerie der Volksmarine
Raketen vom Typ «Sopka». Auch die wichtigsten
herk¶mmlichen Waffengattungen erhielten neue
Milit¤rtechnik, z.B. den Panzer T-55, moderne Ge-
schoBwerfer BM-24,Panzerabwehrlenkraketen, das
Ãberschalljagdflugzeug MiG-21 und Raketen- und
Torpedoschnellboote.
Diese Umrüstung hatte weitreichende Konse-
quenzen für die Organisation und Ausbildung der
Teilstreitkr¤fte der NVA. Bei den Landstreitkräften
entstanden die neuen Waffengattungen Raketen-
truppen/Artillerie und Truppenluftabwehr. Die
Luftstreitkräfte/Luftverteidigung bildeten Luftver.
teidigungsdivisionen, und die Volksmarine schu
auf der Basis neuer selbständiger Raketen- und Torpedoschnellbootformationen StoBverbände. Im
Manöver «Quartett» im September 1963- der er-
sten groBen Übung der NVA, der Polnischen Ar-
mee, der Sowjetarmee und der Tschechoslowaki-
schen Volksarmee im Rahmen der Vereinten
Streitkräfte auf dem Territorium der DDR- be-
wiesen auch die Angehörigen der NVA ihre Bereit-
schaft und Fahigkeit, entsprechend den Anforde-
rungen modernerOperationenund
Gefechte
erfolgreich zu handeln.
Aus diesen Veränderungen in der Armee ergaben
sich auch vielfaltige Konsequenzen für den Beklei-
dungs- und Ausrüstungsdienst der NVA auf dem
Gebiet der Uniformierung in allen Teilstreitkraften.
Zwar blieben die Hauptuniformarten der NVA
auch in den 60er ,Jahren bestehen, doch eine Reihe
von Veränderungen und Neuerungen - zum gro-
ßen Teil noch in den 50er Jahren vorbereitet bzw.
durch Erprobungen eingeleitet - setzte sich gerade
zu Beginn des ,Jahrzchnts durch.