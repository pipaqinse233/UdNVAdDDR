\section{1956年的国家人民军制服}

\begin{figure}
\includegraphics[width = \columnwidth]{./media/page (11).jpg}
\end{figure}

1956年1月18日,东德最高议会成员讨论并一致通过了第63号印刷品《国家人民军和国防部组建法》,同时也确定了第一支社会主义德国军队的制服。而在此之前还采取了一项不同寻常的措施——服装展。
% Als die Mitglieder der obersten Volksvertretung der DDR am 18. Januar 1956 mit der Drucksache Nr.63 das «Gesetz über die Schaffung der Nationalen Volksarmee und des Ministeriums für Nationale Verteidigung» berieten und einstimmig verabschiedeten, hatten sie auch die Uniformen der ersten sozialistischen deutschen Armee zu bestätigen. Dieser Entscheidung ging eine ungewöhnliche Maßnahme voraus —— eine Modenschau.

在人民议会国家会议厅,议员们参观了为工农军成员准备的制服。毫无疑问,这种军装展示引起了极大的兴趣。1月18日下午,人民议会的议员们还一致通过了第64号印刷品,即《关于实行国家人民军制服的决议》。
% Im Länderkammersaal der Volkskammer besichtigten die Abgeordneten die Uniformen, die für die Angehörigen der Arbeiter-und-Bauern-Armee bestimmt waren. Keine Frage, die Vorführung dieser militärischen Bekleidung stieß bei ihnen auf großes Interesse. Die Volkskammerabgeordneten stimmten am Nachmittag dieses 18. Januar einmütig auch der Drucksache Nr. 64, dem «Beschluß über die Einführung der Uniform für die Nationale Volksarmee», zu.

通过关于成立国家人民军和国防部的法律,包括确认武装部队制服的宪法法案,标志着德国本土上第一支正规人民军队历史的开始。这些决定得到了民众的广泛支持。大多数东德人认识到:到了20世纪50年代中期,建立一支强大的社会主义人民军队已势在必行。西德联邦议院的绝大多数议员不顾民众强烈抗议,于1955年2月批准了《巴黎条约》。该条约于同年5月5日生效。通过该条约,北约国家批准了西德在所有涉及德国的事务中拥有唯一代表权的要求,从而批准了西德帝国主义统治势力的复辟政策。他们趁机组建西欧最强大的军队。西德加入北约并迅速组建了50万人的联邦国防军,增加了中欧帝国主义军事条约对社会主义国家的直接军事威胁,使得社会主义国家不得不采取反制措施。
% Mit dem staatsrechtlichen Akt der Verabschiedung des Gesetzes über die Schaffung der NVA und des Ministeriums für Nationale Verteidigung einschließlich der Bestätigung der Uniformierung der Streitkräfte begann die Geschichte der ersten regulären Volksarmee auf deutschem Boden. DieBeschlüsse fanden breite Zustimmung in der Bevölkerung. Die Mehrzahl der Menschen in der DDR erkannte, daß es Mitte der 50er Jahre zwingend notwendig geworden war, eine kampfstarke sozialistische Volksarmee aufzustellen. Eine bedeutende Mehrheit des Bundestages der BRD hatte im Februar 1955 ungeachtet des nachdrücklichen Protestes in der Bevölkerung die Pariser Verträge ratifiziert. Sie waren am 5. Mai des Jahres in Kraft getreten. Die NATO-Staaten billigten mit diesen Verträgen den Alleinvertretungsanspruch der BRD in allen Deutschland betreffenden Fragen und somit die revanchistische Politik der herrschenden Kräfte des BRD-Imperialismus. Diese erhielten die Möglichkeit, die stärkste Armee Westeuropas auf zustellen. Die Einbeziehung der BRD in die NATO und die beschleunigte Schaffung der Bundeswehr als 500000-Mann-Armee verstärkten die direkte militärische Bedrohung der Länder der sozialistischen Gemeinschaft durch den imperialistischen Militärpakt in Mitteleuropa und machten Gegenmaßnahmen der sozialistischen Staaten erforderlich.

正因如此,1955年5月14日,苏联与包括东德在内的其他欧洲社会主义国家在华沙举行的欧洲国家维护欧洲和平与安全会议上签署了《友好合作互助条约》。东德是防务联盟\footnote{译者注:即「华约」。}的平等成员,其目的是为确保和平并保护社会主义做出军事贡献。当时的驻营人民警察并不能替代武装部队。
% Deshalb unterzeichneten die UdSSR und andere europäische sozialistische Staaten, darunter auch die DDR, auf der Konferenz europäischer Länder für die Gewährleistung des Friedens und der Sicherheit in Europa am 14. Mai 1955 in Warschau den «Vertrag über Freundschaft, Zusammenarbeit und gegenseitigen Beistand». Für die DDR, gleichberechtigtes Mitglied des Verteidigungsbündnisses, galt es, einen militärischen Beitrag zur Sicherung des Friedens und für den Schutz des Sozialismus zu leisten. Die bisher bestehende Kasernierte Volkspolizei vermochte Streitkräfte nicht zu ersetzen.

\begin{figure}
\includegraphics[width = \columnwidth]{./media/page (13).jpg}
\end{figure}

这些事实同时也回答了在人民议会作出上述决定前后东德许多人关心的问题:为什么需要一支我们自己的军队?在1956年1月甚至以后的那些日子里,还有许多其他问题困扰着人们:国家人民军将是一支什么样的军队?它与以前的德国军队和西德的联邦国防军有何不同?什么样的人将在国家人民军中担任士兵、士官、尉官、校官和将官?共和国的经济发展是否会因为军队的发展而受到影响?
% Diese Tatsachen waren zugleich die Antwort auf die Frage,  die viele Menschen in der DDR, unmittelbar vor und nach dem genannten Volkskammerbeschluß bewegte: Warum ist eine eigene Armee notwendig? Noch viele andere Fragen beschäftigten die Menschen in jenen Januartagen 1956 und auch noch später: Was für eine Armee wird die NVA sein? Wie unterscheidet sie sich von früheren deutschen Armeen und wie von der Bundeswehr der BRD? Was für Menschen werden in der NVA ihren Dienst als Soldaten, Unteroffiziere, Offiziere und Generale versehen? Leidet die wirtschaftliche Aufwärtsentwicklung der Republik nicht unter dem Aufbau der Armee?

不少人对国家人民军制服的兴趣也同样浓厚。当时人手一份的报刊经常分几期整版刊登国家人民军陆军、空军、海军的军服种类、军衔徽章和配饰。在展示完军服后,许多年轻人要求《年轻人世界报》的编辑们也刊登一下军队的军衔徽章。而从1月23日起,各报刊就这样做了。
% Nicht minder groß war das Interesse der Bevölkerung an der Uniformierung der NVA. Von Hand zu Hand gingen in diesen Tagen die Zeitungen, die oft in mehreren Folgen ganzseitige Zusammenstellungen der Uniformarten, Dienstgradabzeichen und Effekten der Land-, Luft- und Seestreitkräfte der NVA veröffentlichten. So richteten viele Jugendliche an die Redaktion der Zeitung «Junge Welt» nach der Vorstellung der Uniformen die Bitte, auch die Dienstgradabzeichen der Armee abzubilden. Dies geschah in der Ausgabe vom 23. Januar.

\subsection{国家人民军的主要制服}

\begin{figure}
\includegraphics[width = \columnwidth]{./media/page (14).jpg}
\end{figure}

在东德首都柏林,当时有项特殊活动吸引着人们。在位于今日卡尔-马克思大街的德国体育馆的房间里,参观者不断挤到展示的制服和配饰前,了解计划中的国家人民军成员的主要制服类型:陆军和空军的士兵和士官将穿着石灰色工作服、常服、礼服和外出服。陆军和空军的军官也将穿着石灰色常服、礼服和外出服,但没有工作服;将军还将获得两种改良的常服,分别用于参谋和野战。
%In Berlin, der Hauptstadt der DDR, zog damals ein besonderes Ereignis die Menschen an. In den Räumen der Deutschen Sporthalle, in der heutigen Karl-Marx-Allee, drängten sich vor den ausgestellten Uniformen und Effekten ständig Besucher. Sie erfuhren, welche hauptsächlichen Uniformarten für die Angehörigen der Nationalen Volksarmee vorgesehen waren: die Soldaten und Unteroffiziere der Land- und Luftstreitkräfte würden steingraue Drillich-,  Dienst-, Parade- und Ausgangsuniformen erhalten. Die Offiziere beider Teilstreitkräfte sollten ebenfalls über steingraue Dienst-, Parade- und Ausgangsuniformen verfügen, nicht aber über die Drillichuniform, Generale würden außerdem mit zwei Modifikationen der Dienstuniform, für den Stabsund den Felddienst, ausgestattet werden.

在海军部队中,各级官兵还将穿着常服、礼服和水兵服,海军水兵、士官以及军官在舰船上进行某些活动时将穿着水兵服,而不是陆军和空军士兵和士官所穿的工作服。
%In den Seestreitkräften wirde es auch Dienst-, Parade- und Ausgangsuniformen für alle Dienstgradgruppen geben. Anstelle des Drillichs der Soldaten und Unteroffiziere der Land- und Luftstreitkräfte sollten die Matrosen, Maate, Meister sowie auch die Offiziere der Seestreitkräfte für bestimmte Tätigkeiten an Bord der Schiffe und Boote einen Bordanzug tragen.

国家人民军各军种在医疗和行政岗位上工作的女性成员穿着剪裁时尚的常服和礼服。
%Für weibliche Angehörige atler Teilstreitkräfte der NVA, in medizinischen und administrativen Dienststellungen tätig, waren Dienst- und Ausgangsuniformen in einem modischen Kostümschnitt vorgesehen.
\subsection{陆军和空军的制服}

许多参观者都花时间仔细观察了各军种的制服。陆军和空军士兵和士官于夏季(4月1日至9月30日)的工作服包括野战帽、工作服上衣和长裤、半靴和黑色带锁皮带。银色皮带扣上压印着东德国徽。根据各指挥官的命令,室内执勤时也可以穿系带鞋。冬天,军人要戴上沿用几十年的便帽式冬季帽,戴上羊毛勤务手套,并根据命令穿上冬季大衣。工作服上衣无衬里、单排扣,有两个胸袋,领口夏天敞开,冬天则合上,而长裤最初是从前往后塞进靴子里,不久后又从后往前塞进靴子里。
% Viele der Besucher der Ausstellung nahmen sich die Zeit und betrachteten genau die Uniformen der einzelnen Teilstreitkräfte, Die Drillichuniform für die Soldaten und Unteroffiziere der Land- und Luftstreitkräfte bestand in der Sommertrageperiode (1. April bis 30. September) aus Feldmütze, Drillichjacke und -hose, Halbschaftstiefeln und schwarzem Lederkoppel mit Schloß. Auf dem silberfarbenen Koppelschloß war das Staatswappen der DDR eingeprägt. Zum Innendienst konnten auf Befehl des jeweiligen Kommandeurs auch Schnürschuhe getragen werden. Im Winter setzten die Armeeangehörigen eine Wintermütze in Form der viele Jahrzehnte typischen Skimütze auf, zogen Diensthandschuhe aus Wolle und —— wiederum auf Befehl —— den Wintermantel an. Der Kragen der ungefütterten, einreihigen Drillichjacke mit den zwei Brusttaschen wurde im Sommer offen gelassen und im Winter geschlossen. Die Uniformhosen wurden anfangs noch von vorn nach hinten eingeschlagen und in die Stiefel gesteckt, kurze Zeit später von hinten nach vorn.

\begin{figure}
\includegraphics[width = \columnwidth]{./media/page (15).jpg}
\end{figure}

两军士兵和士官的常服包括船型帽、上衣和长裤、黑色半靴(或指挥官命令的系带鞋)和带扣腰带。在冬季,这类制服还配有冬季帽、大衣和勤务手套。
% Die Dienstuniform der Soldaten und Unteroffiziere beider Teilstreitkräfte bestand aus Feldmütze Uniformjacke und -hose, schwarzen Halbschaftstiefeln (oder auf Befehl des Kommandeurs Schnürschuhen) und Koppel mit Schloß. Für die Winterperiode wurde diese Uniformart durch Wintermütze, Uniformmantel und Diensthandschuhe komplettiert.

两军士兵和士官的礼服和外出服略有不同。礼服的特点是头戴大檐帽,身穿礼服上衣、长裤、半靴和带扣腰带,冬季还穿大衣,戴勤务手套;而在外出服中,手套和皮鞋或短靴取代了勤务手套和半靴。
% Nur unwesentlich unterschied sich die Parade- von der Ausgangsuniform der Soldaten und Unteroffiziere der Land- und Luftstreitkräfte. Schirmmütze, Paradejacke, Uniformhose, Halbschaftstiefe und Koppel mit Schloß sowie Uniformmantel und Diensthandschuhe im Winter bestimmten das Aus sehen der Paradeuniform. Bei der Ausgangsuniform traten Handschuhe und Halbschuhe bzw. Stiefeletten an die Stelle der Diensthandschuhe und der Halbschaftstiefel.

两军尉官和校官制服主要是面料材质(精梳纱而非粗纺纱)和制造类型不同:陆军尉官和校官常服在夏季为大檐帽、合领常服上衣、马裤和棕色带扣皮带,在冬季则为冬季帽、黑色或棕色皮手套和大衣;而空军尉官和校官常服包括一件开襟上衣、一件银灰色衬衣和一条深灰色领带。由于冬季大衣也是敞开的,所以还要佩戴一条灰色围巾。
% Die Offiziersuniformen beider Teilstreitkräfte wiesen vor allem eine andere Stoffqualität (Kammgarn statt Streichgarn) und Fertigungsart auf, Offziere der Landstreitkräfte trugen als Dienstuniform im Sommer Schirmmütze, Dienstjacke mit geschlossenem Kragen, Stiefelhose und braunes Lederkoppel mit Schnalle; im Winter Wintermütze und zusätzlich schwarze oder braune Lederhandschuhe sowie Uniformmantel. Dagegen bestand die Dienstuniform der Offiziere der Luftstreitkräfte aus einer Uniformjacke offener Fasson, einem silbergrauen Hemd und einem dunkelgrauen Binder. Da im Winter auch der Uniformmantel oben offen getragen wurde, gehörte ein grauer Schal dazu.

两军尉官和校官的礼服和外出服的设计与常服相同。如前所述,礼服上衣可敞开,可闭合。礼服总是配以黑色铬合金皮靴和银色礼服腰带。黑色皮鞋或短靴与外出服相同。所有陆军和空军尉官和校官都可以穿着礼服上衣和长裤作为外出服,但也可购买或自费请裁缝制作特殊的双排扣外出服上衣。在这种情况下,陆军尉官和校官也可以穿银灰色衬衣,系深灰色领带,冬季使用灰色围巾。尉官和校官们在外出或上下班途中可以穿夏装或雨衣搭配各类制服。
% Analog zu den Uniformstücken der Dienstuniform gestalteten sich die Parade- und die Ausgangsuniformen für die Offiziere beider Teilstreitkrafte. Die Parade-/Ausgangsjacke wurde, wie bereits geschildert, offen bzw.geschlossen gehalten. Bei der Paradeuniform ergänzten stets schwarze Chromlederstiefel und silberfarbene Feldbinde die Anzugsordnung. Zur Ausgangsuniform gehrtenschwarze Halbschuhe oder Stiefeletten. Alle Offiziere der Land- und Luftstreitkräfte konnten ihre Paradejacke und die Uniformhosen als Ausgangsunifor tragen, Sie konnten aber auch eine spezielle zweireihige Ausgangsjacke kaufen bzw. sie beim Schneider auf eigene Kosten anfertigen lassen. In diesem Fall zogen auch die Offiziere der Landstreitkräfte ein silbergraues Uniformhemd mit dunkelgrauem Binder an und verwendeten im Winter den grauen Schal. Zu allen Uniformarten wurde den Offizieren im Ausgang und auf dem Wege zum und vom Dienst das Tragen des Sommer- oder des Regenmantels gestattet.

陆军和空军将官身着白色外穿衬衫,并始终穿着双排扣开领礼服上衣作为外出服的一部分。将官的外出服还需佩戴金色礼服腰带和棕色皮手套。冬季,蓝灰色的大衣与两套将军服相得益彰;外出服还加了一条白色围巾。
% Generale der Land- und Luftstreitkräfte zogen Zur Ausgangsuniform ein weißes Oberhemd an und trugen stets die zweireihige, offene Ausgangsjacke. Zur Paradeuniform der Generale waren zusätzlich die goldfarbene Feldbinde und braune Lederhandschuhe vorgeschrieben. Im Winter ergänzte ein blaugrauer Uniformmantel beide Uniformen der Generale; zur Ausgangsuniform kam noch ein weißer Schal hinzu.

如前所述,将官有两种常服:一是用于野战和部队视察的野战常服;二是用于日常参谋工作的参谋常服。,野战常服上衣根据军种不同分为开领式或合领式,并配有马裤和长筒靴;所有将官的参谋常服包括双排扣开领式上衣、银灰色衬衫配深灰色领带、石灰色长裤和皮鞋或短靴。夏季,可以根据天气条件穿夏季大衣或皮大衣执勤外出;冬季则须穿着灰色大衣、皮手套和灰色围巾。
% Wie an anderer Stelle schon erwähnt, konnten Generale zwischen der Dienstuniform I zum Felddienst und zu Truppenbesichtigungen und der Dienstuniform II zum täglichen Dienst im Stab wählen. Zum je nach Teilstreitkraft offenen oder geschlossenen Kragen der Uniformjacke der Dienstuniform I gehörten Stiefelhose und Schaftstiefel, zur Dienstuniform II bei allen Generalen die zweireihige, offene Uniformjacke, das silbergraue Hemd mit dunkelgrauem Binder, die steingraue Uniformhose und Stiefeletten oder Halbschuhe. Im Sommer konnten sie entsprechend den Witterungsverhältnissen den Sommer- oder einen Ledermantel zum Dienst und zum Ausgang tragen. Im Winter waren der graue Uniformmantel, Lederhandschuhe und ein grauer Schal obligatorisch.
\subsection{海军的制服}

\begin{figure}
\includegraphics[width = \columnwidth]{./media/page (16).jpg}
\end{figure}

在德国体育馆的展览中,参观者发现海军制服种类比陆军和空军的更多。在海军制服前人头攒动。海军制服与陆军和空军制服的最大区别在于配色和剪裁。此外,大多数类型的制服除了蓝色款式外,还允许在夏季(5月1日至9月30日)穿着白色款式。
% Eine größere Vielfalt der Uniformierung als bei den Land- und Luftstreitkräften fanden die Besucher der Ausstellung in der Deutschen Sporthalle bei den Seestreitkräften vor. Der Andrang vor den Uniformen dieser Teilstreitkraft war außerordentlich groß. Die gravierendsten Unterschiede zwischen den Uniformen der Seestreitkräfte und denen der Land- und Luftstreitkräfte bestanden in der Farbgebung und der Schnittgestaltung. Weiterhin ließen die meisten Uniformarten neben der blauen noch eine weiBe Ausführung für die Sommertrageperiode, die bei den Seestreitkräften die Zeit vom 1. Mai bis zum 30. September umfaßte, zu.

水兵、海军下士和海军中士穿白色或蓝色的水兵服。因此,水兵帽、上衣和裤子以及半靴、系带鞋或水手鞋和带扣腰带都是必备的。上课和操练时也要佩戴基尔领。在冬季,水手和船员们会在蓝色水兵服的基础上再搭配一件蓝白条纹长袖背心\footnote{译者注:应指「海魂衫」。}和蓝色羊毛手套。在恶劣天气条件下,指挥官可以命令他们穿外套和高领毛衣。
% Für Matrosen und Maate war das Tragen des weißen oder des blauen Bordanzuges möglich. Vorgeschrieben waren dementsprechend Bordkäppi, -bluse und -hose sowie Halbschaftstiefel, Schnürschuhe oder Bordschuhe und Koppel mit Schloß. Im Unterricht und beim Exerzieren kam der Kieler Kragen hinzu. In der Wintertrageperiode ergänzten die Matrosen und Maate den blauen Bordanzug durch ein blau-weiß gestreiftes Unterhemd mit langen Ärmeln und blaue Wollhandschuhe. Bei schlechten Witterungsverhältnissen konnte der Kommandeur befehlen, Überzieher und Rollkragenpullover anzuziehen.

海军上士、海军尉官和海军校官只穿蓝色水兵服,包括水兵帽、上衣和长裤,以及半靴(只适用于海军上士)或系带鞋或水兵鞋。此外,海军上士还戴着蓝色羊毛手套,海军尉官和海军校官则戴着黑色皮手套,以抵御冬季的寒冷。
% Meister und Offiziere der Seestreitkräfte besaßen ausschließlich den blauen Bordanzug. Dieser bestand aus Bordkäppi, -jacke und -hose sowie Halbschaftstiefeln (nur bei den Meistern) oder Schnürschuhen bzw. Bordschuhen. Zusätzlich schützten sich die Meister mit blauen Wollhandschuhen und die Offiziere mit schwarzen Lederhandschuhen gegen die Winterkälte.

\begin{figure}
\includegraphics[width = \columnwidth]{./media/page (19).jpg}
\end{figure}

水兵、海军下士和海军中士的常服、礼服和外出服分为很多部分(见第16页表格)% TODO
% Die Dienst-, die Parade- und die Ausgangsuniformen der Matrosen und Maate setzten sich aus vielen Uniformstücken zusammen. (Siehe Tabelle S.16)

除了偶有不同设计外,海军上士、海军尉官、海军校官和海军将官的常服、礼服和外出服有许多共同之处。
% Von einer manchmal unterschiedlichen Ausführung abgesehen, wiesen die Dienst-, die Parade- und die Ausgangsuniformen der Meister, Offiziere und Admirale der Seestreitkräfte viele Gemeinsamkeiten auf.

海岸炮兵和高射炮兵、工兵连和保卫连、汽车兵以及训练和警卫部队的海军尉官和海军校官在某些场合穿着蓝色马裤和高筒靴。对于汽车兵来说,这只适用于驾驶任务和车辆训练;对于其他海军尉官和海军校官来说,这适用于警卫任务、操练、射击训练、行军、演习和视察。海军将官在演习和视察陆军部队时也穿马裤和高筒靴。
% Offziere der Küsten- und der Flakartillerie, der Pionier- und der Schutzkompanien, des Kfz-Dienstes sowie Kommandeure und Ausbildungsoffiziere in Ausbildungs- und Wacheinheiten trugen zu bestimmten Anlässen eine blaue Stiefelhose und Schaftstiefel. Dies traf bei den Offizieren des Kfz-Dienstes nur für den Fahrdienst und die Ausbildung an Fahrzeugen, bei den anderen Offizieren für den Wachdienst, das Exerzieren, die Schießausbildung, für Märsche, Übungen und Besichtigungen zu. Admirale zogen bei Übungen und Besichtigungen von Landeinheiten ebenfalls Stiefelhose und Schaftstiefel an.
\subsection{女性军人的制服}

妇女尤其喜欢军队为女性军人发放的制服。常服和外出服两种制服紧跟当时的女性时尚。陆军和空军女兵夏季制服一般包括贝雷帽、单排扣常服上衣或双排扣外出服上衣、银灰色衬衣、深灰色领带、配常服的长筒靴或黑色皮鞋、配外出服的黑色皮鞋、颜色时尚的长袜或白袜和雨衣。尉官和校官还有一件夏季大衣。冬季,陆军和空军的女兵穿常服外套,戴勤务手套(尉官和校官戴黑色或棕色皮手套)、灰色领带和冬季帽。
% Zu den ausgestellten Uniformen für die weiblichen Armeeangehörigen fühlten sich besonders die Frauen und jungen Mädchen hingezogen. Die beiden Uniformarten, die Dienst- und die Ausgangsuniform, folgten im Schnitt der damaligen Damenmode. Die Uniformen für die Frauen setzten sich generell bei den Land- und Luftstreitkräften für die Sommertrageperiode aus Baskenmütze, einreihiger Uniformjacke der Dienstuniform bzw. zweireihiger Ausgangsjacke, silbergrauer Bluse, dunkelgrauem Binder, Schaftstiefeln oder schwarzen Halbschuhen zur Dienstuniform, schwarzen Halbschuhen zur Ausgangsuniform, modefarbenen Strümpfen oder weißen Söckchen und Regenmantel zusammen. Die Offiziere verfügten zusätzlich über einen Sommermantel. Im Winter zogen die Frauen in den Land- und Luftstreitkräften den Uniformmantel und Diensthandschuhe (Offiziere schwarze bzw. braune Lederhandschuhe) an, legten einen grauen Schal ein und setzten zur Dienstuniform eine Wintermütze auf.

海军女兵也有相应的制服,但通常只有蓝色。她们夏天戴白色贝雷帽,冬天戴蓝色贝雷帽。女兵可以选择白色或蓝色的常服上衣,女性海军尉官和女性海军校官也可以选择白色或蓝色的出行服上衣。各级官兵的外出服均为银灰色或白色衬衣。除了女性水兵、女性海军士官、女性海军尉官和女性海军校官所穿的两种雨衣外,后者还有一件雨披作为其常服的一部分。
% In den Seestreitkräften besaßen die Frauen ebenfalls die entsprechenden Uniformstücke, aber meist nur in Blau. Im Sommer trugen sie eine weiße und im Winter eine blaue Baskenmütze. Die Frauen konnten zwischen der weißen und der blauen Dienstjacke und die weiblichen Offiziere dieser Teilstreitkraft außerdem noch zwischen der weißen oder der blauen Ausgangsjacke wählen. Zur Ausgangsuniform nahmen die Angehörigen aller Dienstgradgruppen entweder eine silbergraue oder eine weiße Bluse. Außer dem Regenmantel für beide Uniformarten der weiblichen Matrosen, Maate, Meister und Offiziere besaßen letztere zusätzlich einen Regenumhang zur Dienstuniform.
\subsection{国家人民军的兵种色}

在体育馆的展览中,许多参观者都仔细观察了军服的细节,如空军和海军以及特种部队和陆军各军种的兵种色以及特殊标记。
% Viele Besucher der Ausstellung in der Sporthalle achteten sehr auf die Details der Uniformen wie die Waffenfarben der Luft- und Seestreitkräfte bzw. die der Waffengattungen, Spezialtruppen und Dienste der Landstreitkräfte und spezielle Kennzeichnungen.

领章和滚边(插入制服接缝处呈彩色突起状的窄布条)\footnote{译者注:应指「牙线」。}上精确规定的兵种色是武装部队各军种的特征,或额外标识武装部队各军种、特种部队和陆军各军种。
% Genau vorgeschriebene Waffenfarben auf den Kragenspiegeln und Paspelierungen (in die Nähte der Uniformen eingefügte schmale Stoffstreifen, die als farbige VorstöBe erscheinen) charakterisierten die einzelnen Teilstreitkräfte bzw. kennzeichneten die Waffengattungen, Spezialtruppen und Dienste der Landstreitkräfte zusätzlich.

空军和海军的所有官兵都佩戴一种固定的兵种色\footnote{译者注:海军有三种:普通海军为深蓝色;海军航空兵为天蓝色;岸勤部队和海岸边防均为绿色。\cite{clarionv}},而陆军则有更多的区别。在部队和军官学校,所有士兵、士官、尉官和校官都佩戴一种统一的兵种色,例如坦克团的兵种色为砖红色,步兵学校的兵种色为白色。在师级单位,士兵和士官佩戴师所属兵种的兵种色,而尉官和校官则佩戴其所属兵种或特种部队的兵种色。
% Wahrend in den Luft- und Seestreitkräften alle Dienstgrade eine festgelegte Waffenfarbe führten,  wiesen die Landstreitkräfte noch zusätzliche Differenzierungen auf, In den Truppenteilen wie auch an den Offiziersschulen galt für alle Soldaten,  Unteroffiziere und Offiziere einheitlich eine Waffenfarbe, z.B. Rosa in einem Panzerregiment oder Weiß in einer Infanterieschule. In Divisionsstaben trugen die Soldaten und Unteroffiziere die Waffenfarbe der Waffengattung, der die Division angehörte, Offiziere aber die Farbe ihrer Waffengattung oder Spezialtruppe.
\subsection{各军种的军衔徽章}

\begin{figure}
\includegraphics[width = \columnwidth]{./media/page (20).jpg}
\end{figure}

国家人民军军衔名称的确定和军衔徽章的设计也吸引了展览参观者的注意。在军事生活中,军衔在确定武装部队成员的等级方面有重要作用。1956年1月18日东德部长会议关于实行军服、军衔名称和军衔徽章的决定,确定了国家人民军的军衔。下表提供了士兵、航空兵和水兵、士官、尉官、校官和将官等军衔组别中各个军衔名称的情况。
% Auch die Festlegungen der Dienstgradbezeichnungen und die Gestaltung der Dienstgradabzeichen der NVA fanden aufmerksame Betrachter unter den Ausstellungsbesuchern. Im militärischen Leben spielen Dienstgrade, die die Ranghöhe der Armeeangehörigen bestimmen, eine wichtige Rolle. Mit dem Beschluß des Ministerrates der DDR vom 18. Januar 1956 über die Einführung der Uniform, der Dienstgradbezeichnung und der Dienstgradabzeichen wurden sie fir die NVA festgelegt. Die nachfolgende Tabelle gibt Auskunft über die ein zelnen Dienstgradbezeichnungen in den Dienstgradgruppen der Soldaten, Flieger und Matrosen, der Unteroffiziere, Maate und Meister, der Offiziere und der Generale bzw. Admirale. 

士兵和士官佩戴士兵肩章,士官学员和尉官学员人员也如此。尉官、校官和将官则佩戴军官肩章。\footnote{译者注:军官肩章为双层结构,士兵肩章为单层结构。\cite{clarionv}下均译作「肩章」。}
% Als Dienstgradabzeichen erhielten Soldaten und Unteroffiziere Schulterklappen. Auch die Unteroffiziers- und die Offiziersschüler wurden durch solche gekennzeichnet. Offiziere, Generale und Admirale trugen Schulterstücke.

士兵、航空兵和水兵的肩章是用制服面料的基布制成的,并有相应兵种色的牙线。士官的肩章也是如此,他们的肩章大部分甚至全部围上了编织带。根据军衔不同,从上士和海军上士以上的军衔开始,肩章上会增加四颗13毫米的铝制小星星。星星的尖角指向肩章的扣眼。此外,士官上衣领子的前下边缘或大衣领子上也缝有一条编织带。陆军和空军肩章和上衣领子上的编织带是用铝线制成的,海军的则是金色的。
% Die Schulterklappen für Soldaten, Flieger und Matrosen wurden aus dem Grundtuch des Uniformstoffes gefertigt und mit einer Biesenumrandung aus Paspelband der jeweiligen Waffenfarbe versehen. Gleiches galt für die Unteroffiziere, deren Schulterklappen zum großen Teil oder vollständig mit einer Tresse umgeben waren. Je nach dem Dienstgrad kamen ab Dienstgrad Feldwebel und Meister vierzackige 13-mm-Sterne aus Aluminium hinzu. Sie zeigten mit einer Spitze zum Knopfloch der Schulterklappe. Außerdem war bei den Unteroffizieren am vorderen unteren Rand des Uniformjackenkragens bzw. am Kragen des Überziehers eine Tresse aufgenäht. Die Tressen auf Schulterklappen und Jackenkragen bestanden bei den Land- und Luftstreitkräften aus Aluminiumgespinst, bei den Seestreitkräften aus goldfarbener Tresse.

三个军种的士官学员都佩戴与士兵、航空兵和水兵相同的肩章,但肩章下缘有一条7毫米宽的兵种色镶边。
% Unteroffiziersschüler aller drei Teilstreitkräfte trugen die gleichen Schulterklappen wie die Soldaten, Flieger und Matrosen, führten aber an der unteren Kante der Schulterklappen ein aufgeschobenes, 7 mm breites Paspelband in der Waffenfarbe.

军官学员的徽章与士官学员的相同。不过第一年受训时,他们的徽章下缘是封上的,以后每年受训时,他们的徽章上都会增加一条7毫米宽的兵种色编织带。最初用金属“A”标明军官学员为候选人。但不久之后,就开始使用银色或金色人造丝制成的高17毫米、宽12毫米的“S”。
% Die Abzeichen der Offiziersschüler entsprachen denen der Unteroffiziere. Sie waren jedoch für das 1. Lehrjahr am unteren Rand geschlossen und für jedes weitere Lehrjahr mit einer zusätzlichen 7mm breiten Quertresse versehen.Die Paspelierung er folgte wieder in einer Waffenfarbe. Ein «A» aus Metall kennzeichnete die Offiziersschüler zunächst noch als Anwärter. Bald wurde aber das 17 mm hohe und 12 mm breite «S» aus silber- bzw. gold- farbener Kunstseide verwendet.

\begin{figure}
\includegraphics[width = \columnwidth]{./media/page (21).jpg}
\end{figure}

海军军官学员在基尔衫、水兵服衬衣和外套的左上方袖子上佩戴一枚金边或蓝边的椭圆形徽章,根据受训年份的不同,顶部有一到四个钝角。
% Die Offizierschüler der Seestreitkräfte führten am linken Oberärmel des Kieler Hemdes, der Bordbluse und des Überziehers ein ovales goldfarben- oder blauumrandetes Abzeichen und darunter, je nach Lehrjahr, ein bis vier nach oben offene stumpfe Winkel.

尉官肩章由四条银线并排平铺在兵种色的布底上制成,上有若干11.5毫米金色四角星。这些军衔星的一个尖角朝向肩章的扣眼。校官肩章由四条银线并排交织而成,末端为环状,同样平铺在兵种色的布底上。根据军衔的不同,肩章环上还缀有一至三颗13毫米的四角金星。
% Offiziere bis zum Dienstgrad Hauptmann und Kapitänleutnant waren an Schulterstücken aus vier nebeneinanderliegenden Silberplattschnüren auf der Tuchunterlage mit der jeweiligen Waffenfarbe und der Anzahlvierzackiger goldfarbener 11,5-mm-Sterne zu erkennen. Eine Spitze dieser Dienstgradsterne war zum Knopfloch des Schulterstückes gerichtet. Stabsoffiziere unterschieden sich von diesen Offizieren durch Schulterstücke aus vier nebeneinanderliegenden, viermal geflochtenen Silberplattschnüren. Diese liefen am Ende in eine Schlaufe aus und lagen ebenfalls auf einer Tuchunterlage in der Waffenfarbe. Je nach Dienstgrad wurden ein bis drei vierzackige goldfarbene 13-mm-Sterne mit einer Spitze zur Schlaufe befestigt.

\begin{figure}
\includegraphics[width = \columnwidth]{./media/page (22).jpg}
\end{figure}

将官肩章由三条线组成,外面两层是金的,内层是银的,较平。肩章末端也为环状,以鲜红色(陆军)、天蓝色(空军)或深蓝色(海军)布料为底。19毫米五角银星的数量取决于具体军衔。这些星星的尖角最初是朝外的,但不久也朝向环。
% Generale und Admirale trugen Schulterstücke aus dreimal geflochtenen Goldschnüren (2 Stück) und einer in der Mitte befindlichen Silberplattschnur. Auch sie liefen am Ende in eine Schlaufe aus und befanden sich auf hochroter (Landstreitkräfte), hellblauer (Luftstreitkräfte) oder dunkelblauer (Seestreitkräfte) Tuchunterlage. Der spezielle Dienstgrad wurde durch die Anzahl finfzakkiger 19-mm-Silbersterne bestimmt. Die Spitze dieser Sterne zeigte anfangs noch nach außen, war aber bald ebenfalls zur Schlaufe gerichtet.

除了两类肩章,海军成员的军衔还可以通过不同的金色编织带来识别。二等水兵和一等水兵在基尔衫、水兵服衬衣和外套的左上方袖子上绑上一两条长5.5厘米、宽7毫米的金色或蓝色编织带。海军下士在这些制服的相似位置佩戴一个金色或蓝色的带链船锚;海军中士佩戴的锚顶端有一个开口。海军下士和海军中士的外套上则有一个压花金属锚,而不是刺绣锚。
% Die Dienstgrade der Angehörigen der Seestreitkräfte waren, außer an den Schulterklappen und -stücken, an einem System unterschiedlicher goldfarbener Tressen festzustellen. Ober- und Stabsmatrosen nahten auf dem linken Oberärmel ihres Kieler Hemdes, ihrer Bordbluse und ihres Überziehers eine oder zwei 5,5 cm lange und 7 mm breite goldfarbene bzw. blaue Tressen auf. Der Maat führte an diesen Uniformstücken an gleicher Stelle einen goldfarbenen oder blau gestickten klaren Anker; der Obermaat einen solchen Anker mit einem oben offenen Winkel. Zum Überzieher kam bei den Maaten und Obermaaten statt des gestickten ein metallgeprägter Anker hinzu.

尽管海军二级上士和海军一级上士以及女性军官不佩戴袖条,但如前所述,海军军官将有一整套不同数量和宽度的金色编织带。在蓝色常服上衣和蓝白两色的礼服或外出上,这些编织带缝在距离两袖下缘9厘米的地方。
% Während die Meister und Obermeister sowie weibliche Offiziere keine Ärmelstreifen trugen, gab es für die Offiziere und Admirale, wie schon angeführt, ein ganzes System in Anzahl und Breite unterschiedlicher goldfarbener Tressen. Sie wurden 9 cm von der Unterkante beider Jackenärmel entfernt an der blauen Dienstjacke sowie auf der blauen und der weißen Parade-/Ausgangsjacke auf genäht.

海军将官的上衣袖条上方有一个深蓝色或白色布衬底、带有共和国徽章的五角星。
% An den Uniformjacken der Admirale befand sich über dem obersten Ärmelstreifen ein fünfzackiger Seestern mit Republikemblem auf dunkelblauer bzw. weißer Tuchunterlage. 

海军军官各军衔组也可以通过观察所佩戴的大檐帽来确定。海军尉官的大檐帽帽檐边缘有条宽约7毫米、呈钝锯齿状的金色条带,海军校官的大檐帽帽檐边缘有条宽约18毫米的金色橡树叶条带,海军将官的大檐帽帽檐边缘有条双层橡树叶条带。此外,海军军官的大檐帽上有条风带,海军将官的大檐帽上的风带由金色金属丝制成。
% Die jeweiligen Dienstgradgruppen der Offiziere und die Admirale konnten auch mit einem Blick auf die von ihnen getragene Schirmmütze bestimmt werden. So war der Mützenschirm für die Leutnante und den Kapitänleutnant mit einem am Rand entlangführenden, ungefähr 7mm breiten, stumpf gezackten goldfarbenen Streifen, für die Dienstgrade Korvettenkapitän bis Kapitän zur See mit einer etwa 18 mm breiten goldfarbenen Eichenlaubranke und für die Admirale mit einer doppelten Eichenlaubranke versehen. Außerdem befand sich an der Schirmmütze der Offiziere ein Sturmriemen und an der der Admirale eine goldfarbene Kordel.
\subsection{制服配饰}

军种之间和军种内部的区别在配饰中也很明显,至少在制作方式上是如此。这些配饰包括不同的大檐帽、船形帽和贝雷帽,以及陆军礼服服袖章上不同的兵种色。武装部队各兵种的士兵和士官使用带有皮带扣的皮带,尉官和将官使用银色皮带扣,而将官使用金色皮带扣。军官在礼服上佩戴银色或金色的礼服腰带和带扣宽腰带,海军军官的礼服腰带扣为金色。
% Differenzierungen zwischen den und innerhalb der Dienstgradgruppen aller Teilstreitkräfte zeigten sich auch in speziellen Effekten, zumindest in ihrer Fertigungsart. Dazu zahlten u.a. unterschiedliche Schirm-, Feld- und Baskenmützen sowie die verschiedenen Waffenfarben an den Ärmelpatten der Angehörigen der Landstreitkräfte an der Parade-/Ausgangsjacke. Wahrend Soldaten und Unteroffiziere aller Teilstreitkräfte ein Koppel mit Koppelschloß verwendeten, benutzten Offiziere eine silber-, Generale und Admirale eine goldfarbene Koppelschnalle. Zur Paradeuniform legten Offiziere, Generale und Admirale silber- bzw. goldfarbene Feldbinden und Schärpen mit Schloß an. Das Feldbindenschloß der Generale und Admirale sowie der Offiziere der Volksmarine war goldfarben.

\begin{figure}
\includegraphics[width = \columnwidth]{./media/page (23).jpg}
\end{figure}

将官的礼服或外出服袖子上饰有阿拉伯风花饰;鲜红色或浅蓝色的衣领上绣有金色图案\footnote{译者注:应为矢车菊图案。};长裤上饰有鲜红色或天蓝色的布条,即镶条。将官制服的领角也都绣有橡树叶。起初,空军的各军衔组也是通过领章来区分的。航空兵和士官佩戴的领章带有天蓝色饰边和银色金属片。尉官的领章上有一个半开放式刺绣橡树叶花环和一个铝纺的刺绣双翼。此外,他们的领章还镶有三股2毫米厚的铝丝纺绳。校官的领章也是由一个带双翼的封闭式橡树叶花环组成。
% Arabesken zierten die Armel der Parade-/Ausgangsjacke der Generale; goldfarben bestickt waren ihre hochroten oder hellblauen Kragenspiegel, und an ihren Uniformhosen befanden sich hochrote bzw. hellblaue Tuchstreifen, die Lampassen. Auch die Kragenecken beider Uniformarten der Admirale waren mit einer Eichenlaubstickerei versehen. Von Beginn an unterschieden sich die einzelnen Dienstgradgruppen der Angehörigen der Luftstreitkräfte auch durch ihre Kragenspiegel. Flieger und Unteroffziere führten solche mit hellblauem Besatz und einer silberfarbenen Metallschwinge. Die der Leutnante und des Hauptmanns wiesen einen halben, offenen, gestickten Eichenlaubkranz und eine gestickte Schwinge aus Aluminiumgespinst auf. Zusätzlich umrandete ihre Kragenspiegel eine 2 mm dicke, dreibiesige Aluminiumdrahtgespinstkordel. Die Spiegel der Stabsoffiziere bestanden außerdem aus einem geschlossenen Eichenlaubkranzmit Schwinge.

服役三年以上的士兵和士官在上衣的左袖下部佩戴一条角形服役年限条;服役五年以上的士兵和军士佩戴两条角形服役年限条。对于海军成员,蓝色上衣上的角形服役年限条由金黄色羊毛刺绣制成,白色上衣上的角形服役年限条由蓝色羊毛刺绣制成。
% Soldaten und Unteroffiziere der drei Teilstreitkräfte legten nach mehr als dreijähriger Dienstzeit einen einfachen spitzen Winkel am linken Unterärmel der Uniformjacke an, Für eine mehr als fünfjahrige Dienstzeit war es ein ebensolcher Doppelwinkel, Diese Winkel bestanden für die Angehörigen der Land- und Luftstreitkräfte aus Aluminiumgespinst auf steingrauer Tuchunterlage. Bei den Angehörigen der Seestreitkräfte waren die Winkel für die blaue Bekleidung aus goldgelber und für die weiße Uniform aus blauer Wollstickerei gefertigt.

对于上士,还规定了特殊标签。陆军和空军在上衣和大衣的两只袖子下端(距离下端边缘10厘米)贴上15毫米宽的铝制袖标;海军上士则贴金制袖标。
% Eine spezielle Kennzeichnung war bei den als Hauptfeldwebel eingesetzten Unteroffizieren festgelegt. In den Land-und Luftstreitkräften befestigten sie an beiden Unterärmeln der Uniformjacken und -mäntel (10cm vom unteren Rand entfernt) einen 15 mm breiten Armelstreifen aus Aluminiumgespinst; Meister und Obermeister der Seestreitkräfte nahten derartige Streifen aus Goldgespinst auf.
% 不是,一级上士和二级上士中间还夹着一个Hauptfeldwebel,把我整不会了都

许多参观者参观完展览后,与亲朋好友交流感想。一言以蔽之,他们表示:我们的国家人民军将穿上国家制服!
% Nachdem viele Besucher ihren Rundgang durch die Ausstellung beendet hatten,tauschten sie mit Arbeitskollegen,Familienangehörigen und Freunden ihre Gedanken aus. Auf einen Nenner gebracht, lauteten sie: Unsere Soldaten der NVA werden nationale Uniformen tragen!

然而,他们中也有不少人有其他的想法和疑问,这些想法和疑问也引发了其他市民和许多海外人士的思考:难道这不是那些在一战、二战中给其他国家人民带来无数苦难的德国旧军装吗?
% Nicht wenige von ihnen beschäftigten aber noch weitere Gedanken und Fragen, die auch andere Bürger und zahlreiche Menschen im Ausland bewegten: Waren dies nicht die alten deutschen Uniformen, deren Träger schon im ersten, vor allem jedoch im zweiten Weltkrieg so viel Leid über andere Völker gebracht hatten?
\subsection{国家人民军的制服}

1956 年,国家人民军陆军和空军选择石灰色制服,海军选择深蓝或白色制服,这主要是一个政治决定。
% Die Wahl der steingrauen Uniformen für die Land- und Luftstreitkrafte der NVA und der dunkelblauen bzw. weißen Uniformen für die Seestreitkräfte im Jahre 1956 war vor allem eine politische Entscheidung.

毫无疑问,对国家人民军制服的评价必须基于军队创建时对服装的要求和需要。为了让士兵在公共场合亮相,需要一种既能体现东德国家武装部队成员特征,同时又能将他们与其他穿制服的人员(包括铁路工人或邮政工人等平民以及其他武装机构人员)充分区分开来的制服。
% Eine Bewertung der Uniformen der NVA hat zweifelsohne von den Voraussetzungen und Bedürfnissen auszugehen, die bei der Schaffung der Armee bezüglich deren Bekleidung bestanden. Es wurde für das Auftreten der Soldaten in der Öffentlichkeit eine Uniform benötigt, die ihre Träger als Angehörige nationaler Streitkräfte der DDR charakterisierte und sie damit gleichzeitig von anderen Uniformierten hinreichend unterschied —— sowohl von den zivilen, beispielsweise den Eisenbahnern oder den Postangestellten, als auch von denen anderer bewaffneter Organe.

同时,制服的设计必须考虑军事要求及功能,必须根据中欧的自然环境为训练和可能的野战行动提供良好伪装。因此,从19世纪末开始,国际上用不显眼的灰色制服取代了「彩衣」,防止士兵在战场上暴露。在德国军队中,原野灰色(即灰绿色)制服一直沿用到1945年。在德国和几乎所有其他国家的海军部队中,蓝色和夏季白色制服一直是典型的制服。
% Gleichermaßen sollte die Uniform militärischen Erfordernissen Rechnung tragen und zweckmäßig gestaltet sein, Sie mußte entsprechend den natürlichen Bedingungen in Mitteleuropa für die Ausbildung und mögliche Handlungen im Gelände eine gute Tarnung gewähren. Vor allem deshalb wurde auf einen Grauton zurückgegriffen, der, seit dem international zu verzeichnenden Ersetzen des «bunten Rockes» ab Ende des 19. Jahrhunderts durch eine unauffällige Uniform, das Erkennen des Soldaten auf dem Gefechtsfeld erschwerte. In den deutschen Armeen war bis 1945 eine feldgraue, d. h. graugrüne Tönung der Uniform üblich. In den Seestreitkräften Deutschlands wie nahezu aller anderen Staaten galten blaue und im Sommer weiße Uniformen von jeher als typisch.

\begin{figure}
\includegraphics[width = \columnwidth]{./media/page (24).jpg}
\end{figure}

国家军事遗产中的其他元素也被添加到了石灰色中,石灰色作为陆军和空军的灰色变体得到了保留。这些元素包括上衣的特色剪裁(通常有四个贴袋)、大檐帽、船形帽和冬季帽的形状、兵种色的牙线、坚固的半靴以及保留两种肩章作为军衔徽章。不过,制服设计略有修改。为了促进《华沙条约》缔约国联合武装部队内部的合作,增加了军衔名称和军衔徽章,肩章上的图案使士兵和水兵的特殊军衔更容易辨认。军官军衔中增加了少尉和大将军衔。与大多数社会主义兄弟军队一样,军官肩章上也有数量相同的军衔星,以便统一识别军衔。
% Zum Steingrau als erhalten gebliebene Grauvariante der Land- und Luftstreitkräfte kamen weitere Elemente des nationalen militärischen Erbes hinzu. Dazu gehörten der charakteristische Schnitt der Uniformjacke mit den meist vier aufgesetzten Taschen, die Form der Schirm-, der Feld- und der Wintermütze, die Paspelierung in Waffenfarben, feste Halbschaftstiefel und die Beibehaltung von Schulterklappen und -stücken als Dienstgradabzeichen. Es gab jedoch einige Modifizierungen in der Uniformgestaltung. Um das Zusammenwirken in nerhalb der Vereinten Streitkräfte der Teilnehmerstaaten des Warschauer Vertrages zu erleichtern, wurden Dienstgradbezeichnungen und Dienstgradabzeichen ergänzt, Auf den Schulterklappen angebrachte Tressenstreifen lieBen die speziellen Soldaten- und Matrosendienstgrade besser erkennen. Bei den Offiziersdienstgraden kam der des Unterleutnants und des Armeegenerals hinzu. Auf den Schulterstücken der Offiziere ordnete man die gleiche Anzahl von Dienstgradsternen zur einheitlichen Kennzeichnung des Dienstgrades an wie in den meisten sozialistischen Bruderarmeen.

国家人民军的军服清晰地展现了清算军事历史遗产的意识。在东德,已经与普鲁士-德意志军事史上的军国主义遗产明显决裂。国家人民军的阶级性质和使命与以往所有德国军队有着本质区别,因为在首个德意志工农国家从军的目的,从一开始便完全是为了与兄弟军队一同保卫社会主义和国家公民的和平生活。另一方面,与反动军事史遗产的明确决裂并不意味着对整个德国军事史持虚无主义态度。虽然国家人民军在其传统中提到了群众斗争,特别是工人革命运动,以及1813至1814年进步力量在反抗拿破仑枷锁的斗争中所带来的部分遗产,但其制服却是以早期的德国军队为基础的。
% Mit der Uniformierung der NVA wurde der Sinn für das militärgeschichtliche Erbe der Vergangenheit deutlich bekundet. In der DDR war ein eindeutiger Bruch mit dem militaristischen Erbe preußisch-deutscher Militärgeschichte vollzogen worden. Klassencharakter und Auftrag der Nationalen Volksarmee unterscheiden sich prinzipiell von dem aller früheren deutschen Armeen, besteht doch der Sinn des Soldatseins im ersten deutschen Arbeiter-und-Bauern-Staat von Anfang an ausschließlich darin, an der Seite der Bruderarmeen den Sozialismus und das friedliche Leben der Bürger des Landes zu schützen. Eroberungsabsichten und Bedrohung anderer Völker und Staaten sind den neuen Streitkräften wesensfremd, Eindeutiger Bruch mit dem reaktionären militärgeschichtlichen Erbe bedeutet andererseits kein nihilistisches Verhalten zur deutschen Militärgeschichte in ihrer Gesamtheit. Während sich die Nationale Volksarmee hinsichtlich ihrer Traditionen auf den von den Kämpfen der Volksmassen, besonders der revolutionären Arbeiterbewegung, sowie auch auf den von den progressiven Kräften im Kampf gegen das napoleonische Joch 1813/14 hervorgebrachten Teil des Erbes beruft, lehnt sie sich in ihrer Uniformierung hingegen an frühere deutsche Armeen an.

这也揭示了东德国家人民军与西德联邦国防军的重大区别。联邦国防军的建立者选择了以美军为蓝本的制服样式,以此来吸引北约的合作伙伴,同时也希望转移联邦国防军与德国过去的帝国主义军队在政治上的类同。
% Hierin zeigte sich auch ein wesentlicher Unterschied zur Bundeswehr der BRD. Deren Schöpfer wählten einen an die USA-Armee angelehnten Uniformstil, hofierten so die NATO-Partner und wollten zugleich von der politischen Wesensverwandtschaft der Bundeswehr mit den imperialistischen Armeen der deutschen Vergangenheit ablenken.

同样与国家人民军制服的确定有关的是,东德的社会经济动荡与和平政策使人们有理由相信,东德、社会主义兄弟国家和其他国家的公民会承认德国土地上第一支社会主义军队的真正阶级性质。1956年1月18日上午,东德部长会议副主席在为创建国家人民军的法律草案中有关军服的规定辩护时说:「东德在国家人民军中,德国军服将具有真正的爱国意义,它表达了捍卫我们的民主成就的坚决准备。」历史最终证实了当时决定的正确性。今天,国家人民军的石灰色制服是公认的社会主义军队的象征,其宪法任务是捍卫和平并保护劳动人民的成就。
% 东德部长会议副主席W. Stoph
% Auch im Zusammenhang mit der Festlegung der Uniformen für die NVA ließen die sozialökonomischen Umwälzungen und die Friedenspolitik der DDR zurecht darauf vertrauen, daß die Birger der DDR, der sozialistischen Bruderländer und andere Staaten das wahre Klassenwesen der ersten sozialistischen Armee aufdeutschem Boden erkennen würden. «In der Nationalen Volksarmee wird die deutsche Uniform als Ausdruck der entschlossenen Verteidigungsbereitschaft unserer demokratischer Errungenschaften einen wirklichen patriotischen Sinn erhalten», hatte W. Stoph, Stellvertreter des Vorsitzenden des Ministerrates der DDR, am Vormittag des 18. Januar 1956 zur Begründung der Gesetzesvorlage für die Schaffung der NVA in bezug auf deren Uniformen ausgefihrt. Die Geschichte bestätigte schließlich die Richtigkeit der damaligen Entscheidung. Heute sind die steingrauen Uniformen der NVA anerkanntes Symbol einer sozialistischen Armee, deren Verfassungsauftrag auf die Sicherung des Friedens und des Schutzes der Errungenschaften der Werktätigen gerichtet ist.