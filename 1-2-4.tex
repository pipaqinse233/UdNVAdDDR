

Eine Geschichte der Uniformierung der NVA w¤re
nicht vollständig, ohne die vielfaltige Sonderbeklei-
dung wenigstens zu erwähnen. Nicht immer ist sie
auch eindeutig als Uniform zu bestimmen, d.h., die
Zugehörigkeit des Trägers von Sonderbekleidung
zu einem Dienstgrad bzw,zu ciner Dienstgrad-
gruppe l¤¡ßt sich nicht erkennen. Moderne Armeen
benötigen seit Mitte des 20..Jahrhunderts derartige
spezielle Bekleidungen für extreme klimatische Be-
dingungen, für verschiedene Tatigkeiten im milit¤-
rischen Dienst und für Spezialisten.
In allen Teilstreitkräften der NVA erhielten die
Kraft- und auch die Kradfahrer zus¤tzliche Beklei.
dungsst¼cke: die Kraftfahrer zum Parkdienst und
bei Reparaturen die Arbeitskombination, für Fahr.
ten im Winter ebenso wie die Kradfahrer einen
Watteanzug und Filzschaftstiefel. Zur Ausstattung
der Kradfahrer gehörten eine Kradhose, ein Krad.
regenmantel, Kradhandschuhe,Schutzbrille und
eine graue Kombination.

Die Panzerbesatzungen der Landstreitkräfte der
NVA verfügten zum Parkdienst iiber eine blaue Ar-
beitskombination, hinzu kam zur Ausbildung und
zur Fahrausbildung mit dem Panzer T-34 jene
Kopfhaube, die auch die sowjetischen Waffenbrii.
der besaßen. Des weiteren konnten die Panzerfah-
rer zu besonderen Anlässen cine graue Kombina-
tion und zur Fahrausbildung im Winter eine
Wattekombination anzichen.

Das Werkstattpersonal war generell mit blauen
Arbeitskombinationen ausgestattet. Posten erhiel-
ten im Winter bei Temperaturen unter minus 6 °C
Pelz-bzw.Ãbermäntel und Filzschaftstiefel.

In den Luftstreitkräften gab es differenzierte
Sonder- bzw. Spezialbekleidung für das fliegende
Personal,für Fallschirmspringer und das flieger-
technische Personal sowie sonstiges Flugpersonal.
Die Flugzeugführer trugen im Sommer die Flieger-
kopfhaube mit FT-Teil (darunter eine Leinenkopf.
haube), Fliegerlederjacke, Oberhemd und Binder,
Stiefelhose, Chromlederstiefel, ungefütterte Leder-
handschuhe, einen gelben Fliegerschal, Flieger-
brille, Sauerstoffmaske, ein Kappmesser und eine
Spezialkartentasche, Im Winter traten die Flieger-
kopfhaube(Winter), eine zweiteilige, wattierte
blaue Fliegerkombination, Fliegerpelzstiefel, lammfellgefütterte Stulpenhandschuhe, Fliegerpullover
und Fliegerwollschal an die Stelle entsprechender
Sommerbekleidungsstücke. Die Flugschüler zogen
im Sommer statt der Fliegerlederjacke eine ungefüt-
terte einteilige Fliegerkombination an.
Fallschirmspringer erschienen im Sommer mit
einer ledernen Fliegerkopfhaube ohne FT-Teil,
einer einteiligen Fliegerkombination, Sprungschu-
hen, Fliegerbrille und Kappmesser zum Sprung-
dienst, Im Winter waren sie mit Fliegerkopfhaube
ohne FT-Teil, einer zweiteiligen, wattierten Flieger
kombination,lammfellgefitterten Stulpenhand.
schuhen und Fliegerpullover mit Rollkragen verse-
hen.
Eine besondere Ausstattung erhielt das flieger-
technische Personal für beide ,Jahreszeiten. Flieger,
Unteroffiziere und Offiziere, die als Techniker
ihren Dienst versahen, trugen im Sommer eine un-
gefütterte einteilige schwarze Kombination, Die Offiziere setzten dazu eine Schirmmütze auf. Im Win-
ter arbeiteten sie in einer gefitterten schwarzen
Kombination,
lammfellgefütterten Handschuhen
aus Segeltuch und Filzschaftstiefeln. Start- und
Sonderposten trugen im Winter Pelz- bzw, Ãber-
mantel.
Bei den Seestreitkräften erhielten Boots- und
Schiffsbesatzungen zunächst das Recht, während
des Borddienstes folgende Bekleidungsstücke zu
tragen:Ledermantel oder lange Lederjacke, kurze
Lederjacke und Lederhose für das Maschinen- und
Sperrpersonal beim Dicnst an der Maschine oder
am Sperrgerät, Ölzeug für das Oberdeckpersonal
sowie für diesen Personenkreis auch Pelzmütze,
Watteanzug, Pelz- oder Wachmantel und Filzstiefel je nach Witterung. Unter Berücksichtigung streng
ster Sparsamkeit ordnete der Chef der Seestreit-
kräfte am 7.Juni 1957 an, Lederbekleidung als
Schutzbekleidung nur an das gesamte Personal von
Torpedoschnellbooten und an das Maschinenperso-
nal der Hochdruck-Heißdampfmaschinen auszuge-
ben, Matrosen und Maate des Maschinenpersonals
anderer Schiffe und Boote erhielten zusätzlich zur
bisherigen Ausstattung einen weiteren blauen Bord-
anzug,so daß sie während der Bordzeit über drei
derartige Uniformen verfügten, Nur die Komman-
danten und Flottillenchefs behielten Ledermantel
und -jacke.