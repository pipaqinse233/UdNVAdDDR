

Am 31. Mai 1965 unterzeichnete der Minister für
Nationale Verteidigung der DDR, Armeegeneral
H, Hoffmann, eine neue Uniformvorschrift - die
DV-10/5. Bekleidungsvorschrift der Nationalen
Volksarmee, Sie trat am 1.Dezember des .Jahres in
Kraft, Schon seit dem 1. Mai 1965 galten mit der
DV-98/4 neue Bekleidungs- und Ausrüstungsnor-
men für die Armeeangeh¶rigen.
Ein Blick in beide Grundsatzdokumente zeigt,
daß die wesentlichsten Veränderungen in der Uni-
formierung und Ausstattung der NVA-Angehörigen
mit Bekleidung und Ausrüstung der vergangenen
Jahre wie auch die unmittelbar bevorstehenden Ãn-
derungen eingearbeitet waren, Das betraf die Erset-
zung der farbigen Dienstlaufbahnabzeichen in den
Landstreitkräften und den LSK/LV durch silber.
graue, die Angleichung der Uniformen der Berufs-
soldaten mit Unteroffiziersdienstgraden an die der
Offiziere in beiden Teilstreitkräften und vor allem
die Ausstattung mit neuen Felddienstuniformen.