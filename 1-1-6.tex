\subsection{各军种的军衔徽章}

\begin{figure}
\includegraphics[width = \columnwidth]{./media/page (20).jpg}
\end{figure}

国家人民军军衔名称的确定和军衔徽章的设计也吸引了展览参观者的注意。在军事生活中,军衔在确定武装部队成员的等级方面有重要作用。1956年1月18日东德部长会议关于实行军服、军衔名称和军衔徽章的决定,确定了国家人民军的军衔。下表提供了士兵、航空兵和水兵、士官、尉官、校官和将官等军衔组别中各个军衔名称的情况。
% Auch die Festlegungen der Dienstgradbezeichnungen und die Gestaltung der Dienstgradabzeichen der NVA fanden aufmerksame Betrachter unter den Ausstellungsbesuchern. Im militärischen Leben spielen Dienstgrade, die die Ranghöhe der Armeeangehörigen bestimmen, eine wichtige Rolle. Mit dem Beschluß des Ministerrates der DDR vom 18. Januar 1956 über die Einführung der Uniform, der Dienstgradbezeichnung und der Dienstgradabzeichen wurden sie fir die NVA festgelegt. Die nachfolgende Tabelle gibt Auskunft über die ein zelnen Dienstgradbezeichnungen in den Dienstgradgruppen der Soldaten, Flieger und Matrosen, der Unteroffiziere, Maate und Meister, der Offiziere und der Generale bzw. Admirale. 

士兵和士官佩戴士兵肩章,士官学员和尉官学员人员也如此。尉官、校官和将官则佩戴军官肩章。\footnote{译者注:军官肩章为双层结构,士兵肩章为单层结构。\cite{clarionv}下均译作「肩章」。}
% Als Dienstgradabzeichen erhielten Soldaten und Unteroffiziere Schulterklappen. Auch die Unteroffiziers- und die Offiziersschüler wurden durch solche gekennzeichnet. Offiziere, Generale und Admirale trugen Schulterstücke.

士兵、航空兵和水兵的肩章是用制服面料的基布制成的,并有相应兵种色的牙线。士官的肩章也是如此,他们的肩章大部分甚至全部围上了编织带。根据军衔不同,从上士和海军上士以上的军衔开始,肩章上会增加四颗13毫米的铝制小星星。星星的尖角指向肩章的扣眼。此外,士官上衣领子的前下边缘或大衣领子上也缝有一条编织带。陆军和空军肩章和上衣领子上的编织带是用铝线制成的,海军的则是金色的。
% Die Schulterklappen für Soldaten, Flieger und Matrosen wurden aus dem Grundtuch des Uniformstoffes gefertigt und mit einer Biesenumrandung aus Paspelband der jeweiligen Waffenfarbe versehen. Gleiches galt für die Unteroffiziere, deren Schulterklappen zum großen Teil oder vollständig mit einer Tresse umgeben waren. Je nach dem Dienstgrad kamen ab Dienstgrad Feldwebel und Meister vierzackige 13-mm-Sterne aus Aluminium hinzu. Sie zeigten mit einer Spitze zum Knopfloch der Schulterklappe. Außerdem war bei den Unteroffizieren am vorderen unteren Rand des Uniformjackenkragens bzw. am Kragen des Überziehers eine Tresse aufgenäht. Die Tressen auf Schulterklappen und Jackenkragen bestanden bei den Land- und Luftstreitkräften aus Aluminiumgespinst, bei den Seestreitkräften aus goldfarbener Tresse.

三个军种的士官学员都佩戴与士兵、航空兵和水兵相同的肩章,但肩章下缘有一条7毫米宽的兵种色镶边。
% Unteroffiziersschüler aller drei Teilstreitkräfte trugen die gleichen Schulterklappen wie die Soldaten, Flieger und Matrosen, führten aber an der unteren Kante der Schulterklappen ein aufgeschobenes, 7 mm breites Paspelband in der Waffenfarbe.

军官学员的徽章与士官学员的相同。不过第一年受训时,他们的徽章下缘是封上的,以后每年受训时,他们的徽章上都会增加一条7毫米宽的兵种色编织带。最初用金属“A”标明军官学员为候选人。但不久之后,就开始使用银色或金色人造丝制成的高17毫米、宽12毫米的“S”。
% Die Abzeichen der Offiziersschüler entsprachen denen der Unteroffiziere. Sie waren jedoch für das 1. Lehrjahr am unteren Rand geschlossen und für jedes weitere Lehrjahr mit einer zusätzlichen 7mm breiten Quertresse versehen.Die Paspelierung er folgte wieder in einer Waffenfarbe. Ein «A» aus Metall kennzeichnete die Offiziersschüler zunächst noch als Anwärter. Bald wurde aber das 17 mm hohe und 12 mm breite «S» aus silber- bzw. gold- farbener Kunstseide verwendet.

\begin{figure}
\includegraphics[width = \columnwidth]{./media/page (21).jpg}
\end{figure}

海军军官学员在基尔衫、水兵服衬衣和外套的左上方袖子上佩戴一枚金边或蓝边的椭圆形徽章,根据受训年份的不同,顶部有一到四个钝角。
% Die Offizierschüler der Seestreitkräfte führten am linken Oberärmel des Kieler Hemdes, der Bordbluse und des Überziehers ein ovales goldfarben- oder blauumrandetes Abzeichen und darunter, je nach Lehrjahr, ein bis vier nach oben offene stumpfe Winkel.

尉官肩章由四条银线并排平铺在兵种色的布底上制成,上有若干11.5毫米金色四角星。这些军衔星的一个尖角朝向肩章的扣眼。校官肩章由四条银线并排交织而成,末端为环状,同样平铺在兵种色的布底上。根据军衔的不同,肩章环上还缀有一至三颗13毫米的四角金星。
% Offiziere bis zum Dienstgrad Hauptmann und Kapitänleutnant waren an Schulterstücken aus vier nebeneinanderliegenden Silberplattschnüren auf der Tuchunterlage mit der jeweiligen Waffenfarbe und der Anzahlvierzackiger goldfarbener 11,5-mm-Sterne zu erkennen. Eine Spitze dieser Dienstgradsterne war zum Knopfloch des Schulterstückes gerichtet. Stabsoffiziere unterschieden sich von diesen Offizieren durch Schulterstücke aus vier nebeneinanderliegenden, viermal geflochtenen Silberplattschnüren. Diese liefen am Ende in eine Schlaufe aus und lagen ebenfalls auf einer Tuchunterlage in der Waffenfarbe. Je nach Dienstgrad wurden ein bis drei vierzackige goldfarbene 13-mm-Sterne mit einer Spitze zur Schlaufe befestigt.

\begin{figure}
\includegraphics[width = \columnwidth]{./media/page (22).jpg}
\end{figure}

将官肩章由三条线组成,外面两层是金的,内层是银的,较平。肩章末端也为环状,以鲜红色(陆军)、天蓝色(空军)或深蓝色(海军)布料为底。19毫米五角银星的数量取决于具体军衔。这些星星的尖角最初是朝外的,但不久也朝向环。
% Generale und Admirale trugen Schulterstücke aus dreimal geflochtenen Goldschnüren (2 Stück) und einer in der Mitte befindlichen Silberplattschnur. Auch sie liefen am Ende in eine Schlaufe aus und befanden sich auf hochroter (Landstreitkräfte), hellblauer (Luftstreitkräfte) oder dunkelblauer (Seestreitkräfte) Tuchunterlage. Der spezielle Dienstgrad wurde durch die Anzahl finfzakkiger 19-mm-Silbersterne bestimmt. Die Spitze dieser Sterne zeigte anfangs noch nach außen, war aber bald ebenfalls zur Schlaufe gerichtet.

除了两类肩章,海军成员的军衔还可以通过不同的金色编织带来识别。二等水兵和一等水兵在基尔衫、水兵服衬衣和外套的左上方袖子上绑上一两条长5.5厘米、宽7毫米的金色或蓝色编织带。海军下士在这些制服的相似位置佩戴一个金色或蓝色的带链船锚;海军中士佩戴的锚顶端有一个开口。海军下士和海军中士的外套上则有一个压花金属锚,而不是刺绣锚。
% Die Dienstgrade der Angehörigen der Seestreitkräfte waren, außer an den Schulterklappen und -stücken, an einem System unterschiedlicher goldfarbener Tressen festzustellen. Ober- und Stabsmatrosen nahten auf dem linken Oberärmel ihres Kieler Hemdes, ihrer Bordbluse und ihres Überziehers eine oder zwei 5,5 cm lange und 7 mm breite goldfarbene bzw. blaue Tressen auf. Der Maat führte an diesen Uniformstücken an gleicher Stelle einen goldfarbenen oder blau gestickten klaren Anker; der Obermaat einen solchen Anker mit einem oben offenen Winkel. Zum Überzieher kam bei den Maaten und Obermaaten statt des gestickten ein metallgeprägter Anker hinzu.

尽管海军二级上士和海军一级上士以及女性军官不佩戴袖条,但如前所述,海军军官将有一整套不同数量和宽度的金色编织带。在蓝色常服上衣和蓝白两色的礼服或外出上,这些编织带缝在距离两袖下缘9厘米的地方。
% Während die Meister und Obermeister sowie weibliche Offiziere keine Ärmelstreifen trugen, gab es für die Offiziere und Admirale, wie schon angeführt, ein ganzes System in Anzahl und Breite unterschiedlicher goldfarbener Tressen. Sie wurden 9 cm von der Unterkante beider Jackenärmel entfernt an der blauen Dienstjacke sowie auf der blauen und der weißen Parade-/Ausgangsjacke auf genäht.

海军将官的上衣袖条上方有一个深蓝色或白色布衬底、带有共和国徽章的五角星。
% An den Uniformjacken der Admirale befand sich über dem obersten Ärmelstreifen ein fünfzackiger Seestern mit Republikemblem auf dunkelblauer bzw. weißer Tuchunterlage. 

海军军官各军衔组也可以通过观察所佩戴的大檐帽来确定。海军尉官的大檐帽帽檐边缘有条宽约7毫米、呈钝锯齿状的金色条带,海军校官的大檐帽帽檐边缘有条宽约18毫米的金色橡树叶条带,海军将官的大檐帽帽檐边缘有条双层橡树叶条带。此外,海军军官的大檐帽上有条风带,海军将官的大檐帽上的风带由金色金属丝制成。
% Die jeweiligen Dienstgradgruppen der Offiziere und die Admirale konnten auch mit einem Blick auf die von ihnen getragene Schirmmütze bestimmt werden. So war der Mützenschirm für die Leutnante und den Kapitänleutnant mit einem am Rand entlangführenden, ungefähr 7mm breiten, stumpf gezackten goldfarbenen Streifen, für die Dienstgrade Korvettenkapitän bis Kapitän zur See mit einer etwa 18 mm breiten goldfarbenen Eichenlaubranke und für die Admirale mit einer doppelten Eichenlaubranke versehen. Außerdem befand sich an der Schirmmütze der Offiziere ein Sturmriemen und an der der Admirale eine goldfarbene Kordel.