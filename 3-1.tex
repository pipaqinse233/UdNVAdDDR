

Mit dem Ziel der Erhaltung und Festigung des
Friedens und des sicheren Schutzes des Sozialismus
hatten die Streitkräfte der DDR die Aufgabe, die
Staatsgrenzen, das Territorium, den Luftraum und
das Küstenvorfeld der DDR sowie die verbündeten
sozialistischen Staaten gemeinsam mit der Sowjet
armee und den im Warschauer Vertrag geeinten so-
zialistischen Armeen auch in diesen beiden .Jahr.
zehnten zuverlässig zu schützen.
Die DDR hatte zu berücksichtigen, daß die im
Oktober 1969 gebildete neue BRD-Regierung un-
geachtet ihrer Entspannungspolitik die in der
NATO getroffenen Festlegungen in ihrer Militär-
politik umsetzte. Spürbar wurde dies in einem wei-
teren, vor allem qualitativ orientierten Ausbau der
Bundeswehr in den 70er Jahren.
Die Erfüllung des militärischen Auftrages durch
die NVA als Kern der Landesverteidigung der
DDR bedingte deren planmäßige Vervollkomm-
nung.Sie schloß die kontinuierliche Modernisie-
rung und Neuausrüstung der Teilstreitkräfte, Waf.
fengattungen, Spezialtruppen und Dienste der
NVA genauso ein wie die Verbesserung der Dienst-
und Lebensbedingungen für die Angeh¶rigen der
Streitkräfte, einschlieBlich ihrer persönlichen Be-
kleidung und Ausr¼stung.

Die Landstreitkräfte wurden in der ersten Hälfte
der 70er,Jahre mit weiterentwickelten operativ-tak-
tischen und taktischen Raketen ausgerüstet, Neue
Artilleriesysteme wie die 122-mm-Haubitze D-30,
die 152-mm-Kanonenhaubitze D-20 und der auf
dem Tatra 813 montierte vierzigrohrige Geschoß.
werfer RM-70 mit Nachladeeinrichtung trugen zur
spiürbaren Erhöhung der Feuerkraft bei. Die Ge.
fechtsmöglichkeiten der Truppenluftabwehr wuch-
sen durch die Einführung der Ein-Mann-Fla-Ra-
kete «Strela II» in die mot. Schützeneinheiten, von
Fla-Raketenkomplexen in die mot. Schützentrup-
penteile und von Mehrfach-Raketenstartrampen
einschlieBlich automatischer Feuerleitkomplexe in
die Verbände der Landstreitkräfte. Den Fla-Rake-
tentruppen derLuftstreitkräfte/Luftverteidigung
wurden moderne Fla-Raketen-und Feuerleitsy-
steme zur Vernichtung gegnerischer Luftangriffs-
mittel zugeführt. Die ,Jagdfliegerkräfte erhielten
modifizierte Versionen des Abfang-Jagdflugzeuges
MiG-21 mit st¤rkerer Bewaffnung, höheren Trieb-
werkleistungen und universellen Einsatzm¶glich-
keiten.
Mit Beginn des Ausbildungsjahres 1971/72
wurde die Volksmarine in den gemeinsamen Ge
fechtsdienst der sozialistischen Ostseeflotten einbezogen, Die neugebildeten Raketen-Torpedoschnell.
bootbrigaden erhielten modernere Kampftechnik
Anstelle der leichten Torpedoschnellboote der Ty
pen Wolgast»und Berlin》 traten Anfang der
70er Jahre kleine Torpedoschnellboote mit wesent-
lich verbesserten Kampfeigenschaften.Kampfwert
und Gefechtsmöglichkeiten der Volksmarine erhöh.
ten sich außerdem durch die Indienststellung von
mittleren Landungsschiffen und von Hochseever-
sorgern.
Die planmäßige Vervollkommnung der Landes
verteidigung und die Erfüllung ihres militärischen
Auftrages stellten in erster Linie neue und hóhere
Anforderungen an die Soldaten, Matrosen, Unter.
offiziere,MMaate,eister,Offiziere,Generale und
Admirale der Armee, Jetzt erst recht war es wichtig,
klassenbewußte K¤mpfer zu erziehen, die fahig wa-
ren,das sozialistische ilitärwesen zu meistern, die
modernen Führungsmittel,Waffenkomplexe und
Geräte zu beherrschen, um durch ihren Friedens.
dienst die DDR zuverlässig militärisch zu si
chern.
Die Armeef¼hrung sorgte sich genauso intensiv
wie um die Erhöhung der Kampfkraft und Ge
fechtsbereitschaft um die Verbesserung der Dienst-
und Lebensbedingungen der Angehörigen der
Streitkräfte, Das Präsidium des inisterrates der
DDR legte am 5.Juli 1972 in einem Beschluß über
Maßnahmen zur Verbesserung der Lage der Be
rufssoldaten fest, die Dienst- und Lebensbedingun-
gen dieser Armeeangehörigen weiter zu verbessern
Auf der Grundlage dieses Beschlusses wurden auch
weitgehende Veränderungen in der Uniformierung
der NVA eingeleitet und schrittweise verwirk-
licht.