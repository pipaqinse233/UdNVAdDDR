

Im folgenden werden Festlegungen zur.Trageweise
der Uniformstücke in den Teilstreitkräften für die
Jahre 1957 bis 1960 wiedergegeben.Grundlegendc
Festlegungen haben sich bis in die Gegenwart nicht
geändert. Die einzelnen Mützenarten waren so auf.
zusetzen, daB sich die Kokarde immer in der Ver-
l¤ngerung der Mittellinie des Gesichts befand. Der
untere Schildrand der Schirm- und der Winter
mütze sollte mit den Augenbrauen abschlieBen, die
Feldmütze aber rechts einen Fingerbreit iber der
Augenbraue sitzen. Der waagerecht zu tragende
Stahlhelm muBte mit seinem vorderen Rand in
Höhe der Augenbrauen liegen.
Der Sitz der Uniformjacke ist bereits beschrieben
worden. Alle Angehörigen der Landstreitkräfte
kn'pften aus hygienischen Grinden eine weiBe
Kragenbinde so innen in den ]ackenkragen ein, daß
ein 2 mm breiter Rand gleichm¤ßig iberstand.
Die Soldaten und Unteroffiziere der Land- und
Luftstreitkräfte steckten bei Halbschaftstiefeln die
Uniformhbse zur Drillich-,zur Dienst- und zur Pa
radeuniform in die Stiefel, Dazu schlugen sie die
Hosenbeine von hinten nach vorn außen ein. Ahn.
lich verfuhren die atrosen und die Maate mit
ihren Bordhosen zu den Halbschaftstiefeln beim
Exerzieren und in der Schießausbildung.Dagegen
lieBen sie zum Wachdienst, bei der Stellung von
Ehrenkompanien und auf Befehl ihre Klapphosen
über die Stiefel auf den sogenannten halben Schlag
fallen, d.h., sie legten die Hosenbeine zweimal nach
außen zu einem 4cm breiten [mschlag um.
Komplizierte Bestimmungen regelten die Ver
wendung des Kieler Kragens und des seidenen
Halstuches durch die Matrosen und Maate. Die Dienst-,die Parade- und die Ausgangsuniformen
mit Kieler Hemd wurden durch den Kieler Kragen
und das seidene Halstuch ergänzt. Dagegen wurde
beim Exerzieren, bei der SchieBausbildung und
zum Unterricht zwar nicht auf den Kieler Kragen
zum blauen oder weißen Bordanzug,iedoch auf das
Halstuch verzichtet, Backschafter(das Küchenper-
sonal)erschienen im weißen Bordanzug mit Kieler
Kragen,atrosen zum Arbeitsdienst an Land oder
an Bord aber immer ohne. 【hre Freizeit verbrachten
die Matrosen und Maate im weißen Bordanzug mit
Kieler Kragen.Die Kommandeure konnten zu Kul.
turveranstaltungen sowie an Sonnabenden, Sonnta
gen und gesetzlichen Feiertagen zusätzlich das sei
dene Halstuch befehlen.
Feste Regeln gab es fir das Umschnallen von
Koppel und Feldbinde bzw.Schärpe. Schloß oder
Schnalle saBen stets in der itte der Knopfréihe
oder zwischen beiden Knopfreihen. Soldaten und
Unteroffiziere der Land- und Luftstreitkräfte sowie
Offiziere der Landstreitkräfte trugen das Koppel
zwischen dem vierten und fünften Knopf(von
oben)der Uniformjacke,Offiziere der Luftstreit.
kräfte so, daß der unterste Knopf verdeckt blieb,
Wurde das Koppel über dem Uniformmantel getra
gen, so befand es sich ebenfalls zwischen dem vier.
ten und fünften Knopf von oben. Diese Trageweise
war für alle Dienstgrade der drei Teilstreitkräfte
bindend,Offiziere und Generale der Land- und
Luftstreitkräfte schnallten zur Paradeuniform die
Feldbinde wie das Koppel um; Offiziere und Admi
rale der Seestreitkräfte legten ihre Sch¤rpe zwi
schen dem dritten und vierten Knopf der Parade.
iacke an.
Die Matrosen und Maate zogen das Koppel zur
Dienst-,zur Parade- und zur Ausgangsuniform
durch die Schlaufen der Klapphose bzw. beim
Exerzieren,bei der SchieBausbildung und im Un
terricht durch die Schlaufen der Bordhose. Zum
täglichen Dienst und in der Freizeit konnte das
Koppel weggelassen werden. Im Winterhalbjahr saß
das Koppel zwischen dem vierten und fiinften
Knopf des Uberziehers. Meister trugen es bei bestimmten Anlässen wie Exerzieren, SchieBausbil-
dung, Wachdienst und Paraden; Offiziere bei glei-
chen Gelegenheiten, jedoch nicht zu Paraden.
In den Uniformmantel oder in den berzieher
konnte der Schal glatt, d, h, nicht geknotet, einge.
legt werden, Einige weitere Einzelbestimmungen
seien noch erwähnt: Hauptfeldwebel und Unteroffi-
ziere der Land- und Luftstreitkräfte, die den Dienst
in Offiziersplanstellen versahen (z. B. als Zugführer)setzten zum Dienst, wenn nicht anders befoh-
len, die Schirmmütze auf, Offizieren war es gestat-
tet, zum Innendienst die Uniformhose zur Dienst-
uniform bzw. auch Stiefelhose und Dienstjacke
ohne Koppel zu tragen.