Stark verändert hatten sich nach dem Inkrafttreten
der DV.10/5, Ausgabe 1965, die Dienstlaufbahnab
zeichen der Landstreitkräfte und der LSK/LV
Dies betraf zuallererst ihr äuBeres Aussehen. Für
beide Teilstreitkräfte der NVA fielen die bisherigen
farbigen Dienstlaufbahnabzeichen weg. Sie wurden
durch silbergrau gestickte Abzeichen auf ovaler
steingrauer Tuchunterlage,6cm hoch und 5cm
breit, ersetzt, Für Offiziere des medizinischen Dien.
stes blieben der Askulapstab, für die der Musik
korps die Lyra, aus goldfarbenem Metall geprägt,
als Dienstlaufbahnabzeichen auf den Schulterstük
ken.Soldaten und Unteroffiziere der Musikkorps
f¼hrten die Lyra aus silberfarbenem Metall auf den
Schulterklappen.
Dienstlaufbahnabzeichen fǔrFlugzeugführer
und für Offiziere des ingenieurtechnischen Dien
stes gab es seit 1963 in einer ver¤nderten Ausfiih-
rung.Diese waren im Zusammenhang mit einer
neuen Form von Klassifizierungsabzeichen entwik
kelt worden und aus emailliertem Metall gefertigt.
Sie wurden bis 1983 verliehen. Ganz und gar fie!
das Dienstlaufbahnabzeichen für Steuerleute weg
Offiziere des Fallschirmdienstes bekamen ebenfalls
ein Dienstlaufbahnabzeichen in der bereits be-
schriebenen silbergrauen Ausfiihrung.
Folgende Dienstlaufbahnabzeichen schrieb die
neue Bekleidungsvorschrift für die Landstreitkräfte
und die LSK/LV der NVA vor: Panzer; Artillerie;
Pioniere(Landstreitkräfte);Aufklärer;chemischer
Militärtransportwesen;Kommandanten
Dienst;
dienst; panzertechnischer Dienst; Artillerie- und
waffentechnischer Dienst;kraftfahrzeugtechni
scher Dienst; Fallschirmjäger, Soldaten und Unter.
offiziere; Fallschirmjäger,Offiziere; Funkortung:
Schirrmeister;Feuerwerker; medizinischer Dienst,
Soldaten und Unteroffiziere; Meteorologen (LSK
LV); Flugzeugmechaniker, Fallschirmdienst, Solda-
ten und Unteroffiziere; Fallschirdienst,Offiziere; Flugschüler;fliegertechnische Versorgung;Kano
niere der uftverteidigung; Pioniere und Flugplatz
wartungsdienst (LSK/LV).Schon in die Vorschrift
aufgenommen,aber erst Ende 1966 best¤tigt wur.
den Dienstlaufbahnabzeichen fǔr Offziere der Mi
litärjustizorgane und fir Soldaten und Unteroffi
ziere der ilitärjustizorgane.
Für die Volksmarine blieben die Dienstlaufbahn.
abzeichen für Matrosen, Maate, Meister und Offi.
ziere sowie die Abzeichen für Sonderausbildung der
Matrosen und Maate in den schon bekannten Aus
fihrungen zur blauen und zur weißen Uniform er
halten, Hinzu kam, wie in den beiden anderen Teil.
streitkräften der NVA,das Dienstlaufbahnabzei
chen Justiz für Maate und für die juristische
Laufbahn der Meister und Offiziere. Festgelegt
wurde auch,daß die Offiziersschüler des 1. und
2.Lehrjahres Laufbahnabzeichen in der bisherigen
Form beibehielten. Die Offiziersschüler des 3. und
4.Lehrjahres führten bereits die Dienstlaufbahnab
zeichen der Offiziere, d.h. goldfarbene Stickerei auf
runder Tuchunterlage von 4 cm Durchmesser.
Eine Reihe neuer Abzeichen wurde für die abge.
schlossene Sonderausbildung der Matrosen und
Maate der Volksmarine mit der 1965er Beklei-
dungsvorschrift festgelegt. Hier handelte es sich um
solche für Turbine,Hydroakustik,Funkmeß, Elek-
tro-Nautik,E-MeB,Waffenleit,Ari-Elektriker und
chemischen Dienst. Mit ihrer Einführung spiegel
ten sich auf diesem spezifischen Gebiet der Unifor.
mierung auch schon ¤ußerlich die gewachsenen An.
forderungen an die Meisterung neuer Militärtech.
nik durch die atrosen und Maate der Volksra-
rine wider. Die Abzeichen fǔr Sonderausbildung
wurden um 1cm,d.h.von 6cm auf 7cm Durch-
messer vergróBert.
lm Zusammenhang mit der Darstellung der
ncuen Dienstlaufbahnabzeichen
bietet sich
an,
auch auf die Uniformierung der Bausoldaten der NVA einzugehen, Eine Anordnung des Nationalen
Verteidigungsrates der DDR vom 7.September
1964 regelte die Aufstellung von Baueinheiten im
Bereich des Ministeriums für Nationale Verteidi-
gung der DDR. Sie ermöglicht es jenen wehrpflich-
tigen DDR-Bürgern, die aus religiösen Gründen
den Dienst mit der Waffe ablehnen, durch Arbeits-
leistungen an der Stärkung der Verteidigungsfahig-
keit des Landes teilzunehmen. Die Bausoldaten er-
hielten 1966 die steingrauen Uniformen der
Landstreitkräfte der NVA, die in der Waffenfarbe
Oliv paspeliert waren und sind. Auf den Schulter-
klappen befand sich als weitere Symbolik ein ma-
schinengestickter Spaten.Gegenwärtig ist dieses
Symbol eine Metallprägung.