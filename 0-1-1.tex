\subsection*{制服小课堂}

下文将对属于国家人民军制服的部分服装和装备项目及相应术语进行简要说明。由于名称、设计和尺寸方面的频繁变化,我们不可能也无意对这些细节的历史发展做出完整的说明。
% In der Folge werden ausgewählte, zur Uniformierung der NVA gehörende Bezeichnungen von Bekleidungs- und Ausrüstungsstücken sowie entsprechende Begriffe kurz erläutert. Aufgrund der häufigen Wandlungen gerade hinsichtlich der Benennungen, der Formgestaltung und auch der Abmessungen ist eine nochmalige vollständige Darstellung der historischen Entwicklung dieser Details nicht möglich und auch nicht beabsichtigt.

\subsubsection*{Abzeichen für Sonderausbildung für Matrosen und Maate/岗位袖章}

蓝底或白底\footnote{常、礼服用袖章由深蓝色马裤呢制成,社交服用袖章由白色马裤呢制成。\cite{clarionv}}红色刺绣——1965年前直径为6厘米,后来改为7厘米;人民海军的水兵和海军下士在通过考核后佩戴在→技术袖章下方;最多允许佩戴2枚袖章。见图1
% rote Stickerei auf blauer oder weißer Tuchunterlage — bis 1965 mit 6 cm, dann mit 7cm Durchmesser; von Matrosen und Maaten der Seestreitkräfte/Volksmarine bei bestandener Prüfung unter → dem Dienstlaufbahnabzeichen getragen; bis zu 2 Abzeichen waren gestattet. Abb. 1 

\subsubsection*{Achselschnur/饰绪}% 

由银色或金色金属丝编织而成的两条粗绳环和两条窄绳,末端是两个镀银或镀金的金属尖端,自1976年以来,男性军官都佩戴在礼服、外出服以及社交服中的上衣上。见图2、图3
% dekoratives Geflecht zweier starker Schnurschlingen und zweier schmaler Schnüre aus silber- oder goldfarbenem Metallgespinst, die in zwei versilberten bzw, vergoldeten Metallspitzen auslaufen; von männlichen Offizieren und von Generalen und Admiralen seit 1976 zur Parade- und Ausgangsuniform sowie zum Großen Gesellschaftsanzug der Gesellschaftsuniform angelegt. Abb. 2, 3

\subsubsection*{Anker, klarer und unklarer/船锚(无链或带链)}

人民海军的象征,用作→技术袖章。见图4、图5
% Symbolik der Seestreitkräfte/Volksmarine, insbesondere für die Gestaltung der → Dienstlaufbahnabzeichen genutzt. Abb. 4, 5 

\subsubsection*{Ankerknopf/带锚纽扣}% 

人民海军制服上的压花纽扣,图案为带链船锚。见图6
% besondere Prägung der Knöpfe an den Uniformen der Seestreitkräfte/Volksmarine in Gestalt des unklaren Ankers. Abb. 6 

\subsubsection*{Arabeske/阿拉伯风装饰}

在将官外出服(1981 年之前)和礼服(1986 年之前)袖口上用金色丝线绣制或压印而成的成对卷须状装饰。见图7
% rankenförmige Verzierung aus goldfarbenem Gespinst, paarweise in die Ärmelaufschläge der Ausgangsuniform für Generale (bis 1981) und in die der Paradeuniform für Generale (bis 1986) eingestickt bzw. eingeprägt. Abb. 7

\subsubsection*{Ärmelabzeichen/袖章、准尉臂章}

\begin{description}

    \item[袖章] 人民海军军衔的徽章,又称袖章;自1983年起,水兵、海军士官学员(非长期海军士官学员)和海军士官的袖标明确分开,蓝色制服为金色图案,白色制服为蓝色图案,贴在→基尔衫和→短大衣的左上袖子中间,男性海军准尉和海军尉官、海军校官以及海军将官的袖标(金色)贴在上衣袖子下方。见图8、图9;
    % a) Kennzeichnung des Dienstgrades in den Seestreitkräften/Volksmarine, auch als Ärmelstreifen bezeichnet; seit 1983 deutliche Trennung in Ärmelabzeichen für Matrosen, Unteroffiziersschüler (nicht Berufsunteroffiziersschüler) und Maate in goldfarbener Ausführung für die blaue Uniform und in blauer Ausführung für die weiße Uniform, in der Mitte des linken Oberärmels des → Kieler Hemdes und des → Überziehers angebracht, sowie in Ärmelstreifen (goldfarben) für männliche Fähnriche und Offiziere sowie für Admirale an beiden Unterärmeln der Uniformjacken. Abb. 8, 9;

    \item[准尉臂章] 国家人民军军衔徽章中另一类准尉徽章;由制服布料编织而成,并绣有东德国徽;1979年之前,服役超过10年、15年或20年的少尉可依次获得一至三颗星\footnote{译者注:后期臂章取消四角星。\cite{clarionv}};缝在制服上衣和大衣的左上袖子上。见图10 
    % b) Kennzeichnung der Fähnriche in der NVA zusätzlich zu den → Dienstgradabzeichen; aus Uniformtuch gewebt und mit dem Staatsemblem der DDR versehen; zusätzlich bis 1979 ein bis drei Sterne für mehr als 10-, 15- bzw. 20jährige Dienstzeit; am linken Oberärmel der Uniformjacken und -mäntel aufgenäht. Abb. 10 

\end{description}

\subsubsection*{Ärmelpatte/礼服袖章}

在陆军和空军(1961年起)至1981年期间,成对出现在礼服或出行服的袖口上;士兵、士官学员、士官和尉官学员的制服布料上绣有银色刺绣,且有兵种色牙线(1980年起,陆军一般为白色,而空降兵为橙色);少尉、尉官和校官则是饰有兵种色坎肩(→坎肩)的银色刺绣(自1977年起,陆军一般为白色,而空降兵为橙色)。图11、12
% bei den Landstreitkräften und den LSK/LV (ab 1961) bis 1981 auf den Ärmelaufschlägen der Parade-/Ausgangsjacken paarweise angebrachtes Zierelement; für Soldaten, Unteroffiziersschüler, Unteroffiziere und Offiziersschüler als silberfarbene Stickerei und mit einem Mittelstreifen in der → Waffenfarbe (seit 1980 in den Landstreitkräften generell Weiß bzw. Orange für Fallschirmjäger) auf Uniformtuch; für Fahnriche und Offiziere als silberfarbene Stickerei mit einer Kantillenfüllung (→ Kantille) in der Waffenfarbe (seit 1977 in den Landstreitkräften Weiß bzw. Orange für die Fallschirmjäger). Abb. 11, 12

\subsubsection*{Ärmelstreifen/袖条、袖标}

\begin{description}

    \item[袖条] 用15毫米宽的铝线或金线为海军士官(自1974年起也含海军准尉)制作的徽章;1960年以前为缝在上衣和大衣距双袖下摆10厘米处,此后为13厘米。见图13;
    % Kennzeichnung für die in der Dienststellung Hauptfeldwebel bestätigten Unteroffiziere und Meister (seit 1974 auch Fähnriche) aus 15 mm breitem Aluminium bzw. Goldgespinst; bis 1960 10 cm, danach 13 cm vom Ärmelsaum entfernt an beiden Unterärmeln der Uniformjacken und -mäntel aufgenäht. Abb. 13;

    \item[袖标] 国家人民军警卫团和埃里希-魏纳特军乐团成员以及军乐学院学员的徽章由3厘米宽的底布制成,边缘绣有刺绣,中间绣有相应名称;佩戴于上衣和大衣距左袖下摆13厘米处。见图14
    % b) Kennzeichnung der Angehörigen der Wachregimenter der NVA und des Erich-Weinert-Ensembles sowie der Militärmusikschüler aus einem 3 cm breiten Grundgewebe, gestickten Kanten und entsprechenden Aufschriften; 13 cm vom Ärmelsaum entfernt auf dem linken Unterärmel der Uniformjacken und -mäntel angebracht. Abb. 14

\end{description}

\subsubsection*{Äskulapstab/阿斯克勒庇俄斯之杖}%  

古希腊掌管医药和治疗的神阿斯克勒庇俄斯手持一杖,杖上盘绕着一条蛇;国家人民军医疗人员的技术袖章便采用这一符号。见图15
% in der Antike wurde Äskulap (Asklepios), Gott der Heilkunde, mit einem großen knotigen Stab dargestellt, um den sich eine Schlange windet; Symbolik der → Dienstlaufbahnabzeichen für Angehörige des medizinischen Dienstes der NVA. Abb. 15

\subsubsection*{贝雷帽}% Baskenmütze

1962年之前军队中女性成员佩戴的帽子(自1983年起以现代形式再次成为野战制服的一部分)以及伞兵佩戴的帽子;每种的形状和颜色都不同。见图16
% Kopfbedeckung weiblicher Armeeangehöriger bis 1962 (seit 1983 in zeitgemäßer Form wieder zur Felddienstuniform gehörend) und der Fallschirmjäger; jeweils in unterschiedlicher Form und Farbgebung. Abb. 16

\subsubsection*{国家人民军服装与设备处}% Bekleidungs- und Ausrüstungsdienst (B/A-Dienst) der NVA

国家人民军后方勤务的一部分,提供制服、住宿用亚麻布、帐篷(包括设备、办公机器和材料)船只及旗帜。
% Teil der rückwärtigen Dienste der NVA, der die Truppen mit Uniformen, Unterkunftswäsche, Zelten einschließlich Einrichtungen, Büromaschinen und -materialien versorgt sowie Schiffe und Boote mit Flaggen ausstattet.

\subsubsection*{水手帽}% Bordkäppi

人民海军成员的头饰,形同野战帽。
% Kopfbedeckung der Angehörigen der Seestreitkräfte/Volksmarine in Form der → Feldmütze.

\subsubsection*{军衔徽章}% Dienstgradabzeichen

在制服上编织、压印或刺绣的标志,以袖章、肩章等形式确定军衔;也可作为编织形式的条纹系统(辫子),特别是在作战服(表面印花)、特种服装和运动服、飞行员服和技师服上;自 1986 年起,作为银色和金色设计的条纹和星形系统出现在飞行员服和技师服的胸前。图 17、18、19  TODO
% gewebte, geprägte oder gestickte Kennzeichnungen an der Uniform zur Bestimmung des militärischen Ranges in Gestalt von → Ärmelabzeichen,→ Schulterklappen, →Schulterstücken; auch als System von Streifen in Tressenform (→ Tresse) vor allem am → Kampfanzug (im Flächendruck), an Sonder- und Sportbekleidung, Flieger- und Technikeranzügen sowie seit 1986 als ein System von Streifen und Sternen in silber- und goldfarbener Ausführung auf der Brustseite der Flieger- und Technikeranzüge vorhanden. Abb. 17, 18, 19

\subsubsection*{技术袖章}% Dienstlaufbahnabzeichen

1957年起主要用于士兵和士官的特殊标志,在某些情况下也用于完成特殊训练的准尉和尉官。在陆军和空军中,1965年之前都是彩色,之后是石灰色布底上的银色刺绣,部分也出现在→肩章和→肩章上(军乐团成员和医疗人员),1986年,这两个军种的数量大大减少;而在海军中,蓝色或金色设计几乎没有变化。见图20、21、22、23
% seit 1957 besondere Kennzeichnung vor allem für Soldaten und Unteroffiziere, z.T. aber auch für Fähnriche und Offiziere, die über eine abgeschlossene Spezialausbildung verfügen und entsprechend eingesetzt werden; in den Landstreitkräften und den LSK/LV bis 1965 in farbiger Ausführung, dann in silberfarbener Stickerei auf steingrauer Tuchunterlage, teilweise auch auf den → Schulterklappen und → Schulterstücken (Musik- und Sanitätsdienst); in beiden Teilstreitkräften bis 1986 zahlenmäßig stark reduziert,in der Volksmarine nahezu unverändert in blauer bzw. goldfarbener Ausfuhrung. Abb. 20, 21, 22, 23

\subsubsection*{荣誉短佩剑}% Ehrendolch

1961年出现的男性尉官、校官以及将官的代表性随身武器,有各种材质和颜色,可与外出服、礼服和社交服一起佩戴。见图24
% 1961 eingeführte repräsentative Seitenwaffe der männlichen Offiziere sowie der Generale und Admirale in unterschiedlicher Material- und Farbausfuhrung; zur Ausgangs-, Parade- und Gesellschaftsuniform mitgefihrt. Abb. 24

\subsubsection*{空降兵系带靴}% Fallschirmjägerschnürstiefel

为空降兵特别设计的鞋,可承受跳伞时的压力。
% spezielle, auf die Belastung beim Sprung ausgerichtete Fußbekleidung der Fallschirmjäger. Abb. 25

\subsubsection*{式样、前襟}% Fasson

或指服装的剪裁;或指上衣的前襟,包括衣领和翻领。
% Bezeichnung für a) den allgemeinen Schnitt eines Bekleidungsstückes oder b) die Vorderseite der Uniformjacke, bestehend aus Kragen und Revers.

\subsubsection*{礼服腰带}% Feldbinde

礼服腰间的宽编织带;男性尉官和校官为银色,将官为金色。见图26
% um die Hüfte getragenes breites Tressenband (→ Tresse) zur Paradeuniform; silberfarbene Ausführung für männliche Offiziere, goldfarbene für Generale und Admirale. Abb.26

\subsubsection*{野战服}% Felddienstanzug

见作战服。
% → Kampfanzug

\subsubsection*{船形帽}% Feldmütze

由统一织物制成的轻型低小头盔,适用于武装部队各军种全体官兵(在人民海军中称为水手帽)。见图27
% leichte, niedrige Kopfbedeckung aus Uniformgewebe für alle Dienstgrade aller Teilstreitkräfte (bei den Seestreitkräften/Volksmarine als → Bordkäppi bezeichnet). Abb.27

\subsubsection*{的确良}% Grisuten

涤纶面料的商品名。
% Bezeichnung für Polyesterfaserstoffe.

\subsubsection*{半靴}% Halbschaftstiefel

士兵和士官的皮制鞋子,用于常服、工作服、野战服和礼服。见图28
% lederne Fußbekleidung für Soldaten und Unteroffiziere zur Dienst-, Drillich-, Felddienst- und Paradeuniform. Abb. 28

\subsubsection*{疏水化}% Hydrophobieren

使用添加剂对纺织品进行精加工,以产生防水浸渍效果。
% Veredelung von Textilien durch Hilfsmittel zu einer wasserabweisenden Imprägnierung.

\subsubsection*{勋表}% Interimsspange

长方形的勋章缎带,1973年以前用纺织面料制成,后来改用美术纸,有塑料套保护;用于水手服、常服、参谋常服、外出服和社交服(小社交服)。
% in eine rechteckige Form gebrachtes Band einer Auszeichnung, bis 1973 aus textilem Gewebe, dann Kunstdruckpapier, geschützt durch eine Plastabdeckung; bestimmt für die Bord-, Dienst-, Stabsdienst-, Ausgangs- und Gesellschaftsuniform (Kleiner Gesellschaftsanzug).

\subsubsection*{精梳纱线}% Kammgarn

光滑细长的纤维羊毛纱线,与→粗纺纱线的不同之处在于其更均匀,表面更光滑;用作全体军官制服的材料。
% feines, glattes, langfaseriges Wollgarn, das sich vom → Streichgarn durch eine größere Gleichmäßigkeit und glattere Oberfläche unterscheidet; Material für die Uniformen vor allem der Offiziere.

\subsubsection*{作战服}% Kampfanzug 

自1965年起称作野战服,在夏季、春秋季(自 1983 年起)和冬季都有不同的设计;在20世纪60年代中期之前,主要用于作战训练,但后来士兵和士官也在日常服役中穿着。
% (seit 1965 als Felddienstanzug bezeichnet) bestimmender Teil der Felddienstuniform in einer unterschiedlichen Ausführung für die Sommer-, Übergangs- (seit 1983) und Wintertrageperiode; bis Mitte der 60er Jahre für Handlungen in der Gefechtsausbildung bestimmt,danach aber von Soldaten und Unteroffizieren auf Zeit auch im täglichen Dienst getragen.

\subsubsection*{捻丝}% Kantille

细螺旋状丝线,用于生产礼服袖章和领章。
% feines, spiralig gedrehtes Gespinst, für die Fertigung von → Ärmelpatten und → Kragenspiegel verwendet.

\subsubsection*{便帽}% Kappe

1962年为军中女性成员推出的头饰,作为常服、参谋常服、外出服和社交服的一部分,采用其经典颜色。见图29
% 1962 eingeführte Kopfbedeckung weiblicher Armeeangehöriger zur Dienst-, Stabsdienst-, Ausgangs- und Gesellschaftsuniform in der für die Teilstreitkräfte typischen Farbgebung. Abb. 29

\subsubsection*{基尔衫}% Kieler Hemd

为纪念1918年11月3日的基尔水兵起义,水兵和海军士官的特制衬衣采用了蓝白相间的款式;穿常服、外出服和礼服时,也要与丝绸领巾和基尔领一起穿着。见图30
% in Erinnerung an den Kieler Matrosenaufstand vom 3. November 1918 gewählte Bezeichnung der spezifischen Uniformbluse der Matrosen und Maate in einer blauen und weißen Ausführung; bei der Dienst-, Ausgangs- und Paradeuniform zusammen mit dem seidenen Halstuch und dem Kieler Kragen getragen. Abb. 30

\subsubsection*{帽徽}% Kokarde

头饰上的圆形徽章,表明是国家武装部队或其他武装机构的成员;在国家人民军中最初只有黑红金三色,自1962至1963年起为东德国徽。见图31、32
% an der Kopfbedeckung befindliches kreisförmiges Abzeichen, das die Zugehörigkeit zu den Streitkräften bzw. zu anderen bewaffneten Organen des Landes kennzeichnet; in der NVA zunächst nur schwarz-rot-gold, seit 1962/63 mit Staatsemblem der DDR versehen. Abb. 31, 32

\subsubsection*{腰带}% Koppel

腰带;士兵和士官的由黑色皮革制成(或野战军服腰带,后来职业军人的也是如此),带皮带扣;职业军人的由棕色皮革制成,带皮带扣。见图33、34
% Leibriemen; für Soldaten und Unteroffiziere auf Zeit aus schwarzem Leder (oder Gurt zur Felddienstuniform, dann auch für Berufssoldaten) mit Koppelschloß, für Berufssoldaten aus braunem Leder mit Schnalle. Abb. 33, 34

\subsubsection*{领底}% Kragenbinde

在作战服/野战服夹克的衣领上扣上布条,以保护衣领。
% zur Schonung des Kragens bei Uniformjacken bzw. Jacken des → Kampfanzuges/Felddienstanzuges eingeknöpfter Stoffstreifen.

\subsubsection*{领章}% Kragenspiegel

vor allem auf den Uniformjacken aller Uniformarten der Landstreitkräfte und der LSK/LV (außer am → Kampfanzug/Felddienstanzug)sowie z.T. bei den Seestreitkräften/Volksmarine angebrachtes Zierelement unterschiedlicher Ausführung;

a) für Marschälle und Generale aus Biesentuch in der → Waffenfarbe mit Stickerei in Gestalt der → Arabeske; auch auf den Uniformmänteln getragen. Abb. 35;

b) für Admirale aus goldfarbenem geprägtem Eichenlaub;

c) für Fähnriche und Offiziere der Landstreitkräfte und der Luftverteidigung aus steingrauem Uniformgewebe mit silberfarben geprägter Doppellitze und weißer (bis 1977 in der → Waffenfarbe) bzw. hellgrauer Kantillenfüllung (→ Kantille); für Fallschirmjäger seit 1970 aus orangefarbenem Biesentuch mit silberfarbener Kordelumrandung, symbolisiertem Fallschirm und Schwinge; bis 1986 auch am Uniformmantel getragen, Abb, 36;

d) für Fähnriche und Offiziere der Luftstreitkräfte aus hellblauem Biesentuch mit silberfarbener Kordelumrandung,silberfarbener Schwinge und offenem bzw. geschlossenem Eichenlaubkranz (nur Stabsoffiziere); bis 1986 auch am Uniformmantel getragen;

e) für Soldaten, Unteroffiziere, Unteroffiziers-, Fähnrich- und Offiziersschüler der Landstreitkräfte und der Luftverteidigung aus steingrauem Uniformgewebe mit silberfarben gewebter Doppellitze, weißem (bis 1980 in der
Waffenfarbe) bzw. hellgrauem Mittelstreifen und zwei Außenstreifen; Fallschirmjäger seit 1970 analog der für Offiziere, Abb. 37;

f) für die unter e) genannten Dienstgrade der Luftstreitkräfte aus hellblauem Biesentuch mit silberfarbener Schwinge; bis 1986 auch am Uniformmantel getragen;

g) für Matrosen,Unteroffiziersschüler und Maate aus kornblumenblauem Biesentuch für den → Überzieher.

\subsubsection*{}% Laminieren

mit Schaumstoffolie aus dem synthetischen Material Polyuretan beschichtetes Gewebe für Uniformmäntel der Berufssoldaten.

\subsubsection*{}% Lampassen

rote bzw, hellblaue Tuchstreifen an den Uniformhosen der Generale; ursprünglich der zivilen Herrenmode Anfang des 19. Jahrhunderts entlehnt. Abb. 38

\subsubsection*{}% Lyra

altgriechisches Zupfinstrument; Symbolik der → Dienstlaufbahnabzeichen für Angehörige des Musikdienstes der NVA; auch auf → Schulterklappen und → Schulterstücken, für Militärmusikschüler auf den Kragen der Uniformjacken. Abb. 39

\subsubsection*{}% Matrosenmütze

auch Tellermütze;Kopfbedeckung der Matrosen und Maate der Seestreitkräfte/Volksmarine zur Dienst-, Parade- und Ausgangsuniform; je nach Trageperiode mit weißem oder blauem Mützenbezug versehen. Abb. 40

\subsubsection*{}% Paspel, Paspelierung

auch Biese; in die Nähte der Uniform eingenähter schmaler Stoffstreifen, der als andersfarbiger (meist in der → Waffenfarbe) Vorstoß sichtbar ist.

\subsubsection*{}% Pumps

ausgeschnittener Damenhalbschuh mit mittlerem bis höherem Absatz,ohne Spangen und Verschnürungen. Abb. 41

\subsubsection*{}% Repräsentationsschnur

dekoratives Geflecht silberfarbener Schnüre für Angehörige der Ehrenkompanien, des Zentralen Orchesters der NVA, des Stabsmusikkorps Berlin und seines Spielmannzuges; seit 1976 zu zentralen militärischen Zeremoniellen angelegt(auBer von Offizieren, die die Achselschnur tragen). Abb. 42

\subsubsection*{}% Rinksgurt

Gürtel des Sommermantels der Berufsunteroffiziere, Fähnriche, Offiziere und Generale/Admirale; kommt von Rinken - mundartlich fǔr Schnalle. Abb. 43

\subsubsection*{}% Schaftstiefel

Fußbekleidung vor allem der Berufsunteroffiziere, Fähnriche, Offiziere und Generale sowie der weiblichen Armeeangehörigen (hier auch z.T. mit Reißverschluß an der
Seite). Abb. 44

\subsubsection*{}% Schiffchen

1962 eingeführte und bis Anfang der 80er Jahre verwendete käppiartige Kopfbedeckung weiblicher Armeeangehöriger in der für die Teilstreitkräfte typischen Farbgebung. Abb. 45

\subsubsection*{}% Schirmmütze

Kopfbedeckung männlicher Armeeangehöriger für nahezu alle Uniformarten; unterschiedliche Ausführung für Dienstgradgruppen und Teilstreitkräfte. Abb. 46, 47, 48

\subsubsection*{}% Schulterklappen

spezifische Form der → Dienstgradabzeichen; auf den Schulterpartien der Uniform befestigte Stoffklappen, die die Dienstgrade der Soldaten und Unteroffiziere sowie der Unteroffiziers-, Fähnrich- und Offiziersschüler, durch die → Paspelierung auch die Zugehörigkeit zu einer Teilstreitkraft, Waffengattung, Spezialtruppe oder zu einem Dienst kennzeichnen. Abb. 49, 50, 51, 52, 53

\subsubsection*{}% Schulterstücke

spezifische Form der - Dienstgradabzeichen; auf den Schulterpartien der Uniform befestigter Tressenbesatz (→ Tresse) bzw. befestigtes Schnurgeflecht; dient der Kennzeichnung der Dienstgrade der Fähnriche, Offiziere,Generale und Admirale sowie des Marschalls. Abb. 54, 55, 56, 57

\subsubsection*{}% Schwalbennester

besondere Kennzeichnung der Angehörigen des Zentralen Orchesters der NVA sowie der Stabsmusikkorps (mit angebrachten silberfarbenen Fransen)und der Musikkorps der Landstreitkräfte und der LSK/LV; historisch entstanden aus den Achselwülsten(einer Verstärkung des Oberärmels);befestigt an beiden Oberärmeln der Uniformjacken. Abb. 58, 59

\subsubsection*{}% Seestern

gewissermaßen das→Dienstlaufbahnabzeichen der Admirale; gestaltet mit dem Staatsemblem der DDR; an beiden Unterärmeln der Uniformjacken,2cm über den → Ärmelabzeichen bzw. Ärmelstreifen aufgenäht. Abb. 60

\subsubsection*{}% Serge

Futter- und Kleiderstoff aus verschiedenen Faserstoffen.

\subsubsection*{}% Slingpumps

weit ausgeschnittener Damenhalbschuh, im Unterschied zu den → Pumps mit Fersenriemen. Abb. 61

\subsubsection*{}% Sporta

Damenhalbschnürschuh. Abb. 62

\subsubsection*{}% Stiefelhose

besonders gefertigte Beinbekleidung der Berufsunteroffiziere (seit Mitte der 60er Jahre), Fähnriche, Offiziere und Generale der Landstreitkräfte und der LSK/LV für die Dienst- und Paradeuniform. Abb. 63, 64

\subsubsection*{}% Streichgarn

aus kurzen, ungleichmäßigen Fasern (Wolle, Baumwolle
oder Chemiefasern) hergestelltes,schwach gedrehtes
Garn; fülliger und rauher als → Kammgarn; Material vor
allem für die Uniformen der Soldaten und Unteroffiziere
auf Zeit.

\subsubsection*{}% Tellermütze

→ Matrosenmütze

\subsubsection*{}% Tombak

Kupfer-Zink-Legierung, gut verformbar; als Goldimitation zur Fertigung von Effekten verwendet.

\subsubsection*{}% Tresse

schmales, durchbrochenes, gewebtes silber- bzw. goldfarbenes Band oder Borte; vielfach bei Uniformstücken verwendet, z. B. bei den →Schulterklappen der Unteroffiziere.

\subsubsection*{}% Überzieher

Uniformjacke der Matrosen und Maate. Abb. 65

\subsubsection*{}% Waffenfarbe

farbige Kennzeichnung der Teilstreitkräfte, Waffengattungen, Spezialtruppen und Dienste der NVA vor allem an Schulterklappen, -stücken und Kragenspiegeln, auch an Schirmmützen und lange Zeit an den Armelpatten; im Laufe der Entwicklung der NVA vielfach modifiziert.

\subsubsection*{}% Winkel

verschiedenartige Kennzeichnung der Dauer der Zugehörigkeit zur NVA bzw, seit 1965 eines Dienstverhältnisses; bei Soldaten und Unteroffizieren angewandt; unterschiedliche Farbgebung in den Teilstreitkräften; stets am linken Unterärmel der Uniformjacken bzw. -mäntel aufgenäht. Abb. 66, 67

\subsubsection*{}% Wintermütze

Bezeichnung für verschiedenartige spezielle Kopfbedeckungen der Armeeangehörigen innerhalb der Wintertrageperiode. Abb. 68, 69, 70

\subsubsection*{}% Wirrvlies

Faserschicht mit gleichmäßigen Festigkeits- und Dehnungseigenschaften nach allen Richtungen; die Fasern liegen wirr (zufallsmäßig) durcheinander.

\subsubsection*{}% Wolpryla

Bezeichnung für Polyakrylnitrilfaserstoff.

\subsubsection*{}% Zugstiefel

Fußbekleidung der Fahnriche, Offiziere, Generale und Admirale zur Dienst-, Stabsdienst- und Ausgangsuniform. Abb. 71