

Die Einführung von Uniformhemdblusen durch die
Anweisung Nr.4/60 des Ministers fǔr Nationalc
Verteidigung der DDR vom 7.Juni 1960 brachte
eine wesentliche Verbesserung der Uniformierung
wenn auch zunächst nur fǔr ffiziere,Generale
und Admirale. Dabei orientierte sich der Beklei-
dungs- und Ausrüstungsdienst der NVA an dem
Beispiel der Sowjetarmee, die schon seit Ende der
50er Jahre derartige Blusen eingef¼hrt und die Tra.
geweise in der Uniformvorschrift von 1959 festge.
legt hatte. In der NVA begannen die Tragetests der
Uniformhemdblusen am 1.September 1959 im Mi.
nisterium für Nationale Verteidigung, in den Kom-
mandos der Militärbezirke der Landstreitkräfte und
des Kommandos der LSK/LV, Sie endeten am
15.Oktober desselben .Jahres. Ab Mitte Juni 1960
wurde die Bluse gegen Bezahlung an Offiziere, Ge
nerale und Admirale ausgegeben. Die Uniform.
hemdbluse wurde für alle Teilstreitkräfte in einem
silbergrauen Farbton gefertigt, Die Angehörigen
der Volksmarine trugen an dieser Bluse goldfarbene
Ankerknöpfe,die Offiziere der Land. und Luft-
streitkräfte einfache silberfarbene, die Generale
goldfarbene Aluminiumknöpfe.
In der Bekleidungsvorschrift, der DV-10/5, vom
1.Dezember 1960 wurde die Trageweise der Uni-
formhemdbluse festgelegt. Der genannte Personen-
kreis konnte sie ohne Uniformjacke, mit Schulter-
stücken versehen undoffenem
Kragen zum
Stabsdienst,zum tglichen Dienst- soweit die
Stabsdienstuniform befohlen war -, zur Gefechts-
ausbildung in der Kaserne und im Gelände, wenn
nicht die Felddienstuniforn befohlen war, und auf
dem Wege vom und zum Dienst anziehen. Auf Be.
fehl des Kommandcurs des Truppenteils oder des
Leiters der Dienststelle konnte die Bluse ohne Uni-
formjacke auch mit geschlossenem Kragen und Binder angezogen werden. Sie konnte aber auch
dann mit Binder und ohne Schultersticke - unter
der Uniformjacke getragen werden. Das Koppel
muBte bei der Uniformhemdbluse und der Stiefel.
hose auf dem Bund der Bluse durch die Schlaufen
gezogen werden. In dieser Zeit wurden an der Uni.
formhemdbluse noch keine Orden und Medaillen
in Form der Interimsspangen befestigt.
Nach etwa cinem Jahr wurden Verbesserungen
des Kragenschnittes der Uniformhemdbluse vorge.
nommen.Er wurde als Hemdkragen umgearbeitet.
Damit erhielt die Bluse ein attraktiveres Aussehen
und war bequemer zu tragen.Das Material bestand
aus leichter Hemdenpopeline.
Die Uniformhemdblusen schon in diesen ahren
f¼r alle Armeeangeh'rigen einzufihren konnte aus
ökonomischen Griünden nicht realisiert werden, Für
die Erstausstattung der Soldaten und Unteroffiziere
waären immerhin etwa 400000m*Baumwollpope
line erforderlich gewesen. Hinzu w¤ren dann noch
die Mengen der jährlichen Ergänzungen für die Of
fiziere gekommen, Eine weitere Veränderung voll.
zog sich aber bei den Drillichuniformen. Für ihre
Produktion konnte durch Verwendung eines leich
teren Gewebes aus 60 Prozent Flachs und 40 Pro.
zent Baumwolle die Qualität verbessert werden.
Ab 1962 konnten Offiziere.Generale und Admi.
rale sowie weibliche Armeeangehörige cinen neu
entwickelten Sommermantelkauflich
erwerben.
Aufgrund seiner Materialbeschaffenheit vermochte
er auch als Regenmantel gute Dienste zu leisten
Dieser steingraue bzw,dunkelblaue Mantel bestanc
füir ffiziere und weibliche ArmeeangehÃrige aus
gummicrter Baumwolle;für Generale und Admi
rale aus ebensolchem Wollserge. Die maBgeschnei.
derten einreihigen Mäntel offener Fasson wiesen
eine verdeckte Knopfleiste,zwei schräge Seitenta.
schen an der Vorderseite,einen Rinksgurt und
Schnallen an den Armeln sowie eine Tragevorrich
tung mit Durchgriff in Höhe der Taillenlinie fü
den Ehrendolch auf, Der Mantel war vollst¤ndig
mit Kunstseide abgef¼ttert und mit einer Innenta.
sche verschen.

Der neue Sommerantel ersetzte die vorhande.
nen und teilweise getragenen grauen Regenmänte
aus den Beständen der KVP sowie die 1956 als
Normbekleidungsstücke in die NVA eingeführten
blaugrauen Sommernäntel der KVP. Diese Man.
tel- aus Gabardine hergestellt - knitterten sehr
stark und blieben anfällig gegenüber Regen und
N¤sse. Bei den neuen Mänteln war dies nicht der
Fall, Sie durften im Sommer und in der Ubergangs
zeit auf dem Wege vom und zum Dienst, zum Aus
gang und im Uraub genutzt werden.
Ab 1963 wurden Offiziersuniformen, bisher nur
maBgeschneidert,konfektioniert hergestellt. Dies
ermöglichte schon im folgenden ahr bedeutsame
Einsparungen an Uniformstoffen, Bis zu diesem
Zeitpunkt erhielten neben den Generalen und Ad.
miralen die weiblichen Areeangchörigen und dic
Angehörigen der NVA mit Uber- bzw.UntergróBen
maBgefertigte Uniformen. Dazu wurden jahrlich
1,6 Millionen Mark an Haushaltsmitteln zur Verfi.
gung gestellt. Die MaBnahmen zur Spezialisierung
Mechanisierung und die Anwendung von Stan
dards in der Uniforfertigung hatten zu einem
solch hohen Entwicklungsstand der Paßformen, der
GröBensortimente und der Stoffverarbeitung ge
fihrt,der es ermöglichte,einen groBeren Krcis von
Ôffzieren mit konfektionierter Bekleidung auszu
statten,Mit Wirkung vom 1..]anuar 1964 wurden
an Offiziere bis um Dienstgrad Oberstleutnant
bzw,Fregattenkapitǎn konfektionicrte niformen
ausgegeben, Sie konnten allerdings auch weiterhin
Stoff zur Maßfertigung auf eigene Kosten empfan-
gen.Nur Generale und Admirale,Oberste und
Kapitäne zur See,weibliche Armeeangehörige und
Offiziere mit Über- und UntergröBen erhiel-
ten MaBuniformen ohne Bezahlung. Bei der zah-
lenmäBigen Stärke und dienstgradmäBigen Zu
sammensetzung des Offizierskorps der NVA jener
Jahre betrug die jahrliche Einsparung nahezu
400000 Mark. Zum anderen trug die gröBere Aus
nutzung konfektionierter Bekleidung dazu bei
volkswirtschaftliche Kräfte und Mittel einzusparen
und die Arbeitsproduktivität weiter zu steigern.

Weiterentwicklungen in der Uniformierung,
wenn auch nur fǔr einen kleineren Personenkreis,
kamen seit 1961 zur Geltung. So wurden die Uni.
formen für die Meister der Volksmarine seit dieser
Zeit aus dem Stoff gefertigt, aus dem die Offiziers-
uniformen hergestellt wurden. Damit war verbun-
den, daß für diese Berufsrichtung das ¤uBere Bild
attraktiver erschien.