

Erfahrungen und Hinweise einer Militärratssitzung
des Kommandos Landstreitkräfte berücksichtigend,
wurde für die Felddienstbekleidung der Generale
und Admirale festgelegt, das Dreifasermischgewebe
mit Stricheltarnmotiv als Obergewebe beizubehal-
ten, aber Sommer- und Winter-Felddienstanzug bei
¤ußerlich gleichem Erscheinungsbild zu trennen.
Zugleich wurde eine starke Angleichung an die äu-
Bere Form und Gestaltung der Felddienstanzüge der Offiziere, Fähnriche, Unteroffiziere und Solda-
ten angestrebt. Entsprechende Muster fanden am
21.Oktober 1977 die volle Zustimmung Armeege-
neral H, Hoffmanns. Sie wurden in die Produktion
überführt, Generale und Admirale trugen den
neuen Felddienstanzug ab 1. März 1978. Nun war
es auch möglich,darunter die Dienstuniform mit
Oberhemd und Binder anzuziehen. Der modifi-
zierte Felddienstanzug bestand im Sommer aus
Feldmütze, Felddienstanzug (Sommer) mit geöffne-
tem Kragen, schmal gehaltener Hose und Schaft-
stiefeln. Im Winter trugen Generale und Admirale
den Felddienstanzug (Winter) mit einem Steppfut-
ter aus Wirrflies, aufgekn¶pftem Webpelzkragen
und Wintermütze. Wintermütze und Webpelzkra-
gen waren nur in der Zeit vom 1.Dezember bis zum
28./29,Februar zu verwenden.
Auch in der Entwicklung der Felddienstbekleidung der Offiziere, Fahnriche, Unteroffiziere und
Soldaten wurde eine ¤ußere Angleichung von Som-
mer- und Winterbekleidung eingeleitet, die sich zu-
nächst in der Gestaltung der Taschen und des Ãr.
melverschlusses bemerkbar machte. Offizieren war
es ab 1.März 1978 gestattet, unter dem Felddienst-
anzug die komplette Dienstuniform mit Oberhemd
und Binder zu tragen.
Die Schutzwirkung der Felddienstbekleidung
wurde insgesamt durch die Ausstattung der Armee-
angehörigen mit Vierfingerhandschuhen im Stri.
cheldruck und durch die schrittweise Einfihrung
von Stahlhelmen mit erhöhtem Splitterschutz, die
jetzt im FlieBdruckverfahren hergestellt wurden,
verbessert.