

An der Schwelle zum achten ,Jahrzehnt stellten die
forcierte Hochr¼stung der NATO und die gestei.
gerte Gefechtsbereitschaft und Angriffsfahigkeit
der NATO-Streitkräfte neue und höhere Anforde-
rungen an die sozialistische Landesverteidigung.
Es waren zwei Aufgaben strategischen Ausmaßes,
die gleichzeitig in der DDR zu lösen waren: die Po
tenzen des Sozialismus auch weiterhin zum Wohle
des Volkes zu erschlieBen und unter komplizierten
internationalen Bedingungen zugleich alles Erfor.
derliche für den zuverlässigen Schutz der sozialisti-
schen Entwicklung bereitzustellen.Das war nur
durch neue groBe Leistungen der Werkt¤tigen m¶g
lich, Aber auch in den Streitkräften mußten ökono.
mische Fragen eine noch gröBere Rolle spielen.
Viele Initiativen wurden ausgelöst, um durch viel-
fältige aßnahmen noch sorgsamer mit dem der
NVA anvertrauten Volksvermögen umzugehen, um
durch höhere Effektivität beim Einsatz materieller
und finanzieller Mittel in den Streitkräften ein
Mehr an Kampfkraft und Gefechtsbereitschaft zu
erreichen.

Generalleutnant ].Goldbach, seit 1979 Stellver-
treter des Ministers für Nationale Verteidigung und
Chef Rückwärtige Dienste, forderte von den Ar-
meeangehörigen und Zivilbeschaftigten seines
Dienstbereiches, besonders ab 1983/84, daß gute
Militärs auch gute ilit¤rökonomen sein müs-
sen.
Die erste Zentrale militärökonomische Konfe-
renz des Bekleidungs- und Ausriüstungsdienstes der
NVA am 5.|uli 1984 ordnete sich in den Prozeß der
intensiveren Bewirtschaftung des der NVA zur Ver.
fügung gestellten Volksvermögens organisch ein.
An dieser Konferenz nahmen neben den Angehörigen und Mitarbeitern des Bekleidungs- und Ausrü
stungsdienstes der NVA auch Kommandeure. Polit
arbeiter und Angeh¶rige der rückwärtigen Dienste
aller Stufen teil. Zentrale Themen dieser bedeutsa
men Konferenz waren einerseits die Rechenschafts
legung zur Qualität der Sicherstellung der Landes
verteidigung auf dem Gebiet der Bekleidung und
Ausri¼stung seit 1971, andererseits die Beratung
von ethoden,wie unter Durchsetzung eines ho
hen Sparsamkeitsprinzips in der weiteren Arbei
des B/A-Dienstes ein noch höherer Beitrag zur Ge
fechtsbereitschaft geleistet werden kann. Schwer.
punkte der Beratung waren die sparsamste Verwen
dung der materiellen und finanziellen Mittel. di
weitere Verbesserung der Pflege und Instandhal
tung der Bekleidung und Ausriistung, die fachspe
zifische Ausbildung der Mitarbeiter des B/A-Dien
stes und der Hauptfeldwebel, die Einhaltung und
Unterbietung der Verbrauchs- und Nutzungsnor
men und die vollständige Erfassung und Wieder
verwendung der Sekundärrohstoffe.Die Bilanz des
B/A-Dienstes zur Sicherstellung des militärischen
Dienstes war beeindruckend:27 Hauptpositionen
konnten bei der Neueinführung bzw. Weiterent
wicklung von Bekleidung und Ausrüstung aufge
fiihrt werden.
Allein bei der Einberufung zum Grundwehr
dienst erhielt ieder Soldat der NVA 1983 Beklei.
dung und Ausr¼stung im Wert von ca.2 600 Mark
Durch gute Pflege der persönlichen Bekleidung
und Ausrüstung konnte jeder den Werkt¤tigen für
Fleiß und Initiative danken, Sie waren es, die Er
zeugnisse mit langer Nutzungsdauer,groBer Halt
barkeit,verbessertem Schutz vor ¤ußeren Einflüs
sen und Witterungserscheinungen sowie guten
Trageeigenschaften schufen, die bekleidungshygie
nisches Wohlbefinden bereiteten und den harten
Bedingungen des militärischen Lebens entspra
chen.
Die in der Tabelle auf S.250 ausgewählten
Beispiele für Versorgungsleistungen seit 1971 ver-
deutlichen die umfassende Fürsorge für die Sol-
daten des Volkes. Sie zeigen die WertgröBen, die dazu notwendig waren, und kennzeichnen die Be
kleidung und Ausrüstung der Armeeangehörigen
als beträchtliche Größe des Volksverm¶gens, mit
dem sorgsam umgegangen werden muß.
Die Forderung nach sorgsamem Umgang mit Be-
kleidung und Ausrüstung stand seit jeher, erlangte
aber in der ersten Halfte der 80er Jahre einen
neuen Stellenwert.
Die intensivere gefechtsnahe Ausbildung, die hö-
heren Anspri¼iche an die militärische Pflichterfǔllung im Diensthabenden System und im Gefechts.
dienst sowie die Erhöhung des Anteilsvon
Reservisten und ihrer Aus- und Weiterbildung war
mit einer größeren Beanspruchung und stärkerem
Verschleiß der Uniformen, besonders der Feld
dienstanzüge verbunden.
Die Einführung bzw.Modernisierungvon
Kampftechnik erhöhte die Vielfalt, Differenziert
heit und Spezifik notwendiger Zusatz- und Spezial-
bekleidung, für die bedeutende Mittel aufzubringen
waren. Die Kosten mußten soweit wie m¶glich in-
nerhalb des B/A-Dienstes kompensiert werden,
Deshalb wurde in der NVA,beeinflußt durch die
Analysen der erstenZentralen militärökonomi
schen Konferenz, verstärkt Wert darauf gelegt, Nut.
zungsfristen und Normen zu überbieten, Schäden
und Verluste wirksam zu bekämpfen und unbe
gründete Aussonderungen zu verhindern.Der m-
gang mit Bekleidung und Ausrüstung rickte st¤rker
in das Blickfeld der Partei- und FD]-Organisatio-
nen, spielte in der Erziehungsarbeit in den militäri.
schen Kollektiven und im sozialistischen Wettbe-
werb eine gröBere Rolle.
Die Verantwortung der Mitarbeiter des B/A.
Dienstes wurde noch st¤rker darauf gerichtet, mili.
tärökonomische Wirkungen zu erzielen.Durch ihre
verantwortungsbewuBte, fleißige und oft mühevolle
Kleinarbeit,sei es als Fachoffizier, Lagerverwalter,
Schuhmacher oder Schneiderin,hatten sie sich
hohe Achtung erworben,Sachkundig wirkten sie in
Gespr¤chen, Vorträgen und Lehrvorführungen dar
auf hin, gute Erfahrungen im Umgang mit Beklei.
dung und Ausristung zu verallgemeinern,das Wie
von Pfege und Instandsetzung noch stärker zu po.
pularisieren,um damit langere Nutzungsfristen und
Einsparungen zu erreichen sowie Schäden und Ver-
luste zu verringern.Als verläBliche Ratgeber und
Helfer der Vorgesetzten unterstützen die Mitarbei
ter des B/A-Dienstes in den Einheiten und Trup
penteilen die konsequente Durchführung der Putz
und Flickstunden, der B/A-Appelle sowie die Ein-
haltung der Tausch-,Reinigungs- und Instandsetzungsnormen.