

Entsprechend dem vom Nationalen Verteidigungs
rat der DDR verabschiedeten Beschluß über milita
rische Dienstgrade waren mit Wirkung vom 1. Mai
1982 die Dienstgrade Marschall der DDR und
Flottenadmiral geschaffen worden.
Im Unterschied zu den Generalsdienstgraden wa.
ren der Marschall der DDR durch vierschlaufige.
dickgeflochtene goldfarbene Schulterstückemit
einem 20mmgroBenfinfzackigen vergoldeten
Stern mit eingelassenem Rubin, der Flottenadmiral
durch fünfschlaufige Schulterstücke mit vier Ster.
nen in Reihe gekennzeichnet.Im Zusammenhang
mit der Gestaltung der Schulterstücke für den Mar.
schall der DDR wurden die der Generale und Ad
mirale generell fünfschlaufig wie die Schulterstücke
der Stabsoffiziere, allerdings in der bisherigen Art
gefertigt.
Veränderungen gab es ab 1983 auch in der
Dienstgradkennzeichnung der Fahnrich- und Offi-
ziersschüler, Mit der endgültigen Regelung der
Fahnrichausbildung als Fachschulausbildung wur.
den Schulterklappen für Fahnrichschüler im 1. und
2.Studienjahr eingeführt, Das Studienjahr wurde
durch die entsprechende Anzahl von Ouerstreifen
auf den Schulterklappen kenntlich gemacht. Mit
Beginn des Studiums an den Offiziershochschulen
trugen Offiziersschüler die gleichen Uniformen aus
Kammgarngewebe wie die Berufssoldaten. Damit
konnte die Umuniformierung zu Beginn des 2. bzw.
3.Studienjahres wegfallen, und die Offiziersschüler
wurden äuBerlich noch mehr den Offizieren ange.
glichen.
Für alle Berufsunteroffiziere entfiel mit der Be
kleidungsvorschrift, Ausgabejahr 1983, der dop
pelte Winkel auf dem rechten Armel. Nur Soldaten.
Unteroffiziere und Maate mit dreijähriger bzw. Ma.
trosen und Maate mit vierjähriger Dienstzeit brach
ten an gleicher Stelle einen stumpfen Winkel - sil
ber- bzw,goldfarben-zurKennzeichnung ihres Dienstverhältnisses auf Zeit 1 cm über der rmel-
biese, in der Volksmarine 13 cm vom Ärmelsaum
des Kieler Hemdes oder des Ãberziehers entfernt
an. Bei der Volksmarine befanden sich auBerdem
metallgeprägte goldfarbene Dienstlaufbahnabzei-
chen auf den Schulterklappen der Berufsunteroffi-
ziere und Berufsunteroffiziersschüler.
Vereinheitlicht, erweitert und verändert wurden
die Dienstgradabzeichen an Flieger- und Techni-
keranzügen,Ab 1983 wurden die Dienstgrade an
den Flieger- und Technikeranzügen einheitlich
durch Tressen in Mattsilbergrau für LSK/LV und
Mattgold bei der Volksmarine kenntlich gemacht.
(Siehe nebenstehende Tabelle)