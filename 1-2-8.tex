

Ein besonderer Abschnitt der «Vorl¤ufigen Beklei-
dungsordnung der Nationalen Volksarmee》 von
1957 umfaßt Bestimmungen zur Trageweise der
Auszeichnungen an den Uniformen. Die Tabelle
gibt Auskunft über jene Orden, Medaillen und Abzeichen sowie Ordens- und Medaillenspangen, die
nach der Verleihung in jener Zeit an den Parade-
und Ausgangsuniformen aller Armeeangehöriger
sowie an den Dienstuniformen der Offiziere, Gene-
rale und Admirale getragen wurden.
Diese Regelung galt auch für ausländische Or-
den und Medaillen, die an Bürger der DDR für
Verdienste im Kampf gegen den Faschismus, fiür
den Frieden und den Aufbau des Sozialismus ver-
liehen wurden, Als Medaillen und Abzeichen ge-
sellschaftlicher Organisationen konnten das Ehren-
zeichen für Parteiveteranen der SED,die Ernst-
Thälmann-Medaille, die Ernst-Moritz-Arndt-Me.
daille,die Friedensmedaille des Deutschen Friedensrates, die Friedensmedaille der FD,das
Abzeichen «Für gutes Wissen» der FD] in Gold,
Silber und Bronze, das Sportabzeichen in Gold, Sil-
ber und Bronze,das AbzeichenPartisanen des
Weltbundes der Demokratischen .Jugend und das
FD]-Abzcichen getragen werden.
Komplizierte Bestimmungen, die sich an interna-
tional übliche Gepflogenheiten anlehnten, regelten
die Art und Weise sowie den Platz der jeweiligen
Auszeichnungen an den Uniformen. Grundsätzlich
befanden sich Orden und Medaillen mit Band links
und die ohne Band bzw, Abzeichen rechts auf der Uniformjacke, [hrer Bedeutung nach (siehe Tabelle
S,77)waren sie alle von rechts nach links auf der
Uniformjacke anzubringen. Ausländische Auszeich.
nungen ordneten sich nach Bedeutung und Zweck
bestimmung in diese Reihenfolge ein, folgten aber
immer den entsprechenden Auszeichnungen der
DDR, Für die Trageweise der Orden und Medail.
len am Band an der offenen Jacke der Offiziere galt
die Regelung, daß 6 bis 8 Auszeichnungen hinter-
einander in einer Reihe angelegt .werden durften,
Das Ordensband sollte jedoch keine gröBeren Aus.
maBe als 14 cm L¤nge aufweisen.