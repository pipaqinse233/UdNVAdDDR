

Die Gesellschaftsuniform, die nur den Offizieren,
Generalen und Admiralen vorbehaltene Uniform
art, gibt es seit 1983 als Kleinen und GroBen Ge-
sellschaftsanzug. Die Bekleidungsvorschrift von 1986 l¤ßt das Tragen der hellgrauen Bluse durch
alle männlichen, der weißen Hemdbluse durch die
weiblichen ffiziere sowie der cremefarbenen
Hemdbluse durch Generale und Admirale als Be-
standteil der Gesellschaftsuniform anstelle der Ge.
sellschaftsjacke nicht mehr zu. Für alle verbindlich
ist nun sowohl zum Kleinen als auch zum Großen
Gesellschaftsanzug die Gesellschaftsjacke. Sie ist
für Offiziere der Landstreitkräfte und der LSK/LV
aus graugrüner, für Offiziere der Volksmarine aus
cremefarbener und für Generale und Admirale aus
hellgrauer Gabardine gefertigt.
Offiziere, Generale und Admirale können die
Gesellschaftsuniform als Großen oder Kleinen Ge-
sellschaftsanzug vierfach variieren. Dazu kombinie-
ren sie die Uniformstücke, wie in der Tabelle auf
S.265 dargestellt.
Admirale ziehen in der Zeit vom 1.Mai bis
30.September zur Gesellschaftsuniform einen hellgrauen, Offiziere der Volksmarine einen cremefar-
benen Mützenbezug auf die Schirmmütze.
Die Gesellschaftsuniform weiblicher Offziere
sieht, da diese Achselschnur und Ehrendolch nicht
führen, keine Unterscheidung zwischen GroBem
und Kleinem Gesellschaftsanzug vor. Ihre Gesell-
schaftsuniform besteht zu allen ]ahreszeiten aus der
Gesellschaftsjacke aus gleichem Uniformtuch wie
für männliche ffiziere, Uniformrock, weiBer
Hemdbluse mit Binder und Halbschuhen. Weibli-
che Offiziere der Landstreitkräfte und der LSK/LV
tragen dazu im Sommer und in den Übergangszei-
ten die graue Kappe, Angeh¶rige der Volksmarine
in der Zeit vom 1.Mai bis 30,September die creme-
farbene, sonst die dunkelblaue Kappe. In der Zeit
vom 1.März bis 15.April und vom 1. bis 30.Novem-
ber vervollständigt entweder der Sommermantel
oder der Uniformmantel die Gesellschaftsuniform
der weiblichen Offiziere. Im Winter gehört zum
Uniformmantel die Wintermütze. Diese ist seit
1.Dezember 1986 wieder ohne Schirm ausgelegt.

Offiziere, Generale und Admirale erscheinen im
Großen Gesellschaftsanzug zu Festveranstaltungen
und Empfängen anlaBlich des Nationalfeiertages
der DDR,zu .Jubiläumsveranstaltungen und Emp-
fängen anläBlich des Tages der NVA, zu Auszeich-
nungsveranstaltungen im Staats- und inisterrat
der DDR und zur Verleihung von Preisen.
Der Kleine Gesellschaftsanzug gilt als feierliche
Uniformierung zu Festveranstaltungen und Emp
fängen, zu Theater- und Konzertbesuchen und zu
famili¤ren Festlichkeiten.