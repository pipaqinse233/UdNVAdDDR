

Bereits mit Wirkung vom 1.Juli 1978 wurde es not-
wendig,die DV010/0/005.Uniformen und ihre
Trageweise, Ausgabejahr 1977,durch eine erste
umfangreiche Ãnderung zu pr¤zisieren. Das betraf
Festlegungen zur Trageweise neuer Uniformstücke,
Neueinführungen und Veränderungen von Effekten und die Trageweise von Dienstgradabzeichen.
Besonders für die Volksmarine gab es eine Vielzahl
detaillierter Anderungen.
So galt ab 1.Juli 1978 nicht mehr die zeitliche
Einengung(16.April bis 31.Oktober) für das Tra-
gen des hellgrauen Mützenbezuges für Admirale
und des cremefarbenen Mützenbezuges für Offi-
ziere. Mit der 1. Ãnderung wurde verfügt, daß die
obengenannten Mützenbezüge und auch die creme-
farbene Kappe für weibliche Offiziere der Volksma-
rine immer nur in Verbindung mit der Gesell-
schaftsuniform zu tragen sind. Noch die 1977er
Vorschrift untersagte es Matrosen und Maaten der
fahrenden Einheiten und der Gefechtseinheiten an
Land,Offiziersschülern im 1. und 2. Lehrjahr, die
zu ihrer Ausstattung gehörenden Halbschaftstiefel zur Ausbildung,innerhalb der Kaserne und zum
Innendienst anzuzichen. Diese Einschränkung
wurde im Interesse einer breiteren Nutzung dieses
strapazierfahigen Schuhwerks aufgehoben. Halb-
schaftstiefel konnten nun zur Gefechtsuniform, zur
Dienstuniform und zur Paradeuniform getragen
werden, Es fiel auch die Festlegung weg, daß ge-
narbte Schaftstiefel von Meistern, Fähnrichen und
Offizieren der Gefechtseinheiten an Land nur in
Verbindung mit Stiefelhosen angezogen werden
durften. Jetzt gehörten genarbte Schaftstiefel zum
Felddienstanzug.
Pr¤zisiert wurden die Bestimmungen für das Tra-
gen des Lederkoppels mit Schnalle. Neu war, daß
nur noch Fahnriche und Berufsunteroffiziere der
Volksmarine ab Dienstgrad Meister zur Paradeuni-
form das Lederkoppel mit Schnalle anlegen durf-
ten.

Für alle Berufssoldaten bestand nunmehr dic
Möglichkeit,in der Zeit vom 1.November bis zum
15.April in geschlossenen Räumen die Hemdbluse
ohne niformjacke mit Schulterstücken und Bin
der zu tragen.
Das fliegende Personal erhielt 1978 anstelle des
bis dahin in der Ausstattung befindlichen stein
grauen Fliegeranzuges(mit Webpelz für Winterbe
dingungen)blaugraue Fliegeranzüge in Sommer
und Winterausführung. Gleichzeitig begann die
Ausstattung mit 【ärmschutzhelmen und modernen
Fliegerschutzhelmen.
Das ingenieurtechnische Personal wurde für die
Sommerperiode miteinemneuenzweiteiligen
Technikeranzug und einemTechnikerhemd an
stelle der einteiligen Arbeitskombination ausgestat
tet, Eine gravierende Ver¤nderung in der Trage
weise der Dienstgradabzeichen an Flieger- und
Technikeranzügen trat mit der 1.Anderung der
Uniformvorschrift, Ausgabejahr 1977, ab 1. [uli
1978 in Kraft. Bereits am 22.November 1977 hatte
der Stellvertreter des Ministers fir Nationale er.
teidigung und Chef der LSK/LV,Generalleutnant
W, Reinhold, dem Minister für Nationale Verteidi
gung vorgeschlagen,imZusammenhangmit der
Einführung der verbesserten Sonderbekleidung für
fiegendes und ingenieurtechnisches Personal der
LSK/LV die bis dahin am Oberarm angebrachten
Dienstgradabzeichen an der linken Brustseite der
Sonderbekleidung anzubringen.Dies gestatte ein
besseres Erkennen des Dienstgrades der Armeean
gehörigen.Diesem Vorschlag wurde zugestimmt
Flieger und Techniker fihrten ab 1978 die Dienst
gradabzeichen auf der Mitte der linken Brustseitc
an der acke des Fliegeranzuges,an der acke des
Technikeranzuges und an der Latzhose des Techni.
keranzuges. Die Farbe der Dienstgradabzeichen war
Mattsilbergrau,fǔr Generale Mattgold. Die Anord,
nung der 10 cm langen Tressen blieb, wie 1965 fest.
gelegt. Nur die Fahnriche erhielten eine 12 mm
breite Tresse.Vermerkt sei noch, daß ab 1978 die
Mützenembleme der Generale und Admirale metall.
gepragt und nicht mehr handgestickt wurden.

Auf der Grundlage der Anordnung Nr.19/78 des
Stellvertreters des Ministers für Nationale Verteidi.
gung und Chefs Rückwärtige Dienste wurden 1978
eine Reihe von Uniformstücken für weibliche Ar.
meeangehörige neu gestaltet. So erhielten der Uni-
form- und der Sommermantel durch eine verän-
derte Nahtführung und modischeren Zuschnitt ein
gefälligeres Aussehen. Die Ausgangs- und Dienst-
jacke war jetzt mit drei Kn¶pfen zu schlieBen, be-
tont tailliert geschnitten und hatte nunmehr
schräge Seitentaschen mit Leiste. Auch die Weste
wurde 'strenger tailliert, in der inienf¼hrung der
Jacke angepaßt und erhielt ebenfalls schräge Sei-
tentaschen mit Leiste. Der Rock war modisch leicht
ausgestellt und bedeckte die Knie. Er wurde mit
einem Reißverschluß in der linken Seitennaht ge-
schlossen. In der rechten Seitennaht befand sich
eine ebenfalls mit Reißverschluß versehene Tasche.
Dem Zeitgeschmack angepaBt, war die Hose ge.
rade geschnitten und wies eine mäßige Saumweite
von 27cm bis 28cm auf. Die Hose besaß einen
Reißverschluß in der Vorderhosennaht und eine
Tasche in der rechten Seitennaht, die mit Reißver.
schluB zu schlieBen war. Der weiBe Pullover er-
hielt einen neuen Armel- und Bundabschluß mit
Strickbund, Getragen werden durften die veränder.
ten Uniformstücke ab 1.Dezember 1978. Klei
dungsstücke der bisherigen Ausführung konnten
bis zum 30,November 1979 aufgetragen werden.