

Die Arbeitsanzüge sind für alle Dienstgradgruppen
zweiteilig und aus einem sehr strapazierfahigen
dunkelgrauen Mischgewebe gearbeitet.Der Ar
beitsanzug (Winter) bietet durch die Wirrvliesein-
lage den erforderlichen Kälteschutz. Die .Jacke des
Arbeitsanzuges, die eine Tasche mit Patte und zwei Kn¶pfen auf der linken Seite aufweist, hat eine ver-.
deckte Knopfleiste und einen schmalen Bund. Die
Hose ist körpernah geschnitten und mit Knöpfen
verstellbar zu schlieBen, Sie wird über den Stiefeln
getragen. Auf den Arbeitsanzügen werden matt-
graue Schulterklappen oder Schulterstücke ange-
bracht, Auch die Arbeitsuniform kann in den Tra-
geperioden in ihrer Zusammenstellung
variiert
werden.
Entsprechend den festgelegten Trageperioden ist
die Zusammenstellung der Arbeitsuniform im Som-
mer und im Winter eindeutig durch den entspre-
chenden Arbeitsanzug bestimmt. In der Übergangs-
zeit wird entweder der Sommer-oderder
Winterarbeitsanzug befohlen. Als Kopfbedeckung
dient den Angehörigen der Landstreitkafte und der LSK/LV die Feld- oder die schwarze Arbeitsmütze,
denen der Volksmarine das Bordk¤ppi oder die Ar-
beitsmütze und den weiblichen Armeeangehörigen
und den Fallschirmjägern die dunkelgraue oder
dunkelblaue Baskenmütze. Halbschaftstiefel, bei
Berufssoldaten genarbte Schaftstiefel und bei weib-
lichen Armeeangehörigen Schaftstiefel bilden das
strapazierfähige, zur
Arbeitsuniform
gehörige
Schuhwerk. Angehörige der Volksmarine tragen zur
Arbeitsuniform an Bord anstelle der Halbschaft-
bzw. der 'genarbten Stiefel Bordschuhe. Im Winter
geben Vierfingerhandschuhe und Kopfschützer zu-
sätzlich Kalteschutz.
Die Arbeitsuniform wird von Soldaten im
Grundwehrdienst sowie Soldaten, Unteroffiziers-
schülern und Unteroffizieren auf Zeit zur Ausbil-
dung an der Technik, zum Park- und Arbeitsdienst,
zum Waffen- und Revierreinigen und auch zur Ver-
büßung von Arreststrafen angezogen. Die gleiche Bekleidungsordnung gilt für Berufsunteroffiziers-
Fähnrich- und ffiziersschüler, Berufsunteroffi-
ziere, Fahnriche und Offiziere tragen die Arbeits.
uniform zur Ausbildung an der Technik und zum
Parkdienst. Weibliche Armeeangehörige reinigen
ihre Waffen und die Reviere ebenfalls in der Ar-
beitsuniform.