

Die fortschreitende wissenschaftlich-technische
Entwicklung im Militärwesen wirkte sich in den
50er Jahren auch auf die Uniformierung der NVA
aus, Bei der Weiterentwicklung der Uniformen lieB
sich die Armeefihrung vor allem von den Erforder-
nissen des Gefechts leiten.
Die von den USA begonnene Einführung der
Kernwaffen in die Streitkräfte führte zu grundle-
genden Veränderungen im Milit¤rwesen, Der m萌g
liche Einsatz dieser Waffen in Kampfhandlungen,
den die Militärstrategie der USA und Vorschriften
der USA-Streitkräfte vorsahen, muBte von den so-
wjetischen und den anderen sozialistischen Streit-
kräften berücksichtigt werden. Die zunehmende
Stoß- und Feuerkraft der Truppen zog einen stei-
genden Bedarf unterschiedlichster materieller it-
tel nach sich, Völlig neue Anforderungen stellte der Schutz der Truppen und Technik vor Kernwaffen.
Die Führung der NVA trug nicht nur dem, sondern
auch der Erfahrung aus der ilitärgeschichte
Rechnung, daß für Kämpfende eine spezielle Uni-
form erforderlich ist, So wurden Kampfanzüge ein-
gefihrt, die den Bedingungen des modernen Ge-
fechts weitgehend angepaßt waren. Der Kampfan
zug- wahrend des zweiten Weltkrieges auf der
Grundlage ziviler Arbeitsanzüge entstanden - hatte
sich bewährt und die ¤uBere Erscheinung des Sol-
daten zugleich noch st¤rker verändert, Es war eine
neue Uniformart aufgekommen, die speziell für die
Erfullung von Kampfaufgaben bestimmt war.

Bereits Mitte 1956 gab es Ãberlegungen, in der
NVA eine solche spezielle Bekleidung für die Aus-
bildung im Gel¤nde und für das Gefecht zu schaf-
fen. Bis dahin und noch in den folgenden Jahren
verwendeten vor allem die Aufklärer vielfach sowje-
tische Tarnbekleidung.Im Winter, wenn Schnee
lag,zogen sie weiBe Tarnbekleidung, die Schnee-
hemden, über. Doch dies genügte nicht mehr. Des-
halb erprobte man ab 1957 als Kernstüick ciner
wirklichen Felddienstuniform einen Kampfanzug
sowie in diesem Zusammenhang auch ein Sturmge-
päck anstelle des bisher verwendeten Rucksacks
und des Brotbeutels, Die erste Erprobung dieser Bekleidungs-und Ausrüstungsgegenstände erfolgte
in je einem verstärkten mot,Schützenbataillon aus
beiden Milit¤ärbezirken der Landstreitkräfte in der
Zeit vom 1.M¤rz 1957 bis zum 15.Februar 1958.
In der ersten gemeinsamen Truppenübung im
August 1957 mit der GSSD bewiesen mot. Schit
zen des Verbandes von Oberstleutnant H. Stech.
barth ihre Einsatzbereitschaft und die mit der Er.
probung des Kampfanzuges betrauten Soldaten dic
Wotwendigkeit und Eignung dieser Felddienstuni
formn. Wie aus dem Befehl des Ministers für Natio
nale Verteidigung der DDR vom 2. M¤rz 1957 her
vorging,sollten Soldaten aller Waffengattungen
und Dienste der Landstreitkräfte während des gan
zen Zeitraums an der Prüfung der neuen Beklei
dung und Ausr¼stung beteiligt sein. Ziel der Maß.
nahme war es, den Kampfanzug und das Sturmge
p¤äck unter Mitführung der gesamten persónlichen
Ausrüistung hinsichtlich ZweckmäBigkeit, Haltbar.
keit und Beweglichkeit unter den verschiedensten
Bedingungen zu testen.Unter anderem interessierte
die Fachoffiziere des B/A-Dienstes der NVA neben
dem Wert des Kampfanzuges fiir zuverlässige Tar.
nung(auch bei Infrarotaufkl¤rung)seine Flammen.
festigkeit und sein Schutz vor radioaktiven Einwir
kungen und chemischen Kampfstoffen.Es sollten
ebenfalls Antworten darauf gegeben werden,wie
der Kampfanzug vor Wind und Regen schützt, wie
luftdurchlässig das aterial ist und wie es bei
Schmutzeinwirkung reagiert. Gleichzeitig mit dem
Sturmgep¤ck wurden das Tragegestell aus textilem
Gurtgewebe und ein Gurtkoppel aus Perlongurt ge
prüft, Vom Frühjahr bis zum Oktober 1958 fand
eine zusitzliche Erprobung des Kampfanzuges
statt.
Anfang November 1958 trat im mot, Schiitzenre
giment <Hans Beimler》(der Traditionsname wurde
ihm später verliehen - d,Verf,) die mit der Beurtei.
lung des Kampfanzuges beauftragte Kommission
aus Offizieren der rückwärtigen Dienste, der Ver
waltungen Ausbildung und Artilleriebewaffnung
des Hauptstabes, der Bereiche der Chefs Pionierwe.
sen,chemische Dienste,Panzer,Nachrichten und Aufklärung sowie des Militärbezirkes Neubranden.
burg zur Auswertung der Erprobungen zusammen.
Nochmals wurde zur Bestätigung der Erprobungs
ergebnisse eine Übung im Rahmen einer verst¤rk
ten mot.Schitzenkompanie durchgeführt,Ziel war
es,festzustellen, ob die Art und die Ausführung des
Kampfanzuges und des Sturmgepäcks zweckmäÃig
auf die jeweilige Bewaffnung und Ausr¼stung des
einzelnen Kimpfers abgestimmt war. Dies konnte
im Abschlußbericht an den inister für Nationale
Verteidigung der DDR vom 28. November des Jah.
res bejaht werden.
Nach erfolgreichem Abschluß der Erprobung des
Kampfanzuges und des Sturmgep¤cks setzte auf
Befehl des Ministers für Nationale Verteidigung
der DDR vom 26,Juni 1959,der auf einem Be-
schluß des Präsidiums des Ministerrates der DDR
vom 21.des Vormonats beruhte,die Ausstattung
der Soldaten,Unteroffiziere und Offiziere der
Landstreitkräfte(auBer den Panzerbesatzungen
und der Luftverteidigungstruppen mit dieser Feld
dienstuniform und dem Sturmgep¤ck ein. Bis 1960
war die Ausstattung vollständig abgeschlossen. Die
Drillichuniformen wurden schwarz eingefärbt und
als Arbeitsuniformen aufgetragen.
Der zweiteilige Kampfanzug bestand aus einer
acke mit Kapuze und einer langen Hose, im Fla
chendruck mit den vier Tarnfarben Blaugrau
Grün, Hellgrüin und Braun versehen und wurde in
drei Größen produziert. Er entsprach in seinen Ei.
genschaften den in den Erprobungen gestellten An
forderungen. Die Dienstgrade wurden in der glei
chen Art und Weise wie bei der Sonder- und
Sportbekleidung, d, h. mit Tressen am linken Ober.
ärmel.kenntlich gemacht.
Das Sturmgepäck setzte sich aus den Teilen I
und II sowie dem Tragegestell und dem Gurtkoppel
zusammen,Das Teil 【-eine nach dem Beispiel des
Tornisters gearbeitete Packtasche - wies mit 32 cm
Breite,8 cm Tiefe und 28,5 cm Höhe günstige Ab
messungen für seinen Tra¡ger auf, Ãhnliche Maße
hatte das Teil Il, dessen Tiefe aber 11 cm betrug. Es
konnte auf das Teil 【 aufgeschnallt werden. Wahrend Teil I mittels Tragegestells auf dem Rücken des
Soldaten meist ständig mitgefihrt werden sollte,
verblieb Teil II auf dem Gefechtsfahrzeug. Beide
Teile waren aus Spezialgewebe hergestellt und
ebenfalls im Tarndruck der vier Farben des Kampf-
anzuges gehalten.
Zum selben Zeitpunkt der Einführung
des
Kampfanzuges wurden alle Angehörigen der Land-
und Luftstreitkräfte sowie der Truppen der Luftver-
teidigung mit dunklen, in mattgrauen Farben gehal-
tenen Dienstgradabzeichen, Effekten und Kn¶pfen
ausgestattet, um die Dienstuniform unauffalliger zu
gestalten. Im einzelnen hieß das, die entsprechen-
den Uniformjacken und -äntel erhielten matt-
graue Schulterklappen und -stücke sowie dunkle
Kragenspiegel, ohne Kennzeichnung durch Waf-
fenfarben. Das galt auch für die Generale. Die
Dienstgradsterne waren ebenfalls mattgrau. Die
Kopfbedeckungen der Offiziere und Generale wur-
den mit dunklen Kokarden sowie mattgrau gehalte-
nen Mützenkr¤nzen und Kordeln versehen. Die far-
bigen Lampassen an den Stiefelhosen der Generale
entfielen.