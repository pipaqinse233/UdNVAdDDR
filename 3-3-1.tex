

Die DV010/0/005, Ausgabejahr 1980, ermöglichte
es, dem Anlaß und der .Jahreszeit angepaßt, entwe-
der den GroBen oder den Kleinen Gesellschaftsan-
zug zu tragen, Im GroBen Gesellschaftsanzug mit
Achselschnur, Ehrendolch und Orden am Band er-
schienen und erscheinen ffiziere der NVA zu
Staatsempfängen und Festveranstaltungen anläB
lich dés Jahrestages der NVA und des Nationalfei-
ertages der DDR. Der Kleine Gesellschaftsanzug
ohne Achselschnur und Ehrendolch, aber mit Inte-
rimsspange wurde und wird zu protokollarischen
Empfängen anläßlich von Staatsfeiertagen, zu Fest-
veranstaltungen, Theater- und Konzertbesuchen
und zu famili¤ren Feierlichkeiten angezogen.

Das Anlegen der Achsel- und der Repräsenta-
tionsschnüre wurde in der Bekleidungsvorschrift
1980 neu geregelt. Offizieren war es jetzt gestattet,
die Achselschnur, die vordem nur zur Gesellschafts-
jacke gehörte, auch zur Ausgangsuniform und wäh-
rend des Urlaubs zu tragen. Offiziere der Ehren-
kompanien, der Wachregimenter und Angehörige
der Ehrenformationen legten nunmehr zu zentralen
militärischen Zeremoniellen im Winterhalbjahr die
Achsel- bzw, die Repräsentationsschnüre auch über
dem Uniformmantel an.
Berufssoldaten war es ab 1. März 1981 möglich,
den Sommermantel witterungsbedingt und der Vor-
schrift entsprechend in der Zeit vom 1. März bis
30.November zur Dienst-,Stabsdienst-, Ausgangs-
und Gesellschaftsuniform anzuziehen. Berufsunter-
offiziere, Fahnriche, Offiziere, Generale und Admi-
rale der NVA wurden ab 1980 mit silbergrauen
Hemdblusen in veränderter Ausführung ausgestat-
tet. Die Brusttaschen waren jetzt mit zwei silberfar-
benen Knöpfen versehen und hatten geradege-
schnittene Patten sowie eine abgesteppte ittel-
falte. Außerdem waren Armelschlaufen zum Hoch-
knöpfen der rmel angebracht.Jeweils vom 1. November bis zum 15. April konnte die Hemd.
bluse anstelle der Uniformjacke getragen werden,
allerdings nur ingeschlossenen Räumen.Ab
15.April galt diese Einschränkung nicht mehr. Vom
16.April jeden Jahres an durften die Ãrmel bei ent-
sprechenden hochsommerlichenTemperaturen
hochgeschlagen und mittels der Schlaufen befestigt
werden.
Den Vorgesetzten ab Regimentskommandeur
aufw¤rts wurde in der 1980er Vorschrift das Recht
eingeräumt, auch auBerhalb der festgelegten Zeit
das Tragen oder Mitführen des Uniformmantels zu
befehlen und in der Zeit vom 16.April bis zum
31.Oktober einheitlich für den jeweiligen Trup
penteil Festlegungen zur Trageweise der Ausgangsuniform ohne oder mit Paradejacke für Soldaten im
Grundwehrdienst sowie Soldaten und Unteroffi-
ziere auf Zeit zu treffen.
1981 entfielen an den Parade-/Ausgangsjacken
aller Dienstgradgruppen bis einschließlich Oberst
die Ãrmelpatten, Die Waffenfarben waren nun nur
noch an den Schulterklappen bzw. -stÃicken der An-
gehörigen der Landstreitkräfte, bei den Angehöri-
gen der LSK/LV und bei den Fallschirmjägern
auch an den Kragenspiegeln zu erkennen.
Weibliche Armeeangehörige durften ab 1980 die
lange Hose das ganze Jahr über tragen. Nicht ge-
stattet war es, den Kragen der Hemdbluse geöffnet
über dem Kragen der Uniformjacke zu tragen. Die
Frauen konnten unter der Weste die Hemdbluse
oder den weiBen Pullover in den Rock oder die
lange Hose ziehen.
Die Erstausstattung aller Armeeangehörigen mit
dem wattierten Felddienstanzug im Stricheldruck konnte 1981,mit Webpelzkragen 1982 abgeschlos.
sen werden.Zusammen mit der Wintermütze.die
bereits seit 1980 in einer verbesserten Qualität pro.
duziert wurde,verfügten alle Armeeangehörigen
nun über eine Felddienstuniform, die es ihnen ge
stattete,auch unter extremen Winterbedingungen
die ihnen übertragenen militärischen Aufgaben
zweckmaßig gekleidet zu bewältigen.
Parallel zur Ausstattung mit wattierten Feld.
dienstanzügen imStricheldruck undschmaler
Schnittgestaltung erfolgte auch die schrittweise
Versorgung mit Felddienstanzügen(Sommer)in
k¶rpernahér Ausführung.
Die Bekleidungsvorschrift der NVA,Ausgabe-
jahr 1980,beinhaltete auch eine ganze Reihe nur
für die Volksmarine zutreffende Anderungen. So
durften jetzt auch Offiziersschüler des 3.Lehrjahres
eine eigene weiße Uniformjacke mit weiBem Ober-
hemd und schwarzem Binder im Ausgang und im
Urlaub anziehen.Das war vordem nur Meistern
und Fahnrichen der Volksmarine gestattet. Mei-
stern,Offiziersschülern ab 3.Lehriahr, Fahnrichen
und allen weiblichen Angeh¶rigen der Volksmarine
wurde erlaubt,zum Uniformmantel in Verbindung
mit der Wintermütze einen weißen Schal einzule.
gen. Der Schal war aber nicht mehr verbindlicher
Bestandteil der Winteruniform.
Die Volksmarine hielt auch nach der Ausstattung
aller Dienstgradgruppen mit braunen Trainingsan
z¼gen an der Festlegung fest, daß Maate, Offiziers
schüler und Meister an denrainingsanzugen
einen gewebten goldfarbenen klaren Anker auf ova-
ler Unterlage aus blauem Stoff 2 cm unter dem
ASV-Abzeichen zu tragen haben. Fahnriche und
Offiziere der Volksmarine hatten den gleichen An.
ker, jedoch mit einer goldfarbenen Umrandung.
Neu aufgenommen in die Vorschrift wurden die
Dienstgradabzeichen für die Angehörigen der Flie
gerkräfte der Volksmarine am Flieger- und Techni.
keranzug.Diese Dienstgradabzeichen waren analog
den Dienstgradabzeichen der Angehörigen der
LSK/LV anzubringen.Im Unterschied zu diesen
waren die Tressen iedoch nicht mattsilbergrau, sondern goldfarben, und die Streifenanordnung gestal-
tete sich anders. (Siehe obige Tabelle)
Eingang in die Vorschrift fand auch der neuge-
staltete rmelstreifen für die Angeh¶rigen des
Wachregiments in Berlin. Anläßlich des National-
feiertages der DDR am 7.Oktober 1980 war dem
Wachregiment der Ehrenname <Friedrich Engels»
verlichen worden.In derselben Kaserne im Zen-
trum Berlins,in der das Wachregiment unterge-
bracht ist, hatte Friedrich Engels 1841/42 als Bom-
bardier bei der preußischen Gardeartillerie prakti-
sche militärische Erfahrungen gesammelt.Der
Vorsitzende des Nationalen Verteidigungsrates
stimmte am 4.Februar 1980 dem Vorschlag des Mi-
nisters für Nationale Verteidigung zu, am 7. Oktober 1980 einen rmelstreifen mit der Aufschrift
«Wachregiment Friedrich Engels» für die Angehö-
rigen dieses Truppenteils einzuführen.