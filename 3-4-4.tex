

Neben der Felddienstbekleidung sind es die Aus-
gangsuniformen, die wohl die umfangreichste Wei-
terentwicklung seit Gründung der Nationalen
Volksarmee im Jahre 1956 erfahren haben, Sie pr¤-
gen vor allem das äuBere Erscheinungsbild der
NVA-Angehörigen in der Öffentlichkeit und geben
diesen die Möglichkeit, gut gekleidet am gesell-
schaftlichen und kulturellen Leben des Standortes
oder des Heimatortes teilzunehmen. Daß gerade
zum 25.Jahrestag der DDR die Ausgangsuniformen
der Berufssoldaten und zum 30,Geburtstag der Re-
publik die der Soldaten und Unteroffiziere ent-
scheidend verbessert wurden, wird als symbolische
Geste der Verbundenheit des Volkes und seiner
Soldaten verstanden.
Entsprechend der Bekleidungsvorschrift von
1986 haben Soldaten im Grundwehrdienst, Unter-
offiziersschüler,Soldaten und Unteroffiziere auf
Zeit und im Reservistenwehrdienst der Landstreit-
kräfte und der LSK/LV im Sommer die Möglichkeit, die Ausgangsuniform Nr, 1, Schirmmütze, Uni-
formjacke, Uniformhose, graues Oberhemd, Bin-
der, Halbschuhe und Lederkoppel oder die
Ausgangsuniform Nr.2,dann jedoch ohne Uni-
formjacke und Binder, dafür mit Schulterklappen
auf dem grauen Oberhemd, zu wahlen.
Die Ausgangsuniform Nr,3 für die Ãbergangspe-
riode entspricht in der Zusammenstellung der Aus-
gangsuniform Nr,1. Dazu werden der Uniformman-
tel mit dem schwarzen Lederkoppel und die
Wirkhandschuhe angezogen, Im Winter kommt zur
Ausgangsuniform Nr,4 die Wintermütze hinzu. Es
muß angemerkt werden, daß seit den 80er .Jahren
auch für die Uniformen der $oldaten und Unterof-
fiziere ein hochwertigeres, in der Struktur feineres
Uniformtuch eingesetzt wird, so daß sich auch von dieser Seite her das Aussehen der Uniformen ver
bessert hat.
Die Ausgangsuniformen der Berufssoldaten blie
ben seit dem Jahr 1974, dem Einführungsjahr der
Uniform offener Fasson, bis auf den Wegfall der
Ärmelpatten unverändert. 1986 gab es eine nde.
rung bei den Uniformmänteln, deren Kragen seit.
dem vereinheitlicht aus dem Grundgewebe des
Mantels hergestellt werden. Damit fielen bei den
Luftstreitkräften und den Fallschirmjägern die Kra-
genspiegel an den Uniformmänteln weg. Die Uni.
formmäntel der Generale veränderten sich nicht.
Die Ausgangsuniform Nr.1 für den Sommer setzt
sich bei den Berufssoldaten aus Uniformjacke, Uni.
formhose,weißem Hemd oder weißer Hemdbluse
Binder und schwarzen Halbschuhen zusammen. 【m
Sommer und in der Ubergangsperiode geh'rt im
mer die Schirmmütze dazu.Zur Ausgangsuniform
Nr.1 und Nr.2(Sommer)kann der Sommermantel
getragen werden.Das bedeutet allerdings bei der
Ausgangsuniform Nr.2, die ohne Uniformjacke ist,
daß der Kragen der weißen Hemdbluse geschlossen
sein muß und der Binder getragen wird. Zur Aus
gangsuniform in den Ubergangszeiten kommt wahl.
weise der Uniformmantel (Nr,3) oder der Sommer
mantel(Nr.4)hinzu.Darunter werden alle
Uniformstücke wie bei der Ausgangsuniform Nr. 1
getragen.ZumUniformmantel müssen,zum Som
mermantel k'nnen Lederhandschuhe ibergestreift
werden.
Zur Ausgangsuniform Nr,5 z¤hlen Wintermütze
Uniformmantel,Uniformjacke,Uniformhose,wei
Bes Hemd bzw.weiBe Hemdbluse,Binder, Leder
handschuhe und wahlweise Halbschuhe oder Zug
stiefel.So gekleidet gehen Berufssoldaten in den
Ausgang und in Urlaub,zu Kulturveranstaltungen.
Empfängen und feierlichen Anlässen.
Die Ausgangsuniform für Generale entspricht in
der Zusammenstellung der vorschriftsmäßigen Aus-
gangsuniform der Berufssoldaten fǔr den Sommer.
die Ubergangszeiten und den Winter.
Weibliche Armeeangehörige brauchen im Aus
gang auf modischen Chic nicht zu verzichten. Zu allen ,Jahreszeiten ist es ihnen möglich, sich adrett
und zweckmäßig zu kleiden. hre Ausgangsuniform
kann, wie in der nebenstehenden Tabelle darge-
stellt, variiert werden.
Matrosen im Grundwehrdienst, Unteroffiziers-
schüler und Matrosen und Maate auf Zeit der
Volksmarine besitzen für Ausgang und Urlaub die
marinetypische Uniform.
Grundelemente ihrer
Ausgangsuniformen sind vom 1.Mai bis zum
30.September Tellermütze mit weißem Bezug, Kie-
ler Hemd in weißer Ausführung mit Kieler Knoten,
Klapphose, Lederkoppel mit Schloß und schwarze
Halbschuhe. Vom 16. bis 30.April und vom 1.Ok-
tober bis zum 30, November werden dann die Tel
lermütze mit blauem Bezug, statt des weiBen das
blaue Kieler Hemd, Kieler Kragen, Seemanns-
hemd, Klapphose, Lederkoppel mit Schloß und
Halbschuhe getragen.Vom 1. November bis zum
15. April geh¶ren zur Ausgangsuniform der Ãberzieher und vom 1.Dezember bis Ende Februar Win-
termütze und Schal.
Die Ausgangsuniformen der Berufsunteroffiziers-
schüler,Berufsunteroffiziere, Fahnrichschüler,Of
fiziersschüler, Fahnriche, Offiziere und Admirale
der Volksmarine entsprechen in ihrer Zusammen-
stellung, allerdings in der marinetypischen Ausfüh-
rung, denen der Berufssoldaten der anderen Teil
streitkräfte, Berufssoldaten bis zum Kapitän zur
See tragen vom 1.Mai bis zum 30.$eptember weiBe,
Admirale cremefarbene Mützenbezüge, Weibliche
Angeh¶rige ergänzen ihre Ausgangsuniform im
gleichen Zeitraum mit der weißen Kappe bzw.
einem weiBen Schiffchen. Sie setzen ansonsten eine
blaue Kappe auf.