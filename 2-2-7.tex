

Die Manöver der Vereinten Streitkräfte der Teil-
nehmerstaaten des Warschauer Vertrages in der er-
sten Hälfte der 60er ,Jahre, an denen auch die NVA
teilgenommen hatte, bestätigten, daß das moderne
Gefecht den Armeeangehörigen über längere Zeit-
r¤ume hohe körperliche Anstrengungen abver-
langte und sie wiederholt zu physischen Höchstleistungen zwang.Viele junge Soldaten verfügten
schon bei Dienstantritt über eine gute Kondition,
wie sie im Achtertest zur Ãberprüfung ihrer sportli-
chen Leistungsfähigkeit nachwiesen. Viele Neuein-
berufene erfüllten die Mindestnormen, die z. B.
14 Liegestütze, 4 Klimmzüge und 3:50 Minuten
für den 1000-m-Lauf forderten, Aber immer wie-
der mußten gerade im ersten Ausbildungshalbjahr
Soldaten an das durchschnittliche Leistungsniveau
ihrer itkämpfer herangeführt werden.
Für die militärische Körperertüchtigung stand
mit etwa 7 Prozent der Gesamtausbildungszeit ein
angemessener Teil zum Training zur Verfügung.
Weitere Möglièhkeiten der Konditionierung aller
Armeeangehörigen wurden im rege betriebenen
Freizeitsport erschlossen, Dieser in der Armeesportvereinigung «Vorwärts》 organisierte Freizeitsport
gewann nicht zuletzt durch die sehr guten Ergeb
nisse von Armeesportlern bei nationalen und inter.
nationalen Sportereignissen an Anziehungskraft.
Gefördert wurde die sportliche Betätigung der
Armeeangehörigen im Dienst- und Freizeitsport
Maßnahmen des Beklei-
auch durch wirksame
dungs-und Ausrüstungsdienstes.Mit dem Befehl
Nr.17/66 vom 15.April 1966 wies der Minister fiir
Nationale Verteidigung der DDR an, mit Beginn
des Ausbildungsjahres 1966/67 eine neue einheitli.
che Sportbekleidung an alle Armeeangehörigen
auszugeben und sie in die Grundnorm aufzuneh.
men.Diese bestand aus einem gelben Sporthemd
einer roten Sporthose, einem schwarz-weiß melier.
ten Trainingsanzug fir Soldaten,Unteroffiziere
und Offiziere aus Dederonmischgewebe und einem
dunkelblauen Trainingsanzug aus Silastikgewebe
für Generale und Admirale sowie schwarzen Sport
schuhen aus Rindboxleder. Die Erstausstattung der
Sportbekleidung war kostenlos und wurde nach Be
endigung des aktiven Wehrdienstes ihren Tragern
übereignet, Auch die Reservisten der NVA erhiel.
ten ab 1.[uni 1967 die neue Sportbekleidung.
Die Kennzeichnung der Dienstgrade ande
Trainingsjacken veränderte sich jetzt. Die DV-10/5.
Ausgabe 1965, hatte bereits festgelegt, daß die Un-
teroffiziere der Landstreitkräfte und der LSK/LV
der NVA in der Mitte des linken Oberärmels de
Trainingsjacke eine 7mm breite und 10 cm lange
mattsilbergraue Tresse zu befestigen hatten. Offi
ziere beider Teilstreitkräfte trugen zwei solche Tres-
sen und die Generale zwei Tressen gleicher MaBe
in Mattgold. Der Abstand der Tressen von der
Schulternaht betrug 18 cm.
Maate und Matrosen führten auf der Sporthose
vorn links bzw.an der Trainingsjacke links in
Brusthöhe einen gewebten gelben klaren Anker au
einer ovalen blauen Stoffunterlage.Offiziere de
Volksmarine bekamen das gleiche Abzeichen.Es
war jedoch zur Kennzeichnung dieser Dienstgrad
gruppe mit einer gelben Umrandung versehen.