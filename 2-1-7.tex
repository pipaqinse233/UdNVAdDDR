

In der Uniformentwicklung der NVA wurden seit
Beginn des .Jahrzehnts die Spezialbekleidungen für
die verschiedenen militärischen Aufgaben immer
vielfaltiger.
Für die Flugzeugfuhrer der LSK/LV wurde mit
der beginnenden Einführung von berschalljagd.
flugzeugen ab 1959 eine neue Spezialbekleidung
notwendig,Um die Flugzeugführer bei möglichen
Höhen über 12 000 m unter den dort herrschenden
natürlichen Bedingungen zu schützen, waren der
Höhenschutzanzug(ein Druckanzug)und eine Si-
cherheitsausriüstung unverzichtbar.
Um eine allen Forderungen des milit¤rischen Le-
bens im Flugdienst und in den verschiedenartigsten
Witterungserscheinungen entsprechende zweckmä-
ßige Bekleidung für die Flugzeugführer zu schaffen,
wurde vom 1.August 1961 bis zum 6.Februar 1962
eine neue Fliegerkombination erfolgreich erprobt. [hrer Einführung stimmte die Leitung des Ministe-
riums für Nationale Verteidigung
der DDR am
16. März 1962 zu. Die neue dreiteilige Fliegerkom-
bination bestand aus einer einreihigen .Jacke mit
ReiBverschluß und drei Seitentaschen, in denen Pistole,Dokumente,Verband- und Notpäckchen un
terzubringen waren,sowie cinerkeilfórmigge
schnittenen Hose mit zwei Knietaschen. Jacke und
Hose waren aus steingrauem Dederonmischgewebe
gearbeitet und mit Kunstseide abgefüttert. Im Win
ter konnte ein mit Webpelz gefütterter Unteranzug
eingeknópft werden.
Mit der neuen Fliegerkombination wurden 1962
und 1963 die Flugzeugführer der Luftverteidi-
gungsdivisionen ausgerüstet,Die Flugzeugfihrer
der Jagdflieger- und Transportfliegerschule sowie
die eines Hubschraubergeschwaders trugen die Spe.
zialbekleidung alter Art noch auf.
In der Volksmarine wurde in iener Zeit auch eine
zweckm¤ßige Gefechtsuniform für maritime Bedin.
gungen entwickelt. Ende der 50er |ahre war ein Ar.
beitskreis für die Iösung des Entwicklungsthemas
xKampfanzug f'r Seestreitkräfte》 gebildet worden.
In der Zeit vom 1.Dezember 1960 bis zum 5.[anuar
1962 erfolgte die praktische Erprobung unter allen
Bedingungen des milit¤rischen Lebens an Bord von
Schiffen und Booten, beispielsweise beim Ruderge
hen, bei der bernahme von Wasserbomben oder
beim Klarmachen von Minenräumgeräten und un
ter den unterschiedlichsten Witterungseinflissen
auf See. Nach erfolgreichem Abschluß dieser Er
probungen bestätigte die Leitung des Ministeriums
füir Nationale Verteidigung der DDR am 16, M¤rz
1962 diese Gefechtsuniform der Volksmarine, die
in den nächsten ahren eingeführt wurde.
Dieser dreiteilige Kampfanzug der Angeh¶rigen
der Volksmarine wurde auf den Schiffen und Boo
ten getragen,Er setzte sich aus einer einreihigen
acke,die mit einer Kapuze versehen war,cinc
Rundbundhose und einer in die [acke cinknöpfba.
ren Schwimmweste usammen.Das Material fǔr
acke und Hose bestand aus spezialbeschichtetem
Dederongewebe.das der Schwimmweste aus Dede.
ronseide - alles in der Farbe Orange, Die Taschen
der Hose waren für Pistole,Kappmesser, Farbbeu
tel sowie Verband- und Entgiftungspäckchen vorge
sehen.Diese Gefechtsuniform wurde der Mehr
zweckfunktion als Kampfanzug. Schutzanzug vor radioaktiven, chemischen und bakteriologischen
Kampfstoffen, als Wetterschutz bei stürischer See
und als Rettungsmittel in Notfallen vollauf gerecht.
Bisher übliche Bekleidungs- und Ausrüstungsge
genstände für Boots- und Schiffsbesatzungen, wie
der Schutzanzug der chemischen Dienste, das Öl-
zeug, die Schwimmjacke «Seeb¤r» und der Vorläu-
fer des Kampfanzuges, der TS-Boot-Anzug, entfie-
len. Bedingt durch die Materialzusammensetzung,
muBte unter dem Kampfanzug von den Matrosen
und Maaten im Sommerhalbjahr der Arbeitsanzug
und von den Meistern und Offizieren die Borduni-
form sowie im Winterhalbjahr von allen Dienstgra-
den der Watteanzug getragen werden.

Auch in den Landstreitkräften der NVA wurden
zur gleichen Zeit neue bzw, veränderte spezielle Be-
kleidungen eingeführt, Nach der Erprobung, die
bereits Ende der 50er ,Jahre erfolgte, wurden seit
1961 zweiteilige, steingraue
Kombinationen aus
spezialbeschichtetem Dederongewebe vor allem an
Panzerbesatzungen, aber auch an Techniker und
Kradfahrer ausgegeben. Die Jacke war mit verdeck-
ter Knopfleiste gearbeitet und mit einem auswech-
selbaren Futter versehen.
Ebenfalls in diesen Jahren erhielten die Aufklä-
rer und - als noch sehr junge Waffengattung der
Landstreitkräfte der NVA die Fallschirmjäger
einen für ihre Bedirfnisse entwickelten Kampfan-
zug.Dieser «Kampfanzug für Aufkl¤rer», so seine
Bezeichnung, wurde in der Zeit vom 1.August 1961
bis zum 31..Januar 1962 erprobt und dann an die
Aufkl¤rer und Fallschirmjäger, aber auch an die
Angehörigen des Fallschirmdienstes der Luftstreitkräfte der NVA ausgeliefert. Damit entfielen fir sie
der Kampfanzug der Landstreitkräfte und die ein-
teilige Sprungkombination.
Dieser neue, spezielle Kampfanzug bestand aus
einer Jacke und ciner Hose aus Dederonmischge-
webe. Er war im Vierfarbentarndruck gearbeitet.
Polsterungen aus Lanonsteppwattine an den Ellbo-
gen- und Knieverstärkungen minderten die Gefahr
von Verletzungen beim Sprung, d.h. beim Aufkom-
men auf dem Erdboden. Form und Ausfihrung des
Kampfanzuges fǔr Aufkl¤rer garantierten seinem
Träger hohe Beweglichkeit, schnelle und cinfache
Handhabung,ausreichende Warmehaltung sowie
ein geringes Gewicht und Volumen. AuBerdem war
er wasserabweisend. Dieser Kampfanzug wurde im Sommerhalbjahr mit der untergezogenen Sommer-
uniform und im Winter mit dem Watteanzug getra-
gen.