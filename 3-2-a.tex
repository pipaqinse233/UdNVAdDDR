

Als Ende der 70er Jahre die NATO-Staaten dazu
übergingen, den Entspannungsprozeß durch Kon
frontationspolitik und forcierte Hochrüistung zu
torpedieren, hatte die NVA der DDR im Rahmen
der Vereinten Streitkräfte ihren nationalen Bei-
trag zur Gewahrleistung des militärstrategischen
Gleichgewichts zu leisten, der einen weit in die 80er
Jahre reichenden Modernisierungsprozeß, vor allem
der ilitärtechnik, einleitete. Er ließ sich in seinen
Dimensionen nur mit der. Einführung von Raketen
in alle Teilstreitkräfte zu Beginn der 60er ,Jahre ver-
gleichen. So begann die Aurüstung der Landstreit-
kräfte mit dem mittleren Panzer T-72 sowjetischer
Produktion,Weitere moderne Waffensysteme der
Landstreitkräfte, so die 152-mm-SFL-Haubitze, die
122-mm-SFL-Haubitze und moderne tragbare Pan-
zerabwehrkomplexe, erhöhten denGefechtswert
der Hauptteilstreitkraft beträchtlich. Der Ausbau
der Hubschrauberkr¤fte und die Einführung des
Kampfhubschraubers Mi-24D anstelle des Mi-
8TB vergrößerten die Gefechtsm¶glichkeiten der
Landstreitkräfte, Die zunehmende Luftbeweglich-
keit der Landstreitkräfte prägte das neue Profil der
mot, Schützen weiter aus, das sich zu Beginn des
Jahrzehnts mit der Indienststellung des Schitzen-
panzers BP und von Fla-Raketen herauszubilden
begonnen hatte.

Gravierende Veränderungen gab es auch bei den
LSK/LV, Die MiG-23 konnte infolge der verstell-
baren Tragflügelgeometrie und einer modernen
Bordausrüstung in variablen Höhen- und Ge-
schwindigkeitsbereichen komplizierte Kampfaufga-
ben erfüllen,Sie benötigte nur kurze Start- und
Landestrecken, hatte eine umfangreichere Bewaff-
nung und Bordausrüstung und konnte 500 km wei-
ter und 300 km/h schneller fliegen als die MiG-21. Die Truppen der Luftverteidigung erhielten Fla-
Raketen auf manövrierfähigen Selbstfahrlafetten.
In der Volksmarine waren ab Mitte 1976 mittlere
Landungsschiffe vom Typ Hoyerswerda》 zuge-
f¼hrt worden, die es nunmehr in weitaus größerem
Umfang ermöglichten, Landungskrafte mit ihrer
Technik zum Einsatz zu bringen. Mit der Indienst-
stellung der neuen Küstenschutzschiffe «Rostock»
am 25.Juli 1978 und «Berlin- Hauptstadt der
DDR》am 10. Mai 1979 verfügte die Volksmarine
über moderne, universell einsetzbare Kampfschiffe
mit großem Fahrbereich, Hauptaufgabe dieser Kü-
stenschutzschiffe war die Sicherung
der Flottenkräfte und der Seetransportmittel sowie die
Boot-Abwehr. Mit Hilfe der modernen Artillerie.
und Raketenbewaffnung konnten sie Seelandeope
rationen decken und erfolgreich gegnerische See
streitkräfte, Luftangriffsmittel und U-Boote be
kampfen.
Mit der Einführung leistungsfahigerer Bewaff
nung und Kampftechnik wuchsen Rolle und Ver
antwortung des einzelnen und der kleinen Kampf
kollektive. Der wissenschaftlich-technische Fort
schritt bedingte eine weitere Differenzierung der
militärischen Tatigkeit und forderte die Spezialisie
rung.Ende der 70er Jahre existierte in der NVA
eine groBe Anzahl verschiedener militärischer Spe
zialrichtungen,Viele Arbeiten zur Funktionsüber
wachung,Wartung und Instandhaltung von auto
matisierten Waffensystemen und moderner Militär.
technik setzten ingenieurtechnischesWissen vor
aus.
Bis 1977 gab es in der NVA aber nur Offiziere
mit einer militartechnisch ausgerichtetenInge
nieurausbildung.Eine neue,den objektiven Erfor-
dernissen angepaßte Ingenieurlaufbahn war notwen.
dig. Im November 1977wurden deshalb nach
entsprechenden Vereinbarungen mit dem Ministe-
rium für Hoch- und Fachschulwesen die ersten Un
teroffiziere und Fahnriche zu speziellen F¤hnrich
lehrgängen delegiert. Der Abschluß als Fachschul.
ingenieure versetzte die Absolventen dieser Lehr
gänge in die Lage,ingenieurtechnische und
ingenieurokonomische Aufgaben mit Sachkenntnis
und in der geforderten Oualität zu erfüllen.Im Zu
sammenhang mit dem erreichten bzw. angestrebten
höheren Qualifizierungsniveau des Fahnrichkorps
der NVA beschloB der Staatsrat der DDR am
23,Juli 1979,neue Fahnrichdienstgrade zu schaf
fen.Zu den seit 1974 vorhandenen Dienstgraden
Fahnrichschüler und Fahnrich kamen Oberfähn
rich,Stabsfähnrich und Stabsoberfähnrich hinzu.
Die Anordnung Nr.13/79 des Stellvertreters des
Ministers für Nationale Verteidigung und Chefs
R¼ckwärtige Dienste vom 8. September 1979 legte
die Kennzeichnung der Dienstgrade der Fähnriche verbindlich fest. Danach trugen Fahnrichschüler
Schulterklappen aus Uniformtuch mit einer Biesen-
umrandung in der jeweiligen Waffenfarbe. Sie wa-
ren außer an den Schmalseiten mit Litze umgeben
und wiesen ein 17 mm langes silberfarbenes, bei der
Volksmarine goldfarbenes «F» auf, Fähnriche wa-
ren an Schulterstücken aus Silberplattschnüren auf
einer Tuchunterlage in der jeweiligen Waffenfarbe,
wobei die äußeren Plattschnüre steingrau bzw. bei
der Volksmarine dunkelblau waren, und einem
Stern zu erkennen; Oberfähnriche befestigten zwei
Sterne, Stabsfähnriche drei Sterne und Stabsober-
fahnriche vier Sterne auf ihren Schulterstücken.
Die Anordnung der Sterne erfolgte im Unterschied
zu den ffzieren in einer Reihe hintereinander.
Die Sterne waren zunächst bei den F¤hnrichen der
Landstreitkräfte und der LSK/LV silberfarben und
bei der Volksmarine goldfarben. Diese Festlegung
wurde jedoch kurzfristig noch geändert. Bis zum
6.Oktober 1979 wechselten alle Fahnriche die silberfarbenen Sterne durch goldfarbene Sterne, wie
sie Offiziere hatten, aus.
Nach wie vor führten alle Fahnrichdienstgrade
ein rmelabzeichen mit dem Staatswappen der
DDR, jedoch jetzt ohne Sterne zur Kennzeichnung
des Dienstalters. Durch die Anordnung der Sterne
auf den Schulterst¼cken der Fahnriche war eine
Neuregelung zum Tragen der Dienstlaufbahnabzei-
chen erforderlich, Dazu legte die Anordnung
Nr.13/79 fest, daß Fahnriche der Landstreitkrafte
und der LSK/LV Dienstlaufbahnabzeichen aus sil-
berfarbener Stickerei auf ovaler Unterlage, mit
1 mm starker Silberkordel umrandet, zu tragen hat-
ten. Fahnriche der Volksmarine brachten Dienst-
laufbahnabzeichen auf beiden rmeln der Uni
formjacke in der Mitte des Ãrmels, 2 cm über dem
Armelstreifen.an.
F¤hnriche der LSK/LV und der Fliegerkräfte der
Volksmarine im Flieger- bzw,Technikeranzug lie-
Ben sich an mattsilbernen, bei der Volksmarine
goldfarbenen Tressen auf der Mitte der linken
Brustseite der |acke erkennen. Diese Tressen waren
10 cm lang und 9 mm breit, Ein Fahnrich trug eine,
ein Oberfähnrich zwei, ein Stabsfähnrich drei und
ein Stabsoberfahnrich vier dieser Tressen zur
Kennzeichnung seines Dienstgrades. Zus¤tzlich zu
den Schulterstücken erhielten Fahnriche der Volks-
marine, wie die Offiziere dieser Teilstreitkraft, auch
als Armelstreifen Dienstgradabzeichen.Dazu wur-
den 10 cm lange und 7 mm breite goldfarbene Tres-
sen entsprechend der Sternenanzahl auf den Schul.
terstücken an beiden Unter¤rmeln aufgenäht.